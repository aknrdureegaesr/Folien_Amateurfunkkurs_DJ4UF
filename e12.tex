% Foliensatz: "AFu-Kurs nach DJ4UF" von DK0TU, Amateurfunkgruppe der TU Berlin
% Lizenz: CC BY-NC-SA 3.0 de (http://creativecommons.org/licenses/by-nc-sa/3.0/de/)
% Autoren: Felix Baum <baum@campus.tu-berlin.de>
% Korrekturen: Lars Weiler <dc4lw@darc.de>


preamble.dk0tu.tex
\subtitle{Technik 12: \\
  Halbleiter, Diode \\[2em]}
\date{Stand 18.09.2017}
 \begin{document}

\begin{frame}
    \titlepage
    \vfill
    \begin{center}
        \ccbyncsaeu\\
        {\tiny This work is licensed under the \em{Creative Commons Attribution-NonCommercial-ShareAlike 3.0 License}.}\\[0.5ex]
         \tiny Amateurfunkgruppe der Technische Universität Berlin (AfuTUB), DKØTU
         %\includegraphics[scale=0.5]{img/DK0TU_Logo.pdf}
    \end{center}
\end{frame}


\section*{Einleitung}

\begin{frame}
  \frametitle{Halbleiter}
  \begin{itemize}
    \item Was ist ein Halbleiter? (Erinnern aus alter Technik)
  \end{itemize}
\end{frame}

\section*{Bohr}

\begin{frame}
  \frametitle{Bohr Atommodell}
  \begin{center}
    \begin{figure}
      \includegraphics[width=\textwidth,height=.75\textheight,keepaspectratio]{e12/Bohr.png}
      \attribcaption{Electron Energy Levels}{BruceBlaus}{https://en.wikipedia.org/wiki/File:Blausen_0342_ElectronEnergyLevels.png}{\ccby}
    \end{figure}
  \end{center}
\end{frame}

\begin{frame}
  \frametitle{Halbleiter}
  \begin{center}
    \begin{itemize}
      \item Heutzutage meist Silizium oder Germanium
      \item 4 Elektronen auf äußerer Schale
      \item Elektronen auf der äußersten Schale nennt man Valenzelektronen
    \end{itemize}
  \end{center}
\end{frame}

\section*{Dotierung}

\begin{frame}
  \frametitle{Dotierung}
  \begin{columns}
    \column{.5\textwidth}
    \begin{figure}
      \includegraphics[height=.6\textheight,width=\textwidth,keepaspectratio]{e12/p-dot.png}
      \attribcaption{p-dotiertes Silizium}{Markus A. Hennig; SVG-Umsetzung: Cepheiden}{https://commons.wikimedia.org/wiki/File:Schema_-_p-dotiertes_Silicium.svg}{\href{https://www.gnu.org/licenses/fdl.html}{GNU Free Documentation License}}
    \end{figure}
    \column{.5\textwidth}
    p-dotiert
    \begin{itemize}
      \item Element mit 3 Valenzelektronen in äußerster Schale
      \item z.B. Aluminium
    \end{itemize}
  \end{columns}
\end{frame}

\begin{frame}
  \frametitle{Dotierung}
  \begin{columns}
    \column{.5\textwidth}
    \begin{figure}
      \includegraphics[height=.6\textheight,width=\textwidth,keepaspectratio]{e12/n-dot.png}
      \attribcaption{n-dotiertes Silizium}{Markus A. Hennig; SVG-Umsetzung: Cepheiden}{https://commons.wikimedia.org/wiki/File:Schema_-_n-dotiertes_Silicium.svg}{\href{https://www.gnu.org/licenses/fdl.html}{GNU Free Documentation License}}
    \end{figure}
    \column{.5\textwidth}
    n-dotiert
    \begin{itemize}
      \item Element mit 5 Valenzlektronen in äußerster Schale
      \item z.B. Phosphor
    \end{itemize}
  \end{columns}
\end{frame}


%\begin{frame}
%    \frametitle{Prüfungsfrage}

%    \begin{center}
%    \begin{tabular}{l||l}\hline
%        TC501 &P-dotiertes Halbleitermaterial ist solches, das \\
%         " "  &mit einem zusätzlichen Stoff versehen wurde, der \\ \hline\hline
%         A & mehr als vier Valenzelektronen enthält.\\\hline
%         B & weniger als vier Valenzelektronen enthält. \\\hline
%         C & keine Valenzelektronen enthält. \\ \hline
%         D & genau vier Valenzelektronen enthält.\\\hline
%    \end{tabular}
% 	\end{center}
%\end{frame}

%\begin{frame}
%    \frametitle{Prüfungsfrage}

%    \begin{center}
%    \begin{tabular}{l||l}\hline
%        TC501 &P-dotiertes Halbleitermaterial ist solches, das \\
%         " "  &mit einem zusätzlichen Stoff versehen wurde, der \\ \hline\hline
%         " " & mehr als vier Valenzelektronen enthält.\\\hline
%         X & weniger als vier Valenzelektronen enthält. \\\hline
%         " " & keine Valenzelektronen enthält. \\ \hline
%         " " & genau vier Valenzelektronen enthält.\\\hline
%    \end{tabular}
% 	\end{center}
% 	Video: \url{https://www.youtube.com/watch?v=IcrBqCFLHIY}
%\end{frame}

\section*{p-n-Übergang}

\begin{frame}
  \frametitle{p-n-Übergang}
  \begin{center}
    Video: \url{https://www.youtube.com/watch?v=IcrBqCFLHIY}
  \end{center}
\end{frame}

\begin{frame}
  \frametitle{p-n-Übergang}
  \begin{center}
    \begin{figure}
      \includegraphics[height=.75\textheight,width=\textwidth,keepaspectratio]{e12/pn-Diagram.png}
      \attribcaption{pn-Diagramm}{TheNoise}{https://commons.wikimedia.org/wiki/File:Pn-junction-equilibrium-graphs.png}{\ccbysa}
    \end{figure}
  \end{center}
\end{frame}

\section*{Diode}

\begin{frame}
  \frametitle{Diode}
  \begin{center}
    \begin{itemize}
      \item Strom nur von p-dotiertem nach n-dotiertem Material, nicht andersherum
      \item In Gegenrichtung erst bei sehr großen Spannungen
      \item Durchlass bei Germanium ca $0.3V$
      \item Durchlass bei Silizium ca $0.7V$
    \end{itemize}
    \begin{figure}
      \includegraphics[width=.7\textwidth,height=.5\textheight,keepaspectratio]{e12/diode_with_electrical_symbol.png}
      \attribcaption{pn-Diode}{Raffamaiden}{https://en.wikipedia.org/wiki/File:PN_diode_with_electrical_symbol.svg}{\ccbysa}
    \end{figure}
  \end{center}
\end{frame}

\begin{frame}
  \frametitle{Aussehen}
  \begin{center}
    \begin{figure}
      \includegraphics[width=.9\textwidth,height=.75\textheight,keepaspectratio]{e12/Diodenalt2.png}
      \attribcaption{Weitere, auch ältere Bauformen von (Gleichrichter-)Dioden}{PeterFrankfurt}{https://de.wikipedia.org/wiki/Datei:Diodenalt2.png}{\ccpd}
    \end{figure}
  \end{center}
\end{frame}

\begin{frame}
  \frametitle{Diode aufgeschnitten}
  \begin{center}
    \begin{figure}
      \includegraphics[width=.9\textwidth,height=.75\textheight,keepaspectratio]{e12/Schottky_Diode_Section.jpg}
      \attribcaption{Schnitt durch eine Schottky-Diode}{Laeman}{https://de.wikipedia.org/wiki/Datei:Schottky_Diode_Section.JPG}{\ccby}
    \end{figure}
  \end{center}
\end{frame}

\begin{frame}
  \frametitle{Kennlinie Diode}
  \begin{center}
    \begin{figure}
      \includegraphics[height=.75\textheight,width=\textwidth,keepaspectratio]{e12/Kennlinie_1N4001.png}
      \attribcaption{Kennlinie aus Messdaten einer 1N4001 Diode}{Cepheiden}{https://commons.wikimedia.org/wiki/File:Dioden-Kennlinie_1N4001.svg}{\ccpd}
    \end{figure}
  \end{center}
\end{frame}

\begin{frame}
  \frametitle{Leitet die Diode?}
  \begin{center}
    \begin{figure}
      \includegraphics[width=1\textwidth,height=.75\textheight,keepaspectratio]{e12/Leit_Diode.png}
      \caption{von DB4UM}
    \end{figure}
  \end{center}
\end{frame}


\subsection*{Z - Diode}

\begin{frame}
  \frametitle{Schaltzeichen Z-Diode}
  \begin{center}
    \begin{figure}
      \includegraphics[width=.8\textwidth,height=.75\textheight,keepaspectratio]{e12/z-diode.png}
      \caption{Z-Diode}
    \end{figure}
  \end{center}
\end{frame}

\begin{frame}
  \frametitle{Kennlinie Z-Diode}
  \begin{center}
    \begin{figure}
      \includegraphics[height=.8\textheight,height=.75\textheight,keepaspectratio]{e12/Kennlinie_Z-Diode.png}
      \attribcaption{Kennlinie Z-Diode}{Biezl}{https://commons.wikimedia.org/wiki/File:Kennlinie_Z-Diode.svg}{\ccpd}
    \end{figure}
  \end{center}
\end{frame}

\begin{frame}
  \frametitle{Anwendung}
  \begin{itemize}
    \item für Squelch (benötigt ein einstellbares Offset)
  \end{itemize}
  \begin{center}
    \begin{figure}
      \includegraphics[width=\textwidth,height=.7\textheight,keepaspectratio]{e12/U-Stab-Z-Diode.jpg}
      \attribcaption{Z-Diode für Squelch}{Pmehrwald}{https://commons.wikimedia.org/wiki/File:U-Stab-Z-Diode.JPG}{\ccbysa}
    \end{figure}
  \end{center}
\end{frame}

\begin{frame}
  \frametitle{Anwendung}
  \begin{itemize}
    \item zur Spannungsstabilisierung
  \end{itemize}
  \begin{center}
    \begin{figure}
      \begin{figure}
        \includegraphics[width=\textwidth,height=.7\textheight,keepaspectratio]{e12/Spannungsstabilisierung.png}
        \attribcaption{Z-Diode zur Spannungsstabilisierung}{MovGP0}{https://de.wikipedia.org/wiki/Datei:Spannungsstabilisierung.png}{\ccpd}
      \end{figure}
    \end{figure}
  \end{center}
\end{frame}

\subsection*{Fotodiode}
\begin{frame}
  \frametitle{Fotodiode}
  \begin{columns}[c]
    \column[c]{4cm}
    \begin{center}
      \begin{figure}
        \includegraphics[width=1\textwidth,height=.25\textheight,keepaspectratio]{e12/Symbol_Photodiode.png}
        \caption{Fotodiode}
      \end{figure}
      \begin{figure}
        \includegraphics[width=1\textwidth,height=.25\textheight,keepaspectratio]{e12/Fotodio.jpg}
        \attribcaption{Fotodioden}{Ulfbastel}{https://commons.wikimedia.org/wiki/File:Fotodio.jpg}{\ccbysa}
      \end{figure}
    \end{center}
    \column{5cm}
    \begin{itemize}
      \item Ändert Widerstand abhängig von Lichteinfall
      \item Wird z.B. als Helligkeitssensor verwendet
    \end{itemize}
  \end{columns}
\end{frame}

\subsection*{LED}

\begin{frame}
  \frametitle{LED}
  \begin{itemize}
    \item LED -- light-emitting diode
    \item Leuchtet, wenn in Durchlassrichtung betrieben
    \item Betriebsspannung je nach LED-Farbe von etwa $1.5V$ -- $3.2V$
    \item Nobelpreis 2014: Galliumnitrid (blaues und damit weißes Licht)
  \end{itemize}
  \begin{center}
    \begin{figure}
      \includegraphics[height=.5\textheight,width=\textwidth,keepaspectratio]{e12/Symbol_LED.png}
      \caption{LED}
    \end{figure}
  \end{center}
\end{frame}

\begin{frame}
  \frametitle{Aufbau LED}
  \begin{columns}[c]
    \column[c]{5cm}
    \begin{center}
      \begin{figure}
        \includegraphics[width=.9\textwidth,height=.75\textheight,keepaspectratio]{e12/Uvled_highres_macro.jpg}\\
        \attribcaption{Macro image of an ultraviolet (UVA) LED}{Grapetonix}{https://commons.wikimedia.org/wiki/File:Uvled_highres_macro.jpg}{\ccby}
      \end{figure}
    \end{center}
    \column{5cm}
    Halbleiter nur ganz klein im Reflektor. Der Rest ist ``Verpackung''\\[1.5em]
    \begin{figure}
      \includegraphics[width=.75\textwidth,height=.6\textheight,keepaspectratio]{e12/Led_reflector.jpg}
      \attribcaption{Reflektor einer grünen Standard-LED}{Thomas Wydra}{https://commons.wikimedia.org/wiki/File:Led_reflector.jpg}{\ccpd}
    \end{figure}
  \end{columns}
\end{frame}

\subsection*{Optokoppler}
\begin{frame}
  \frametitle{Optokoppler}
  \begin{columns}[c]
    \column[c]{5cm}
    \begin{center}
      \begin{figure}
        \includegraphics[width=1\textwidth,height=.25\textheight,keepaspectratio]{e12/Optokoppler_Aus.png}
        \caption{Optokoppler Aus}
      \end{figure}
      \begin{figure}
        \includegraphics[width=1\textwidth,height=.25\textheight,keepaspectratio]{e12/Optokoppler_An.png}
        \caption{Optokoppler An}
      \end{figure}
    \end{center}
    \column{5cm}
    \begin{itemize}
      \item LED und Photodiode in einem Bauelement
      \item Galvanische Trennung von 2 Schaltkreisen
      \item Probleme bei Hochfrequenzschaltungen
      \item Glasfaser ist ein ``lang gezogener'' Optokoppler
    \end{itemize}
  \end{columns}
\end{frame}


\renewcommand{\refname}{Referenzen}

\hypertarget{refs}{}
\textcolor{white}{} \\ %\vspace{} geht nicht
\Large Referenzen/Links
\footnotesize

\begin{thebibliography}{}
  \bibitem{darc}  DARC Online-Lehrgang Lektion E12:
    \url{https://www.darc.de/der-club/referate/ajw/lehrgang-te/e12/}
  \bibitem{wp}    Wikipedia - Die freie Enzyklopädie:
    \url{https://de.wikipedia.org/wiki/Gleichrichter}
  \bibitem{bna}   Fragenkatalog Bundesnetzagentur Technik Klasse A:
    \url{https://www.bundesnetzagentur.de/SharedDocs/Downloads/DE/Sachgebiete/Telekommunikation/Unternehmen_Institutionen/Frequenzen/Amateurfunk/Fragenkatalog/TechnikFragenkatalogKlasseAf252rId9014pdf.pdf?__blob=publicationFile&v=3}
  \bibitem{fi}    Freie Inhalte (DK0TU):
    \url{http://www.dk0tu.de/Projekte/Freie_Inhalte/}
  \bibitem{yu}    Youtube Video:
    \url{https://www.youtube.com/watch?v=IcrBqCFLHIY}

\end{thebibliography}

% Hier könnte noch eine Kontaktfolie stehen

\end{document}

