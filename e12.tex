% Foliensatz: "AFu-Kurs nach DJ4UF" von DK0TU, Amateurfunkgruppe der TU Berlin
% Lizenz: CC BY-NC-SA 3.0 de (http://creativecommons.org/licenses/by-nc-sa/3.0/de/)
% Autoren: Felix Baum <baum@campus.tu-berlin.de>

preamble.dk0tu.tex
\subtitle{Technik 12: \\
           Halbleiter, Diode \\[2em]}
\date{Stand 8.12.2014}
 \begin{document}

\begin{frame}
    \titlepage
    \vfill
    \begin{center}
        \ccbyncsaeu\\
        {\tiny This work is licensed under the \em{Creative Commons Attribution-NonCommercial-ShareAlike 3.0 License}.}\\[0.5ex]
         \tiny Amateurfunkgruppe der Technische Universität Berlin (AfuTUB), DKØTU
         %\includegraphics[scale=0.5]{img/DK0TU_Logo.pdf}
    \end{center}
\end{frame}


%fixme Referenzen/Fußnoten-Systematik vereinheitlichen

\section*{Einleitung}

\begin{frame}
    \frametitle{Halbleiter}
      	\begin{itemize}
			\item Was ist ein Halbleiter? (Erinnern aus alter Technik)
    \end{itemize}
\end{frame}
  
\section*{Bohr}

\begin{frame}
    \frametitle{Bohr Atommodell}
    \begin{center}
        \includegraphics[width=.7\textwidth]{e12/Bohr.png}
        \footnote{\tiny \url{https://en.wikipedia.org/wiki/File:Blausen_0342_ElectronEnergyLevels.png}}
	\end{center}
\end{frame}

\begin{frame}
    \frametitle{Halbleiter}
    \begin{center}
    \begin{itemize}
			\item Heutzutage meist Silizium
			\item 4 Elektronen auf äußerer Schale
			\item Elektronen auf der äußersten Schale nennt man Valenzelektronen
    \end{itemize}
	\end{center}
\end{frame}

\section*{Dotierung}

\begin{frame}
    \frametitle{Dotierung}
      	\begin{itemize}
			\item p-dotiert
      		\begin{itemize}
				\item Element mit 3 Valenzelektronen in äußerster Schale
				\item \includegraphics[width=.65\textwidth]{e12/p-dot.png}
				\item z.B. Aluminium
        \footnote{\tiny \url{https://commons.wikimedia.org/wiki/File:Schema_-_p-dotiertes_Silicium.svg}}
    	\end{itemize}
    \end{itemize}
\end{frame}

\begin{frame}
    \frametitle{Dotierung}
      	\begin{itemize}
			\item n-dotiert
      		\begin{itemize}
				\item Element mit 5 Valenzlektronen in äußerster Schale
				 \includegraphics[width=.65\textwidth]{e12/n-dot}
				 \item z.B. Phosphor
        \footnote{\tiny \url{https://commons.wikimedia.org/wiki/File:Schema_-_n-dotiertes_Silicium.svg}}
    	\end{itemize}
    \end{itemize}
\end{frame}


\begin{frame}
    \frametitle{Prüfungsfrage}

    \begin{center}
    \begin{tabular}{l||l}\hline
        TC501 &P-dotiertes Halbleitermaterial ist solches, das \\
         " "  &mit einem zusätzlichen Stoff versehen wurde, der \\ \hline\hline
         A & mehr als vier Valenzelektronen enthält.\\\hline
         B & weniger als vier Valenzelektronen enthält. \\\hline
         C & keine Valenzelektronen enthält. \\ \hline
         D & genau vier Valenzelektronen enthält.\\\hline
    \end{tabular}
 	\end{center}
\end{frame}

\begin{frame}
    \frametitle{Prüfungsfrage}

    \begin{center}
    \begin{tabular}{l||l}\hline
        TC501 &P-dotiertes Halbleitermaterial ist solches, das \\
         " "  &mit einem zusätzlichen Stoff versehen wurde, der \\ \hline\hline
         " " & mehr als vier Valenzelektronen enthält.\\\hline
         X & weniger als vier Valenzelektronen enthält. \\\hline
         " " & keine Valenzelektronen enthält. \\ \hline
         " " & genau vier Valenzelektronen enthält.\\\hline
    \end{tabular}
 	\end{center}
 	Video: \url{https://www.youtube.com/watch?v=IcrBqCFLHIY}
\end{frame}

\section*{pn Übergang}

\begin{frame}
    \frametitle{pn Übergang}
    \begin{center}
        \includegraphics[width=.7\textwidth]{e12/pn-Diagram.png}
        \footnote{\tiny \url{https://commons.wikimedia.org/wiki/File:Pn-junction-equilibrium-graphs.png}}
	\end{center}
\end{frame}

\section*{Diode}

\begin{frame}
    \frametitle{Diode}
    \begin{center}
    \begin{itemize}
			\item Strom nur von P nach N, nicht andersherum
			\item In Gegenrichtung erst bei sehr großen Spannungen
			\item Durchlass bei Germanium ca $0.3V$
			\item Durchlass bei Silizium ca $0.7V$
    \end{itemize} " "\\
        \includegraphics[width=.7\textwidth]{e12/diode_with_electrical_symbol.png}
        \footnote{\tiny \url{https://en.wikipedia.org/wiki/File:PN_diode_with_electrical_symbol.svg}}
	\end{center}
\end{frame}

\begin{frame}
    \frametitle{Diode Aufgeschnitten}
    \begin{center}
        \includegraphics[width=1\textwidth]{e12/Schottky_Diode_Section.jpg}
        \footnote{\tiny \url{https://de.wikipedia.org/wiki/Datei:Schottky_Diode_Section.JPG}}
	\end{center}
\end{frame}

\begin{frame}
    \frametitle{Aussehen}
    \begin{center}
        \includegraphics[width=.8\textwidth]{e12/Diodenalt2.png}
        \footnote{\tiny \url{https://de.wikipedia.org/wiki/Datei:Diodenalt2.png}}
	\end{center}
\end{frame}

\begin{frame}
    \frametitle{Kennlinie}
    \begin{center}
        \includegraphics[width=.65\textwidth]{e12/Kennlinie_1N4001.png}
        \footnote{\tiny \url{https://commons.wikimedia.org/wiki/File:Dioden-Kennlinie_1N4001.svg}}
	\end{center}
\end{frame}

\begin{frame}
    \frametitle{Leitet die Diode?}
    \begin{center}
        \includegraphics[width=1\textwidth]{e12/Leit_Diode.png}         \footnote{\tiny DB4UM} \\ " "\\ " " \\ " "
	\end{center}
\end{frame}


\section*{Z - Diode}

\begin{frame}
    \frametitle{Kennlinie}
    \begin{center}
        \includegraphics[width=.7\textwidth]{e12/Kennlinie_Z-Diode.png}
        \footnote{\tiny \url{https://commons.wikimedia.org/wiki/File:Kennlinie_Z-Diode.svg}}
	\end{center}
\end{frame}

\begin{frame}
    \frametitle{Schaltzeichen}
    \begin{center}
        \includegraphics[width=.8\textwidth]{e12/z-diode.png}
	\end{center}
\end{frame}

\begin{frame}
    \frametitle{Nutzen}
      	\begin{itemize}
			\item Als Squelch (benötigt ein einstellbares Offset)
    \end{itemize}    
    \begin{center}
        \includegraphics[width=.8\textwidth]{e12/U-Stab-Z-Diode.jpg}
        \footnote{\tiny \url{https://commons.wikimedia.org/wiki/File:U-Stab-Z-Diode.JPG}}
	\end{center}
\end{frame}

\begin{frame}
    \frametitle{Nutzen}
      	\begin{itemize}
			\item Als Spannungsstabilisierung
    \end{itemize}    
    \begin{center}
        \includegraphics[width=.8\textwidth]{e12/Spannungsstabilisierung.png}
        \footnote{\tiny \url{https://de.wikipedia.org/wiki/Datei:Spannungsstabilisierung.png}}
	\end{center}
\end{frame}


\section*{Referenzen}

\begin{frame}
    \frametitle{Referenzen/Links}
    
    \footnotesize
    \begin{itemize}
        \item Moltrecht E 12: \\
              \url{http://www.dj4uf.de/lehrg/e12/e12.html}
        \item Diode (Wikipedia): \\
              \url{https://de.wikipedia.org/wiki/Diode}
        \item Z-Diode (Wikipedia): \\
              \url{https://de.wikipedia.org/wiki/Z-Diode}
    \end{itemize}

\end{frame}

% Hier könnte noch eine Kontaktfolie stehen

\end{document}

