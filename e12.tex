% Foliensatz: "AFu-Kurs nach DJ4UF" von DK0TU, Amateurfunkgruppe der TU Berlin
% Lizenz: CC BY-NC-SA 3.0 de (http://creativecommons.org/licenses/by-nc-sa/3.0/de/)
% Autoren: Felix Baum <baum@campus.tu-berlin.de>

preamble.dk0tu.tex
\subtitle{Technik 12: \\
           Halbleiter, Diode \\[2em]}
\date{Stand 19.11.2015}
 \begin{document}

\begin{frame}
    \titlepage
    \vfill
    \begin{center}
        \ccbyncsaeu\\
        {\tiny This work is licensed under the \em{Creative Commons Attribution-NonCommercial-ShareAlike 3.0 License}.}\\[0.5ex]
         \tiny Amateurfunkgruppe der Technische Universität Berlin (AfuTUB), DKØTU
         %\includegraphics[scale=0.5]{img/DK0TU_Logo.pdf}
    \end{center}
\end{frame}


%fixme Referenzen/Fußnoten-Systematik vereinheitlichen

\section*{Einleitung}

\begin{frame}
    \frametitle{Halbleiter}
      	\begin{itemize}
			\item Was ist ein Halbleiter? (Erinnern aus alter Technik)
    \end{itemize}
\end{frame}
  
\section*{Bohr}

\begin{frame}
  \frametitle{Bohr Atommodell}
    \begin{center}
      \begin{figure}
        \includegraphics[width=\textwidth,height=.8\textheight,keepaspectratio]{e12/Bohr.png}
        \attribcaption{Bohr'sches Atommodell}{BriceBlaus}{https://en.wikipedia.org/wiki/File:Blausen_0342_ElectronEnergyLevels.png}{\ccby}
      \end{figure}
	\end{center}
\end{frame}

\begin{frame}
    \frametitle{Halbleiter}
    \begin{center}
    \begin{itemize}
			\item Heutzutage meist Silizium
			\item 4 Elektronen auf äußerer Schale
			\item Elektronen auf der äußersten Schale nennt man Valenzelektronen
    \end{itemize}
	\end{center}
\end{frame}

\section*{Dotierung}

\begin{frame}
    \frametitle{p-Dotierung}

	 \begin{figure}
         \includegraphics[width=\textwidth,height=.7\textheight,keepaspectratio]{e12/p-dot.png}       
	 \attribcaption{mit Aluminium p-dotiertes Silizium (3-Valenzelektronen in äußerster Schale)}{Markus A. Henning}{https://commons.wikimedia.org/wiki/File:Schema_-_p-dotiertes_Silicium.svg}{GNU Free Documentation License}
    	 \end{figure}
\end{frame}

\begin{frame}
    \frametitle{n-Dotierung}
    \begin{figure}
	\includegraphics[height=.7\textheight]{e12/n-dot}
       \attribcaption{mit Phosphor n-dotiertes Silizium (5-Valenzelektronen in äußerster Schale)}{Markus A. Henning}{https://commons.wikimedia.org/wiki/File:Schema_-_n-dotiertes_Silicium.svg}{\ccbysa}
    \end{figure}
\end{frame}


%\begin{frame}
%    \frametitle{Prüfungsfrage}

%    \begin{center}
%    \begin{tabular}{l||l}\hline
%        TC501 &P-dotiertes Halbleitermaterial ist solches, das \\
%         " "  &mit einem zusätzlichen Stoff versehen wurde, der \\ \hline\hline
%         A & mehr als vier Valenzelektronen enthält.\\\hline
%         B & weniger als vier Valenzelektronen enthält. \\\hline
%         C & keine Valenzelektronen enthält. \\ \hline
%         D & genau vier Valenzelektronen enthält.\\\hline
%    \end{tabular}
% 	\end{center}
%\end{frame}

%\begin{frame}
%    \frametitle{Prüfungsfrage}

%    \begin{center}
%    \begin{tabular}{l||l}\hline
%        TC501 &P-dotiertes Halbleitermaterial ist solches, das \\
%         " "  &mit einem zusätzlichen Stoff versehen wurde, der \\ \hline\hline
%         " " & mehr als vier Valenzelektronen enthält.\\\hline
%         X & weniger als vier Valenzelektronen enthält. \\\hline
%         " " & keine Valenzelektronen enthält. \\ \hline
%         " " & genau vier Valenzelektronen enthält.\\\hline
%    \end{tabular}
% 	\end{center}
% 	Video: \url{https://www.youtube.com/watch?v=IcrBqCFLHIY}
%\end{frame}

\section*{pn Übergang}

\begin{frame}
    \frametitle{pn Übergang}
    \begin{center}
Video: \url{https://www.youtube.com/watch?v=IcrBqCFLHIY}
	\end{center}
\end{frame}

\begin{frame}
    \frametitle{pn Übergang}
    \begin{center}
    \begin{figure}
        \includegraphics[width=\textwidth,height=.8\textheight, keepaspectratio]{e12/pn-Diagram.png}
        \attribcaption{pn-Übergang}{TheNoise}{https://commons.wikimedia.org/wiki/File:Pn-junction-equilibrium-graphs.png}{\ccbysa}
    \end{figure}
	\end{center}
\end{frame}

\section*{Diode}

\begin{frame}
    \frametitle{Diode}
    \begin{center}
    \begin{itemize}
			\item Strom nur von P nach N, nicht andersherum
			\item In Gegenrichtung erst bei sehr großen Spannungen
			\item Durchlass bei Germanium ca $0.3V$
			\item Durchlass bei Silizium ca $0.7V$
    \end{itemize} " "\\
	\begin{figure}
        \includegraphics[width=.7\textwidth,height=.8\textheight,keepaspectratio]{e12/diode_with_electrical_symbol.png}
        \attribcaption{pn-Diode}{Raffamaiden}{https://en.wikipedia.org/wiki/File:PN_diode_with_electrical_symbol.svg}{\ccbysa}
        \end {figure}
	\end{center}
\end{frame}

\begin{frame}
    \frametitle{Diode Aufgeschnitten}
    \begin{center}
	\begin{figure}
        \includegraphics[width=1\textwidth,height=.8\textheight,keepaspectratio]{e12/Schottky_Diode_Section.jpg}
        \attribcaption{aufgeschnittene Schottky-Diode}{Laeman}{https://de.wikipedia.org/wiki/Datei:Schottky_Diode_Section.JPG}{\ccby}
	\end{figure}
	\end{center}
\end{frame}

\begin{frame}
    \frametitle{Aussehen}
    \begin{center}
	\begin{figure}
        \includegraphics[width=.8\textwidth,height=.8\textheight,keepaspectratio]{e12/Diodenalt2.png}
        \attribcaption{Dioden}{PeterFrankfurt}{https://de.wikipedia.org/wiki/Datei:Diodenalt2.png}{\ccpd}
     \end{figure}
	\end{center}
\end{frame}

\begin{frame}
    \frametitle{Kennlinie}
    \begin{center}
        \begin{figure}
        \includegraphics[width=\textwidth,height=.8\textheight,keepaspectratio]{e12/Kennlinie_1N4001.png}
        \attribcaption{Kennlinie einer Diode}{Cepheiden}{https://commons.wikimedia.org/wiki/File:Dioden-Kennlinie_1N4001.svg}{\ccpd}
	\end{figure}
	\end{center}
\end{frame}

\begin{frame}
    \frametitle{Leitet die Diode?}
    \begin{center}
        \begin{figure}
        \includegraphics[width=1\textwidth,height=.6\textheight]{e12/Leit_Diode.png}
        \caption{Welche Diode leite? DB4UM}
	\end{figure}
	\end{center}
\end{frame}


\section*{Z - Diode}

\begin{frame}
    \frametitle{Schaltzeichen}
    \begin{center}
      \begin{figure}
        \includegraphics[width=.8\textwidth]{e12/z-diode.png}
        \caption{Z-Diode}
      \end{figure}
	\end{center}
\end{frame}

\begin{frame}
    \frametitle{Kennlinie Z-Diode}
    \begin{center}
	\begin{figure}
        \includegraphics[width=\textwidth,height=.8\textheight,keepaspectratio]{e12/Kennlinie_Z-Diode.png}
        \attribcaption{Kennlinie einer Z-Diode}{Biezl}{https://commons.wikimedia.org/wiki/File:Kennlinie_Z-Diode.svg}{\ccpd}
	\end{figure}
	\end{center}
\end{frame}

\begin{frame}
    \frametitle{Nutzen}
      	\begin{itemize}
			\item Als Squelch (benötigt ein einstellbares Offset)
    \end{itemize}    
    \begin{center}
	\begin{figure}
        \includegraphics[width=.8\textwidth,height=.8\textheight,keepaspectratio]{e12/U-Stab-Z-Diode.jpg}
        \attribcaption{Vewendung einer Z-Diode}{Pmehrwald}{https://commons.wikimedia.org/wiki/File:U-Stab-Z-Diode.JPG}{\ccbysa}
	\end{figure}
	\end{center}
\end{frame}

\begin{frame}
    \frametitle{Nutzen}
      	\begin{itemize}
			\item Als Spannungsstabilisierung
    \end{itemize}    
    \begin{center}
	\begin{figure}
        \includegraphics[width=.8\textwidth,height=.6\textheight,keepaspectratio]{e12/Spannungsstabilisierung.png}
        \attribcaption{Spannungsstabilisierung}{MovGP0}{https://de.wikipedia.org/wiki/Datei:Spannungsstabilisierung.png}{\ccpd}
	\end{figure}
	\end{center}
\end{frame}

\begin{frame}
    \frametitle{Fotodiode}
    \begin{columns}[c]
        \column[c]{5cm}
        \begin{center}
            \includegraphics[width=1\textwidth]{e12/Symbol_Photodiode.png}\\
            \includegraphics[width=0.8\textwidth]{e12/Fotodio.jpg}
            \tiny \hyperlink{refs}{\cite{wm}}
        \end{center}
        \column{5cm}
    \begin{itemize}
			\item Ändert Widerstand abhängig von Lichteinfall 
			\item Wird z.B. als Helligkeitssensor verwendet
    \end{itemize}
    \end{columns}
\end{frame}

\section*{LED}

\begin{frame}
    \frametitle{LED}
      	\begin{itemize}
			\item LED - light-emitting diode
			\item Leuchtet wenn in Durchlassrichtung betrieben
			\item Betriebsspannung je nach LED-Farbe von etwa $1.5V$ - $3.2V$
            \item Nobelpreis 2014: galium nitrid (blaues Licht)
    \end{itemize}    
    \begin{center}
	\begin{figure}
        \includegraphics[height=.5\textheight]{e12/Symbol_LED.png}
	\end{figure}
	\end{center}
\end{frame}

\begin{frame}
    \frametitle{Diode Aufbau}
    \begin{columns}[c]
        \column[c]{5cm}
        \begin{center}
	\begin{figure}
            \includegraphics[width=0.7\textwidth]{e12/Uvled_highres_macro.jpg}\\
            \tiny \hyperlink{refs}{\cite{wm}}
	\end{figure}
        \end{center}
        \column{5cm}
            Halbleite nur ganz klein im Reflektor. Der Rest ist "Verpackung"\\ 
	 \begin{figure}
            \includegraphics[width=0.7\textwidth]{e12/Led_reflector.jpg}
	    \caption{}
	\end{figure}
    \end{columns}
\end{frame}

\begin{frame}
    \frametitle{Optokoppler}
    \begin{columns}[c]
        \column[c]{5cm}
        \begin{center}
            \includegraphics[width=1\textwidth]{e12/Optokoppler_Aus.png}\\
            \includegraphics[width=1\textwidth]{e12/Optokoppler_An.png}
            \tiny \hyperlink{refs}{\cite{wm}}
        \end{center}
        \column{5cm}
    \begin{itemize}
			\item LED und Photodiode in einem Bauelement
			\item Galvanische Trennung von 2 Schaltkreisen
			\item Probleme bei Hochfrequenzschaltungen
    \end{itemize}
    \end{columns}
\end{frame}


\renewcommand{\refname}{Referenzen}

\hypertarget{refs}{}
\textcolor{white}{} \\ %\vspace{} geht nicht
\Large Referenzen/Links
\footnotesize

\begin{thebibliography}{}
    \bibitem{darc}  DARC Online-Lehrgang Lektion E12:
                    \url{http://www.darc.de/referate/ajw/ausbildung/darc-online-lehrgang/technik-klasse-e/technik-e12/}
    \bibitem{wm} 	Wikimedia:
                    \url{https://en.wikipedia.org/wiki/File:PN_diode_with_electrical_symbol.svg}
                    \url{https://commons.wikimedia.org/wiki/File:Pn_Junction_Diffusion_and_Drift.svg}
                    \url{http://commons.wikimedia.org/wiki/File:AusfuerungsformenSchottkyDiode.png}
                    \url{http://de.wikipedia.org/wiki/Datei:Diode-Schottky-EN_A-K.svg}
                    \url{https://commons.wikimedia.org/wiki/File:Uvled_highres_macro.jpg}
                    \url{}
    \bibitem{wp}    Wikipedia - Die freie Enzyklopädie:
                    \url{https://de.wikipedia.org/wiki/Gleichrichter}
	\bibitem{bna}   Fragenkatalog Bundesnetzargentur Technik Klasse A:                   
                    \url{https://www.bundesnetzagentur.de/SharedDocs/Downloads/DE/Sachgebiete/Telekommunikation/Unternehmen_Institutionen/Frequenzen/Amateurfunk/Fragenkatalog/TechnikFragenkatalogKlasseAf252rId9014pdf.pdf?__blob=publicationFile&v=3}
    \bibitem{fi}    Freie Inhalte (DK0TU):
                    \url{http://www.dk0tu.de/Projekte/Freie_Inhalte/}
     \bibitem{yu}    Youtube Video:                   
                    \url{https://www.youtube.com/watch?v=IcrBqCFLHIY}

\end{thebibliography} 

% Hier könnte noch eine Kontaktfolie stehen

\end{document}

