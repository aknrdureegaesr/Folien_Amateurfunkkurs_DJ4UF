% Foliensatz: "AFu-Kurs nach DJ4UF" von DK0TU, Amateurfunkgruppe der TU Berlin
% Lizenz: CC BY-NC-SA 3.0 de (http://creativecommons.org/licenses/by-nc-sa/3.0/de/)
% Autoren: Felix Baum <baum@campus.tu-berlin.de>
% Korrekturen: Lars Weiler <dc4lw@darc.de>, Sebastian Lange <dl7bst@dk0tu.de>

preamble.dk0tu.tex
\subtitle{Technik Klasse E 06: \\
          Spule und Transformator \\[2em]}
\date{Stand 12.11.2015}
 \begin{document}

\begin{frame}
    \titlepage
    \vfill
    \begin{center}
        \ccbyncsaeu\\
        {\tiny This work is licensed under the \em{Creative Commons Attribution-NonCommercial-ShareAlike 3.0 License}.}\\[0.5ex]
         \tiny Amateurfunkgruppe der Technische Universität Berlin (AfuTUB), DKØTU
         %\includegraphics[scale=0.5]{img/DK0TU_Logo.pdf}
    \end{center}
\end{frame}


%FIXME Referenzen/Fußnoten-Systematik vereinheitlichen

\section*{Einleitung}

\begin{frame}
    \frametitle{Einleitung / Spule}
    \begin{center}
        \Large{Wofür braucht man Spulen?} \\
    \end{center}
\end{frame}

\begin{frame}
  \frametitle{Diverse Anwendungsmöglichkeiten}
  \begin{itemize}
    \item Erzeugung von Magnetfeldern
    \item Detektion von Magnetfeldern
    \item Transformator
    \item Relais
    \item Elektromotor
    \item Lautsprecher
    \item Mikrofon
    \item LC-Schwingkreis
    \item Tief-, Hoch-, Bandpass
    \item Stromflussglättung
    \item Energiespeicher
    \item Drossel
    \item NFC, RFID-Transponder
  \end{itemize}
\end{frame}

\begin{frame}
    \frametitle{Einleitung / Spulen-Beispiele}

    \begin{center}
        \includegraphics[height=0.8\textheight]{e06/Spule.jpg}
        \footnote{\tiny \url{https://commons.wikimedia.org/wiki/File:Electronic_component_inductors.jpg}}
    \end{center}
 	
\end{frame}

\begin{frame}
    \frametitle{Einleitung / SMD-Spule}

    \begin{center}
        \includegraphics[height=0.8\textheight]{e06/smd-Spule.png}
        \footnote{\tiny \url{https://de.wikipedia.org/wiki/Datei:Planar_core_assembly_exploded.png}}
    \end{center}
 	
\end{frame}

\section*{Induktivitäten}
\begin{frame}
    \frametitle{Induktivität}
	
	\begin{center}
        \includegraphics[width=1\textwidth]{e06/H-Feld.png}
        \footnote{\tiny \url{https://commons.wikimedia.org/wiki/File:VFPt_Solenoid_correct2.svg}}
    \end{center}
    
\end{frame}

\begin{frame}
    \frametitle{3D Magnetisches Feld}
    \begin{center}
        \includegraphics[width=1\textwidth]{e06/3-D_HFeld.png}
        \footnote{\tiny \url{https://commons.wikimedia.org/wiki/File:Solenoid.png}}
    \end{center}

\end{frame}

\section*{Zylinderspule}

\begin{frame}
    \frametitle{Berechnung der Induktivität}
    \begin{center}
      $$L = \frac{\mu \cdot A}{l}\cdot N^2$$
        \includegraphics[width=1\textwidth]{e06/Luftspule.png}
    \end{center}
\end{frame}

\begin{frame}
    \frametitle{Prüfungsfrage}

    \begin{center}
    \begin{tabular}{l||l}\hline
        TC302 & Wie ändert sich die Induktivität einer Spule \\
         " "  & von $12 \mu H$ wenn die Wicklung auf dem \\ 
         " "  & Wickelkörper bei gleicher Windungszahl auf\\
           " "  & den doppelten Wert auseinander gezogen wird? \\\hline\hline
        A & Die Induktivität sinkt auf $3 \mu H$. \\ \hline
        B & Die Induktivität steigt auf $48 \mu H$. \\ \hline
        C & Die Induktivität steigt auf $24 \mu H$. \\ \hline
        D & Die Induktivität sinkt auf $6 \mu H$. \\ \hline
    \end{tabular}
 	    \end{center}
\end{frame}

\begin{frame}
    \frametitle{Prüfungsfrage}

    \begin{center}
    \begin{tabular}{l||l}\hline
        TC302 & Wie ändert sich die Induktivität einer Spule \\
         " "  & von $12 \mu H$ wenn die Wicklung auf dem \\ 
         " "  & Wickelkörper bei gleicher Windungszahl auf\\
           " "  & den doppelten Wert auseinander gezogen wird? \\\hline\hline
        " " & Die Induktivität sinkt auf $3 \mu H$. \\ \hline
        " " & Die Induktivität steigt auf $48 \mu H$. \\ \hline
        " " & Die Induktivität steigt auf $24 \mu H$. \\ \hline
        X & Die Induktivität sinkt auf $6 \mu H$. \\ \hline
    \end{tabular}
 	    \end{center}
\end{frame}

\begin{frame}
    \frametitle{Prüfungsfrage}

    \begin{center}
    \begin{tabular}{l||l}\hline
        TC303 & Wie kann man die Induktivität\\
         " "  & einer Spule vergrößern?\\\hline\hline
        A & Durch Auseinanderziehen der Spule \\
        " " & (Vergrößerung der Spulenlänge). \\ \hline
    	B & Durch Stauchen der Spule \\
        " " & (Verkürzen der Spulenlänge). \\ \hline
        C &  Durch Einführen eines Kupferkerns in die Spule. \\ \hline
        D & Durch Einbau der Spule in einen Abschirmbecher. \\ \hline
    \end{tabular}
 	    \end{center}
\end{frame}

\begin{frame}
    \frametitle{Prüfungsfrage}

    \begin{center}
    \begin{tabular}{l||l}\hline
        TC303 & Wie kann man die Induktivität\\
         " "  & einer Spule vergrößern?\\\hline\hline
        " " & Durch Auseinanderziehen der Spule \\
        " " & (Vergrößerung der Spulenlänge). \\ \hline
    	X & Durch Stauchen der Spule \\
        " " & (Verkürzen der Spulenlänge). \\ \hline
        " " &  Durch Einführen eines Kupferkerns in die Spule. \\ \hline
        " " & Durch Einbau der Spule in einen Abschirmbecher. \\ \hline
    \end{tabular}
 	    \end{center}
\end{frame}

\section*{Schaltungen mit Spulen}

\begin{frame}
    \frametitle{Schaltzeichen}
    \begin{center}
        \includegraphics[width=1\textwidth]{e06/Spule-schaltZ.png}
    \end{center}
\end{frame}

\begin{frame}
    \frametitle{Reihenschaltung}
    
    $$L_{gesamt} = L_1 + L_2 + L_3 + ...$$

    
    \begin{center}
        \includegraphics[width=1\textwidth]{e06/L_Reihe.png}
    \end{center}
	
\end{frame}

\begin{frame}
    \frametitle{Parallelschaltung}
    $$\frac{1}{L_{gesamt}} = \frac{1}{L_1} + \frac{1}{L_2} + \frac{1}{L_3} + ...$$
    \begin{center}
        \includegraphics[width=1\textwidth]{e06/L_Parallel.png}
    \end{center}
\end{frame}

\section*{Blindwiderstand}

\begin{frame}
    \frametitle{Spule bei Wechselspannung (Blindwiderstand)}
    Für komplexe Widerstände gilt allgemein:
    $$Z = R + iX$$
    
    Das X berechnet sich bei der Spule durch:
    $$X_L = 2\cdot \pi \cdot f\cdot L $$

\end{frame}

\begin{frame}
    \frametitle{Prüfungsfrage}    
    \begin{center}
    \begin{tabular}{l||l}\hline
        TC306 & Mit zunehmender Frequenz\\\hline\hline
        A & sinkt der Wechselstromwiderstand der Spule bis \\
        " " & zu einem Minimum und steigt dann wieder. \\ \hline
    	B & steigt der Wechselstromwiderstand der Spule bis\\
        " " & zu einem Maximum und sinkt dann wieder. \\ \hline
        C &  steigt der Wechselstromwiderstand der Spule. \\ \hline
        D & sinkt der Wechselstromwiderstand der Spule. \\ \hline
    \end{tabular}
 	    \end{center}
\end{frame}

\begin{frame}
    \frametitle{Prüfungsfrage}    
    \begin{center}
    \begin{tabular}{l||l}\hline
        TC306 & Mit zunehmender Frequenz\\\hline\hline
        " " & sinkt der Wechselstromwiderstand der Spule bis \\
        " " & zu einem Minimum und steigt dann wieder. \\ \hline
    	" " & steigt der Wechselstromwiderstand der Spule bis\\
        " " & zu einem Maximum und sinkt dann wieder. \\ \hline
        X &  steigt der Wechselstromwiderstand der Spule. \\ \hline
        " " & sinkt der Wechselstromwiderstand der Spule. \\ \hline
    \end{tabular}
 	\end{center}
\end{frame}

\section*{Transformator}
\begin{frame}
    \frametitle{Transformator}
        \begin{center}
        \includegraphics[width=0.9\textwidth]{e06/trafo-Real.jpg}
        \footnote{\tiny \url{https://commons.wikimedia.org/wiki/File:Trafo_6.jpg}}
         \end{center}
\end{frame}

\begin{frame}
    \frametitle{Transformator}
    $$\text{\"u} = \frac{N_1}{N_2} = \frac{U_1}{U_2}$$
    \begin{center}
        \includegraphics[width=.7\textwidth]{e06/Trafo.png}
    \end{center}
\end{frame}

\begin{frame}
    \frametitle{Prüfungsfrage}
    
    \begin{center}
    \begin{tabular}{l||l}\hline
        TC306 & Ein Trafo liegt an 230 Volt und gibt $11,5V$\\
        " "  & ab. Seine Primärwicklung hat 600 Windungen. \\ 
         " "  & Wie groß ist seine Sekundärwindungszahl?\\\hline\hline
        A & 20 Windungen\\ \hline
    	B & 180 Windungen \\ \hline
        C &  30 Windungen \\ \hline
        D & 520 Windungen \\ \hline
    \end{tabular}
 	    \end{center}
\end{frame}

\begin{frame}
    \frametitle{Prüfungsfrage}
    
    \begin{center}
    \begin{tabular}{l||l}\hline
        TC306 & Ein Trafo liegt an 230 Volt und gibt $11,5V$\\
        " "  & ab. Seine Primärwicklung hat 600 Windungen. \\ 
         " "  & Wie groß ist seine Sekundärwindungszahl?\\\hline\hline
        " " & 20 Windungen\\ \hline
    	" " & 180 Windungen \\ \hline
        X &  30 Windungen \\ \hline
        " " & 520 Windungen \\ \hline
    \end{tabular}
 	    \end{center}
\end{frame}

\section*{Referenzen}

\begin{frame}
    \frametitle{Referenzen/Links}
    
    \footnotesize
    \begin{itemize}
        \item Moltrecht E 06: \\
              \url{http://www.darc.de/referate/ajw/ausbildung/darc-online-lehrgang/technik-klasse-e/technik-e06/}
         \item WP Trafo: \\
              \url{https://de.wikipedia.org/wiki/Transformator}
 		\item WP Spule: \\
              \url{https://de.wikipedia.org/wiki/Spule_(Elektrotechnik)}
    \end{itemize}

\end{frame}

% Hier könnte noch eine Kontaktfolie stehen

\end{document}

