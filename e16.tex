% Foliensatz: "AFu-Kurs nach DJ4UF" von DK0TU, Amateurfunkgruppe der TU Berlin
% Lizenz: CC BY-NC-SA 3.0 de (http://creativecommons.org/licenses/by-nc-sa/3.0/de/)
% Autoren: Sebastian Lange <dl7bst@dk0tu.de>

preamble.dk0tu.tex
\subtitle{Technik Klasse E 16 \& Betriebstechnik/Vorschriften 12: \\
          (Digitale) Betriebsarten \\[2em]}
\date{Stand 14.01.2016}
 \begin{document}

\begin{frame}
    \titlepage
    \vfill
    \begin{center}
        \ccbyncsaeu\\
        {\tiny This work is licensed under the \em{Creative Commons Attribution-NonCommercial-ShareAlike 3.0 License}.}\\[0.5ex]
         \tiny Amateurfunkgruppe der Technische Universität Berlin (AfuTUB), DKØTU
         %\includegraphics[scale=0.5]{img/DK0TU_Logo.pdf}
    \end{center}
\end{frame}


% todo Gern noch mehr Bilder - es ist ein schönes praktisches Thema.
% todo bei viel Text Buzzwords fett hervorheben oder Text kürzen
% todo da der Talk ca. 2h lang ist: Guten Pausenzeitpunkt finden
% todo Eine schöne Datenbank an Digisounds: http://www.nonstopsystems.com/radio/radio-sounds.html
% todo Audioquiz am Ende
% todo Spektrum nochmal mit Audacity/GNU Radio

\section{Einleitung}

\begin{frame}
    \frametitle{Einleitung / Umleitung}

    Aufgrund sehr großer inhaltlicher Überschneidungen der beiden
    \emph{Moltrecht}-Lektionen, ist die Lektion
    \texttt{BV12}\hyperlink{refs}{\cite{bv12}} in diesen Foliensatz der Lektion
    \texttt{Technik E16}\hyperlink{refs}{\cite{e16}} integriert. \\[2em]

    Freut euch auf eine Lektion ``Klingeln in den Ohren'' :-)

\end{frame}

\begin{frame}
    \frametitle{Einleitung / Betriebsarten}

    Grundsätzlich unterscheidet man zwischen:

    \begin{itemize}
		\item analoge Betriebsarten
		\item digitale Betriebsarten (Digimodes)
    \end{itemize}

    Vorweg: In der \emph{Klasse E} liegt der Fokus auf das Kennenlernen und
    betriebstechnische Grundlagen. Die technischen Grundlagen werden dann im
    Technikteil der \emph{Klasse A} ausgebaut.

\end{frame}

\section{analog}

\subsection[Sprechfunk]{Sprechfunk (Wiederholung)}

\begin{frame}
    \frametitle{Sprechfunk (Wiederholung)}

    In Lektion \texttt{E14} wurde das Thema bereits besprochen.

    Der Vollständigkeit halber seien sie noch einmal erwähnt - auch weil die
    meisten Digimodes im Amateurfunk \emph{SSB} oder \emph{FM} als Hauptträger
    benutzen.

\end{frame}

\begin{frame}
    \frametitle{Sprechfunk (Wdh.) / SSB}

    \begin{center}
        \includegraphics[width=0.8\textwidth,height=.65\textheight,keepaspectratio]{e16/Ssb-de.png}
        \tiny \hyperlink{refs}{\cite{wc}}
    \end{center}

    Durchgängiges (da analoges) Spektrum ca. $300 Hz$ bis $3 kHz$

\end{frame}

\begin{frame}
    \frametitle{Sprechfunk (Wdh.) / FM}

    \begin{center}
        \includegraphics[width=0.8\textwidth,height=.55\textheight,keepaspectratio]{e16/Dpx-fm-radio.png}
        \tiny \hyperlink{refs}{\cite{wc}}
    \end{center}

    Je $2x$ Hub ($3 kHz$) + NF-Signal ($2,7 kHz$) um Träger herum
    (durchgängiges Spektrum) \\[1em]
    
    $\rightarrow$ Gesamtbandbreite eines NBFM-Signals ca. $12 kHz$ ($12,5 kHz$
    Kanalraster)

\end{frame}

% FIXME Hellschreiber?

\subsection[SSTV]{Slow Scan Television (SSTV)}

\begin{frame}
    \frametitle{SSTV / Historie}

    \textbf{S}low \textbf{S}can \textbf{T}elevision\\[1.5em]

%    \begin{center}
%        \includegraphics[width=0.6\textwidth]{e16/Mechanical_glow_drum_slow_scan_television_monitor.jpg}
%        \tiny \hyperlink{refs}{\cite{wc}}
%    \end{center}

    \begin{columns}
      \column[c]{.45\textwidth}
       \includegraphics[width=\textwidth,height=.4\textheight,keepaspectratio]{e16/Mechanical_glow_drum_slow_scan_television_monitor.jpg}
       \tiny \hyperlink{refs}{\cite{wc}}
      \column[c]{.45\textwidth}
       \includegraphics[width=\textwidth,height=.4\textheight,keepaspectratio]{e16/SSTV-RS0ISS_2015-07-19.jpg}
    \end{columns}


    \begin{itemize}
        \item 1958 durch US-HAMs veröffentlicht
        \item Übertragung von Bildern in $3 kHz$ SSB-Kanal
        \item damals noch Darstellung auf Katodenstrahlröhre mit hoher Nachleuchtdauer
        \begin{itemize}
            \item 120x120 Bildpunkte (s/w) in acht Sekunden
            \item \emph{elektromechanischer SSTV-Empfänger} siehe \texttt{GIF}
            \item \emph{elektromechanischer SSTV-Empfänger} siehe \texttt{GIF}
                  \footnote{\url{https://upload.wikimedia.org/wikipedia/commons/1/1e/Mechanical_glow_drum_slow_scan_television_monitor.gif}}
        \end{itemize}
    \end{itemize}

\end{frame}

\begin{frame}
    \frametitle{SSTV / Technik}

    \begin{center}
        \includegraphics[width=0.4\textwidth]{e16/Sstv_frequences.png}
        \includegraphics[width=0.4\textwidth]{e16/SSTV_signal.jpg}
        \tiny \hyperlink{refs}{\cite{wc}}
    \end{center}

    \begin{itemize}%[<+->]
        \item heute Software-MODEM via Soundkarte
        \begin{itemize}
            \item Quasi-Standard: 320x240 Bildpunkte (RGB) in 120s
            \item zunehmend ziehen digitale Verfahren mit Fehlerkorrektur ein,
                  z.B. \emph{MT63}\hyperlink{refs}{\cite{wp}}
        \end{itemize}
        \item SSTV-Norm: NF-Frequenz 2300 Hz weiß, 1500 Hertz schwarz, 1200 Hz Sync
        \item am Bildsynchronisierimpuls kann man Modulation erkennen
    \end{itemize}

\end{frame}

\begin{frame}
    \frametitle{SSTV / Betriebstechnik}

    \begin{center}
        \includegraphics[width=0.8\textwidth]{e16/Transmittingtubes.jpg}
    \end{center}

    %todo ordentliche QSL-Karte (Kontrast, Textgröße etc) und Antwort einfügen

    \begin{itemize}
        \item Nachricht meist Fotos von der Funkstation und Text im Bild codiert
    \end{itemize}

\end{frame}

\begin{frame}
    \frametitle{SSTV / Rapportsystem}
   
    %todo Anmerkung: Rapportsystem kommt erst in BV13 dran - Foto einfügen?

    Rapportsystem \textbf{RSV}\hyperlink{refs}{\cite{bv12}}: Readability, Signal Strength, Video

    \begin{itemize}
        \item V1 = Nur Synchronisation zu sehen
        \item V2 = Großes Call lesbar
        \item V3 = Große Details zu erkennen
        \item V4 = Kleine Details zu erkennen
        \item V5 = rauschfreies Bild
    \end{itemize}

    Rapport wird ebenfalls im Bild eingefügt $\rightarrow$ siehe DKØTU
    SSTV-RX-Sammlung

\end{frame}

\begin{frame}
    \frametitle{SSTV / ''NF over Ackerschnacker''}

    \begin{center}
        \includegraphics[width=0.5\textwidth]{e16/SSTV-over-Ackerschnacker_LNDW2014.jpg}
        \footnote{SSTV über Feldtelefon \emph{Lange Nacht der Wissenschaften} 2014}
    \end{center}

\end{frame}

\begin{frame}
    \frametitle{SSTV / Demo}

    \Large{Ohren gespitzt... es spricht für Sie: \emph{QSSTV}}

\end{frame}

\subsection[ATV]{Amateurfunk-Fernsehen, analog ATV}

\begin{frame}
    \frametitle{ATV}

    Amateurfunk-Fernsehen, analog ATV

    \begin{center}
        \includegraphics[width=0.6\textwidth]{e16/SVECS-atv16.jpg}
        \tiny \hyperlink{refs}{\cite{atv}}
    \end{center}

    \begin{itemize}
        \item technisch wie klassisches analoges Fernsehen auch
        \item $BW = 6,5 MHz$ - logischerweise nichts für HF und VHF
        \item auch hier liegt der Fokus inzwischen auf digitaler Übertragung
        \begin{itemize}
            \item Schmalband ATV (\emph{SATV}, $1 MHz$) im Prinzip == \emph{DVB}
        \end{itemize}
    \end{itemize}

\end{frame}

\section{digital}

\subsection[CW]{Morsetelegrafie (CW)}

\begin{frame}
    \frametitle{Morsetelegrafie (CW)}

    \begin{center}
        \includegraphics[width=1\textwidth]{e16/CQWW_2013_CW_Waterfall.png}
        \footnote{Wasserfalldiagramm \emph{CQ WW Contest} 2013}
    \end{center}

    \emph{A1A} aka \emph{ASK (amplitude shift keying)} - die Mutter aller
    Betriebsarten - ja, digital!

    \begin{itemize}
        \item ''menschenlesbarer'' Digimode mit ausreichend Training
        \item Bandbreite je nach Geschwindigkeit \& Qualität ca. $200-500 Hz$
              \footnote{Interessante Ausführungen zu Bandbreite und SNR von
              DK5KE: \url{http://www.qsl.net/dk5ke/a1a.html}}
              $\rightarrow$ besserer $SNR$ bei gleicher Leistung als z.B. \emph{SSB}
        %\item durch bekannte Abkürzungen einfache QSOs ohne
        %      Fremdsprachenkenntnisse möglich 
    \end{itemize}

\end{frame}

\begin{frame}
    \frametitle{CW / Demo}

    \Large{Ohren gespitzt... es spricht für Sie: \emph{Fldigi}}

\end{frame}

\subsection[RTTY]{Funkfernschreib-Telegrafie (RTTY)}

\begin{frame}
    \frametitle{Funkfernschreib-Telegrafie (RTTY)}

    \textbf{R}adio \textbf{T}ele\textbf{TY}pe

    \begin{center}
        \includegraphics[width=0.5\textwidth]{e16/RTTY_LNDW2014.jpg}
        \footnote{RTTY-RX \emph{Lange Nacht der Wissenschaften} 2014}
    \end{center}

    %todo Bild Fernschreiber

    \begin{itemize}
        \item simplex oder semiduplex auf einer Frequenz
        \item früher:
            \begin{itemize}
                \item TX: elektromechanische Fernschreiber mit kontaktierter
                      Schreibmaschinentastatur
                \item RX: Fernschreibdrucker
            \end{itemize}
        \item heute: Software - klar.
    \end{itemize}

\end{frame}

\begin{frame}
    \frametitle{RTTY / Technik}

    \begin{itemize}
        \item fünf Kontakte $\rightarrow$ 5-Bit-Code
        \item \emph{AFSK}\footnote{Audio Frequency Shift Keying}:
              $2^5 = 32 Bit$ in $0$ und $1$ $\rightarrow$ Mark und Space
        \item Abstand = Shift
        \item Kurzwellenshift 170 Hz, UKW 850 Hz
              % todo Skizze bzw. Screenshot Spektrum
        \item technische Details: Technik Klasse A (A15)
    \end{itemize}

\end{frame}

\begin{frame}
    \frametitle{RTTY / Demo}

    \Large{Ohren gespitzt... es spricht für Sie: \emph{Fldigi}}

\end{frame}

 
\subsection{(B)PSK31}

\begin{frame}
    \frametitle{(B)PSK31}
 
    (\textbf{B}inary) \textbf{P}hase \textbf{S}hift \textbf{K}eying (31)

    \begin{columns}[c]
        \column[c]{8cm}
            \begin{itemize}
                \item BPSK31 eine der Standard-QRP-Betriebsarten
                \begin{itemize}
                    \item zwei Phasenlagen ($0^{\circ}$, $180^{\circ}$)
                    \item Bitrate von $31,25 Bit/s$
                    \item Bandbreite entspricht ca. Bitrate pro Sekunde, $31Hz$
                \end{itemize}
                \item im Vgl. zu CW ca. $\frac{1}{10}$ Bandbreite $\rightarrow$ \\
                      $SNR \approx +10dB$ (Filterung \& selbe Leistung)
                \item Modulation/Demodulation mit Software (Soundcard SDR)
                \item auch QPSK (Quadratur), 8-PSK, ...
            \end{itemize}
        \column{2cm}
        \begin{center}
            \includegraphics[width=1\textwidth]{e16/BPSK_31_63_125.jpg}
            \tiny \hyperlink{refs}{\cite{wc}}
            \scriptsize BPSK31/63/125
        \end{center}
    \end{columns}

\end{frame}

\begin{frame}
    \frametitle{PSK31 / Demo}

    \Large{Ohren gespitzt... es spricht für Sie: \emph{Fldigi}}

\end{frame}

\subsection{Baudrate}

\begin{frame}
    \frametitle{Baudrate}

    Der Vollständigkeit halber: \\[2em]

    \begin{center}
        $1 Bd = 1 \frac{Symbol}{s} = Symboldauer^{-1}$
    \end{center}

    Bei binären Modulationsverfahren entspricht das der Bitrate. Gibt es mehr
    als zwei Symbole ist die Bitrate höher als die Baudrate. Beispiele:

    \begin{itemize}
        \item RTTY: Bitrate = ?x Baudrate
        \item BPSK31: Bitrate = ?x Baudrate
        \item BPSK63: Bitrate = ?x Baudrate
        \item QPSK: Bitrate = ?x Baudrate
    \end{itemize}

\end{frame}

\begin{frame}
    \frametitle{Baudrate / Lösungen}

    \begin{itemize}
        \item RTTY: Bitrate = 1x Baudrate
        \item BPSK31: Bitrate = 1x Baudrate
        \item BPSK63: Bitrate = 1x Baudrate
        \item QPSK: Bitrate = 2x Baudrate
    \end{itemize}

\end{frame}

\subsection{WSJT}

\begin{frame}
    \frametitle{WSJT (FSK441, JT65, JT9)}

    \textbf{W}eak \textbf{S}ignal communication by K1\textbf{JT} \\[2em]

    Gruppe von Übertragungsprotokollen\footnote{Open-Source-Projekt} für
    Soundcard SDR in SSB-Bandbreite, z.B.:
    
    \begin{itemize}
        \item Meteorscatter (schnell): FSK441\footnote{Vierton Frequenzumtastung
              (\emph{FSK}) mit 441 Baud}
        \item EME, Troposcatter (schwache Signale): JT65\footnote{\emph{MFSK
              (Multiple Frequency Shift Keying)} mit 65 Tönen}
        \item Mittel- und Langwelle (schmalbandig): JT9\footnote{ähnlich JT65}
    \end{itemize}

\end{frame}

\begin{frame}
    \frametitle{WSJT / FSK441}

    Beispiel FSK441:

    \begin{itemize}
        \item Zeichen besteht aus nacheinandergesendeten drei von den vier Tönen
        \item Übertragungsgeschwindigkeit 147 Buchstaben/s $\rightarrow$
              \textbf{Meteorscatter}
        \item Ping ca. $\frac{1}{10}s$ (100km über der Erde) $\rightarrow$ 15 Zeichen
        \item 144,370 MHz Anruf-QRG, ab Kontaktaufnahme QSY
        \item Rapporte nach speziellem Kurzzystem
    \end{itemize}

\end{frame}

\subsection{ARQ-Protokolle}

\subsubsection{Verfahren}

\begin{frame}
    \frametitle{ARQ-Protokolle / Verfahren}

    \textbf{A}utomatic \textbf{R}epeat re\textbf{Q}uest

    zuverlässige Datenübertragung durch Sendewiederholungen:

    \begin{itemize}
        \item erfordert ein Verfahren zur Fehlererkennung, z.B. Checksummen
        \item Feedback: \emph{ACK/NAK}-Signale (\emph{Acknowledgement} / {Negative Acknowledgement})
        \item ggf. Wiederholung der Nachricht
    \end{itemize}

    In Kombination mit Kanalcodierung (Hinzufügen von Redundanz, wie z.B.
    \emph{FEC}\footnote{Forward Error Correction}) genauer: Hybride \emph{ARQ-Protokolle}.

\end{frame}

\subsubsection{AMTOR}

\begin{frame}
    \frametitle{AMTOR}

    \textbf{A}mateur \textbf{T}eleprinting \textbf{O}ver \textbf{R}adio

    \begin{itemize}
        \item 1978 veröffentlicht, in der Seefahrt wurden ähnliche Verfahren benutzt
        \item erstes einfaches ARQ-Protokoll nach Schema ''Stop-and-Wait''
        \begin{itemize}
            \item hohe Übertragungssicherheit: in $450ms$-Sendelücke TX von drei
                  Kontrollzeichen ($240ms$) Quittung
            \item ergo: Halbduplexbetrieb auf einer ''Simplex-QRG'' unter
                  Berücksichtigung vom \emph{TX Delay}
        \end{itemize}
        \item Modulation sonst im Prinzip wie RTTY: FSK mit gleichen Tönen und gleicher Shift
        \item allerdings 7-Bit-Code und Geschwindigkeit 100 Baud
        \item implementiert auch bereits eine \emph{FEC}
    \end{itemize}

\end{frame}

\subsubsection{AX.25}

\begin{frame}
    \frametitle{AX.25}

    8-Bit-Standard\footnote{1984 v2.0 standardisiert} für Pakete (Frames) mit Adressierung und Prüfsumme:

    \begin{center}
        \includegraphics[width=1\textwidth]{e16/Ax25-US-Paket.png}
        \tiny \hyperlink{refs}{\cite{wc}}
    \end{center}

    \begin{itemize}
        \item Sicherungsschicht (Data Link Layer)
              \footnote{\emph{OSI (Open Systems Interconnection Model)}-Layer 2}
        \item Anpassung des  \emph{ITU-T X.25}-Standards (Layer
              1-3)\footnote{Bitübertragungsschicht, Sicherungsschicht,
              Vermittlungsschicht} aus den 1970ern
        \item verbindungsorientiert, aber Übertragung von verbindungslosen Daten zulässig
    \end{itemize}

\end{frame}

\subsubsection {Packet Radio (PR)}

\begin{frame}
    \frametitle{Packet Radio (PR)}

    Paketbasiertes ''Radio'' - Grundlage: \emph{AX.25}
        
    \begin{itemize}
        \item Kennt einer (noch) \emph{GPRS}?
    \end{itemize}
   
\end{frame}

\begin{frame}
    \frametitle{Packet Radio / Technische Grundlagen (Duplex)}

    Duplex (Zeit- und ggf. Frequenzduplex), mit Fehlerkorrektur
    %todo http://de.wikipedia.org/wiki/Duplex_%28Nachrichtentechnik%29
    %todo Wiederholung Kapitel BV11: Simplex, HD, Duplex mit Skizze

    \begin{itemize}
        \item mehrere Stationen auf einer \emph{QRG} (Timeslots)
        \item verschiedene Hin- und Rückfrequenzen möglich \\ $\rightarrow$
              Unterscheidung:
        \begin{itemize}
            \item Simplex\footnote{Afu-Simplex ''Senden bzw. Empfangen auf
                  der gleichen Frequenz''!}-Digipeater (Halbduplex, selbe QRG)
            \item Duplex-Digipeater (2 QRGs\footnote{70-cm: Ablage $7,6 MHz$
                  oder $9,4 MHz$ höher})
        \end{itemize}
        \item Digipeater-''User'' im Halbduplex:
        \begin{itemize}
            \item RX (Ausgabefrq.)
            \item TX (Eingabefrq.)
        \end{itemize}
    \end{itemize}

\end{frame}

\begin{frame}
    \frametitle{Packet Radio / Technische Grundlagen (Baudraten)}

    Viel höhere Datenraten als RTTY:

    \begin{itemize}
        \item \textbf{1k2 Bd}: AFSK\footnote{$1200$ \& $2200 Hz$}-Subträger \\
              (Bandbreite 1x $12,5kHz$-Kanal, NF-BW $3000 Hz$)
        \item \textbf{9k6 Bd}: FSK direkt aufmoduliert
              \footnote{Leitungscodierung ist auch anders (Manchester-Code?)} \\
              (Bandbreite 20kHz $\rightarrow$ 1x $25kHz$-Kanal)
              %todo Skizzen der Bandbreiten von DJ4UF?
              %     http://www.darc.de/referate/ajw/ausbildung/darc-online-lehrgang/technik-klasse-e/technik-e16/#Packet-Radio
    \end{itemize}

    $\rightarrow$ VHF/UHF (eher unüblich: HF $300 Bd$)

\end{frame}

\begin{frame}
    \frametitle{Packet Radio / Packaging}

    Zusammensetzen der Pakete am Ziel zur Nachricht \\[1em]

    \begin{itemize}
        \item Leerzeiten ohne Aussendung, kann von anderen Stationen genutzt werden
        \begin{itemize}      
            \item eine Übertragungsstrecken für viele gleichzeitige Verbindungen
        \end{itemize}
        \item Problem: Kollisionen verschiedener Pakete trotz
              ''sensing''\footnote{wenn QRG frei: TX}
        \item Lösungen:
        \begin{itemize}
            \item zufällige Wartezeit beider TX
            \item DAMA\footnote{Demand Assigned Multiple Access}: Digipeater fragt Stationen ab
            \begin{itemize}
                \item Overhead, aber Kollisionen werden vollständig vermieden
            \end{itemize}
        \end{itemize}
    \end{itemize}

\end{frame}

\begin{frame}
    \frametitle{Packet Radio / Vernetzung \& Routing}

    Netzstruktur durch Digipeater\footnote{''unbesetzte, fernbediente feste
    Amateurfunkstellen für Packet Radio''} (Digital Repeater), digitale
    Zwischenstationen/Relais

    \begin{itemize}
        \item teilweise mit ''Linkstrecken''\footnote{Prüfungsfrage:
              ''Linkstrecken sind fest eingerichtete Funkverbindungen zur
              Vernetzung von Relaisfunkstellen oder Digipeatern.''} oder
              weltweit (Internet \& Co) untereinander vernetzt
        \item Datenpakete von Sender zu Empfänger und ggf weiter im Verbindungsnetz
        \item es gibt z.B.
        \begin{itemize}
            \item Speicher für Nachrichten (Mailboxen)
            \item DX-Cluster via telnet
            \item IRC Channel
        \end{itemize}
    \end{itemize}

\end{frame}

\begin{frame}
    \frametitle{Packet Radio / Vernetzung \& Routing}

    Fragen an unsere ''Digital Natives'': \\[2em]

    \begin{block}{Wozu dient ein ''Auto-Router'' im Packet-Radio-Betrieb?}
        \only<2>{Eine Einrichtung, die es ermöglicht automatisch ein
        Zielrufzeichen zu erreichen.}
    \end{block}
        \begin{block}{Was versteht man unter ''Forwarding'' im
        Packet-Radio-Betrieb?}
            \only<2>{Automatisches Weiterleiten von Nachrichten an andere
            Mailboxen}
    \end{block}

\end{frame}

\begin{frame}
    \frametitle{Packet Radio / Begriffe}

    \begin{block}{Zum Merken: Begriffe im Amateurfunk-Sprachgebrauch}
        \begin{itemize}
            \item Repeater: ?
            \item Digipeater: ?
            \item Mailbox: ?
            \item Relais: ?
        \end{itemize}
    \end{block}

\end{frame}

\begin{frame}
    \frametitle{Packet Radio / Begriffe}

    \begin{block}{Zum Merken: Begriffe im Amateurfunk-Sprachgebrauch}
        \begin{itemize}
            \item Repeater: unbesetzte, fernbediente feste Amateurfunkstellen für
                  Telefoniebetrieb
            \item Digipeater: unbesetzte, fernbediente feste Amateurfunkstellen für Packet
                  Radio
            \item Mailbox: Datenbank mit allgemeinen Zugriff zum Abspeichern und Auslesen
                  von Informationen
            \item Relais: Funkstelle zur Umsetzung von Funksignalen
        \end{itemize}
    \end{block}

\end{frame}

\begin{frame}
    \frametitle{Packet Radio / Praxis}

    Man braucht: PTT, TX, RX - Hardware-MODEM und Controller, ein \emph{TNC
    (Terminal Node Controller)} oder Software-\emph{TNC} \\[2em]

    %todo Bild TNC bzw. Blockschaltbild Setup - interaktiv an Tafel ging aber bisher

    \begin{block}{TX-Delay des PTT $50$ bis $250ms$ - Wozu?}
        \only<2>{
        \begin{itemize}
            \item so kurz wie möglich, sonst wird Übertragungszeit verschwendet -
                  ökonomische Nutzung der Kanalkapazität auf der Frequenz
            \item so lang wie nötig, da sonst Daten bei der Umschaltung ''verschluckt''
                  werden
        \end{itemize}
        }
    \end{block}

\end{frame}

\subsubsection{APRS}

\begin{frame}
    \frametitle{APRS}

    \textbf{A}utomatic \textbf{P}acket \textbf{R}eporting \textbf{S}ystem

    \begin{center}
        \includegraphics[width=0.5\textwidth]{e16/APRS.png}
        \footnote{\url{http://aprs.fi} Screenshot vom 22.01.2015}
    \end{center}

    \begin{itemize}
        \item Positionsmeldungen, Wetterdaten, Messwerte, ...
        \item Modulation Packet Radio, in ''echtes Simplex''
        \item Backbone: Packet Radio Digipeater-Netzwerk bis zum erreichen eines
              APRS-Digipeaters
    \end{itemize}

\end{frame}

\begin{frame}
    \frametitle{APRS}

    \begin{center}
        \includegraphics[width=0.8\textwidth]{e16/APRS.png}
        \footnote{\url{http://aprs.fi} Screenshot vom 22.01.2015} \\
        \Large Demo!
    \end{center}

\end{frame}

\subsubsection{ARQ Heute}

\begin{frame}
    \frametitle{ARQ Heute}

    Hervorgegangen aus \emph{AMTOR} und \emph{Packet Radio}, auf Kurzwelle z.B.
    \emph{PACTOR}\footnote{\textbf{PAC}ket \textbf{T}eleprinting \textbf{O}ver
    \textbf{R}adio} und \emph{WINMOR}\footnote{\textbf{Win}Link \textbf{M}ail
    \textbf{O}ver \textbf{R}adio}

    \begin{itemize}
        \item 8 Bit als Basis für alle Betriebsarten
        \item + verschiedene Fehlerkorrekturverfahren
        \item + verschiedene Modulationsarten je nach Kanaleigenschaften
        \item auch bei sehr schwachen Signalen im Rauschen noch nutzbar
    \end{itemize}

    \vspace{0.5cm}

    \emph{PACTOR} Rant: Man braucht einen teuren TNC, der
    Gebrauchsmustergeschützt ist - Selbstbau nicht möglich.  

\end{frame}

\begin{frame}
    \frametitle{ARQ Heute / Netzwerke}

    \begin{itemize}
        \item globalee HF-Mailboxsysteme mit der Möglichkeit Forwarding per
              Internet-E-Mail, z.B. \emph{WinLink2000}
        \item \emph{HAMNET (Highspeed Amateurradio Multimedia NETwork)} benutzt
              \emph{IEEE 802.11}-Technologie im $GHz$-Bereich
    \end{itemize}

    \begin{center}
        \includegraphics[width=0.6\textwidth]{e16/db0fuz.png}
        \tiny \hyperlink{refs}{\cite{db0fuz}}
    \end{center}

\end{frame}

\begin{frame}
    \frametitle{ARQ Heute / WinLink2000}

    \begin{center}
        \includegraphics[width=1\textwidth]{e16/WinLink_Topology.jpg}
        \tiny \hyperlink{refs}{\cite{wl2k}}
    \end{center}

\end{frame}

\begin{frame}
    \frametitle{ARQ Heute / HAMNET}

    \begin{center}
        \includegraphics[width=1\textwidth]{e16/backbone_berlin1201.jpg}
        \footnote{\url{http://hamnet.funkzentrum.de} (Stand 24.03.2012)}
    \end{center}

\end{frame}

\subsection{Sprechfunk}

\begin{frame}
    \frametitle{Sprechfunk (digital)}

    %todo ausbauen

    Der Vollständigkeit halber: Digitale Betriebsarten für den Sprechfunk gibt
    es zunehmen, z.B. \emph{FreeDV}\footnote{Free Digital Voice} für Kurzwelle.

\end{frame}

\subsection{Zusammenfassung}

\begin{frame}
    \frametitle{Zusammenfassung}

    \begin{center}
        \includegraphics[width=1\textwidth]{e16/Digital_Rosetta_Stone.jpg}
        \footnote{''Digital Rosetta Stone'' by K2NCC}
    \end{center}

\end{frame}

\section{Quiz}

\begin{frame}
    \frametitle{Quiz}

    \begin{block}{Welche Betriebsarten sind für QRP-DX-Betrieb auf Kurzwelle am
    besten geeignet?}
        \only<2>{In den Prüfungsantworten: CW, Pactor, PSK31
        \begin{itemize}
            \item Bandbreiten?
            \item Wie viele habt ihr herausgefunden?
        \end{itemize}
        }
    \end{block}

% todo Aufgabe Nach Bandbreite sortieren?
%     * Telegrafie benötigt u.U. weniger als 200 Hz Bandbreite
%     * PSK31, MSK63, JT65                                    
%     *alle die vorher dran waren

% todo Jeopardy Sektion zu Betriebsarten (Spektrum und Bandbreiten) machen

\end{frame}

\section{NF-QSOs}

\begin{frame}
    \frametitle{NF-QSOs: Fldigi, QSSTV, ...}

    \Large \textbf{Pause.}
    \normalsize Anschließend Erklärung der Bedienung. \\[2em]

    %todo Screenshots der Programme

    Wer die Tools nicht installiert hat, nutzt die Zeit. ;-)

    %todo WSJT, ARQ-Protokolle und FreeDV via NF testen?
    %     ... Vielleicht bei Klasse A dann.

\end{frame}

\renewcommand{\refname}{Referenzen}

\hypertarget{refs}{}
\textcolor{white}{} \\ %\vspace{} geht nicht
\Large Referenzen/Links
\footnotesize

\begin{thebibliography}{}
    \bibitem{bv12}  Moltrecht B/V 12: \\
                    \url{http://www.amateurfunkpruefung.de/lehrg/bv12/bv12.html}
    \bibitem{e16}   Moltrecht E 16: \\
                    \url{http://www.darc.de/referate/ajw/ausbildung/darc-online-lehrgang/technik-klasse-e/technik-e16/}
    \bibitem{wp}    Wikipedia DE: \\
                    \url{https://de.wikipedia.org/wiki/Einseitenbandmodulation}\\
                    \url{http://de.wikipedia.org/wiki/Slow_Scan_Television}\\
                    \url{http://de.wikipedia.org/wiki/MT63}\\
                    \url{http://de.wikipedia.org/wiki/Amateurfunk-Fernsehen}\\
                    \url{http://de.wikipedia.org/wiki/PSK31}\\
                    \url{https://de.wikipedia.org/wiki/Baud}\\
                    \url{https://de.wikipedia.org/wiki/WSJT}\\
                    \url{http://de.wikipedia.org/wiki/ARQ-Protokoll}\\
                    \url{http://de.wikipedia.org/wiki/AMTOR}\\
                    \url{http://de.wikipedia.org/wiki/AX.25}
    \bibitem{we}    Wikipedia EN: \\
                    \url{https://en.wikipedia.org/wiki/Terminal_node_controller}
    \bibitem{wc}    Wikimedia Commons \\
                    \url{http://commons.wikimedia.org/wiki/File:Ssb-de.png}\\
                    \url{http://commons.wikimedia.org/wiki/File:Dpx-fm-radio.png}\\
                    \url{http://commons.wikimedia.org/wiki/File:Mechanical_glow_drum_slow_scan_television_monitor.gif}\\
                    \url{http://commons.wikimedia.org/wiki/File:Sstv_frequences.svg}\\
                    \url{http://commons.wikimedia.org/wiki/File:SSTV_signal.jpg}\\
                    \url{http://commons.wikimedia.org/wiki/File:BPSK_31_63_125.jpg}\\
                    \url{http://commons.wikimedia.org/wiki/File:Ax25-US-Paket.png}
    \bibitem{atv}   \url{http://www.svecs.net/SVECS-atv16.JPG}
    \bibitem{wl2k}  \url{http://letarc.org/main/2011/06/08/packet-modes-and-winlink-2000/}
    \bibitem{db0fuz}\url{http://hamnet.funkzentrum.de/berliner-hamnet/backbone/hamnet-knoten.html}

\end{thebibliography} 

% Hier könnte noch eine Kontaktfolie stehen

\end{document}

