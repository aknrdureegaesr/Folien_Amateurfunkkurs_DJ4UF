%-*- coding: utf-8; -*-
\documentclass[a4paper,10pt]{article}
\usepackage[a4paper, top=3cm]{geometry}
\usepackage[utf8]{inputenc}
\usepackage[german]{babel}
\usepackage{fancyhdr}
\usepackage{graphicx}
\usepackage{caption}

\captionsetup{font=scriptsize,labelfont=scriptsize}
\pagestyle{fancy}

\lhead{}
\chead{}
\rhead{}
\lfoot{\footnotesize \it DK0TU Amateurfunkkurs, 2015}
\cfoot{}
\rfoot{\thepage}

\renewcommand{\headrulewidth}{0.0pt}
\renewcommand{\footrulewidth}{0.4pt}

\begin{document}

\begin{center}
    \textbf{DK0TU \\ \parskip 8pt
            \Large Erstes FM QSO
            } \\[4em]
\end{center}

\subsection*{Don't Panic}

Die ersten paar Male am Mikrofon sind immer eine Überwindung. Auch wenn jedes QSO immer etwas anders ist, ist der grundlegende Ablauf bei „Standard“-QSOs immer gleich. Damit der Einstieg leichter fällt, kannst du etwas auf UKW (430MHz) mit anderen Leuten von DK0TU üben.

\subsection*{Rufzeichen}

Alle Lizensierten Teilnehmer des Amateurfunks besitzen ein personengebundenes individuelles Rufzeichen z.B. DL7BST, DM1RI oder DB4UM. Da Ihr noch kein eigenes Rufzeichen habt dürft Ihr zum Funken ein Ausbildungsrufzeichen nehmen. So besitzt DB4UM z.B. das Rufzeichen DN4BA, welches ein Ausbildungsrufzeichen ist. Damit dürft Ihr in Anwesenheit von DB4UM funken um etwas zu üben und ein Gefühl für Funkkontakte zu entwickeln.

\subsection*{Logbuch}

Eueren Funkkontakt müsst ihr beim nutzen eines Ausbildungsrufzeichen in einem Logbuch festhalten. Wichtig ist dabei die Uhrzeit in UTC, das Datum und das Rufzeichen der anderen Station/Person. Des weiteren könnt Ihr noch euren Namen, und euren Standort(QTH) übergeben. Meist noch eine beschriebung wie gut das Signal emfangen wird, doch dies lassen wir fürs erste. Der Rest ist offen gehalten.\\

\large Wichtig:
\begin{itemize}
    \item Rufzeichen der anderen Person
    \item Vorname der anderen Person
    \item Standort der anderen Person
    \item Uhrzeit
\end{itemize}

\newpage

\subsection*{Beispiel QSO}

\textbf{Ausbildungsrufzeichen: DN4BA}

\begin{itemize}
    \item \textbf{\large Dies ist ein allgemeiner Anruf von DN4BA, Delta November 4 Bravo Alpha mit einem algemeinen Anruf.}
    \item \normalsize DN4BA von DK0TU
    \item \textbf{\large DK0TU von DN4BA, Hallo. Mein Name ist \$name und ich bin gerade an der TU Berlin im Mar Gebäude. Dies ist mein erster Funkkontakt. DK0TU von DN4BA.}
    \item \normalsize DN4BA von DK0TU, Hallo \$name. Schön dich zu hören. Mein name ist \$name2 und ich bin momentan auf dem Dach vom Hauptgebäude. Dein Empfangsraport ist 5 und 9. Ich bla..bla
    \item \textbf{\large DK0TU von DN4BA, Antworts Bla.. Mein Lieblingseis ist Kirsche... Bla.. Bla DK0TU von DN4BA.}
    \item \normalsize DN4BA von DK0TU, bla bla bla... DN4BA DK0TU
    \item \textbf{\large DK0TU von DN4BA, Danke für dieses nette Gespräch. Sicherlich hören wir uns bei wieder. 73! DK0TU von DN4BA.}
\end{itemize}

\end{document}
