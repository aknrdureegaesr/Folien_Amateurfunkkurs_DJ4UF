\thispagestyle{empty}

\hspace{.1cm}
\begin{minipage}[t]{13cm}

  \textbf{Impressum}\\

  \begin{tabular}{p{1cm} p{11cm}}
    Titel: &  Skript AfuTUB-Kurs (basierend auf dem Praxisskript der
    Projektwerkstatt "`Amateurfunk verbindet"', 2. Auflage März 2017)\\
    & \\
    Autoren: & Sebastian Lange, M.Sc. (Konzeption \& Praktische Anwendungen)\\
    & Christian Stoll, M.Ed. (Theorie- und Prüfungsfragen)\\
    %Abbildung:&\\
    & \\
    \multicolumn{2}{l}{Erschienen an der Technischen Universität Berlin:}\\
    &\\
    & Fachgebiet Hochfrequenztechnik (Sekr. HFT 4)\\
    & Institut für Hochfrequenz- und Halbleiter-Systemtechnologien\\
    & Fakultät IV -- Elektrotechnik und Informatik \\
    & Einsteinufer 25 \\
    & D-10587 Berlin \\
    & \\
    & Fachgebiet Raumfahrttechnik (Sekr. F 6)\\
    & Institut für Luft- und Raumfahrt\\
    & Fakultät V -- Verkehrs- und Maschinensysteme\\
    & Marchstraße 12-14\\
    & D-10587 Berlin\\
    &\\
    \multicolumn{2}{l}{1. Auflage Oktober 2018}\\
  \end{tabular}

  \vspace{3cm}

  Aktuelle Informationen der Amateurfunkgruppe der TU Berlin finden Sie unter:\\
  \url{https://dk0tu.de}

  \vspace{1cm}

  Die vorliegende Fassung des Skripts wurde sorgfältigst auf Fehler hin
  überprüft. Um das Skript dennoch laufend verbessern zu können, würden wir uns
  über Hinweise auf etwaig vorhandene Fehler sowie Verbesserungsvorschläge sehr
  freuen. Wenden Sie sich dazu bitte an die zuständigen Tutor*innen und
  wissenschaftlichen Mitarbeiter*innen.

\end{minipage}

% TODO acknowledgements Prof. Dr. Klaus Petermann / Brieß / Neuer Prof.
