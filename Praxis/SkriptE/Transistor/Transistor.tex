

\begin{wrapfigure}[0]{r}[-1cm]{4cm}
 \vspace{-6cm}
  \includegraphics[scale=0.4]{Transistor/Bilder/Transistors-white.jpg}
 \vspace{-6cm}
\end{wrapfigure}

\section*{Theorie- und Prüfungsfragen}

\begin{enumerate}
\itemsep1pt\parskip0pt\parsep0pt
\item[1] Skizziere die Schaltzeichen eines NPN- und eines PNP-Transistors. Beschrifte  entsprechend die Anschlüsse.
\end{enumerate}

\loesung{
\begin{figure}[H]
\centering
\includegraphics[scale=3]{Transistor/Bilder/PNP_NPN.pdf}
\caption{1 und 2 - NPN- und PNP-Transistor}
\end{figure}
}

\mucho{2}{TC601}
{Was versteht man unter Stromverstärkung beim Transistor?}%Frage
{Mit einem geringen Strom (Basisstrom) wird ein großer Strom (Kollektorstrom) gesteuert.}%A
{Mit einem geringen Strom (Emitterstrom) wird ein großer Strom (Kollektorstrom) gesteuert.}%B
{Mit einem geringen Strom (Emitterstrom) wird ein großer Strom (Basisstrom) gesteuert.}%C
{Mit einem geringen Strom (Kollektorstrom) wird ein großer Strom (Emitterstrom) gesteuert.}%D
{A}%Lösung

\begin{enumerate} 
\itemsep1pt\parskip0pt\parsep0pt
\item[3] \emph{\textbf{TC605}} Welche Kollektorspannungen haben NPN- und PNP-Transistoren?
	\begin{enumerate}
	\itemsep1pt\parskip0pt\parsep0pt
		\item[A] NPN- und PNP-Transistoren benötigen negative Kollektorspannungen.
		\item[B] PNP-Transistoren benötigen positive, NPN-Transistoren negative Kollektorspannung.
		\item[C] PNP- und NPN-Transistoren benötigen positive Kollektorspannungen.
		\item[D] NPN-Transistoren benötigen positive, PNP-Transistoren negative Kollektorspannungen.
		\loesung{Lösung D}
	\end{enumerate}
\end{enumerate}

\begin{enumerate} 
\item[4] \emph{\textbf{TC602}}  Das Verhältnis von Kollektorstrom zum Basisstrom eines Transistors liegt üblicherweise im Bereich von
	\begin{enumerate}
	\itemsep1pt\parskip0pt\parsep0pt
		\item[A] 1 zu 50 bis 1 zu 100.
		\item[B] 10 zu 1 bis 900 zu 1.
		\item[C] 1000 zu 1 bis 5000 zu 1.
		\item[D] 1 zu 100 bis 1 zu 500.
		\loesung{Lösung B}
	\end{enumerate}
\end{enumerate}

\mucho{5}{TC609}
{Ein bipolarer Transistor ist}%Frage
{spannungsgesteuert.}%A
{thermisch gesteuert.}%B
{ein Gleichspannungsverstärker.}%C
{stromgesteuert.}%D
{D}%Lösung

\mucho{6}{TC611}
{Wie erfolgt die Steuerung des Stroms im Feldeffekttransistor (FET)?}%Frage
{Die Gatespannung ist allein verantwortlich für den Drainstrom.}%A
{Die Gatespannung steuert den Widerstand des Kanals zwischen Source und Drain.}%B
{Der Gatestrom ist allein verantwortlich für den Drainstrom.}%C
{Der Gatestrom steuert den Widerstand des Kanals zwischen Source und Drain.}%D
{B}%Lösung

\mucho{7}{TC612}
{Wie bezeichnet man die Anschlüsse des folgenden Transistors?\\ \includegraphics[scale=0.5]{Transistor/Bilder/TC612.png}}%Frage
{1 Drain, 2 Source, 3 Gate.}%A
{1 Source, 2 Drain, 3 Gate.}%B
{1 Anode,  2 Katode, 3 Gate.}%C
{1 Kollektor, 2 Emitter, 3 Basis.}%D
{A}%Lösung

\mucho{8}{TD401}
{In welcher der folgenden Zeilen werden nur Verstärker-Bauelemente genannt?}%Frage
{Transistor, Halbleiterdiode, Operationsverstärker, Röhre}%A
{Transistor, MOSFET, Operationsverstärker, Röhre}%B
{Transistor, Varicap-Diode, Operationsverstärker, Röhre}%C
{Transistor, MOSFET, Halbleiterdiode, Röhre}%D
{B}%Lösung

\mucho{9}{TD402}
{Was versteht man in der Elektronik unter Verstärkung? Man spricht von Verstärkung, wenn ...}%Frage
{das Eingangssignal gegenüber dem Ausgangssignal in der Leistung größer ist.}%A
{z.B. beim Transformator die Ausgangsspannung größer ist als die Eingangsspannung.}%B
{das Ausgangssignal gegenüber dem Eingangssignal in der Leistung größer ist.}%C
{das Eingangssignal gegenüber dem Ausgangssignal in der Spannung größer ist.}%D
{C}%Lösung

\mucho{10}{TD403}
{Was ist ein Operationsverstärker? Operationsverstärker sind ...}%Frage
{Gleichstrom gekoppelte Verstärker mit sehr hohem Verstärkungsfaktor und großer Linearität.}%A
{Wechselstrom gekoppelte Verstärker mit niedrigem Eingangswiderstand und großer Linearität.}%B
{in Empfängerstufen eingebaute Analogverstärker mit sehr niedrigem Verstärkungsfaktor aber großer Linearität.}%C
{digitale Schaltkreise mit hohem Verstärkungsfaktor.}%D
{A}%Lösung

\mucho{11}{TD404}
{ Ein IC (integrated circuit) ist...}%Frage
{eine aus vielen einzelnen Bauteilen aufgebaute Schaltung auf einer Platine.}%A
{eine miniaturisierte, aus SMD-Bauteilen aufgebaute Schaltung.}%B
{eine Zusammenschaltung verschiedener Baugruppen zu einer Funktionseinheit.}%C
{eine komplexe Schaltung auf einem Halbleiterkristallplättchen.}%D
{D}%Lösung

\mucho{12}{TD405}
{Worauf beruht die Verstärkerwirkung von Elektronenröhren?}%Frage
{Die Anodenspannung steuert das magnetische Feld an der Anode und damit den Anodenstrom.}%A
{Das von der Gitterspannung hervorgerufene elektrische Feld steuert den Anodenstrom.}%B
{Die Heizspannung steuert das elektrische Feld an der Kathode und damit den Anodenstrom.}%C
{Die Katodenvorspannung steuert das magnetische Feld an der Katode und damit den Gitterstrom.}%D
{B}%Lösung

\newpage

\section*{Praktische Anwendung}

\loesung{In diesem Kapitel geht es um grundlegende Transistorschaltungen. Bei
der Gruppenzusammensetzung von vornherein darauf achten, dass in jeder Gruppe
mind. eine Person mit Löterfahrung zu haben -- solche Leute hat man immer dabei.}

\subsection*{Transistorschaltung 01 - Der Bipolar-Transistor als Schalter}

\begin{enumerate}
\itemsep1pt\parskip0pt\parsep0pt
\item Schaut auch den gegebenen Transistor als Bauteil an. Welche Bezeichnung
  hat er und um welchen Transistortyp handelt es sich? \loesung{BC547C,
    NPN-Bipolartransistor}
\item Ordnet mit Hilfe des Datenblattes die Bezeichnungen der einzelnen Beinchen
  zu. \loesung{CBE $\rightarrow$ Kollektor, Basis und Emitter}
\item Verwendet den Komponententester des Multimeters. Was zeigt dieser an?
  \loesung{Stromverstärkungsfaktor des Transistors, $\beta = h_{FE} \approx 600$}
\item Baut die Transistor-Schaltung aus Abbildung \ref{s01} auf.
\item Legt die Versorgungsspannung an die Schaltung an.
\item Entfernt unter Last die Leuchtdiode 1. Welche Auswirkungen hat das auf die Schaltung und warum?
\end{enumerate}

\begin{figure}[H]
	\centering
	%\subfigure[Schaltplan]{
  \includegraphics[scale=1.4]{Transistor/Schaltungen/NotBeleuchtung_Schaltplan.pdf}
  %}
	%\subfigure[Mögliche Breadboard-Ansicht]{\includegraphics[scale=1]{Transistor/Schaltungen/NotBeleuchtung_Steckplatine.pdf}}
	\caption{Schaltplan Transistorschaltung 01 - Der Bipolar-Transistor als Schalter}
	\label{s01}
\end{figure}

%----------------------------------------------

\subsection*{Transistorschaltung 02 - Der Bipolar-Transistor als Sensor}

\loesung{Der Sensor ist natürlich auch ein Verstärker. Diese
Darlington-Schaltung verstärkt allerdings mit der multiplizierten
Stromverstärkung beider Transistoren.}

\begin{enumerate}
    \itemsep1pt\parskip0pt\parsep0pt
    \item Baut die Transistor-Schaltung aus Abbildung \ref{s02} auf.
    \item Legt die Versorgungsspannung an die Schaltung an.
    \item Berührt die Basis des Transistors Q1 mit dem Finger. Was passiert und
      warum? Der Effekt wird besser sichtbar, wenn man mit der anderen Hand das
      Potential der Spannungsversorgung berührt. \loesung{Der vom Körper als
      Antenne eingefangene Elektrosmok reicht aus um den Transistor
      anzusteuern.}
    \item \textbf{Zusatz:} Ersetze die LED durch einen Lautsprecher? Was
      passiert und warum? \loesung{Man hört den besagten Elektrosmok.}
\end{enumerate}

\begin{figure}[H]
	\centering
	\includegraphics[scale=1.6]{Transistor/Schaltungen/NPN_Sensor.pdf}
	\caption{Transistorschaltung 02 - Der Bipolar-Transistor als Sensor}
	\label{s02}
\end{figure}

%----------------------------------------------

\subsection*{Transistorschaltung 03 - Der Bipolar-Transistor als Verstärker}

\begin{enumerate}
    \itemsep1pt\parskip0pt\parsep0pt
    \item Baut die Transistor-Schaltung aus Abbildung \ref{s03} auf.
    \item Legt die Versorgungsspannung an die Schaltung an.
    \item Legt ein Audiosignal an den Eingang der Schaltung an. Was passiert und warum?
\end{enumerate}

\begin{figure}[H]
	\centering
	\includegraphics[scale=1.6]{Transistor/Schaltungen/NPN_Verstaerker.pdf}
	\caption{Transistorschaltung 03 - Der Bipolar-Transistor als Verstärker}
	\label{s03}
\end{figure}

%----------------------------------------------

\subsection{Die Lizenz zum Löten}

Übertragt die letzte Schaltung auf eine Lochrasterplatine (ca. $5~x~2.5 cm$).
Für die Anschlüsse (Spannungsversorgung, Signalein- und ausgang) sind beliebig
teilbare Buchsenleisten vorhanden. Denkt an die Beschriftung.
