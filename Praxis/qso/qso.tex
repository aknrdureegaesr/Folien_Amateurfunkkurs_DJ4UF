%-*- coding: utf-8; -*-
\documentclass[a4paper,10pt]{article}
\usepackage[a4paper, top=3cm]{geometry}
\usepackage[utf8]{inputenc}
\usepackage[german]{babel}
\usepackage{fancyhdr}
\usepackage{graphicx}
\usepackage{caption}
\usepackage{amsmath}
\usepackage{amssymb}
\usepackage{pdfpages}
\usepackage{sectsty}
\usepackage{hyperref}

\sectionfont{\fontsize{12}{13}\selectfont}
\captionsetup{font=scriptsize,labelfont=scriptsize}
\pagestyle{fancy}
\setlength{\parindent}{0pt}

\lhead{\small Amateurfunkgruppe der TU Berlin (AfuTUB) \\
       \normalsize \textbf{Ausbildungs-Logbook}}
\chead{}
\rhead{\includegraphics[height=0.9cm]{img/dk0tu-logo_2016-1_rooftop+call_bw.png}}
\lfoot{\footnotesize \it CC BY-NC-SA Sebastian Lange (DL7BST@bastla.net),
       Stand 2017-10-25}
\cfoot{}
\rfoot{\thepage}

\renewcommand{\headrulewidth}{0.0pt}
\renewcommand{\footrulewidth}{0.4pt}

%%%%%%%%%%%%%%%%%%%%%%%%%%%%%%%%%%%%%%%%%%%%%%%%%%%%%%%%%%%%%%%%%%%%%%%%
%% Document

\begin{document}

\begin{center}
    \Large ``Mein erstes QSO''
\end{center}

\vspace{-1.5em}

\section*{\textit{Vorbereitung für Ausbilder/innen}}

\textit{
  \begin{itemize}
    \setlength\itemsep{0em}
    \item mind. zwei, aber eher mehr UHF-Funkgeräte (da am weitesten verbreitet)
      mitbringen
    \item Programmierung (falls notwendig): 430.200, 430.225, 430.250, 430.275
      (CQ TU QRGs)
    \item Ausbilder/innen QRV @ 430.200
  \end{itemize}
}

\section{Abstract}

Um gleich von Beginn an in die Praxis reinzuschnuppern geht es bei den ersten
Funkkontakten (QSO) darum im Nahbereich auf UKW und gut verständlicher
FM-Modulation erste Funksprüche abzusetzen. Im Fokus steht hierbei die
Betriebstechnik, also der Ablauf eines Funkspruchs, Funkdisziplin sowie das Üben
des internationalen Buchstabieralphabets. Dies dient auch als erste Vorbereitung
für die Funkpraxis auf Kurzwelle in der Clubstation und des "`CQ TU"'-Contests.

\section{Aufgabe}

% Wort -> Satz?

\textbf{Vorbereitung:} Sucht euch in der Gruppe einen Standort (QTH) und ein
Lösungswort mit der Länge der Anzahl der Teilnehmenden aus. Jede/r bekommt einen
Buchstaben zugewiesen und hat die Aufgabe diesen den anderen Stationen in
Sprechweise des internationalen Buchstabieralphabets zu übermitteln. \bigskip

\textbf{Ablauf:} In jeden Durchgang sind das Rufzeichen (Call), der Buchstabe
und die Stelle im Wort zu übertragen. Auf Nachfrage gilt es ebenso das QTH (z.B.
Raumnummer) sowie den eigenen Vorname zu buchstabieren. Jedes QSO ist zu loggen.
Eine beliebige Station startet mit einem allgemeinen Anruf. Nach jedem Durchgang
ist die QRG freizugeben und die zuvor nicht rufende Station darf allgemein oder
direkt anrufen. Alles weitere ergibt sich. \bigskip

\textit{Hinweis für die Ausbilder: Es ist kein Contest, also bitte piano!}
\bigskip

\textbf{Ziel:} Der Funkplan ist -- inlusive der Lösungssätze -- vollständig
auszufüllen.
%Tip: Zieht euch Unterstriche für die Anzahl der Personen der Gegenstelle und
%füllt es nicht vollständig aus, auch wenn ihr es bereits erraten könnt.

\section{ITU Phonetic Alphabet}

\begin{minipage}[t]{0.5\textwidth}
  \large
  \textbf{A}lpha\\
  \textbf{B}ravo\\
  \textbf{C}harly\\
  \textbf{D}elta\\
  \textbf{E}cho\\
  \textbf{F}oxtrott\\
  \textbf{G}olf\\
  \textbf{H}otel\\
  \textbf{I}ndia\\
  \textbf{J}uliett\\
  \textbf{K}ilo\\
  \textbf{L}ima\\
  \textbf{M}ike\\
\end{minipage}
\begin{minipage}[t]{0.5\textwidth}
  \large
  \textbf{N}ovember\\
  \textbf{O}scar\\
  \textbf{P}apa\\
  \textbf{Q}uebec\\
  \textbf{R}omeo\\
  \textbf{S}ierra\\
  \textbf{T}ango\\
  \textbf{U}niform\\
  \textbf{V}ictor\\
  \textbf{W}hiskey\\
  \textbf{X}-Ray\\
  \textbf{Y}ankee\\
  \textbf{Z}ulu\\
\end{minipage}

\clearpage

\section{Funkplan}

  \begin{center}
  %\footnotesize
  \renewcommand{\arraystretch}{1.5}
  \begin{tabular}{|c|c|c|p{9cm}|}\hline
    \textbf{Call} & \textbf{QTH} & \textbf{\#Pers.} &
    \textbf{Buchstaben/Lösungswort \& Namen} \\ \hline \hline
    DK0TU & H $>$9000 & 5 & 1: T : Hyman \newline 2: U : Almay
    \newline 3: B : Murray \newline 4: B : Anton \newline 5: Y : Berta \\ \hline
     & & & 1: \_ : \newline 2: \_ : \newline 3: \_ : \newline 4: \_ : \newline 5: \_ :\\ \hline
     & & & 1: \_ : \newline 2: \_ : \newline 3: \_ : \newline 4: \_ : \newline 5: \_ :\\ \hline
     & & & 1: \_ : \newline 2: \_ : \newline 3: \_ : \newline 4: \_ : \newline 5: \_ :\\ \hline
     & & & 1: \_ : \newline 2: \_ : \newline 3: \_ : \newline 4: \_ : \newline 5: \_ :\\ \hline
     & & & 1: \_ : \newline 2: \_ : \newline 3: \_ : \newline 4: \_ : \newline 5: \_ :\\ \hline
     & & & 1: \_ : \newline 2: \_ : \newline 3: \_ : \newline 4: \_ : \newline 5: \_ :\\ \hline
     & & & 1: \_ : \newline 2: \_ : \newline 3: \_ : \newline 4: \_ : \newline 5: \_ :\\ \hline
     & & & 1: \_ : \newline 2: \_ : \newline 3: \_ : \newline 4: \_ : \newline 5: \_ :\\ \hline
  \end{tabular}
  \end{center}

  \clearpage

\section{Logbook (vereinfacht)}
  \label{att:log0}

  %% HEADER
  \bigskip
  \hspace{1cm}
  \begin{minipage}[t]{0.33\textwidth}
    \hspace{3cm} {\Large FM}\vspace{0.5em}
      \centering
      \hrule
      \vspace{0.5ex}
      \small QRG \& Mode
  \end{minipage}
  \hfill
  \begin{minipage}[t]{0.33\textwidth}
    \hspace{-2cm} {\Large DN}\vspace{0.5em}
      \centering
      \hrule
      \vspace{0.5ex}
      \small My Call
  \end{minipage}
  \hspace{1cm}

  %% LOG
  \bigskip
  \begin{minipage}[t]{0.5\textwidth}
  \begin{center}
  %\footnotesize
  \renewcommand{\arraystretch}{1.5}
  \begin{tabular}{|c|c|}\hline
      \textbf{UTC} & \textbf{Call} \\ \hline \hline
       ~~1530~~ & DKØTU \\ \hline
       \hspace{2cm} & \hspace{3.0cm} \\ \hline
       & \\ \hline
       & \\ \hline
       & \\ \hline
       & \\ \hline
       & \\ \hline
       & \\ \hline
       & \\ \hline
       & \\ \hline
       & \\ \hline
       & \\ \hline
       & \\ \hline
       & \\ \hline
       & \\ \hline
       & \\ \hline
       & \\ \hline
       & \\ \hline
       & \\ \hline
       & \\ \hline
       & \\ \hline
       & \\ \hline
       & \\ \hline
       & \\ \hline
       & \\ \hline
       & \\ \hline
       & \\ \hline
       & \\ \hline
  \end{tabular}
  \end{center}
  \end{minipage}
  \hfill
  \begin{minipage}[t]{0.5\textwidth}
  \begin{center}
  %\footnotesize
  \renewcommand{\arraystretch}{1.5}
  \begin{tabular}{|c|c|}\hline
      \textbf{UTC} & \textbf{Call} \\ \hline \hline
       \hspace{2cm} & \hspace{3.0cm} \\ \hline
       & \\ \hline
       & \\ \hline
       & \\ \hline
       & \\ \hline
       & \\ \hline
       & \\ \hline
       & \\ \hline
       & \\ \hline
       & \\ \hline
       & \\ \hline
       & \\ \hline
       & \\ \hline
       & \\ \hline
       & \\ \hline
       & \\ \hline
       & \\ \hline
       & \\ \hline
       & \\ \hline
       & \\ \hline
       & \\ \hline
       & \\ \hline
       & \\ \hline
       & \\ \hline
       & \\ \hline
       & \\ \hline
       & \\ \hline
       & \\ \hline
  \end{tabular}
  \end{center}
  \end{minipage}

  %% FOOTER
  \bigskip
  \begin{minipage}[t]{0.33\textwidth}
    \hspace{1cm} SWL-Namen:
  \end{minipage}
  \hfill
  \begin{minipage}[t]{0.35\textwidth}
  \vspace{8ex}
      \hrule
      \vspace{0.5ex}
      {\small{Datum,}}\hfill{\small{Unterschrift Rufzeicheninh.}}
  \end{minipage}

\end{document}
