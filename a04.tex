% Foliensatz: "AFu-Kurs nach DJ4UF" von DK0TU, Amateurfunkgruppe der TU Berlin
% Lizenz: CC BY-NC-SA 3.0 de (http://creativecommons.org/licenses/by-nc-sa/3.0/de/)
% Autoren: Martin Deutschmann, Lars Weiler <dc4lw@darc.de>
% Korrekturen: Sebastian Lange <dl7bst@dk0tu.de>

preamble.dk0tu.tex
\subtitle{Technik Klasse A 04: \\
          Schwingkreise \& Filter \\[2em]}
\date{Stand 19.05.2016}
 \begin{document}

\begin{frame}
    \titlepage
    \vfill
    \begin{center}
        \ccbyncsaeu\\
        {\tiny This work is licensed under the \em{Creative Commons Attribution-NonCommercial-ShareAlike 3.0 License}.}\\[0.5ex]
         \tiny Amateurfunkgruppe der Technische Universität Berlin (AfuTUB), DKØTU
         %\includegraphics[scale=0.5]{img/DK0TU_Logo.pdf}
    \end{center}
\end{frame}


%fixme Links zur bibliography einfuegen

\section*{Schwingkreis}
\begin{frame}
\frametitle{Schwingkreise}
	\begin{center}
		\includegraphics[scale=0.4]{a04/Schwingkreis_reihe.png}
		\includegraphics[scale=0.4]{a04/Schwingkreis_parallel.png}\\
		Abb.1: Serien- \& Parallelschwingkreis
	\end{center}
\end{frame}

\begin{frame}
\frametitle{Schwingungserzeugung}
\begin{center}
	\includegraphics[scale=0.35]{a04/Schwingkreis.png}\\
	{\tiny Abb.2: Energie in einem LC-Schwingkreis \cite{wmde}} \\
	\vspace{3mm}
	\begin{itemize}
		\item durch Verluste kommt es zur gedämpfte Schwingung\\
		\item animierte Darstellung (\url{http://en.wikipedia.org/wiki/File:Tuned_circuit_animation_3.gif})
	\end{itemize}
\end{center}
\end{frame}

\subsection*{Reihen\-schwing\-kreis}
\begin{frame}
\frametitle{Reihenschwingkreis}
\begin{center}
	\begin{minipage}{0.4\textwidth}
	\includegraphics[height=.5\textheight,width=\textwidth,keepaspectratio]{a04/Serirenschw.png}\\
	\tiny{Abb.3: Serienschwingkreis \cite{wmen}}
	\end{minipage}
	\begin{minipage}{0.4\textwidth}
	\includegraphics[height=.5\textheight,width=\textwidth,keepaspectratio]{a04/SerirenschwSig.png}\\
	\tiny{Abb.4: Resonanzwiderstand \cite{wmen}} 
	\end{minipage}
\end{center}
\begin{itemize}
	\item Im Verlauf der Frequenzänderung ändert sich der Gesamtwellenwiderstand Z des Schwingkreises
	\item Der Schwingkreis hat als minimale Impedanz seinen ohmschen Wert, da sich bei der Resonanzfrequenz $f_R$ die induktiven und kapazitiven Anteile gegenseitig aufheben
\end{itemize}
\end{frame}

\subsection*{Parallel\-schwing\-kreis}
\begin{frame}
\frametitle{Parallelschwingkreis}
\begin{center}
	\begin{minipage}{0.4\textwidth}
	\includegraphics[scale=1]{a04/Parallelschw.png}\\
	\tiny{Abb.5: Parallelschwingkreis \cite{wmen}}
	\end{minipage}
	\begin{minipage}{0.4\textwidth}
	\includegraphics[scale=0.2]{a04/ParallelschwSig.png}\\
	\tiny{Abb.6: Resonanzwiderstand \cite{wmen}} 
	\end{minipage}
\end{center}
\begin{itemize}
	\item Der Parallelschwingkreis verhält sich genau entgegen gesetzt zum Reihenschwingkreis
	\item Dieser zeigt bei niedrigen und hohen Frequenzen das Verhalten eines Leiters
	\item Bei der Resonanzfrequenz hingegen steigt der Wellenwiderstand an, da hier nur noch der ohmsche Widerstand wirkt
\end{itemize}
\end{frame}

\subsection*{Resonanz\-frequenz}
\begin{frame}
  \frametitle{Resonanzfrequenz}
  \begin{block}{Resonanzfrequenz}
    Frequenz der äußeren Anregung, bei der die resultierende Amplitude maximal wird.
  \end{block}

  Das gilt, wenn der induktive Blindwiderstand $X_L$ gleich dem kapazitiven Blindwiderstand $X_C$ ist. Damit ergibt sich für die Resonanzfrequenz $f_0$:
  \begin{block}{Resonanzfrequenz}
    \begin{center}
      $f_0 = \cfrac{1}{2\pi \cdot \sqrt{L \cdot C}}$
    \end{center}
  \end{block}
\end{frame}

%% TODO alte Herleitung - vielleicht mit in die neue einarbeiten?
% \begin{frame}
% \frametitle{ESB eines Quartzes}
% \begin{center}
%     \includegraphics[scale=0.3]{a04/Quartz-ESB.png}\\
%     Abb.9: ESB eines Quartzes \cite{wpde}
% \end{center}
% \begin{itemize}
%     \item   Für den Serienschwingkreis gilt:
% \end{itemize}
% \begin{Large}
%     \begin{center}
%         $f_S = \frac{1}{2 \pi \sqrt{L C_{s}}}$
%     \end{center}
% \end{Large}
% \begin{itemize}
%     \item   Für den Parallelschwingkreis gilt:
% \end{itemize}
% \begin{Large}
%     \begin{center}
%         $f_P = \frac{1}{2 \pi \sqrt{L C_{ges}}}$
%     \end{center}
% \end{Large}
% \end{frame}
% 
% \begin{frame}
% \frametitle{Resonanzfrequenz}
% \begin{itemize}
%     \item Für Resonanz muss gelten $X_L = X_C$
% \end{itemize}
% \begin{LARGE}
% \begin{center}
%     $j \omega L = \frac{-j}{\omega C}$
% \end{center}
% \end{LARGE}
% \begin{itemize}
%     \item Nun stellen wir nach $\omega$ um
% \end{itemize}
% \begin{LARGE}
% \begin{center}
%     $\omega^2 = \frac{-j}{j L C}$
% \end{center}
% \end{LARGE}
% \begin{itemize}
%     \item   Nun lösen wir noch das Quadrat auf und mult mit $\frac{j}{j}$
% \end{itemize}
% \begin{LARGE}
% \begin{center}
%     $\omega = \frac{-1}{1 \cdot \sqrt{L C}}$
% \end{center}
% \end{LARGE}
% \begin{itemize}
%     \item   Nachdem wir $\omega$ durch $2 \pi f$ ersetzt und durch $2 \pi$ geteilt haben, erhalten wir die entgültige Form:
% \end{itemize}
% \begin{center}
%     \LARGE{$ |f_{res}| = \frac{1}{2 \pi \sqrt{L C}} $}
% \end{center}
% \end{frame}

\begin{frame}
  \frametitle{Resonanzfrequenz}
  Herleitung:
  \begin{align*}
    X_L &= X_C \\
    \omega \cdot L &= \frac{1}{\omega \cdot C} & \text{mit } \omega = 2\pi \cdot f \\
    2\pi \cdot f \cdot L &= \frac{1}{2\pi \cdot f \cdot C} & \cdot\  2\pi \cdot f \\
    4\pi^2 \cdot f^2 \cdot L &= \frac{1}{C} & \div\  L \\  
    4\pi^2 \cdot f^2 &= \frac{1}{L \cdot C} & \sqrt{\ } \\
    2\pi \cdot f &= \frac{1}{\sqrt{L \cdot C}} & \div\  2\pi \\
    f &= \frac{1}{2\pi \cdot \sqrt{L \cdot C}}
  \end{align*}
\end{frame}


\begin{frame}
  \begin{tabular}{l||p{.8\textwidth}}\hline
    \textbf{TD207} & \textbf{Wie groß ist die Resonanzfrequenz dieser Schaltung, wenn $C_1 = 0,1nF$, $C_2 = 1,5nF$, $C_3 = 220pF$ und $L = 1mH$ beträgt?} \\
    & \includegraphics[width=.8\textwidth,height=.3\textheight,keepaspectratio]{a04/td207.png} \\ \hline\hline
    A & $1,18 kHz$ \\ \hline
    B \only<2>\checkmark & $117,973 kHz$ \\ \hline
    C & $1,17973 MHz$ \\ \hline
    D & $11,979 kHz$ \\ \hline
  \end{tabular}
\end{frame}

%% -> siehe Aufgabenblatt
% \begin{frame}
%   \begin{tabular}{l||p{.8\textwidth}}\hline
%     \textbf{TD209} & \textbf{Welche Resonanzfrequenz $f_{res}$ hat die Parallelschaltung einer Spule von $2 \mu H$ mit einem Kondensator von $60 pF$ und einem Widerstand von $10 k\Omega$?} \\ \hline\hline
%     A & $145,288 kHz$ \\ \hline
%     B & $1,45288 MHz$ \\ \hline
%     C & $145,288 MHz$ \\ \hline
%     D \only<2>\checkmark & $14,5288 MHz$ \\ \hline
%   \end{tabular}
% \end{frame}



\subsection*{Bandbreite}
\begin{frame}
\frametitle{Bandbreite eines Schwingkreises}
\begin{center}
	\includegraphics[scale=0.3]{a04/bandwidth.png}\\
	\tiny{Abb.7: Bandbreite \cite{wmen}}
\end{center}
Untere $f_L$ und obere Grenzfrequenz $f_H$ festgelegt beim $-3dB$-Punkt.
\end{frame}

\subsection*{Güte}

\begin{frame}
  \frametitle{Die Güte}
  \begin{itemize}
    \item Bandbreite hängt von der Güte des Schwingkreises ab
    \item Güte hängt vom (reellen) Widerstand der Spule $X_L$ ab
    \item Kondensatorverluste sind bei niedrigen und mittleren Frequenzen vernachlässigbar klein
  \end{itemize}
  \vspace{1em}
  \begin{block}{Reihenschwingkreis}
    \begin{center}
      $Q = \cfrac{X_L}{R_S}$
    \end{center}
  \end{block}
  \begin{block}{Parallelschwingkreis}
    \begin{center}
      $Q = \cfrac{R_P}{X_L}$
    \end{center}
  \end{block}
\end{frame}

\begin{frame}
  \frametitle{Die Güte}
  Kennt man die Güte und die Resonanzfrequenz $f_0$ eines Schwingkreises, so lässt sich die Bandbreite bestimmen:
  \begin{block}{Bandbreite}
    \begin{center}
      $B = \cfrac{f_0}{Q}$
    \end{center}
  \end{block}
  Und damit ergibt sich dieser Zusammenhang:
  \begin{block}{Güte}
    \begin{center}
      $Q = \cfrac{f_0}{B} = \cfrac{R_P}{X_L} = \cfrac{X_L}{R_S}$
    \end{center}
  \end{block}
\end{frame}

\begin{frame}
  \begin{tabular}{l||p{.8\textwidth}} \hline
    \textbf{TD214} & \textbf{Welchen Gütefaktor $Q$ hat die Reihenschaltung einer Spule von $100 \mu H$ mit einem Kondensator von $0,01 \mu F$ und einem Widerstand von $10 \Omega$?} \\ \hline\hline
    A & 1 \\ \hline
    B & 0,1 \\ \hline
    C \only<2>\checkmark & 10 \\ \hline
    D & 100 \\ \hline
  \end{tabular}
\end{frame}

\section*{Quarz}
\begin{frame}
  \frametitle{Der Quarz als Schwingkreis}
  \begin{center}
    \includegraphics[width=\textwidth,height=.4\textheight,keepaspectratio]{a04/Quartz.jpg}\\
    {\tiny Abb.8: Verschiedene Bauformen von Quarzen}
  \end{center}
  \begin{itemize}
    \item Englisch: quar\textbf{t}z
    \item Besteht aus reinem Siliziumdioxid und wird aus einem Quarzkristall als dünnes Plättchen herausgeschnitten
    \item Verhalten ist durch den umgekehrten piezoelektrischen Effekt gekennzeichnet
    \item Ist ein Schwingkreis von hoher Güte und geringer Bandbreite
    \item Bessere Frequenzstabilität als LC-Oszillatoren
  \end{itemize}
\end{frame}

\begin{frame}
\frametitle{Zusatzwissen für Interessierte: ESB eines Quarzes}
\begin{center}
    \includegraphics[scale=0.25]{a04/Quartz-ESB.png}\\
    Abb.9: ESB eines Quartzes \cite{wpde}
\end{center}
\begin{itemize}
    \item   Für den Serienschwingkreis gilt:
\end{itemize}
\begin{Large}
    \begin{center}
        $f_S = \frac{1}{2 \pi \sqrt{L C_{s}}}$
    \end{center}
\end{Large}
\begin{itemize}
    \item   Für den Parallelschwingkreis gilt:
\end{itemize}
\begin{Large}
    \begin{center}
        $f_P = \frac{1}{2 \pi \sqrt{L C_{ges}}}$
    \end{center}
\end{Large}
\end{frame}

\begin{frame}
  \begin{tabular}{l||p{.8\textwidth}} \hline
    \textbf{TF412} & \textbf{Ein Frequenzmarken-Generator in einem Empfänger sollte möglichst}\\ \hline \hline
    A & ein BFO sein. \\ \hline
    B & ein RC-Oszillator sein. \\ \hline
    C \only<2>\checkmark & ein Quarzoszillator sein. \\ \hline
    D & ein LC-Oszillator sein. \\ \hline
  \end{tabular}
\end{frame}

\begin{frame}
  \begin{tabular}{l||p{.8\textwidth}} \hline
    \textbf{TD234} & \textbf{Ein Quarzfilter mit einer der \emph{(sic!)} 3-dB-Bandbreite von 500~Hz eignet sich besonders zur Verwendung in einem Sendeempfänger für} \\ \hline\hline
    A & SSB. \\ \hline
    B & FM. \\ \hline
    C & AM. \\ \hline
    D \only<2->\checkmark & CW. \\ \hline
  \end{tabular}

  \pause
  \vspace{2em}
  Die Frage gibt für alle Antwortmöglichkeiten, aber unterschiedlichen Bandbreiten:
  \begin{description}
    \item[2,3~kHz $\rightarrow$] \only<3>{SSB}
    \item[6~kHz $\rightarrow$] \only<3>{AM}
    \item[12~kHz $\rightarrow$] \only<3>{FM}
  \end{description}
\end{frame}

\section*{Filter}
\subsection*{Tiefpass}
\begin{frame}
\frametitle{Tiefpass}
\begin{center}
	\includegraphics[width=\textwidth,height=.5\textheight,keepaspectratio]{e07/LC-Tiefpass.png}\\
	Abb.8: LC-Tiefpass
\end{center}
\begin{itemize}
	\item Bei steigender Frequenz sinkt der Blindwiderstand $X_L$ und der Blindwiderstand $X_C$ steigt
	\item Bei sinkender Frequenz hingegen steigt $X_L$ und $X_C$ sinkt
	\item Dadurch werden nur niedrige Frequenzen durchgelassen 
\end{itemize}
\end{frame}

\subsection*{Hochpass}
\begin{frame}
\frametitle{Hochpass}
\begin{center}
	\includegraphics[width=\textwidth,height=.5\textheight,keepaspectratio]{e07/LC-Hochpass.png}\\
	Abb.9: LC-Hochpass
\end{center}
\begin{itemize}
	\item Bei steigender Frequenz steigt der Blindwiderstand $X_L$ und der Blindwiderstand $X_C$ sinkt
	\item Bei sinkender Frequenz hingegen sinkt $X_L$ und $X_C$ steigt
	\item Dadurch werden nur hohe Frequenzen durchgelassen 
\end{itemize}
\end{frame}

\subsection*{Bandpass}
\begin{frame}
  \frametitle{Bandpass}
  \begin{center}
    \includegraphics[width=\textwidth,height=.5\textheight,keepaspectratio]{a04/BandpassSpulen.png}\\
    Abb.12: Bandpass aus induktiv gekoppelten Schwingkreisen \cite{wpde}
  \end{center}
  \begin{itemize}
    \item Mehrere Parallelschwingkreise können zu Bandpässen gekoppelt werden
    \item Je nachdem wie fest\,/\,lose die Schwingkreise gekoppelt sind, ändert sich die Bandbreite des Bandpasses
  \end{itemize}
\end{frame}

\subsection*{Saugkreis}
\begin{frame}
  \frametitle{Saugkreis}
  \begin{center}
    % https://upload.wikimedia.org/wikipedia/commons/c/c9/Saugkreis.png
    \includegraphics<1>[width=\textwidth,height=.5\textheight,keepaspectratio]{e07/Saugkreis.png}
    \includegraphics<2>[width=\textwidth,height=.5\textheight,keepaspectratio]{e07/SerirenschwSig.png}
  \end{center}
  \pause
  \begin{itemize}
    \item vor und nach der Resonanzfrequenz hoher Widerstand
    \item nur Wechselspannungen mit Frequenzen in der Nähe der Resonanzfrequenz werden durchgelassen
    \item Anwendung: Audiotechnik
  \end{itemize}
\end{frame}

\subsection*{Sperrkreis}
\begin{frame}
  \frametitle{Sperrkreis}
  \begin{center}
    % https://upload.wikimedia.org/wikipedia/commons/e/ef/Sperrkreis.png
    \includegraphics<1>[width=\textwidth,height=.5\textheight,keepaspectratio]{e07/Sperrkreis.png}
    \includegraphics<2>[width=\textwidth,height=.5\textheight,keepaspectratio]{e07/ParallelschwSig.png}
  \end{center}
  \pause
  \begin{itemize}
    \item bei der Resonanzfrequenz hoher Widerstand
    \item die Resonanzfrequenz wird blockiert
    \item Anwendungen: Mehrbandantennen; Filtern von starken Sendern
  \end{itemize}
\end{frame}

\subsection*{Resonanz\-trans\-formation}
\begin{frame}
  \frametitle{Resonanztransformation}
  \begin{center}
    \includegraphics[scale=0.2]{a04/Pi-Filter.png}\\
    Abb.13: Pi- oder auch Collinsfilter \cite{wpde}
  \end{center}
  \begin{itemize}
    \item Schwingkreise in Resonanz eignen sich gut zum Anpassen von Impedanzen
    \item Nicht die Induktivität, sondern die Kapazitäten sind für die Anpassung verantwortlich
    \item Oft werden Drehkondensatoren benutzt, um stufenlos anpassen zu können
    \item Eingesetzt in Tunern oder Verstärkern (mit den zwei Drehkondensatoren \emph{Load} und \emph{Plate}).
  \end{itemize}
\end{frame}

%% Passt eher zur Kopplung
% \begin{frame}
%   \begin{columns}
%     \column{.65\textwidth}
%     \begin{tabular}{l||p{.8\textwidth}} \hline
%       \textbf{TD230} & \textbf{Das folgende Bild zeigt ein typisches ZF-Filter und vier seiner möglichen Übertragungskurven (a bis d). Welche Kurve ergibt sich bei kritischer Kopplung und welche bei überkritischer Kopplung?} \\ \hline\hline
%       A \only<2>\checkmark & Die b-Kurve zeigt kritische, die a-Kurve zeigt überkritische Kopplung. \\ \hline
%       B & Die a-Kurve zeigt kritische, die b-Kurve zeigt überkritische Kopplung. \\ \hline
%       C & Die c-Kurve zeigt kritische, die b-Kurve zeigt überkritische Kopplung. \\ \hline
%       D & Die d-Kurve zeigt kritische, die c-Kurve zeigt überkritische Kopplung. \\ \hline
%     \end{tabular}
%     \column{.3\textwidth}
%       \includegraphics[width=\textwidth,height=.95\textheight,keepaspectratio]{a04/td230.png}
%   \end{columns}
% \end{frame}


\renewcommand{\refname}{Referenzen}

\hypertarget{refs}{}
\textcolor{white}{} \\ %\vspace{} geht nicht
\Large Referenzen/Links
\footnotesize

\begin{thebibliography}{}
    \bibitem{a04}   Moltrecht A 04: \\
                    \url{https://www.darc.de/der-club/referate/ajw/lehrgang-ta/a04/}
    
    \bibitem{wpde}    Wikipedia DE: \\
                    \url{http://de.wikipedia.org/wiki/Elektrische_Energie#Elektrische_Energie_in_einem_elektrischen_Feld}\\ 
                    \url{http://de.wikipedia.org/wiki/Datei:Verschiedene_Schwingquarze.jpg}\\
                    \url{http://de.wikipedia.org/wiki/Datei:Schwingquarz-Ersatzschaltbild.png}\\
                    \url{http://de.wikipedia.org/wiki/Datei:BandpassSpulen.png}\\
    				\url{http://de.wikipedia.org/wiki/Datei:ResoTrafo_3.svg}\\
    				
    \bibitem{wpen}	Wikipedia EN:\\
    				\url{http://en.wikipedia.org/wiki/File:Tuned_circuit_animation_3.gif}\\
    				
    \bibitem{wmde}	Wikimedia DE:\\
    				\url{http://commons.wikimedia.org/wiki/File:LC_circuit_4_times_new_version.svg?uselang=de}\\
   \vspace{1cm}
   \bibitem{wmen}	Wikimedia EN:\\
   					\url{http://commons.wikimedia.org/wiki/File:RLC_series_circuit_v1.svg}\\
   					\url{http://commons.wikimedia.org/wiki/File:Resonanzwiderstand_serie.svg}\\
   					\url{http://commons.wikimedia.org/wiki/File:KondiSpuleWiderstandParallel.svg}\\
   					\url{http://commons.wikimedia.org/wiki/File:Resonanzwiderstand_parallel.svg}\\
   				  					
\end{thebibliography} 

% Hier könnte noch eine Kontaktfolie stehen

\end{document}

