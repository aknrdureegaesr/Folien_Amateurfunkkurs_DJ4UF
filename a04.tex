% Foliensatz: "AFu-Kurs nach DJ4UF" von DK0TU, Amateurfunkgruppe der TU Berlin
% Lizenz: CC BY-NC-SA 3.0 de (http://creativecommons.org/licenses/by-nc-sa/3.0/de/)
% Autoren: Martin Deutschmann, Lars Weiler <dc4lw@darc.de>
% Korrekturen: Sebastian Lange <dl7bst@dk0tu.de>

preamble.dk0tu.tex
\subtitle{Technik Klasse A 04: \\
  Schwingkreise \& Filter \\[2em]}
\date{Stand 11.05.2017}
 \begin{document}

\begin{frame}
    \titlepage
    \vfill
    \begin{center}
        \ccbyncsaeu\\
        {\tiny This work is licensed under the \em{Creative Commons Attribution-NonCommercial-ShareAlike 3.0 License}.}\\[0.5ex]
         \tiny Amateurfunkgruppe der Technische Universität Berlin (AfuTUB), DKØTU
         %\includegraphics[scale=0.5]{img/DK0TU_Logo.pdf}
    \end{center}
\end{frame}



\section*{Schwingkreis}
\begin{frame}
  \frametitle{Schwingkreise}
  \begin{center}
    \begin{figure}
      \includegraphics[width=\textwidth,height=.75\textheight,keepaspectratio]{a04/Schwingkreis_reihe.png}
      \includegraphics[width=\textwidth,height=.75\textheight,keepaspectratio]{a04/Schwingkreis_parallel.png}
      \caption{Serien- \& Parallelschwingkreis}
    \end{figure}
  \end{center}
\end{frame}

\begin{frame}
  \frametitle{Schwingungserzeugung}
  \begin{columns}
    \column{.6\textwidth}
    \begin{center}
      \begin{figure}
        \includegraphics[width=\textwidth,height=.75\textheight,keepaspectratio]{a04/Schwingkreis.png}
        \attribcaption{Energie in einem LC-Schwingkreis}{X3ntar}{https://commons.wikimedia.org/wiki/File:LC_circuit_4_times_new_version.svg}{\ccpd}
      \end{figure}
    \end{center}
    \column{.4\textwidth}
    \begin{itemize}
      \item durch Verluste kommt es zur gedämpften Schwingung\\
      \item \href{http://en.wikipedia.org/wiki/File:Tuned_circuit_animation_3.gif}{\ExternalLink animierte Darstellung}
    \end{itemize}
  \end{columns}
\end{frame}

\subsection*{Reihen\-schwing\-kreis}
\begin{frame}
  \frametitle{Reihenschwingkreis}
  \begin{center}
    \begin{minipage}{0.5\textwidth}
      \begin{figure}
        \includegraphics[height=.45\textheight,width=\textwidth,keepaspectratio]{a04/Serirenschw.png}
        \attribcaption{Serienschwingkreis}{V4711}{https://commons.wikimedia.org/wiki/File:RLC_series_circuit_v1.svg}{\ccbysa}
      \end{figure}
    \end{minipage}
    \begin{minipage}{0.49\textwidth}
      \begin{figure}
        \includegraphics[height=.45\textheight,width=\textwidth,keepaspectratio]{a04/SerirenschwSig.png}
        \attribcaption{Resonanzwiderstand}{Unknown}{https://commons.wikimedia.org/wiki/File:Resonanzwiderstand_serie.svg}{\ccpd}
      \end{figure}
    \end{minipage}
  \end{center}
  \begin{itemize}
    \item Im Verlauf der Frequenzänderung ändert sich der Gesamtwellenwiderstand Z des Schwingkreises
    \item Der Schwingkreis hat als minimale Impedanz seinen ohmschen Wert, da sich bei der Resonanzfrequenz $f_R$ die induktiven und kapazitiven Anteile gegenseitig aufheben
  \end{itemize}
\end{frame}

\subsection*{Parallel\-schwing\-kreis}
\begin{frame}
  \frametitle{Parallelschwingkreis}
  \begin{center}
    \begin{minipage}{0.5\textwidth}
      \begin{figure}
        \includegraphics[height=.45\textheight,width=\textwidth,keepaspectratio]{a04/Parallelschw.png}
        \attribcaption{Parallelschwingkreis}{Tillmann Walther}{https://commons.wikimedia.org/wiki/File:KondiSpuleWiderstandParallel.svg}{\ccpd}
      \end{figure}
    \end{minipage}
    \begin{minipage}{0.49\textwidth}
      \begin{figure}
        \includegraphics[height=.45\textheight,width=\textwidth,keepaspectratio]{a04/ParallelschwSig.png}
        \attribcaption{Resonanzwiderstand}{Unknown}{https://commons.wikimedia.org/wiki/File:Resonanzwiderstand_parallel.svg}{\ccpd}
      \end{figure}
    \end{minipage}
  \end{center}
  \begin{itemize}
    \item Der Parallelschwingkreis verhält sich genau entgegen gesetzt zum Reihenschwingkreis
    \item Dieser zeigt bei niedrigen und hohen Frequenzen das Verhalten eines Leiters
    \item Bei der Resonanzfrequenz hingegen steigt der Wellenwiderstand an, da hier nur noch der ohmsche Widerstand wirkt
  \end{itemize}
\end{frame}

\subsection*{Resonanz\-frequenz}
\begin{frame}
  \frametitle{Resonanzfrequenz}
  \begin{block}{Resonanzfrequenz}
    Frequenz der äußeren Anregung, bei der die resultierende Amplitude maximal wird.
  \end{block}

  Das gilt, wenn der induktive Blindwiderstand $X_L$ gleich dem kapazitiven Blindwiderstand $X_C$ ist. Damit ergibt sich für die Resonanzfrequenz $f_0$:
  \begin{block}{Resonanzfrequenz}
    \begin{center}
      $f_0 = \cfrac{1}{2\pi \cdot \sqrt{L \cdot C}}$
    \end{center}
  \end{block}
\end{frame}

\begin{frame}
  \frametitle{Resonanzfrequenz}
  Herleitung:
  \begin{align*}
    X_L &= X_C \\
    \omega \cdot L &= \frac{1}{\omega \cdot C} & \text{mit } \omega = 2\pi \cdot f \\
    2\pi \cdot f \cdot L &= \frac{1}{2\pi \cdot f \cdot C} & \cdot\  2\pi \cdot f \\
    4\pi^2 \cdot f^2 \cdot L &= \frac{1}{C} & \div\  L \\
    4\pi^2 \cdot f^2 &= \frac{1}{L \cdot C} & \sqrt{\ } \\
    2\pi \cdot f &= \frac{1}{\sqrt{L \cdot C}} & \div\  2\pi \\
    f &= \frac{1}{2\pi \cdot \sqrt{L \cdot C}}
  \end{align*}
\end{frame}


\begin{frame}
  \begin{tabular}{l||p{.8\textwidth}}\hline
    \textbf{TD207} & \textbf{Wie groß ist die Resonanzfrequenz dieser Schaltung, wenn $C_1 = 0,1nF$, $C_2 = 1,5nF$, $C_3 = 220pF$ und $L = 1mH$ beträgt?} \\
    & \includegraphics[width=.8\textwidth,height=.3\textheight,keepaspectratio]{a04/td207.png} \\ \hline\hline
    A & $1,18 kHz$ \\ \hline
    B \only<2>\checkmark & $117,973 kHz$ \\ \hline
    C & $1,17973 MHz$ \\ \hline
    D & $11,979 kHz$ \\ \hline
  \end{tabular}
\end{frame}

\subsection*{Bandbreite}
\begin{frame}
  \frametitle{Bandbreite eines Schwingkreises}
  \begin{center}
    \begin{figure}
      \includegraphics[width=\textwidth,height=.6\textheight,keepaspectratio]{a04/bandwidth.png}
      \attribcaption{Bandbreite}{Inductiveload}{https://commons.wikimedia.org/wiki/File:Bandwidth_2.svg}{\ccpd}
    \end{figure}
  \end{center}
  Untere $f_L$ und obere Grenzfrequenz $f_H$ festgelegt beim $-3dB$-Punkt.
\end{frame}

\subsection*{Güte}

\begin{frame}
  \frametitle{Die Güte}
  \begin{itemize}
    \item Bandbreite hängt von der Güte des Schwingkreises ab
    \item Güte hängt vom (reellen) Widerstand der Spule $X_L$ ab
    \item Kondensatorverluste sind bei niedrigen und mittleren Frequenzen vernachlässigbar klein
  \end{itemize}
  \vspace{1em}
  \begin{block}{Reihenschwingkreis}
    \begin{center}
      $Q = \cfrac{X_L}{R_S}$
    \end{center}
  \end{block}
  \begin{block}{Parallelschwingkreis}
    \begin{center}
      $Q = \cfrac{R_P}{X_L}$
    \end{center}
  \end{block}
\end{frame}

\begin{frame}
  \frametitle{Die Güte}
  Kennt man die Güte und die Resonanzfrequenz $f_0$ eines Schwingkreises, so lässt sich die Bandbreite bestimmen:
  \begin{block}{Bandbreite}
    \begin{center}
      $B = \cfrac{f_0}{Q}$
    \end{center}
  \end{block}
  Und damit ergibt sich dieser Zusammenhang:
  \begin{block}{Güte}
    \begin{center}
      $Q = \cfrac{f_0}{B} = \cfrac{R_P}{X_L} = \cfrac{X_L}{R_S}$
    \end{center}
  \end{block}
\end{frame}

\begin{frame}
  \begin{tabular}{l||p{.8\textwidth}} \hline
    \textbf{TD214} & \textbf{Welchen Gütefaktor $Q$ hat die Reihenschaltung einer Spule von $100 \mu H$ mit einem Kondensator von $0,01 \mu F$ und einem Widerstand von $10 \Omega$?} \\ \hline\hline
    A & 1 \\ \hline
    B & 0,1 \\ \hline
    C \only<2>\checkmark & 10 \\ \hline
    D & 100 \\ \hline
  \end{tabular}
\end{frame}

\section*{Quarz}
\begin{frame}
  \frametitle{Der Quarz als Schwingkreis}
  \begin{center}
    \begin{figure}
      \includegraphics[width=\textwidth,height=.35\textheight,keepaspectratio]{a04/Quartz.jpg}
      \attribcaption{Verschiedene Bauformen von Quarzen}{Stefan Riepl (Quark48)}{https://commons.wikimedia.org/wiki/File:Verschiedene_Schwingquarze.jpg?uselang=de}{\ccpd}
    \end{figure}
  \end{center}
  \begin{itemize}
    \item Englisch: quar\textbf{t}z
    \item Besteht aus reinem Siliziumdioxid und wird aus einem Quarzkristall als dünnes Plättchen herausgeschnitten
    \item Verhalten ist durch den umgekehrten piezoelektrischen Effekt gekennzeichnet
    \item Ist ein Schwingkreis von hoher Güte und geringer Bandbreite
    \item Bessere Frequenzstabilität als LC-Oszillatoren
  \end{itemize}
\end{frame}

\begin{frame}
  \frametitle{Zusatzwissen für Interessierte: ESB eines Quarzes}
  \begin{columns}
    \column{.6\textwidth}
    \begin{center}
      \begin{figure}
        \includegraphics[width=\textwidth,height=.75\textheight,keepaspectratio]{a04/Quartz-ESB.png}
        \attribcaption{Ersatzschaltbild eines Schwingquarzes}{Elcap, Jens Both}{https://commons.wikimedia.org/wiki/File:Schwingquarz-Ersatzschaltbild.png?uselang=de}{\ccpd}
      \end{figure}
    \end{center}
    \column{.39\textwidth}
    \begin{block}{Serienschwingkreis}
      $f_S = \frac{1}{2 \pi \sqrt{L C_{s}}}$
    \end{block}
    \begin{block}{Parallelschwingkreis}
      $f_P = \frac{1}{2 \pi \sqrt{L C_{ges}}}$
    \end{block}
  \end{columns}
\end{frame}

\begin{frame}
  \begin{tabular}{l||p{.8\textwidth}} \hline
    \textbf{TF412} & \textbf{Ein Frequenzmarken-Generator in einem Empfänger sollte möglichst}\\ \hline \hline
    A & ein BFO sein. \\ \hline
    B & ein RC-Oszillator sein. \\ \hline
    C \only<2>\checkmark & ein Quarzoszillator sein. \\ \hline
    D & ein LC-Oszillator sein. \\ \hline
  \end{tabular}
\end{frame}

\begin{frame}
  \begin{tabular}{l||p{.8\textwidth}} \hline
    \textbf{TD234} & \textbf{Ein Quarzfilter mit einer der \emph{(sic!)} 3-dB-Bandbreite von 500~Hz eignet sich besonders zur Verwendung in einem Sendeempfänger für} \\ \hline\hline
    A & SSB. \\ \hline
    B & FM. \\ \hline
    C & AM. \\ \hline
    D \only<2->\checkmark & CW. \\ \hline
  \end{tabular}

  \pause
  \vspace{2em}
  Die Frage gibt es für alle Antwortmöglichkeiten, aber unterschiedlichen Bandbreiten:
  \begin{description}
    \item[2,3~kHz $\rightarrow$] \only<3>{SSB}
    \item[6~kHz $\rightarrow$] \only<3>{AM}
    \item[12~kHz $\rightarrow$] \only<3>{FM}
  \end{description}
\end{frame}

\section*{Filter}
\subsection*{Tiefpass}
\begin{frame}
  \frametitle{Tiefpass}
  \begin{center}
    \begin{figure}
      \includegraphics[width=\textwidth,height=.45\textheight,keepaspectratio]{e07/LC-Tiefpass.png}
      \caption{LC-Tiefpass}
    \end{figure}
  \end{center}
  \begin{itemize}
    \item Bei steigender Frequenz sinkt der Blindwiderstand $X_L$ und der Blindwiderstand $X_C$ steigt
    \item Bei sinkender Frequenz hingegen steigt $X_L$ und $X_C$ sinkt
    \item Dadurch werden nur niedrige Frequenzen durchgelassen
  \end{itemize}
\end{frame}

\subsection*{Hochpass}
\begin{frame}
  \frametitle{Hochpass}
  \begin{center}
    \begin{figure}
      \includegraphics[width=\textwidth,height=.45\textheight,keepaspectratio]{e07/LC-Hochpass.png}
      \caption{LC-Hochpass}
    \end{figure}
  \end{center}
  \begin{itemize}
    \item Bei steigender Frequenz steigt der Blindwiderstand $X_L$ und der Blindwiderstand $X_C$ sinkt
    \item Bei sinkender Frequenz hingegen sinkt $X_L$ und $X_C$ steigt
    \item Dadurch werden nur hohe Frequenzen durchgelassen
  \end{itemize}
\end{frame}

\subsection*{Bandpass}
\begin{frame}
  \frametitle{Bandpass}
  \begin{center}
    \begin{figure}
      \includegraphics[width=\textwidth,height=.5\textheight,keepaspectratio]{a04/BandpassSpulen.png}
      \attribcaption{Bandfilter mit magnetisch gekoppelten Spulen}{PeterFrankfurt}{https://commons.wikimedia.org/wiki/File:BandpassSpulen.png?uselang=de}{\ccpd}
    \end{figure}
  \end{center}
  \begin{itemize}
    \item Mehrere Parallelschwingkreise können zu Bandpässen gekoppelt werden
    \item Je nachdem wie fest\,/\,lose die Schwingkreise gekoppelt sind, ändert sich die Bandbreite des Bandpasses
  \end{itemize}
\end{frame}

\subsection*{Saugkreis}
\begin{frame}
  \frametitle{Saugkreis}
  \begin{columns}
    \column{.47\textwidth}
    \begin{center}
      \begin{figure}
        \includegraphics[width=\textwidth,height=.5\textheight,keepaspectratio]{e07/Saugkreis.png}
        \attribcaption{Saugkreis}{Herbertweidner}{https://commons.wikimedia.org/wiki/File:Saugkreis.png}{\ccpd}
      \end{figure}
    \end{center}
    \column{.47\textwidth}
    \begin{center}
      \begin{figure}
        \includegraphics[width=\textwidth,height=.5\textheight,keepaspectratio]{e07/SerirenschwSig.png}
        \attribcaption{Resonanzwiderstand}{Unknown}{https://commons.wikimedia.org/wiki/File:Resonanzwiderstand_serie.svg}{\ccpd}
      \end{figure}
    \end{center}
  \end{columns}
  \pause
  \begin{itemize}
    \item bei Resonanzfrequenz besonders geringer Gesamtwiderstand
    \item Wechselspannung umgeht bei Resonanzfrequenz den Widerstand
    \item Anwendung: Kurzschluss einer bestimmten Frequenz; Unterdrücken unerwünschter Signale; 50Hz-Filter
  \end{itemize}
\end{frame}

\subsection*{Sperrkreis}
\begin{frame}
  \frametitle{Sperrkreis}
  \begin{columns}
    \column{.47\textwidth}
    \begin{center}
      \begin{figure}
        \includegraphics[width=\textwidth,height=.5\textheight,keepaspectratio]{e07/Sperrkreis.png}
        \attribcaption{Sperrkreis}{Herbertweidner}{https://commons.wikimedia.org/wiki/File:Sperrkreis.png}{\ccpd}
      \end{figure}
    \end{center}
    \column{.47\textwidth}
    \begin{center}
      \begin{figure}
        \includegraphics[width=\textwidth,height=.5\textheight,keepaspectratio]{e07/ParallelschwSig.png}
        \attribcaption{Parallelschwingkreis}{Tillmann Walther}{http://commons.wikimedia.org/wiki/File:KondiSpuleWiderstandParallel.svg}{\ccpd}
      \end{figure}
    \end{center}
  \end{columns}
  \pause
  \begin{itemize}
    \item bei der Resonanzfrequenz hoher Widerstand
    \item die Resonanzfrequenz wird blockiert
    \item Anwendungen: Mehrbandantennen; Filtern von starken Sendern
  \end{itemize}
\end{frame}

\subsection*{Resonanz\-trans\-formation}
\begin{frame}
  \frametitle{Resonanztransformation}
  \begin{columns}
    \column{.5\textwidth}
    \begin{center}
      \begin{figure}
        \includegraphics[width=\textwidth,height=.75\textheight,keepaspectratio]{a04/Pi-Filter.png}\\
        \attribcaption{Pi- oder auch Collinsfilter}{Frank Murmann}{https://commons.wikimedia.org/wiki/File:ResoTrafo_3.svg?uselang=de}{\ccpd}
      \end{figure}
    \end{center}
    \column{.5\textwidth}
    \begin{itemize}
      \item Schwingkreise in Resonanz eignen sich gut zum Anpassen von Impedanzen
      \item Nicht die Induktivität, sondern die Kapazitäten sind für die Anpassung verantwortlich
      \item Oft werden Drehkondensatoren benutzt, um stufenlos anpassen zu können
      \item Eingesetzt in Tunern oder Verstärkern (mit den zwei Drehkondensatoren \emph{Load} und \emph{Plate}).
    \end{itemize}
  \end{columns}
\end{frame}


\renewcommand{\refname}{Referenzen}

\hypertarget{refs}{}
\textcolor{white}{} \\ %\vspace{} geht nicht
\Large Referenzen/Links
\footnotesize

\begin{thebibliography}{}
  \bibitem{darc}   Moltrecht A 04: \\
    \url{https://www.darc.de/der-club/referate/ajw/lehrgang-ta/a04/}

  \bibitem{wpde}    Wikipedia DE: \\
    \url{http://de.wikipedia.org/wiki/Elektrische_Energie#Elektrische_Energie_in_einem_elektrischen_Feld}

\end{thebibliography}

% Hier könnte noch eine Kontaktfolie stehen

\end{document}

