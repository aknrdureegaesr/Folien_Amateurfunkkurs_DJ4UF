% Foliensatz: "AFu-Kurs nach DJ4UF" von DK0TU, Amateurfunkgruppe der TU Berlin
% Lizenz: CC BY-NC-SA 3.0 de (http://creativecommons.org/licenses/by-nc-sa/3.0/de/)
% Autoren: Martin Deutschmann

preamble.dk0tu.tex
\subtitle{Technik Klasse A 04: \\
          Schwingkreise \& Filter \\[2em]}
\date{Stand \today}
 \begin{document}

\begin{frame}
    \titlepage
    \vfill
    \begin{center}
        \ccbyncsaeu\\
        {\tiny This work is licensed under the \em{Creative Commons Attribution-NonCommercial-ShareAlike 3.0 License}.}\\[0.5ex]
         \tiny Amateurfunkgruppe der Technische Universität Berlin (AfuTUB), DKØTU
         %\includegraphics[scale=0.5]{img/DK0TU_Logo.pdf}
    \end{center}
\end{frame}


%fixme Links zur bibliography einfuegen

\section*{Schwingungsvorgang}
\begin{frame}
\frametitle{Schwingkreise}
	\begin{center}
		\includegraphics[scale=0.4]{a04/Schwingkreis_reihe.png}
		\includegraphics[scale=0.4]{a04/Schwingkreis_parallel.png}\\
		Abb.1: Serien- \& Paralellschwingkreise
	\end{center}
\end{frame}

\begin{frame}
\frametitle{Schwingungserzeugung}
\begin{center}
	\includegraphics[scale=0.35]{a04/Schwingkreis.png}\\
	Abb.2: Energie in einem LC-Schwingkreis \cite{wmde} \\
	\vspace{3mm}
	\begin{itemize}
		\item durch Verluste kommt es zur gedämpfte Schwingung\\
		\item animierte Darstellung (\url{http://en.wikipedia.org/wiki/File:Tuned_circuit_animation_3.gif})
	\end{itemize}
\end{center}
\end{frame}

\begin{frame}
\frametitle{Resonanzfrequenz}
	\begin{itemize}
		\item Wir benötigen die Blindwiderstände $X_C$ und $X_L$
	\end{itemize}

	\begin{LARGE}
		$X_C = \frac{1}{j \omega C}$ 
		\hspace{3cm}
		$X_L = j \omega L$ \\
		\vspace{1cm}
		$X_C = \frac{j}{j^2 \omega C}$ \\
		\vspace{1cm}
		$X_C = \frac{-j}{\omega C}$ \\
	\end{LARGE}
\end{frame}

\begin{frame}
\frametitle{Resonanzfrequenz}
\begin{itemize}
	\item Für Resonanz muss gelten $X_L = X_C$
\end{itemize}
\begin{LARGE}
\begin{center}
	$j \omega L = \frac{-j}{\omega C}$
\end{center}
\end{LARGE}
\begin{itemize}
	\item Nun stellen wir nach $\omega$ um
\end{itemize}
\begin{LARGE}
\begin{center}
	$\omega^2 = \frac{-j}{j L C}$
\end{center}
\end{LARGE}
\begin{itemize}
	\item	Nun lösen wir noch das Quadrat auf und mult mit $\frac{j}{j}$
\end{itemize}
\begin{LARGE}
\begin{center}
	$\omega = \frac{-1}{1 \cdot \sqrt{L C}}$
\end{center}
\end{LARGE}
\begin{itemize}
	\item	Nachdem wir $\omega$ durch $2 \pi f$ ersetzt und durch $2 \pi$ geteilt haben, erhalten wir die entgültige Form:
\end{itemize}
\begin{center}
	\LARGE{$ |f_{res}| = \frac{1}{2 \pi \sqrt{L C}} $}
\end{center}
\end{frame}

\section*{Reihenschwingkreis}
\begin{frame}
\frametitle{Reihenschwingkreis}
\begin{center}
	\begin{minipage}{0.4\textwidth}
	\includegraphics[scale=0.8]{a04/Serirenschw.png}\\
	\tiny{Abb.3: Serienschwingkreis \cite{wmen}}
	\end{minipage}
	\begin{minipage}{0.4\textwidth}
	\includegraphics[scale=0.2]{a04/SerirenschwSig.png}\\
	\tiny{Abb.4: Resonanzwiderstand \cite{wmen}} 
	\end{minipage}
\end{center}
\begin{itemize}
	\item Im Verlauf der Frequenzänderung ändert sich der Gesamtwellenwiderstand Z des Schwingkreises
	\item Der Schwingkreis hat als minimale Impedanz seinen ohmschen Wert, da sich bei der Resonanzfrequenz $f_R$ die induktiven und kapazitiven Anteile gegenseitig aufheben
\end{itemize}
\end{frame}

\section*{Parallelschwingkreis}
\begin{frame}
\frametitle{Parallelschwingkreis}
\begin{center}
	\begin{minipage}{0.4\textwidth}
	\includegraphics[scale=1]{a04/Parallelschw.png}\\
	\tiny{Abb.5: Parallelschwingkreis \cite{wmen}}
	\end{minipage}
	\begin{minipage}{0.4\textwidth}
	\includegraphics[scale=0.2]{a04/ParallelschwSig.png}\\
	\tiny{Abb.6: Resonanzwiderstand \cite{wmen}} 
	\end{minipage}
\end{center}
\begin{itemize}
	\item Der Parallelschwingkreis verhält sich genau entgegen gesetzt zum Reihenschwingkreis
	\item Dieser zeigt bei niedrigen und hohen Frequenzen das Verhalten eines Leiters
	\item Bei der Resonanzfrequenz hingegen steigt der Wellenwiderstand an, da hier nur noch der ohmsche Widerstand wirkt
\end{itemize}
\end{frame}

\section*{Bandreite}
\begin{frame}
\frametitle{Bandbreite}
\begin{center}
	\includegraphics[scale=0.3]{a04/bandwidth.png}\\
	\tiny{Abb.7: Bandweite \cite{wmen}}
\end{center}
\end{frame}

\section*{Die Güte}

\begin{frame}
\frametitle{Die Güte}
\begin{itemize}
	\item Bandbreite hängt von der Güte des Schwingkreises ab
	\item Güte hängt von der Widerstand der Spule $X_L$ ab
	\item	Kondensatorverluste sind bei niedrigen und mittleren Frequenzen vernachlässigbar klein
\end{itemize}
\vspace{3mm}
\begin{itemize}
	\item	Für den Reihenschwingkreis gilt:
\end{itemize}
\begin{Large}
	\begin{center}
		$Q = \frac{X_L}{R_S}$
	\end{center}
\end{Large}
\begin{itemize}
	\item	Für den Parallelschwingkreis gilt:
\end{itemize}
\begin{Large}
	\begin{center}
		$Q = \frac{R_P}{X_L}$
	\end{center}
\end{Large}
\begin{itemize}
	\item	Kennt man die Güte und die Resonanzfrequenzg $f_0$ eines Schwingkreises, so lässt sich die Bandbreite bestimmen:
\end{itemize}
\begin{Large}
	\begin{center}
		$B = \frac{f_0}{Q}$
	\end{center}
\end{Large}
\end{frame}

\section*{Der Quartz als Schwingkreis}
\begin{frame}
\frametitle{Der Quartz als Schwingkreis}
\begin{center}
	\includegraphics[scale=0.3]{a04/Quartz.jpg}\\
	Abb.8: Verschiedene Bauformen von Quartzen
\end{center}
\begin{itemize}
	\item	Besteht aus reinem Siliziumdioxid und wird aus einem Quarzkristall als dünnes Plätchen herausgeschnitten
	\item	Verhalten ist du den umgekehrten piezoelektrischen Effekt gekennzeichnet
	\item	Ist ein Schwingkreis von hoher Güte
\end{itemize}
\end{frame}

\begin{frame}
\frametitle{ESB eines Quartzes}
\begin{center}
	\includegraphics[scale=0.3]{a04/Quartz-ESB.png}\\
	Abb.9: ESB eines Quartzes \cite{wpde}
\end{center}
\begin{itemize}
	\item	Für den Serienschwingkreis gilt:
\end{itemize}
\begin{Large}
	\begin{center}
		$f_S = \frac{1}{2 \pi \sqrt{L C_{s}}}$
	\end{center}
\end{Large}
\begin{itemize}
	\item	Für den Parallelschwingkreis gilt:
\end{itemize}
\begin{Large}
	\begin{center}
		$f_P = \frac{1}{2 \pi \sqrt{L C_{ges}}}$
	\end{center}
\end{Large}
\end{frame}

\section*{Tiefpass}
\begin{frame}
\frametitle{Tiefpass}
\begin{center}
	\includegraphics[scale=1.2]{a04/LC-Tiefpass.png}\\
	Abb.10: LC-Tiefpass
\end{center}
\begin{itemize}
	\item Bei steigender Frequenz sinkt der Blindwiderstand $X_L$ und der Blindwiderstand $X_C$ steigt
	\item Bei sinkender Frequenz hingegen steigt $X_L$ und $X_C$ sinkt
	\item Dadurch werden nur niedrige Frequenzen durchgelassen 
\end{itemize}
\end{frame}

\section*{Hochpass}
\begin{frame}
\frametitle{Hochpass}
\begin{center}
	\includegraphics[scale=1.2]{a04/LC-Hochpass.png}\\
	Abb.11: LC-Hochpass
\end{center}
\begin{itemize}
	\item Bei steigender Frequenz steigt der Blindwiderstand $X_L$ und der Blindwiderstand $X_C$ sinkt
	\item Bei sinkender Frequenz hingegen sinkt $X_L$ und $X_C$ steigt
	\item Dadurch werden nur hohe Frequenzen durchgelassen 
\end{itemize}
\end{frame}

\section*{Der Bandpass}
\begin{frame}
\frametitle{Der Bandpass}
\begin{center}
	\includegraphics[scale=0.5]{a04/BandpassSpulen.png}\\
	Abb.12: Bandpass aus induktiv gekoppelten Schwingkreisen \cite{wpde}
\end{center}
\begin{itemize}
	\item	Mehrere Parallelschwingkreise können zu Bandpässen gekoppelt werden
	\item	Je nachdem wie fest/ lose die Schwingkreise gekoppelt sind, ändert sich die Bandbreite des Bandpasses
\end{itemize}
\end{frame}

\section*{Resonanztranzformation}
\begin{frame}
\frametitle{Resonanztransformation}
\begin{center}
	\includegraphics[scale=0.2]{a04/Pi-Filter.png}\\
	Abb.13: Pi- oder auch Collinsfilter \cite{wpde}
\end{center}
\begin{itemize}
	\item	Schwingkreise in Resonanz eignen sich gut zum Anpassen von Impedanzen
	\item	Nicht die Induktivität, sondern die Kapazitäten sind für die Anpassung verantwortlich
	\item	Oft werden Drehkondensatoren benutzt, um stufenlos anpassen zu können
\end{itemize}
\end{frame}

\begin{frame}
\frametitle{ein kleines Beispiel}
\begin{itemize}
	\item	$R_2$ sei der Ausgangswiderstand
	\item	Es soll gelten Eingangsleistung $P_1$ gleich Ausgangsleistung $P_2$
\end{itemize}
\begin{Large}
	\begin{center}
		$P_1 = P_2 \rightarrow \frac{U^2_1}{R_1} = \frac{U^2_2}{R_2} \rightarrow \frac{U_1}{U_2} = \sqrt{\frac{R_1}{R_2}} $
	\end{center}
\end{Large}
\begin{itemize}
	\item	Außerdem verhalten sich im abgestimmten Schwingkreis die Spannungen wie die Wechselstromwiderstände
\end{itemize}
\begin{Large}
	\begin{center}
		$\frac{U_1}{U_2} = \frac{X_{C1}}{X_{C2}} = \sqrt{\frac{R_1}{R_2}} \rightarrow \frac{C_2}{C_1} = \sqrt{\frac{R_1}{R_2}}$
	\end{center}
\end{Large}
\begin{itemize}
	\item	Kapazitäten verhalten sich umgekehrt proportional zu den zu transformierenden Widerständen
\end{itemize}
\end{frame}

\renewcommand{\refname}{Referenzen}

\hypertarget{refs}{}
\textcolor{white}{} \\ %\vspace{} geht nicht
\Large Referenzen/Links
\footnotesize

\begin{thebibliography}{}
    \bibitem{a04}   Moltrecht A 04: \\
                    \url{http://www.darc.de/referate/ajw/ausbildung/darc-online-lehrgang/technik-klasse-a/technik-a04/}
    
    \bibitem{wpde}    Wikipedia DE: \\
                    \url{http://de.wikipedia.org/wiki/lektrische_Energie#Elektrische_Energie_in_einem_elektrischen_Feld}\\ 
                    \url{http://de.wikipedia.org/wiki/Datei:Verschiedene_Schwingquarze.jpg}\\
                    \url{http://de.wikipedia.org/wiki/Datei:Schwingquarz-Ersatzschaltbild.png}\\
                    \url{http://de.wikipedia.org/wiki/Datei:BandpassSpulen.png}\\
    				\url{http://de.wikipedia.org/wiki/Datei:ResoTrafo_3.svg}\\
    				
    \bibitem{wpen}	Wikipedia EN:\\
    				\url{http://en.wikipedia.org/wiki/File:Tuned_circuit_animation_3.gif}\\
    				
    \bibitem{wmde}	Wikimedia DE:\\
    				\url{http://commons.wikimedia.org/wiki/File:LC_circuit_4_times_new_version.svg?uselang=de}\\
   \vspace{1cm}
   \bibitem{wmen}	Wikimedia EN:\\
   					\url{http://commons.wikimedia.org/wiki/File:RLC_series_circuit_v1.svg}\\
   					\url{http://commons.wikimedia.org/wiki/File:Resonanzwiderstand_serie.svg}\\
   					\url{http://commons.wikimedia.org/wiki/File:KondiSpuleWiderstandParallel.svg}\\
   					\url{http://commons.wikimedia.org/wiki/File:Resonanzwiderstand_parallel.svg}\\
   				  					
\end{thebibliography} 

% Hier könnte noch eine Kontaktfolie stehen

\end{document}

