% Foliensatz: "AFu-Kurs nach DJ4UF" von DK0TU, Amateurfunkgruppe der TU Berlin
% Lizenz: CC BY-NC-SA 3.0 de (http://creativecommons.org/licenses/by-nc-sa/3.0/de/)
% Autoren: Felix Baum <baum@campus.tu-berlin.de>

preamble.dk0tu.tex
\subtitle{Technik 09: \\
           Die Wellenausbreitung \\[2em]}
\date{Stand 02.12.2015}
 \begin{document}

\begin{frame}
    \titlepage
    \vfill
    \begin{center}
        \ccbyncsaeu\\
        {\tiny This work is licensed under the \em{Creative Commons Attribution-NonCommercial-ShareAlike 3.0 License}.}\\[0.5ex]
         \tiny Amateurfunkgruppe der Technische Universität Berlin (AfuTUB), DKØTU
         %\includegraphics[scale=0.5]{img/DK0TU_Logo.pdf}
    \end{center}
\end{frame}


%fixme Referenzen/Fußnoten-Systematik vereinheitlichen

\section*{Einleitung}

\begin{frame}
    \frametitle{Wie kommen die Wellen um die Welt?}
    \begin{center}
        \includegraphics[width=1\textwidth,height=.8\textheight,keepaspectratio]{e09/cqww-kontakte-2015.png}
        \footnote{\tiny Funkkontakte beim CQWW-SSB 2015 von DK\O TU}
    \end{center}
\end{frame}

\begin{frame}
    \frametitle{Raum und Bodenwelle}
	\begin{center}
        \includegraphics[width=1\textwidth,height=.8\textheight,keepaspectratio]{e09/Ionospheric_reflectionpng.png}
         \footnote{\tiny \url{https://commons.wikimedia.org/wiki/File:Ionospheric_reflection-de.svg}}
    \end{center}
\end{frame}

\begin{frame}
    \frametitle{Boden- und Raumwelle}
    \textbf{Bodenwelle}
    \begin{itemize}
	    \item Folgt der Erdkrümmung
	    \item Stark vorhanden bei Langwelle und Mittelwelle
	    \item Viele Verluste durch Berge, Wälder und Städte
    \end{itemize}
    \textbf{Raumwelle}
    \begin{itemize}
	    \item Meist genutzt bei Kurzwelle (manchmal auch UHF, VHF)
	    \item Reflektion an der Ionosphäre 
	    \item Nicht zuverlässig für Überreichweiten
    \end{itemize}
\end{frame}

\section*{Bodenwelle}
\begin{frame}
    \frametitle{Bodenwelle}
    \begin{itemize}
      \item Bei LF über $400km$
      \item $80m$ Wellenlänge ($3.5MHz$) $\rightarrow~\thicksim150km$ Reichweite
      \item $10m$ Wellenlänge ($28MHz$) $\rightarrow~\thicksim30km$ Reichweite
      \item Reichweite hängt auch stark vom Untergrund ab
    \end{itemize}
\end{frame}

\section*{Raumwelle}
    
\begin{frame}
    \frametitle{Raumwelle}
	\begin{center}
        \includegraphics[width=\textwidth,height=.8\textheight,keepaspectratio]{e09/schichten_behelf_43.png}
        \footnote{\tiny Amateurfunkbehelf S.43 \url{http://ham.granjow.net/builds/Amateurfunkbehelf.pdf}}
    \end{center}
\end{frame}

\begin{frame}
    \frametitle{Sonnenflecken-Zyklus}
    \begin{itemize}
    			\item Bei Sonneneinstrahlung werden Moleküle in der Ionosphäre durch EUV-Strahlung ionisiert
			\item Maximum ungefähr alle 11 Jahre
                        \item HF Frequenzen ab $\thicksim20MHz$ sind bei einem Minimum nicht verwendbar
       		 	\item Bei einem Maximum quasi täglich DX-Verbindungen mit weniger als $10W$ machbar 
        		\item Nächstes Maximum vermutlich 2022
    \end{itemize}
	\begin{center}
        \includegraphics[width=.8\textwidth,height=.4\textheight,keepaspectratio]{e09/Predictions_sunspot.png}
        \footnote{\tiny By Scientific data, based on prediction by David Hathaway, Public domain, via Wikimedia Commons \url{https://commons.wikimedia.org/wiki/File_3APredictions3_strip.jpg}} %\url{https://commons.wikimedia.org/wiki/File%3APredictions3_strip.jpg}
    \end{center}
\end{frame}

\begin{frame}
    \frametitle{D-Schicht}
    \begin{itemize}
    			\item Tagsüber und verschwindet nach Sonnenuntergang sehr schnell
			\item Dämpft Frequenzen unter $5MHz$ (160m und 80m unbenutzbar)
                        \item HF Frequenzen ab $\thicksim20MHz$ sind bei einem Minimum nicht verwendbar
       		 	\item Bei hoher Sonnenaktivität Möbel-Dellinger-Effekt (Kurzzeitig ganzes KW-Band unbenutzbar)
    \end{itemize}
    \begin{center}
        \includegraphics[width=.6\textwidth,height=.5\textheight,keepaspectratio]{e09/schichten_behelf_43.png}
        \footnote{\tiny Amateurfunkbehelf s.43 \url{http://ham.granjow.net/builds/Amateurfunkbehelf.pdf}}
    \end{center}
\end{frame}

\begin{frame}
    \frametitle{E-Schicht}
    \begin{itemize}
    		\item Manchmal Tagsüber
			\item Reflektiert HF-Bänder 10m, 6m
            \item Reflektiert gelegentlich 2m (Sporadic-E) \footnote{\tiny \url{http://www.dk0tu.de/blog/2012/11/27_Sporadic-E_QSO_mit_Spanien/}}
            \item mit sehr starken Signalen zwischen 750km und 2200 km (Short-Skip)
    \end{itemize}
    \begin{center}
        \includegraphics[width=.8\textwidth,height=.4\textheight,keepaspectratio]{e09/schichten_behelf_43.png}
        \footnote{\tiny Amateurfunkbehelf s.43 \url{http://ham.granjow.net/builds/Amateurfunkbehelf.pdf}}
    \end{center}
\end{frame}

\begin{frame}
    \frametitle{F-Schichten}
    \begin{itemize}
                        \item F\textsubscript{1} und F\textsubscript{2} Schicht
                        \item F\textsubscript{2} Schicht besteht auch nachts (langsame Rekombination)
                        \item Für KW wichtig, da beständige Schichten
                        \item mit sehr starken Signalen zwischen 750km und 2200 km (Short-Skip)
    \end{itemize}
	\begin{center}
        \includegraphics[width=.8\textwidth,height=.4\textheight,keepaspectratio]{e09/schichten_behelf_43.png}
        \footnote{\tiny Amateurfunkbehelf s.43 \url{http://ham.granjow.net/builds/Amateurfunkbehelf.pdf}}
    \end{center}
\end{frame}

\section*{DX-Propergation}

\begin{frame}
  \frametitle{Einige Einflüsse auf die Ionosphere}
  \begin{center}
    \includegraphics[width=.8\textwidth,height=.8\textheight,keepaspectratio]{e09/Ionosphere-Thermosphere_Processes.jpg}
    \footnote{\tiny \url{https://commons.wikimedia.org/wiki/File:Ionosphere-Thermosphere_Processes.jpg}}\\
  \end{center}
\end{frame}

\begin{frame}
    \frametitle{Aurora}
	\begin{center}
        \begin{block}{DX-Propergation}
        \item Einige Seiten bieten eine Vorschau oder aktuelle Infos über die Ausbreitungsbedingungen.
        \item \url{http://www.dr1a.com/pages/en/dx-propagation.php}
        \item \url{https://dxheat.com/dxc/}
        \item \url{https://www.youtube.com/watch?v=TQkzzJqMA3g}
        \end{block}
    \end{center}
\end{frame}

\section*{Besonderes}

\begin{frame}
    \frametitle{Sonstiges}
    \begin{description}
      \item[MUF] maximum usable frequency bei welcher die Wellen noch reflektiert werden
      \item[Tote Zone] Bereich, bei dem ein Signal nicht zu hören ist, da keine Welle dort niederschlägt
      \item[Fading] Schicht baut sich ab und das Signal schwindet oder destruktive Interferenz von Raum und Bodenwelle
      \item[Grey Line] Zone des Sonnenauf- und -untergangs mit besonderen Ausbreitungsbedingungen
    \end{description}
\end{frame}

\begin{frame}
  \frametitle{Fading bei Sonnensturm}
  \begin{center}
    \includegraphics[width=.8\textwidth,height=.6\textheight,keepaspectratio]{e09/planetary-k-index.png}
    \footnote{\tiny Space Weather Prediction Center \url{http://www.swpc.noaa.gov/impacts/hf-radio-communications}}\\
    Planetary K-Index gibt Stärke des Sonnensturms an.\\
    Beim Funken in JT65/JT9 auf 20m verschwand das Signal binnen einer Minute im Wasserfall und war danach komplett tot.
  \end{center}
\end{frame}

\begin{frame}
    \frametitle{Aurora}
	\begin{center}
        \includegraphics[width=.9\textwidth,height=.8\textheight,keepaspectratio]{e09/Aurora_Seen_From_Space_by_NASA.jpg}
        \footnote{\tiny \url{https://commons.wikimedia.org/wiki/File:Aurora_Seen_From_Space_by_NASA.jpg}}
    \end{center}
\end{frame}

\begin{frame}
    \frametitle{Troposphärische Überreichweiten}
    \begin{itemize}
      \item Beugt UKW (dadurch Überreichweite)
      \item bekannt als Tropo da nicht Ionosphäre sondern Troposphäre (15km Höhe)
      \item Kalte Luft unten, warme Luft oben sorgt für Brechung der UKW Wellen
      \item Entfernungen bis ca 700km
    \end{itemize}
    	\begin{center}
        \includegraphics[width=.65\textwidth,height=.55\textheight,keepaspectratio]{e09/tropo.jpg}
        \footnote{\tiny by Adam Hauner via Wikimedia}
    \end{center}
\end{frame}

%\begin{frame}
%    \frametitle{Prüfungsfrage}

%    \begin{center}
%    \begin{tabular}{l||p{.8\textwidth}}\hline
%      \textbf{TI101} & \textbf{Welche ionosphärischen Schichten bestimmen die Wellenausbreitung am Tage?} \\\hline\hline
%         A 	  & Die E- und D-Schicht \\\hline
%         B 	  & Die F\textsubscript{1}- und F\textsubscript{2}-Schicht \\\hline
%         C	  & Die E- und F-Schicht \\\hline
%         D \only<2>\checkmark 	  & Die D-, E-, F\textsubscript{1}- und F\textsubscript{2}-Schicht\\\hline
%    \end{tabular}
% 	\end{center}
%\end{frame}
%
%\begin{frame}
%    \frametitle{Prüfungsfrage}

%   \begin{center}
%   \begin{tabular}{l||p{.8\textwidth}}\hline
%      \textbf{TI102} & \textbf{Welche ionosphärischen Schichten bestimmen die Wellenausbreitung in der Nacht?} \\\hline\hline
%         A 	  & Die D-, E- und F\textsubscript{2}-Schicht \\\hline
%         B \only<2>\checkmark 	  & Die F\textsubscript{2}-Schicht \\\hline
%         C	  & Die F\textsubscript{1}- und F\textsubscript{2}-Schicht \\\hline
%         D 	  & Die D- und E-Schicht\\\hline
%    \end{tabular}
% 	\end{center}
%\end{frame}



\section*{Referenzen}

\begin{frame}
    \frametitle{Referenzen/Links}
    
    \footnotesize
    \begin{itemize}
        \item Moltrecht E 09: \\
              \url{http://www.dj4uf.de/lehrg/e09/e09.html}
        \item Aurora (Youtube): \\
              \url{https://www.youtube.com/watch?v=izYiDDt6d8s}
        \item Spradic E QSO \\
              \url{http://www.dk0tu.de/blog/2012/11/27_Sporadic-E_QSO_mit_Spanien/}
	\item Space Weather Prediction Center -- HF Radio Communications \\
	      \url{http://www.swpc.noaa.gov/impacts/hf-radio-communications}
    \end{itemize}

\end{frame}

% Hier könnte noch eine Kontaktfolie stehen

\end{document}

