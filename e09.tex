% Foliensatz: "AFu-Kurs nach DJ4UF" von DK0TU, Amateurfunkgruppe der TU Berlin
% Lizenz: CC BY-NC-SA 3.0 de (http://creativecommons.org/licenses/by-nc-sa/3.0/de/)
% Autoren: Felix Baum <baum@campus.tu-berlin.de>

preamble.dk0tu.tex
\subtitle{Technik 09: \\
           Die Wellenausbreitung \\[2em]}
\date{Stand 22.11.2014}
 \begin{document}

\begin{frame}
    \titlepage
    \vfill
    \begin{center}
        \ccbyncsaeu\\
        {\tiny This work is licensed under the \em{Creative Commons Attribution-NonCommercial-ShareAlike 3.0 License}.}\\[0.5ex]
         \tiny Amateurfunkgruppe der Technische Universität Berlin (AfuTUB), DKØTU
         %\includegraphics[scale=0.5]{img/DK0TU_Logo.pdf}
    \end{center}
\end{frame}


%fixme Referenzen/Fußnoten-Systematik vereinheitlichen

\section*{Einleitung}

\begin{frame}
    \frametitle{Wie kommen die Wellen um die Welt?}
    \begin{center}
        \includegraphics[width=1\textwidth]{e09/cqww-kontakte.png}
        \footnote{\tiny Funkkontakte beim CQWW-SSB 2014 von DK0TU}
    \end{center}
\end{frame}

\begin{frame}
    \frametitle{Raum und Bodenwelle}
	\begin{center}
        \includegraphics[width=.8\textwidth]{e09/Ionospheric_reflectionpng.png}
         \footnote{\tiny \url{https://commons.wikimedia.org/wiki/File:Ionospheric_reflection-de.svg}}
    \end{center}
\end{frame}

\begin{frame}
    \frametitle{Raum und Bodenwelle}
	\begin{itemize}
        \item Bodenwelle
			\begin{itemize}
				\item Folgt der Erdkrümmung
       		 	\item Stark vorhanden bei Langwelle und Mittelwelle
        		\item Viele Verluste durch Berge, Wälder und Städte
    		\end{itemize}
        \item Raumwelle
			\begin{itemize}
       		 	\item Meist genutzt bei Kurzwelle (Manchmal auch UHF, VHF)
        		\item Reflektion an der Ionosphäre 
        		\item Nicht Zuverlässig für Überreichweiten
    		\end{itemize}
    \end{itemize}
\end{frame}

\section*{Bodenwelle}
\begin{frame}
    \frametitle{Bodenwelle}
	\begin{itemize}
				\item Bei LF über $400km$
       		 	\item $80m$ Wellemlänge ($3.5MHz$)  - $150km$ Reichweite
        		\item $10m$ Wellenlänge ($28MHz$) - $30km$ Reichweite
        		\item Reichweite hängt auch stark vom Untergrund ab.
    \end{itemize}
\end{frame}

\begin{frame}
    \frametitle{Prüfungsfrage}

    \begin{center}
    \begin{tabular}{l||l}\hline
        TI203 & Welche der folgenden Aussagen trifft für KW-Funk- \\
         " "  & verbindungen zu, die über Bodenwellen erfolgen? \\ 
         " "  & Die Bodenwelle folgt der Erdkrümmung und ...\\\hline\hline
         A 	  & geht nicht über den geografischen Horizont hinaus. \\
         " "  & Sie wird in niedrigeren Frequenzbereichen stärker \\ 
         " "  & gedämpft als in höheren Frequenzbereichen.\\\hline
         B 	  & geht über den geografischen Horizont hinaus. Sie \\
         " "  & wird in höheren Frequenzbereichen stärker gedämpft \\ 
         " "  & als in niedrigeren Frequenzbereichen.\\\hline
         C	  & geht über den geografischen Horizont hinaus. Sie \\
         " "  & wird in niedrigeren Frequenzbereichen stärker \\ 
         " "  & gedämpft als in höheren Frequenzbereichen.\\\hline
         D 	  & geht nicht über den geografischen Horizont hinaus. \\
         " "  & Sie wird in höheren Frequenzbereichen stärker \\ 
         " "  & gedämpft als in niedrigeren Frequenzbereichen.\\\hline
    \end{tabular}
 	\end{center}
\end{frame}

\begin{frame}
    \frametitle{Prüfungsfrage}

    \begin{center}
    \begin{tabular}{l||l}\hline
        TI203 & Welche der folgenden Aussagen trifft für KW-Funk- \\
         " "  & verbindungen zu, die über Bodenwellen erfolgen? \\ 
         " "  & Die Bodenwelle folgt der Erdkrümmung und ...\\\hline\hline
         " " 	  & geht nicht über den geografischen Horizont hinaus. \\
         " "  & Sie wird in niedrigeren Frequenzbereichen stärker \\ 
         " "  & gedämpft als in höheren Frequenzbereichen.\\\hline
         " " 	  & geht über den geografischen Horizont hinaus. Sie \\
         x  & wird in höheren Frequenzbereichen stärker gedämpft \\ 
         " "  & als in niedrigeren Frequenzbereichen.\\\hline
         " "	  & geht über den geografischen Horizont hinaus. Sie \\
         " "  & wird in niedrigeren Frequenzbereichen stärker \\ 
         " "  & gedämpft als in höheren Frequenzbereichen.\\\hline
         " " 	  & geht nicht über den geografischen Horizont hinaus. \\
         " "  & Sie wird in höheren Frequenzbereichen stärker \\ 
         " "  & gedämpft als in niedrigeren Frequenzbereichen.\\\hline
    \end{tabular}
 	\end{center}
\end{frame}

\section*{Raumwelle}
    
\begin{frame}
    \frametitle{Raumwelle}
	\begin{center}
        \includegraphics[width=.9\textwidth]{e09/schichten_behelf_43.png}
        \footnote{\tiny Amateurfunkbehelf s.43 \url{http://ham.granjow.net/builds/Amateurfunkbehelf.pdf}}
    \end{center}
\end{frame}

\begin{frame}
    \frametitle{Sonnenflecken Zyklus}
    \begin{itemize}
    			\item Bei Sonneneinstrahlung werden Moleküle in der Ionosphäre durch EUV-Strahlung ionisiert
				\item Maximum ca alle 11 Jahre
       		 	\item HF Frequenzen ab ca $20MHz$ sind bei einem Minimum nicht verwendbar
       		 	\item Bei einem Maximum quasi Täglich DX-Verbindungen mit weniger als $10W$ machbar 
        		\item Nächstes Maximum vermutlich 2022
    \end{itemize}
	\begin{center}
        \includegraphics[width=0.8\textwidth]{e09/Predictions_sunspot.png}
        \footnote{\tiny By Scientific data, based on prediction by David Hathaway, Public domain, via Wikimedia Commons \url{https://commons.wikimedia.org/wiki/File_3APredictions3_strip.jpg}} %\url{https://commons.wikimedia.org/wiki/File%3APredictions3_strip.jpg}
    \end{center}
\end{frame}

\begin{frame}
    \frametitle{D-Schicht}
    \begin{itemize}
    			\item Tagsüber und verschwindet nach Sonnenuntergang sehr schnell
				\item Dämpft Frequenzen unter $5MHz$ (160m und 80m unbenutzbar)
       		 	\item HF Frequenzen ab ca $20MHz$ sind bei einem Minimum nicht verwendbar
       		 	\item Bei hoher Sonnenaktivität Möbel-Dellinger-Effekt (Kurzzeitig ganzes KW-Band unbenutzbar)
    \end{itemize}
    \begin{center}
        \includegraphics[width=.6\textwidth]{e09/schichten_behelf_43.png}
        \footnote{\tiny Amateurfunkbehelf s.43 \url{http://ham.granjow.net/builds/Amateurfunkbehelf.pdf}}
    \end{center}
\end{frame}

\begin{frame}
    \frametitle{E-Schicht}
    \begin{itemize}
    			\item Tagsüber
				\item Reflektiert HF-Bänder 10m, 6m
       		 	\item Refelktiert gelegendlich 2m (Sporedic-E) \footnote{\tiny \url{https://www.youtube.com/watch?v=xSWTkuSekhE}}
       		 	\item (Short Skip)mit sehr starken Signalen zwischen 750 und 2200 km (Short-Skip)
    \end{itemize}
    \begin{center}
        \includegraphics[width=.75\textwidth]{e09/schichten_behelf_43.png}
        \footnote{\tiny Amateurfunkbehelf s.43 \url{http://ham.granjow.net/builds/Amateurfunkbehelf.pdf}}
    \end{center}
\end{frame}

\begin{frame}
    \frametitle{F-Schichten}
    \begin{itemize}
    			\item F1 und F2 Schicht
				\item F2 Schicht besteht auch Nachts (langsame Rekombination)
       		 	\item Wichtigstens da beständigste Schichten für KW
       		 	\item (Short Skip)mit sehr starken Signalen zwischen 750 und 2200 km (Short-Skip)
    \end{itemize}
	\begin{center}
        \includegraphics[width=.75\textwidth]{e09/schichten_behelf_43.png}
        \footnote{\tiny Amateurfunkbehelf s.43 \url{http://ham.granjow.net/builds/Amateurfunkbehelf.pdf}}
    \end{center}
\end{frame}

\section*{Besonderes}

\begin{frame}
    \frametitle{Sonstiges}
    \begin{itemize}
    			\item MUF: maximum usable frequency
				\item Tote Zone
       		 	\item Fading
       		 	\item Grey Line
    \end{itemize}
\end{frame}

\begin{frame}
    \frametitle{Aurora}
	\begin{center}
        \includegraphics[width=1\textwidth]{e09/Aurora_Seen_From_Space_by_NASA.jpg}
        \footnote{\tiny \url{https://commons.wikimedia.org/wiki/File:Aurora_Seen_From_Space_by_NASA.jpg}}
    \end{center}
\end{frame}

\begin{frame}
    \frametitle{Troposphärische Überreichweiten}
    \begin{itemize}
    			\item Beugt UKW (dadurch Überreichweite)
				\item bekannt als Tropo da nicht Ionosphäre sondern Troposphäre (15km höhe)
       		 	\item Kalte Luft unten, Warme Luft oben sorgt für Brechnung der UKW Wellen
       		 	\item Entfernungen bis ca 700km
    \end{itemize}
    	\begin{center}
        \includegraphics[width=.65\textwidth]{e09/tropo.jpg}
        \footnote{\tiny by Adam Hauner via Wikimedia}
    \end{center}
\end{frame}

\begin{frame}
    \frametitle{Prüfungsfrage}

    \begin{center}
    \begin{tabular}{l||l}\hline
        TI101 & Welche ionosphärischen Schichten bestimmen \\
         " "  & die Wellenausbreitung am Tage? \\\hline\hline
         A 	  & Die E- und D-Schicht \\\hline
         B 	  & Die F1- und F2-Schicht \\\hline
         C	  & Die E- und F-Schicht \\\hline
         D 	  & Die D-, E-, F1- und F2-Schicht\\\hline
    \end{tabular}
 	\end{center}
\end{frame}

\begin{frame}
    \frametitle{Prüfungsfrage}

    \begin{center}
    \begin{tabular}{l||l}\hline
        TI101 & Welche ionosphärischen Schichten bestimmen \\
         " "  & die Wellenausbreitung am Tage? \\\hline\hline
         " " 	  & Die E- und D-Schicht \\\hline
         " " 	  & Die F1- und F2-Schicht \\\hline
         " "	  & Die E- und F-Schicht \\\hline
         X 	  & Die D-, E-, F1- und F2-Schicht\\\hline
    \end{tabular}
 	\end{center}
\end{frame}

\begin{frame}
    \frametitle{Prüfungsfrage}

    \begin{center}
    \begin{tabular}{l||l}\hline
        TI102 & Welche ionosphärischen Schichten bestimmen \\
         " "  & die Wellenausbreitung in der Nacht? \\\hline\hline
         A 	  & Die D-, E- und F2-Schicht \\\hline
         B 	  & Die F2-Schichtt \\\hline
         C	  & Die F1- und F2-Schicht \\\hline
         D 	  & Die D- und E-Schicht\\\hline
    \end{tabular}
 	\end{center}
\end{frame}

\begin{frame}
    \frametitle{Prüfungsfrage}

    \begin{center}
    \begin{tabular}{l||l}\hline
        TI102 & Welche ionosphärischen Schichten bestimmen \\
         " "  & die Wellenausbreitung in der Nacht? \\\hline\hline
         " " 	  & Die D-, E- und F2-Schicht \\\hline
         X 	  & Die F2-Schichtt \\\hline
         " "	  & Die F1- und F2-Schicht \\\hline
         " " 	  & Die D- und E-Schicht\\\hline
    \end{tabular}
 	\end{center}
\end{frame}


\section*{Referenzen}

\begin{frame}
    \frametitle{Referenzen/Links}
    
    \footnotesize
    \begin{itemize}
        \item Moltrecht E 09: \\
              \url{http://www.dj4uf.de/lehrg/e09/e09.html}
        \item Aurora (Youtube): \\
              \url{https://www.youtube.com/watch?v=izYiDDt6d8s}
        \item Spradic E QSO \\
              \url{http://www.dk0tu.de/blog/2012/11/27_Sporadic-E_QSO_mit_Spanien/}
    \end{itemize}

\end{frame}

% Hier könnte noch eine Kontaktfolie stehen

\end{document}

