% Foliensatz: "AFu-Kurs nach DJ4UF" von DK0TU, Amateurfunkgruppe der TU Berlin
% Lizenz: CC BY-NC-SA 3.0 de (http://creativecommons.org/licenses/by-nc-sa/3.0/de/)
% Autoren: Sebastian Lange <dl7bst@dk0tu.de>

preamble.dk0tu.tex
\subtitle{Betriebstechnik/Vorschriften 03: \\
          Der "Q-Schlüssel"                \\[2em]}
\date{Stand 30.10.2014}
 \begin{document}

\begin{frame}
    \titlepage
    \vfill
    \begin{center}
        \ccbyncsaeu\\
        {\tiny This work is licensed under the \em{Creative Commons Attribution-NonCommercial-ShareAlike 3.0 License}.}\\[0.5ex]
         \tiny Amateurfunkgruppe der Technische Universität Berlin (AfuTUB), DKØTU
         %\includegraphics[scale=0.5]{img/DK0TU_Logo.pdf}
    \end{center}
\end{frame}


%fixme Referenzen/Fußnoten-Systematik vereinheitlichen

\section*{Einleitung}

\begin{frame}
    \frametitle{Einleitung / Q Code}
    \begin{center}
        \Large{Was ist das?} \\
        \Large{Kennt ihr ggf. welche?}
    \end{center}
\end{frame}

\begin{frame}
    \frametitle{Einleitung / Q Code}

    Bekannt aus Film und Fernsehen: \emph{QAM} \\[2em]

    Ursprung: Seit 1912 organisationsübergreifende Vereinfachung von Nachrichten
    in Morsetelegrafie, eingeführt durch \emph{International Radiotelegraph
    Convention}.\footnote{\tiny Heute geregelt durch die Radio Regulations (RR)}
    \\[2em]
    
    Heute $>$250 Schlüssel:

    \begin{itemize}
        \item QAA bis QNZ: Flugfunkdienst
        \item QOA bis QQZ: Seefunkdienst
        \item QRA bis QUZ: allen Funkdienste
        \item QVA bis QZZ: andere, teilw. militärisch
    \end{itemize}

\end{frame}

\section*{Anwendung}

\begin{frame}
    \frametitle{Q Code / Verwendung}

    \begin{itemize}
        \item außer im AFu heute kaum noch genutzt
        \item Verwendung in der Morsetelegrafie \\
              (umgangssprachlich teilw. im Sprechfunk, sollte vermieden werden)
        \item Fragen einfach durch angehängtes Fragezeichen
        \item beliebige Ergänzungen durch Ziffern (1..5), Zahlen, Orte, Rufzeichen, ...
        \item Uhrzeiten grundsätzlich in UTC
              \footnote{\tiny Universal Time, Coordinated}
    \end{itemize}
    
\end{frame}

\begin{frame}
    \frametitle{Q Code / Beispiele}

    \only<1>{Statt \emph{Wie lautet das Rufzeichen?} \\[2em]
             \emph{QRZ?}}

    \only<2>{Statt \emph{Soll ich die Sendeleistung verringern?} \\[2em]
             \emph{QRP?}}

    \only<3>{Statt \emph{Mein Standort ist Berlin.} \\[2em]
             \emph{QTH Berlin}}

    \only<4>{Statt \emph{Ich werde stark durch eine andere Station gestört.} \\[2em]
             \emph{QRM 4}}
    
    \only<5>{Q Codes teils beliebige Buchstabengruppen - \\
             es gibt aber \emph{Eselsbrücken}}

\end{frame}

\begin{frame}
    \frametitle{Q Code / Amateurfunk {\small (grau: prüfungsrelevant)}}
  
    %todo schlecht lesbar :-(

    \only<1>{
    \begin{center}
        \includegraphics[height=0.85\textheight]{bv03/Q-1.png}
        \footnote{\tiny \url{http://www.amateurfunkpruefung.de/lehrg/bv03/bv03.html}}
    \end{center}}

    \only<2>{
    \begin{center}
        \includegraphics[height=0.85\textheight]{bv03/Q-2.png}
        \footnote{\tiny \url{http://www.amateurfunkpruefung.de/lehrg/bv03/bv03.html}}
    \end{center}}

\end{frame}

\begin{frame}
    \frametitle{Noch mehr Beispiele}

    \only<1>{\emph{QRX 1500 UTC} - bis 15 Uhr Pause}

    \only<2>{\emph{QRX 5 min} - ?}
    
    \only<3>{\emph{QTF 240} - Richtantenne: 240 Grad}
    
    \only<4>{\emph{QTR 1225 UTC} - Es ist ...}

    \only<5>{\emph{QSP DJ4UF} - Weiterleitung an ...}

\end{frame}

\section*{Lernhilfen}

\begin{frame}
    \frametitle{Lernhilfen}

    Baut euch am Anfang möglichst \emph{Eselsbrücken}. Am Besten selbst
    ausdenken, aber Anregungen findet man auch bei anderen HAMs, z.B. hier: \\[2em]

    \url{http://www.funkwelle.com/amateurfunk/q-schluessel-fuer-amateurfunkpruefung.html}

\end{frame}

\section*{Referenzen}

\begin{frame}
    \frametitle{Referenzen/Links}
    
    \footnotesize
    \begin{itemize}
        \item Moltrecht B/V 03: \\
              \url{http://www.amateurfunkpruefung.de/lehrg/bv03/bv03.html}
        \item Wikipedia DE: \\
              \url{http://de.wikipedia.org/wiki/Q-Schl\%C3\%BCssel}
        \item Wikipedia EN: \\
              \url{http://en.wikipedia.org/wiki/Q_code}
        \item List of Q-codes: \\
              \url{http://www.kloth.net/radio/qcodes.php}
    \end{itemize}

\end{frame}

% Hier könnte noch eine Kontaktfolie stehen

\end{document}

