% Foliensatz: "AFu-Kurs nach DJ4UF" von DK0TU, Amateurfunkgruppe der TU Berlin
% Lizenz: CC BY-NC-SA 3.0 de (http://creativecommons.org/licenses/by-nc-sa/3.0/de/)
% Autoren: Sebastian Lange <dl7bst@dk0tu.de>, Lars Weiler <dc4lw@darc.de>

preamble.dk0tu.tex
\subtitle{Betriebstechnik/Vorschriften 03: \\
          Der "Q-Schlüssel"                \\[2em]}
\date{Stand 28.10.2015}
 \begin{document}

\begin{frame}
    \titlepage
    \vfill
    \begin{center}
        \ccbyncsaeu\\
        {\tiny This work is licensed under the \em{Creative Commons Attribution-NonCommercial-ShareAlike 3.0 License}.}\\[0.5ex]
         \tiny Amateurfunkgruppe der Technische Universität Berlin (AfuTUB), DKØTU
         %\includegraphics[scale=0.5]{img/DK0TU_Logo.pdf}
    \end{center}
\end{frame}


%fixme Referenzen/Fußnoten-Systematik vereinheitlichen

\section*{Einleitung}

\begin{frame}
    \frametitle{Einleitung / Q Code}
    \begin{center}
        \Large{Was ist das?} \\
        \Large{Kennt ihr ggf. welche?}
    \end{center}
\end{frame}

\begin{frame}
    \frametitle{Einleitung / Q Code}

    Bekannt aus Film und Fernsehen: \emph{QAM} \\[2em]

    Ursprung: Seit 1912 organisationsübergreifende Vereinfachung von Nachrichten
    in Morsetelegrafie, eingeführt durch \emph{International Radiotelegraph
    Convention}.\footnote{\tiny Heute geregelt durch die Radio Regulations (RR)}
    \\[2em]
    
    Heute $>$250 Schlüssel:

    \begin{itemize}
        \item QAA bis QNZ: Flugfunkdienst
        \item QOA bis QQZ: Seefunkdienst
        \item QRA bis QUZ: allen Funkdienste
        \item QVA bis QZZ: andere, teilw. militärisch
    \end{itemize}

\end{frame}

\section*{Anwendung}

\begin{frame}
    \frametitle{Q Code / Verwendung}

    \begin{itemize}
        \item außer im AFu heute kaum noch genutzt
        \item Verwendung in der Morsetelegrafie \\
              (umgangssprachlich teilw. im Sprechfunk)
        \item Fragen einfach durch angehängtes Fragezeichen
        \item beliebige Ergänzungen durch Ziffern (1..5), Zahlen, Orte, Rufzeichen, \ldots
        \item Uhrzeiten grundsätzlich in UTC
              \footnote{\tiny Universal Time, Coordinated}
    \end{itemize}
    
\end{frame}

\begin{frame}
       \frametitle{Q Code / Beispiele}

    \only<1>{
    \begin{exampleblock}{Wie lautet das Rufzeichen?}
      \emph{QRZ?}
    \end{exampleblock}
    }

    \only<2>{
    \begin{exampleblock}{Soll ich die Sendeleistung verringern?}
      \emph{QRP?}
    \end{exampleblock}
    }

    \only<3>{
    \begin{exampleblock}{Mein Standort ist Berlin.}
      \emph{QTH Berlin}
    \end{exampleblock}
    }

    \only<4>{
    \begin{exampleblock}{Ich werde stark durch eine andere Station gestört.}
      \emph{QRM 4}
    \end{exampleblock}
    }

    \only<5>{
    \begin{block}{Q-Codes teils beliebige Buchstabengruppen}
      Es gibt aber \emph{Eselsbrücken}
    \end{block}
    }
\end{frame}

\begin{frame}
  \frametitle{(prüfungsrelevante) Q Codes / Amateurfunk}
  
  \begin{center}
  \tiny
  \only<1>{
  \begin{tabular}{lp{3cm}lp{2.5cm}p{1.5cm}}\hline
    Q   & Bedeutung & Q?   & Bedeutung & Eselsbrücke \\ \hline \hline
    QRA & Name der Station & QRA? & Wie ist der Name der Station? & \textbf{A}nrede \\
    QRG & Frequenz & QRG? & Wie ist die Frequenz? & \textbf{G}enaue Frequenz \\
    QRK & Lesbarkeit der Zeichen in RST-Stufen 1\ldots5 & QRK? & Wie hören Sie mich? & \\
    QRL & Operator beschäftigt & QRL? & Sind Sie beschäftigt & \textbf{R}eal-\textbf{L}ife \\
    QRM & Gestört in Stufen 1\ldots5 & QRM? & Werden Sie gestört? & \textbf{R}eceive \textbf{M}ist \\
    QRN & Atmosphärische Störung in Stufen 1\ldots5 & QRN? & Haben Sie atmosphärische Störungen? & \textbf{N}oise / \textbf{N}atural \\
    QRP & Sendeleistung verringern & QRP? & Soll ich die Sendeleistung verringern? & \textbf{R}educe \textbf{P}ower \\
    QRO & Sendeleistung erhöhen & QRO? & Soll ich die Sendeleistung erhöhen? & Gegenteil von QRP \\
    QRX & Ich rufe Sie wieder um \ldots Uhr & QRX? & Wann rufen Sie mich wieder? & \\
    QRT & Übermittlung einstellen & QRT? & Soll ich die Übermittlung einstellen? & \textbf{T}erminate \\
  \end{tabular}
  }
  
  \only<2>{
  \begin{tabular}{lp{3cm}lp{2.5cm}p{1.5cm}}\hline
    Q   & Bedeutung & Q?   & Bedeutung & Eselsbrücke \\ \hline \hline
    QRV & Ich bin bereit & QRV? & Sind Sie bereit? & \textbf{V}erkehrsbereit \\
    QRZ & Sie werden von \ldots / auf \ldots gerufen & QRZ? & Von wem werde ich gerufen? & \\
    QSB & Signalstärke schwankt (Fading) & QSB? & Schwankt die Signalstärke? & \\
    QSL & Empfangsbestätigung & QSL? & Können Sie mir eine Empfangsbestätigung geben? & \\
    QTH & Standort & QTH? & Wie ist Ihr Standort? & \textbf{H}ome \\
    QSY & Frequenzwechsel auf \ldots & QSY? & Soll ich zum Senden auf eine andere Frequenz gehen? & \textbf{S}hift Frequenc\textbf{y}\\
    QSO & Funk-Verbindung / Gespräch & & & \\
  \end{tabular}
  }
  \end{center}
\end{frame}


\begin{frame}
    \frametitle{Noch mehr Beispiele}
    
    \only<1-2>{
    \begin{exampleblock}{QRX 1500 UTC}
      \only<1>{\vspace{1em}}
      \only<2>{Pause bis 15 Uhr UTC.}
    \end{exampleblock}
    }

    \only<3-4>{
    \begin{exampleblock}{QRX 5 min}
      \only<3>{\vspace{1em}}
      \only<4>{Pause für 5 Minuten.}
    \end{exampleblock}
    }

    \only<5-6>{
    \begin{exampleblock}{QRP?}
      \only<5>{\vspace{1em}}
      \only<6>{Können Sie die Sendeleistung verringern?}
    \end{exampleblock}
    }

    \only<7-8>{
    \begin{exampleblock}{QTH Essen}
      \only<7>{\vspace{1em}}
      \only<8>{Mein Standort ist Essen.}
    \end{exampleblock}
    }

    \only<9-10>{
    \begin{exampleblock}{tnx fer qso}
      \only<9>{\vspace{1em}}
      \only<10>{Danke für das Funkgespräch}
    \end{exampleblock}
    }
\end{frame}

\section*{Lernhilfen}

\begin{frame}
    \frametitle{Lernhilfen}

    Baut euch am Anfang möglichst \emph{Eselsbrücken}. Am Besten selbst
    ausdenken, aber Anregungen findet man auch bei anderen HAMs, z.B. hier: \\[2em]

    \url{http://www.funkwelle.com/amateurfunk/q-schluessel-fuer-amateurfunkpruefung.html}
\end{frame}

\begin{frame}
  \begin{alertblock}{Hausaufgabe}
    Relevante Q-Gruppen für die Prüfung anschauen: Prüfungsfragen ``Betriebliche Kenntnisse'' Kapitel 2.2.2 ``Q-Schlüssel'' (BB201--BB209).
  \end{alertblock}
\end{frame}

\section*{Referenzen}

\begin{frame}
    \frametitle{Referenzen/Links}
    
    \footnotesize
    \begin{itemize}
        \item Moltrecht B/V 03: \\
              \url{http://www.amateurfunkpruefung.de/lehrg/bv03/bv03.html}
        \item Wikipedia DE: \\
              \url{http://de.wikipedia.org/wiki/Q-Schl\%C3\%BCssel}
        \item Wikipedia EN: \\
              \url{http://en.wikipedia.org/wiki/Q_code}
        \item List of Q-codes: \\
              \url{http://www.kloth.net/radio/qcodes.php}
    \end{itemize}

\end{frame}

% Hier könnte noch eine Kontaktfolie stehen

\end{document}

