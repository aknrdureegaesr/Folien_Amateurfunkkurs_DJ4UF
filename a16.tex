% Foliensatz: "AFu-Kurs nach DJ4UF" von DK0TU, Amateurfunkgruppe der TU Berlin
% Lizenz: CC BY-NC-SA 3.0 de (http://creativecommons.org/licenses/by-nc-sa/3.0/de/)
% Autoren: Martin Deutschmann <martin.deutschmann@campus.tu-berlin.de>
% Korrekturen: Lars Weiler <dc4lw@darc.de>, Sebastian Lange <dl7bst@dk0tu.de>

preamble.dk0tu.tex
\subtitle{Technik A 16: \\
  Messtechnik \\[2em]}
\date{Stand 01.06.2017}
 \begin{document}

\begin{frame}
    \titlepage
    \vfill
    \begin{center}
        \ccbyncsaeu\\
        {\tiny This work is licensed under the \em{Creative Commons Attribution-NonCommercial-ShareAlike 3.0 License}.}\\[0.5ex]
         \tiny Amateurfunkgruppe der Technische Universität Berlin (AfuTUB), DKØTU
         %\includegraphics[scale=0.5]{img/DK0TU_Logo.pdf}
    \end{center}
\end{frame}


\section*{Einleitung}

\begin{frame}
  \frametitle{Messgeräte}
  {\Large Was wisst ihr noch aus Kapitel \emph{E17}?}
\end{frame}

\begin{frame}
  \frametitle{Stehwellenmessgerät}
  \begin{center}
    \begin{figure}
      \includegraphics[width=1\textwidth,height=.7\textheight,keepaspectratio]{a16/RS_SWR.jpg}
      \attribcaption{SWR-Meter zum messen des Stehwellenverhältnisses}{HBD20}{https://commons.wikimedia.org/wiki/File:RS_SWR.jpg}{\ccbysa}
    \end{figure}
  \end{center}
\end{frame}

\section*{Analog}

\begin{frame}
  \frametitle{Drehspulenmessgerät (Antik)}
  \begin{columns}
    \column{.3\textwidth}
    \begin{figure}
      \includegraphics[width=.95\textwidth,height=.65\textheight,keepaspectratio]{a16/drehspulenMess.png}
      \attribcaption{Technische Ausführung eines Drehspulinstruments}{Søren Peo Pedersen}{https://commons.wikimedia.org/wiki/File:Moving_coil_instrument_principle.png}{\ccbysa}
    \end{figure}
    \column{.65\textwidth}
    \begin{footnotesize}
      \begin{enumerate}
        \item Weicheisenkern der Drehspule
        \item Permanentmagnet
        \item Polschuh zur Bündelung des Magnetfeldes
        \item Skala
        \item Hilfsspiegel zur genauen Ablesung
        \item Rückstellfeder
        \item Drehspule
        \item Drehspule in Nulllage
        \item Drehspule bei Maximalausschlag
        \item Joch der Spule
        \item Stellschraube für Nullpunkteinstellung
        \item Zeiger
        \item Zeiger in Nulllage
        \item Zeiger bei Maximalausschlag
      \end{enumerate}
    \end{footnotesize}
  \end{columns}
\end{frame}

\begin{frame}
  \frametitle{Funktionsprinzip analoger Messgeräte}
  \begin{itemize}
    \item Analoge Messgeräte funktionieren nach dem elektrodynamischen, oder dem elektrostatischen Prinzip
    \item Dabei erzeugt die zu messende Größe ein mechanisches Drehmoment zwischen dem feststehendem Messwerk und dem beweglichen Organ
    \item Die Empfindlichkeit wird in $k \Omega /V$ angegeben
    \item Für Gleichspannung sollte die Empfindlichkeit bei mindestens $20 k \Omega / V$ liegen
  \end{itemize}
\end{frame}

\begin{frame}
  %\frametitle{Symbole auf analogen Messgeräten}
  \begin{center}
    \begin{figure}
      \includegraphics[width=\textwidth,height=.9\textheight,keepaspectratio]{a16/Symbole.png}
      \attribcaption{Symbole auf analogen Messgeräten}{(Quelle unbekannt)}{}{}
    \end{figure}
  \end{center}
\end{frame}

\begin{frame}
  \frametitle{Messgeräteklassen}
  Klasse gibt den prozentualen Fehler bezogen auf den Skalenendwert an
  \begin{center}
    \begin{tabular}{c|c}
      \textbf{Feinmessgeräte} & \textbf{Betriebsmessgeräte}\\ \hline
      Klasse 0,1 & Klasse 1,0 \\
      Klasse 0,2 & Klasse 1,5 \\
      Klasse 0,5 & Klasse 2,5 \\
      " " & Klasse 5,0 \\
    \end{tabular}
  \end{center}

  \pause

  \begin{exampleblock}{\textbf{TJ805} Mit einem Voltmeter der Klasse 1.5, das einen Skalenendwert von 300 Volt hat, messen Sie an einer Spannungsquelle 230 Volt. In welchem Bereich liegt der wahre Wert?}
    \only<2>{\vspace{1em}}
    \only<3>{Er liegt zwischen 225,5 und 234,5 Volt.}
  \end{exampleblock}
\end{frame}

\begin{frame}
  \frametitle{Messbereichserweiterung}
  \begin{columns}
    \column{.47\textwidth}
    \begin{center}
      \begin{figure}
        \includegraphics[width=\textwidth,height=.4\textheight,keepaspectratio]{a16/Messbereichserweiterung-Spannung.png}
        \caption{Spannungsmessgerät}
      \end{figure}
    \end{center}
    \only<1>{\vspace{4em}}
    \only<2->{Zu große Spannung fällt am Vorwiderstand ab $\rightarrow$ Spannungsteiler \\
    (Messung höherer Spannungen möglich -- abhängig von $\frac{R_M}{R_{gesamt}}$)}
    \column{.47\textwidth}
    \begin{center}
      \begin{figure}
        \includegraphics[width=\textwidth,height=.4\textheight,keepaspectratio]{a16/Messbereichserweiterung-Strom.png}
        \caption{Strommessgerät}
      \end{figure}
    \end{center}
    \only<1-2>{\vspace{4em}}
    \only<3>{Zu großer Strom wird parallel durch einen Widerstand geleitet\vspace{2em}}
  \end{columns}
\end{frame}

\begin{frame}
  \frametitle{Messgleichrichter}
  \begin{columns}
    \column{.35\textwidth}
    \begin{center}
      \begin{figure}
        \includegraphics[width=\textwidth,height=.4\textheight,keepaspectratio]{a16/Messgleichrichter1.png}
        \includegraphics[width=\textwidth,height=.4\textheight,keepaspectratio]{a16/Messgleichrichter2.png}
        \caption{Mögliche Schaltungen für Messgleichrichter}
      \end{figure}
    \end{center}
    \column{.65\textwidth}
    \begin{itemize}
      \item Drehspulmesswerke eignen sich nur zum messen von Gleichstrom
      \item Um auch Wechselspannungen messen zu können, nutzt man Messgleichrichter
      \item Diese funktionieren aber nur für sinusförmige Signale
      \item Für andere Signalformen gibt es das Dreheisenmesswerk
      \item Diese brauchen keinen Gleichrichter, benötigen aber recht viel Leistung und sind nicht als Feinmessgeräte tauglich
      \item Will man im GHz-Bereich messen, benötigt man einen speziellen Thermoumformer
    \end{itemize}
  \end{columns}
\end{frame}

\section*{Digital}

\begin{frame}
  \frametitle{Digitales Multimeter}
  \begin{columns}
    \column{.3\textwidth}
    \begin{center}
      \begin{figure}
        \includegraphics[width=\textwidth,height=.7\textheight,keepaspectratio]{a16/digitalmultimeter.jpg}
        \attribcaption{Digitales Multimeter}{MichaelHaeckel}{https://commons.wikimedia.org/wiki/File:Digitalmultimeter.jpg}{\ccpd}
      \end{figure}
    \end{center}
    \column{.65\textwidth}
    \begin{itemize}
      \item Verringern Gefahr von Ablesefehlern
      \item Brauchen Strom zum messen
      \item Auflösung ist die kleinste Einteilung der Anzeige
      \item Neben Auflösung auch Genauigkeit beachten
      \item Einfache DMM messen:\newline Spannung, Stromstärke, Widerstand
      \item Bessere:\newline Durchgangsprüfung (auf Geschwindigkeit achten!), Kapazität, (niedrige) Frequenzen, Dioden- und Transistortest
      \item Selten auch:\newline Induktivität, Temperatur, Temperatur, Verstärkung von Bipolartransistoren, Lichtstärke, Schalldruck
    \end{itemize}
  \end{columns}
\end{frame}

\begin{frame}
  \frametitle{Was wo anschließen?}
  \begin{columns}
    \column{.32\textwidth}
    \begin{center}
      \begin{figure}
        \includegraphics[width=\textwidth,height=.7\textheight,keepaspectratio]{a16/digitalmultimeterMess.jpg}
        \attribcaption{Digitales Multimeter}{MichaelHaeckel}{https://commons.wikimedia.org/wiki/File:Digitalmultimeter.jpg}{\ccpd}
      \end{figure}
    \end{center}
    \column{.65\textwidth}
    \begin{itemize}
      \item Was kann alles gemessen werden?
      \item Wo anschließen zum Strom messen?
      \item Wo anschließen zum Spannung messen?
      \item Welcher Messbereich?
    \end{itemize}
  \end{columns}
\end{frame}

\begin{frame}
  \frametitle{Messfehler}
  \begin{center}
    \begin{tabular}{ccc}
      Nullpunktabweichung & Empfindlichkeitsabweichung & Linearitätsabweichung
    \end{tabular}
    \begin{figure}
      \includegraphics[width=1\textwidth,height=.6\textheight]{a16/werMisstMisst.png}
      \attribcaption{Nullpunktsabweichung, Empfindlichkeitsabweichung, Linearitätsabweichung}{Saure}{https://commons.wikimedia.org/wiki/File:AMT_Fehler.svg}{\ccbysa}
    \end{figure}
  \end{center}
\end{frame}

\section*{Oszilloskop}


\begin{frame}
  \frametitle{Oszilloskop}
  \begin{center}
    \begin{figure}
      \includegraphics[width=1\textwidth,height=.7\textheight,keepaspectratio]{a16/osziModern.jpg}
      \attribcaption{Modernes Speicheroszilloskop}{Björn Heller}{https://commons.wikimedia.org/wiki/File:Modernes_Speicheroszilloskop.jpg}{\ccbysa}
    \end{figure}
  \end{center}
\end{frame}

\begin{frame}
  \frametitle{Oszilloskop}
  \begin{itemize}
    \item Können zeitliche Verläufe von Spannungen darstellen
    \item Anzeige mit Elektronenstrahlröhre (ein Strahl, sehr genau, direkte Spannungsumsetzung) oder LCD (moderner, Computer-Schaltung bereitet das Bild auf)
    \item Kann stehende Bilder von Wellen darstellen, indem die Welle immer an einem bestimmten Amplitudenwert getriggert (gestartet) wird
    \item Benötigt dafür eine Triggereinrichtung
  \end{itemize}
\end{frame}

\begin{frame}
  \frametitle{Oszilloskop}
  \begin{center}
    \begin{figure}
      \includegraphics[width=.75\textwidth,height=.75\textheight,keepaspectratio]{e17/WTPCOscilloscopeBeschreiben.jpg}
      \attribcaption{Bedienungselemente eines Oszilloskops}{Brian S. Elliott}{https://en.wikipedia.org/wiki/File:WTPC_Oscilloscope-1.jpg}{\ccbysa}
    \end{figure}
  \end{center}
\end{frame}

\begin{frame}
  \frametitle{Oszilloskop ablesen}
  \begin{center}
    $100mV / Div$ und $0,1ms / Div$\\
    \begin{figure}
      \includegraphics[width=.8\textwidth,height=.65\textheight,keepaspectratio]{a16/OsziTon.png}
      \attribcaption{Schematisches Oszilloskopbild eines Tons}{Klaus-Dieter Keller}{https://commons.wikimedia.org/wiki/File:Oszi_Ton.svg}{\ccbysa}
    \end{figure}
  \end{center}
\end{frame}

\begin{frame}
  \frametitle{Verschiedene Signalformen}
  \begin{center}
    \begin{figure}
      \includegraphics[width=0.95\textwidth,height=.75\textheight,keepaspectratio]{a16/Signalformen.png}
      \attribcaption{Verschiedene periodische Wellenformen}{Mik81}{https://commons.wikimedia.org/wiki/File:Waveforms_de.svg}{\ccbysa}
    \end{figure}
  \end{center}
\end{frame}

\begin{frame}
  \frametitle{Spannungen}
  \begin{center}
    \begin{itemize}
      \item Peak-to-Peak
      \item RMS (root mean square) -- Effektivwert
      \item Bei Sinus: $u_{Spitze} = \sqrt{2} \cdot U_{eff}$ \\
    \end{itemize}
  \end{center}
  \begin{exampleblock}{Aufgabe}
    Berechnet die Spitze-Spitze-Spannung der in Europa üblichen Netzspannung!
  \end{exampleblock}
\end{frame}

\section*{Absorptions"-frequenz"-messgerät}
\begin{frame}
  \frametitle{Absorptionsfrequenzmessgerät}
  \begin{columns}
    \column{.6\textwidth}
    \begin{center}
      \begin{figure}
        % FIXME Quelle der Grafik
        \includegraphics[width=\textwidth,height=.75\textheight,keepaspectratio]{a16/Absorptionsfrequenzmesser.png}
        \caption{Schaltung eines Absorptionsfrequenzmessgeräts}
      \end{figure}
    \end{center}
    \column{.4\textwidth}
    \begin{itemize}
      \item Besteht aus einer Antenne, einem passiven Schwingkreis hoher Güte und  einem AM-Demodulator
      \item Damit lassen sich passiv Senderfrequenzen und Oberwellen feststellen
      \item Anzeigegenauigkeit liegt bei etwa 5\%
    \end{itemize}
  \end{columns}
\end{frame}

\section*{Feld"-stärke"-anzeiger}
\begin{frame}
  \frametitle{Feldstärkeanzeiger}
  \begin{columns}
    \column{.6\textwidth}
    \begin{center}
      \begin{figure}
        % FIXME Quelle der Grafik
        \includegraphics[width=\textwidth,height=.7\textheight,keepaspectratio]{a16/Feldstaerkeanzeiger.png}
        \caption{Schaltung eines Feldstärkeanzeigers (nach Prüfungsfrage TJ706)}
      \end{figure}
    \end{center}
    \column{.4\textwidth}
    \begin{itemize}
      \item Wird zum Prüfen (\emph{nicht zum Messen!}) der Feldstärke genutzt
      \item Besteht aus einer HF-Diode, HF-Drosseln und einem Kondensator
    \end{itemize}
    \begin{flushright}
      \begin{figure}
        \includegraphics[width=.7\textwidth,height=.35\textheight,keepaspectratio]{a16/ghostbusters_egon.jpg}
        \attribcaption{Feldstärkeanzeiger}{unbekannt}{http://static.tvtropes.org/pmwiki/pub/images/real_ghostbusters_egon_3468.jpg}{http://tvtropes.org \ccbyncsaeu}
      \end{figure}
    \end{flushright}
  \end{columns}
\end{frame}

\section*{Dipmeter}

\begin{frame}
  \frametitle{Dipmeter}
  \begin{columns}
    \column{.35\textwidth}
    \begin{center}
      \begin{figure}
        \includegraphics[width=\textwidth,height=.7\textheight,keepaspectratio]{a16/Dipmeter.jpg}
        \attribcaption{Dipmeter}{(Japanische Zeichen nicht darstellbar)}{https://commons.wikimedia.org/wiki/File:Dipmeter_and_its_probe_coils.jpg}{\ccbysa}
      \end{figure}
    \end{center}
    \column{.65\textwidth}
    \begin{itemize}
      \item prinzipiell ein Oszillator mit nach außen geführter Schwingkreisspule
      \item Schwingkreis wird durch das Messobjekt beeinflusst
      \item Rückgang der Schwingungsamplitude wird mit einer Anzeige sichtbar gemacht
      \item Anzeigegenauigkeit liegt bei etwa 10\%
      \item kann indirekt eine Induktivität messen $\rightarrow$ Kondensator parallel zur Induktivität schalten, Resonanzfrequenz ermitteln und umrechnen
      \item zum Messen wird eine relativ lose Kopplung zwischen Dipmeter und Messobjekt benötigt, da sonst das Messobjekt verstimmt wird und das Messergebnis dadurch verfälscht
    \end{itemize}
  \end{columns}
\end{frame}

\section*{Frequenzzähler}

\begin{frame}
  \frametitle{Frequenzzähler}
  \begin{columns}
    \column{.4\textwidth}
    \begin{center}
      \begin{figure}
        \includegraphics[width=\textwidth,height=.8\textheight,keepaspectratio]{a16/Frequenzzaehler.jpg}
        \attribcaption{Frequenzzähler von ELV}{Coaster J}{https://commons.wikimedia.org/wiki/File:Frequency_counter.JPG}{\ccbysa}
      \end{figure}
    \end{center}
    \column{.6\textwidth}
    \begin{itemize}
      \item zählt während einer eingestellten Torzeit die ankommenden Impulse
      \item zur Erhöhung der Genauigkeit müssen mehr Impulse gezählt werden $\rightarrow$ längere Torzeit einstellen
      \item kann keine Oberwellen messen, wenn die Grundfrequenz noch vorhanden ist
      \item sehr genaue Messungen mit einer hohen Auflösung und einer temperaturstabilen Quarzzeitbasis möglich
      \item zum Messen von höheren Frequenzen einen Vorteiler verwenden $\rightarrow$ meistens Faktor 10
    \end{itemize}
  \end{columns}
\end{frame}

\section*{Spektrum"-analysator}
\begin{frame}
  \frametitle{Spektrumanalysator}
  \begin{columns}
    \column{.5\textwidth}
    \begin{center}
      \begin{figure}
        \includegraphics[width=\textwidth,height=.75\textheight,keepaspectratio]{a16/Spektrumanalysator.jpg}
        \attribcaption{Beispiel eines Spektrumanalysators}{Rohde \& Schwarz GmbH \& Co. KG}{http://commons.wikimedia.org/wiki/File:FSL.jpg}{\ccby}
      \end{figure}
    \end{center}
    \column{.5\textwidth}
    \begin{itemize}
      \item kann die Amplitude eines Signals in Abhängigkeit der Frequenz darstellen
      \item besitzt dafür einen ``Wobbeloszillator'', der schnell seine Frequenz ändern kann
      \item Oberwellen mit sehr breiter Wobbelbandbreite messen
      \item Senderbandreite mit sehr schmaler Wobbelbandbreite messen
    \end{itemize}
  \end{columns}
\end{frame}

\begin{frame}
  \frametitle{Spektrumanalysator}
  \begin{center}
    \begin{figure}
      \includegraphics[width=\textwidth,height=.75\textheight,keepaspectratio]{a16/Spektrumanalysator-Display.png}
      \attribcaption{Display eines Spektrumanalysators zeigt Frequenzgang}{Wirepath}{https://commons.wikimedia.org/wiki/File:SpectrumAnalyzerDisplay.png}{\ccbysa}
    \end{figure}
  \end{center}
\end{frame}

\renewcommand{\refname}{Referenzen}

\hypertarget{refs}{}
\textcolor{white}{} \\ %\vspace{} geht nicht
\Large Referenzen/Links
\footnotesize

\begin{thebibliography}{}
  \bibitem{a17}  Moltrecht A 16: \\
    \url{https://www.darc.de/der-club/referate/ajw/lehrgang-ta/a16/}

\end{thebibliography}

% Hier könnte noch eine Kontaktfolie stehen

\end{document}

