% Foliensatz: "AFu-Kurs nach DJ4UF" von DK0TU, Amateurfunkgruppe der TU Berlin
% Lizenz: CC BY-NC-SA 3.0 de (http://creativecommons.org/licenses/by-nc-sa/3.0/de/)
% Autoren: Sebastian Lange <dl7bst@dk0tu.de>
% Korrekturen: Lars Weiler <dc4lw@darc.de>

preamble.dk0tu.tex
\subtitle{Technik Klasse E 01: \\
  Mathematische Grundlagen und Einheiten \\[2em]}
\date{Stand 10.10.2016}
 \begin{document}

\begin{frame}
    \titlepage
    \vfill
    \begin{center}
        \ccbyncsaeu\\
        {\tiny This work is licensed under the \em{Creative Commons Attribution-NonCommercial-ShareAlike 3.0 License}.}\\[0.5ex]
         \tiny Amateurfunkgruppe der Technische Universität Berlin (AfuTUB), DKØTU
         %\includegraphics[scale=0.5]{img/DK0TU_Logo.pdf}
    \end{center}
\end{frame}


\section{Einleitung}

\begin{frame}
  \frametitle{Einleitung}

  Zu Beginn eine kurze Wiederholung der benötigten mathematischen Grundlagen,
  um das Schulwissen kurz aufzufrischen.

\end{frame}

\section{Größen und Einheiten}

\subsection{SI-Basissystem}

\begin{frame}
  \frametitle{SI-Basissystem}

  SI\footnote{Système international d’unités, ab 1790 von franz. Akademie der
  Wissenschaften entwickelt, immer wieder erweitert}-Einheiten: Weitest
  verbreitetes System seit \emph{Meterkonvention} 1875 durch 17~Staaten.
  \\[1em]

  Eigenschaften:

  \begin{itemize}
    \item basiert auf metrischen Größen
    \item dezimal (Basis 10)
    \item kohärentes Einheitensystem\footnote{alles aus Basiseinheiten
      ableitbar ohne zusätzliche Faktoren}
    \item sieben Basiseinheiten -- Welche?
  \end{itemize}

  \bigskip \pause
  Welche Einheiten gibt es und welche Größen beschreiben sie?

\end{frame}

\begin{frame}
  \frametitle{SI-Basissystem}

  \begin{center}
    \begin{figure}
      \includegraphics[height=0.75\textheight]{e01/SI_base_unit.png}
      \attribcaption{SI Basiseinheiten}{Dono}{https://commons.wikimedia.org/wiki/File:SI_base_unit.svg}{\ccbysa}
    \end{figure}
  \end{center}

\end{frame}

\begin{frame}
  \begin{block}{Beschreibung der Einheiten}
    \begin{description}
      \item[m/Meter] Länge
      \item[A/Ampere] Stromstärke
      \item[mol/Mol] Stoffmenge/Substanzmenge
      \item[kg/Kilogramm] Masse
      \item[K/Kelvin] Temperatur
      \item[cd/Candela] Lichtstärke
      \item[s/Sekunde] Zeit
    \end{description}
  \end{block}

  \pause
  Kleiner Ausflug: SI vs. Imperial System...
\end{frame}

\begin{frame}
  \begin{center}
    \begin{figure}
      \includegraphics[height=.9\textheight]{e01/imperial_vs_metric.jpeg}
      \attribcaption{Imperial units vs. Metrical units}{n/a}{http://asset-f.soup.io/asset/2816/7317_fb37.jpeg}{}
    \end{figure}
  \end{center}

\end{frame}

\subsection{Abgeleitete Einheiten}

\begin{frame}
  \frametitle{Abgeleitete Einheiten}

  \begin{center}
    \footnotesize
    \renewcommand{\arraystretch}{1.5}
    \begin{tabular}{|c|l|l|}\hline
      \textbf{Formelzeichen} & \textbf{Maßeinheit} & \textbf{Größe} \\ \hline \hline
      $Q$ & $C = A\cdot s$            & \only<2>{Ladung}           \\ \hline
      $U$ & $V$                       & \only<2>{Spannung}         \\ \hline
      $P$ & $W = V\cdot A$            & \only<2>{Leistung}         \\ \hline
      $E$ & $\frac{V}{m}$             & \only<2>{El. Feldstärke}   \\ \hline
      $H$ & $\frac{A}{m}$             & \only<2>{Magn. Feldstärke} \\ \hline
      $f$ & $Hz = \frac{1}{s}$        & \only<2>{Frequenz}         \\ \hline
      $R$ & $\Omega = \frac{V}{A}$    & \only<2>{Widerstand}       \\ \hline
      $G$ & $S = \frac{1}{\Omega}$    & \only<2>{Leitwert}         \\ \hline
      $C$ & $F = \frac{A\cdot s}{V}$  & \only<2>{Kapazität}        \\ \hline
      $L$ & $H = \frac{V\cdot s}{A}$  & \only<2>{Induktivität}     \\ \hline
    \end{tabular}

    Vorsicht: Einheit vs. Formelzeichen vs. Zehnerpotenzen!
  \end{center}

\end{frame}

\begin{frame}

  \begin{center}
    \begin{figure}
      \includegraphics[height=.9\textheight]{e01/imperial_vs_metric++.png}
      \attribcaption{Noch ein Ausflug zu SI vs. Imperial System}{n/a}{http://th03.deviantart.net/fs70/PRE/i/2012/358/1/f/imperial_vs__metric_by_nekit1234007-d5p0ou5.png}{}
    \end{figure}
  \end{center}

\end{frame}

\section{Zehnerpotenzen}

\begin{frame}
  \frametitle{Zehnerpotenzen}

  Zur einfacheren Anwendung: Verwendung von Einheitenpräfixe als Potenzen zur Basis 10

  \begin{center}
    \footnotesize
    \begin{tabular}{|c|l|l|}\hline
      \textbf{Symbol} & \textbf{Name} & \textbf{Potenz} \\ \hline \hline
      $P$   & Peta  & $10^{15}$  \\ \hline
      $T$   & Tera  & $10^{12}$  \\ \hline
      $G$   & Giga  & $10^{09}$  \\ \hline
      $M$   & Mega  & $10^{06}$  \\ \hline
      $k$   & Kilo  & $10^{03}$  \\ \hline
      $h$   & Hekto & $10^{02}$  \\ \hline
      $da$  & Deka  & $10^{01}$  \\ \hline
      $d$   & Dezi  & $10^{-01}$ \\ \hline
      $c$   & Zenti & $10^{-02}$ \\ \hline
      $m$   & Milli & $10^{-03}$ \\ \hline
      $\mu$ & Mikro & $10^{-06}$ \\ \hline
      $n$   & Nano  & $10^{-09}$ \\ \hline
      $p$   & Piko  & $10^{-12}$ \\ \hline
      $f$   & Femto & $10^{-15}$ \\ \hline
    \end{tabular}
  \end{center}

\end{frame}

\begin{frame}
  \frametitle{Zehnerpotenzen / Zum Nachdenken}

  Sind alle Einheiten und ihre Präfixe ``case insensitive'' und zur Basis 10?

  \begin{block}{\begin{center}\Large Mb vs. MB vs. MiB vs. mb\end{center}}
    \only<2>{
    \begin{itemize}
      \item Mb = Megabit, Basis $10$
      \item MB = Megabyte, Basis $10$
      \item MiB = Mebibyte, Basis $2^{(10)}$
      \item mb = Millibit!? ;-)
    \end{itemize}
    }
  \end{block}

  \vspace{2em}

  \only<2>{Auch nochmal zur Erinnerung: Einheiten, Formelzeichen und Präfixe nicht
  durcheinanderwürfeln!}

  % todo ggf. Beispiele mit Zehnerpotenzen

\end{frame}

\begin{frame}
  \begin{alertblock}{Fakultative Hausaufgabe}
    Zehnerpotenzen mit Name und Potenz auswendig lernen!

    \bigskip \tiny (Praktisch für den Alltag - notwendige stehen allerdings
    in der Formelsammlung)
  \end{alertblock}
\end{frame}


\section{Formeln umstellen}

\begin{frame}
  \frametitle{Formeln umstellen}

  ...sollte grob beherrscht werden. \\[2em]

  Für \emph{Klasse E} geht es nicht über einfache Umstellungen\footnote{es
  gibt eine umfangreiche Formelsammlung für die Prüfung} heraus, die man sich über ein
  $U = R\cdot I$- oder $P = U\cdot I$-Dreieck herleiten kann.

  \begin{center}
    \begin{figure}
      \includegraphics[height=.4\textheight]{e03/Ohm_law_triangle.png}
      \attribcaption{das ohmsche Dreieck}{Eirik}{https://commons.wikimedia.org/wiki/File:Ohm's_law_triangle.PNG}{\ccpd}
    \end{figure}
  \end{center}

\end{frame}

\renewcommand{\refname}{Referenzen}

\hypertarget{refs}{}
\textcolor{white}{} \\ %\vspace{} geht nicht
\Large Referenzen/Links
\footnotesize

\begin{thebibliography}{}
  \bibitem{dj4uf} Moltrecht E 01: \\
    \url{https://www.darc.de/der-club/referate/ajw/lehrgang-te/e01/}
  \bibitem{wp}    Wikipedia DE: \\
    \url{https://de.wikipedia.org/wiki/Internationales_Einheitensystem}\\
\end{thebibliography}

% Hier könnte noch eine Kontaktfolie stehen

\end{document}

