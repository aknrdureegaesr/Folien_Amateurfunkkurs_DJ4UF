% Foliensatz: "AFu-Kurs nach DJ4UF" von DK0TU, Amateurfunkgruppe der TU Berlin
% Lizenz: CC BY-NC-SA 3.0 de (http://creativecommons.org/licenses/by-nc-sa/3.0/de/)
% Autoren: Sebastian Lange <dl7bst@dk0tu.de>

preamble.dk0tu.tex
\subtitle{Technik Klasse E 01: \\
          Mathematische Grundlagen und Einheiten \\[2em]}
\date{Stand 30.10.2014}
 \begin{document}

\begin{frame}
    \titlepage
    \vfill
    \begin{center}
        \ccbyncsaeu\\
        {\tiny This work is licensed under the \em{Creative Commons Attribution-NonCommercial-ShareAlike 3.0 License}.}\\[0.5ex]
         \tiny Amateurfunkgruppe der Technische Universität Berlin (AfuTUB), DKØTU
         %\includegraphics[scale=0.5]{img/DK0TU_Logo.pdf}
    \end{center}
\end{frame}


\section{Einleitung}

\begin{frame}
    \frametitle{Einleitung}

    Zu Beginn kurze Wiederholung der benötigten mathem. Grundlagen um das
    Schulwissen kurz aufzufrischen.

\end{frame}

\section{Größen und Einheiten}

\subsection{SI-Basissystem}

\begin{frame}
    \frametitle{SI-Basissystem}

    SI\footnote{Système international d’unités, ab 1790 von franz. Akademie der
    Wissenschaften entwickelt, immer wieder erweitert}-Einheiten weitest
    verbreitetes System seit \emph{Meterkonvention} 1875 durch 17 Staaten.
    \\[1em]

    Eigenschaften:

    \begin{itemize}
        \item basiert auf metrischen Größen
        \item dezimal (Basis 10)
        \item kohärentes Einheitensystem\footnote{alles aus Basiseinheiten
              ableitbar ohne zusätzliche Faktoren}
        \item sieben Basiseinheiten - Welche?
    \end{itemize}

\end{frame}

\begin{frame}
    \frametitle{SI-Basissystem}

    \begin{center}
        \includegraphics[width=0.7\textwidth]{e01/SI_base_unit.png}
        \tiny \hyperlink{refs}{\cite{wc}}
    \end{center}

    Kleiner Ausflug: SI vs. Imperial System...

\end{frame}

\begin{frame}

    \begin{center}
        \includegraphics[height=1\textheight]{e01/imperial_vs_metric.jpeg}
        \tiny \hyperlink{refs}{\cite{soup}}
    \end{center}

\end{frame}

\subsection{Abgeleitete Einheiten}

\begin{frame}
    \frametitle{Abgeleitete Einheiten}

    \begin{center}
    \footnotesize
    \begin{tabular}{|c|l|c|}\hline
        \textbf{Formelzeichen} & \textbf{Maßeinheit} & \textbf{Größe} \\ \hline \hline
        $Q$ & $C = As$                & \textbf{?} \\ \hline
        $U$ & $V$                     & \textbf{?} \\ \hline
        $P$ & $W = VA$                & \textbf{?} \\ \hline
        $E$ & $\frac{V}{m}$           & \textbf{?} \\ \hline
        $H$ & $\frac{A}{m}$           & \textbf{?} \\ \hline
        $f$ & $Hz = \frac{1}{s}$      & \textbf{?} \\ \hline
        $R$ & $\Omega = \frac{V}{A}$  & \textbf{?} \\ \hline
        $G$ & $S = \frac{1}{\Omega}$  & \textbf{?} \\ \hline
        $C$ & $F = \frac{As}{V}$      & \textbf{?} \\ \hline
        $L$ & $H = \frac{Vs}{A}$      & \textbf{?} \\ \hline
    \end{tabular}
    \end{center}

    Vorsicht: Einheit vs. Formelzeichen vs. Zehnerpotenzen!

\end{frame}

\begin{frame}
    \frametitle{Abgeleitete Einheiten}

    \begin{center}
    \footnotesize
    \begin{tabular}{|c|l|l|}\hline
        \textbf{Formelzeichen} & \textbf{Maßeinheit} & \textbf{Größe} \\ \hline \hline
        $Q$ & $C = As$                & Ladung          \\ \hline
        $U$ & $V$                     & Spannung        \\ \hline
        $P$ & $W = VA$                & Leistung        \\ \hline
        $E$ & $\frac{V}{m}$           & El. Feldstärke  \\ \hline
        $H$ & $\frac{A}{m}$           & Magn. Feldstärke\\ \hline
        $f$ & $Hz = \frac{1}{s}$      & Frequenz        \\ \hline
        $R$ & $\Omega = \frac{V}{A}$  & Widerstand      \\ \hline
        $G$ & $S = \frac{1}{\Omega}$  & Leitwert        \\ \hline
        $C$ & $F = \frac{As}{V}$      & Kapazität       \\ \hline
        $L$ & $H = \frac{Vs}{A}$      & Induktivität    \\ \hline
    \end{tabular}
    \end{center}

    %todo ggf. Frage/Antwort-Spiel mit Beispielen

    Und weil es so schön war ein weiterer Ausflug: SI vs. Imperial System...

\end{frame}

\begin{frame}

    \begin{center}
        \includegraphics[height=1\textheight]{e01/imperial_vs_metric++.png}
        \tiny \hyperlink{refs}{\cite{devi}}
    \end{center}

\end{frame}

\section{Zehnerpotenzen}

\begin{frame}
    \frametitle{Zehnerpotenzen}

    Zur einfacheren Anwendung Verwendung von Einheitenpräfixe als Potenzen zur Basis 10

    \begin{center}
    \footnotesize
    \begin{tabular}{|c|l|}\hline
        \textbf{Symbol} & \textbf{Potenz} \\ \hline \hline
        $T$   & $10^{12}$  \\ \hline
        $G$   & $10^{09}$  \\ \hline
        $M$   & $10^{06}$  \\ \hline
        $k$   & $10^{03}$  \\ \hline
        $h$   & $10^{02}$  \\ \hline
        $da$  & $10^{01}$  \\ \hline
        $d$   & $10^{-01}$ \\ \hline
        $c$   & $10^{-02}$ \\ \hline
        $m$   & $10^{-03}$ \\ \hline
        $\mu$ & $10^{-06}$ \\ \hline
        $n$   & $10^{-09}$ \\ \hline
        $p$   & $10^{-12}$ \\ \hline
    \end{tabular}
    \end{center}

    Einheiten und ihre Präfixe ''case insensitive'' und Basis 10?

\end{frame}

\begin{frame}
    \frametitle{Zehnerpotenzen / Zum Nachdenken}

    \begin{block}{\begin{center}\Large Mb vs. MB vs. MiB vs. mb\end{center}}
        \only<2>{
        \begin{itemize}
            \item Mb = Megabit, Basis $10$
            \item MB = Megabyte, Basis $10$
            \item MiB = Mebibyte, Basis $2^{(10)}$
            \item mb = Millibit!? ;-)
        \end{itemize}
        }
    \end{block}

    \vspace{2em}

    \only<2>{Auch nochmal zur Erinnerung: Einheiten, Formelzeichen und Präfixe nicht
    durcheinanderwürfeln!}

    % todo ggf. Beispiele mit Zehnerpotenzen

\end{frame}


\section{Formeln umstellen}

\begin{frame}
    \frametitle{Formeln umstellen}

    ...sollte grob beherrscht werden. \\[2em]
    
    Für \emph{Klasse E} geht es nicht über einfache Umstellungen\footnote{es
    gibt eine umfangreiche Formelsammlung für die Prüfung} die man sich über ein
    ''U=RI''- oder ''P=UI''-Dreieck herleiten kann hinaus.

    %todo Beispiele - ernsthaft?

\end{frame}

\renewcommand{\refname}{Referenzen}

\hypertarget{refs}{}
\textcolor{white}{} \\ %\vspace{} geht nicht
\Large Referenzen/Links
\footnotesize

\begin{thebibliography}{}
    \bibitem{dj4uf} Moltrecht E 01: \\
                    \url{http://www.darc.de/referate/ajw/ausbildung/darc-online-lehrgang/technik-klasse-e/technik-e01/}
    \bibitem{wp}    Wikipedia DE: \\
                    \url{https://de.wikipedia.org/wiki/Internationales_Einheitensystem}\\
    \bibitem{wc}    Wikimedia Commons \\
                    \url{https://commons.wikimedia.org/wiki/File:SI_base_unit.svg}\\
    \bibitem{soup}  \url{http://asset-f.soup.io/asset/2816/7317_fb37.jpeg}
    \bibitem{devi}  \url{http://th03.deviantart.net/fs70/PRE/i/2012/358/1/f/imperial_vs__metric_by_nekit1234007-d5p0ou5.png}
\end{thebibliography} 

% Hier könnte noch eine Kontaktfolie stehen

\end{document}

