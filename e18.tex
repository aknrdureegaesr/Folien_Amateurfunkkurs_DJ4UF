% Foliensatz: "AFu-Kurs nach DJ4UF" von DK0TU, Amateurfunkgruppe der TU Berlin
% Lizenz: CC BY-NC-SA 3.0 de (http://creativecommons.org/licenses/by-nc-sa/3.0/de/)
% Autoren: Martin Deutschmann <martin.deutschmann@campus.tu-berlin.de>

preamble.dk0tu.tex
\subtitle{Technik Klasse E 18 \& Betriebstechnik/Vorschriften 14: \\
          EMV, EMVG, EMVU, Sicherheit \& Störungen  \\[2em]}
\date{Stand 05.02.2015}
 \begin{document}

\begin{frame}
    \titlepage
    \vfill
    \begin{center}
        \ccbyncsaeu\\
        {\tiny This work is licensed under the \em{Creative Commons Attribution-NonCommercial-ShareAlike 3.0 License}.}\\[0.5ex]
         \tiny Amateurfunkgruppe der Technische Universität Berlin (AfuTUB), DKØTU
         %\includegraphics[scale=0.5]{img/DK0TU_Logo.pdf}
    \end{center}
\end{frame}


%fixme Bildreferenzen aus den Folien raus und in den Anhang

\section{Einleitung}

\begin{frame}
    \frametitle{Einleitung / Umleitung}

    Aufgrund sehr großer inhaltlicher Überschneidungen der beiden
    \emph{Moltrecht}-Lektionen, ist die Lektion
    \texttt{BV14}\hyperlink{refs}{\cite{bv14}} in diesen Foliensatz der Lektion
    \texttt{Technik E18}\hyperlink{refs}{\cite{e18}} integriert. \\[2em]
\end{frame}

\begin{frame}
	\frametitle{Einleitung}
	In diesem Kapitel geht es um Elektromagnetische Verträglichkeit (EMV) von Geräten(EMVG) und 	der Umwelt (EMVU), Störungen und Sicherheit während des Funkes im Rahmen der gesetzlichen 		Vorgaben zu halten.
\end{frame}

\begin{frame}
	\frametitle{Warum kümmert uns EMV eigentlich?}
	\begin{itemize}
		\item Unsere Sendeanlagen erzeugen elektromagnetische Strahlung
		\item Diese kann Geräte in der Nähe stören (z.B. Fernseher des Nachbarn)
		\item Aber auch unsere Geräte können gestört werden
		\item Deswegen müssen diese geschützt werden
	\end{itemize}
\end{frame}

\section{EMVG}
\begin{frame}
	\frametitle{\textbf{E}lektro-\textbf{M}agnetische \textbf{V}erträglichkeit von \textbf{G}eräten}
	\Large{\textbf{EMVG §3 Schutzanforderungen}}
	\begin{normalsize}
	(1) Geräte müssen so beschaffen sein, dass bei vorschriftsgemäßer Installierung, 				angemessener Wartung und bestimmungsgemäßem Betrieb gemäß den Angaben des Herstellers in 		der Gebrauchsanweisung	
	\begin{itemize}
		\item 1. die Erzeugung elektromagnetischer Störungen begrenzt wird, dass ein 							bestimmungsfähiger Betrieb von Funk und Telekommunikationsgeräten sowie 						sonstigen Geräten möglich ist, 
		\item 2. die Geräte eine angemessene Festigkeit gegen elektromagnetische Störungen 						aufweisen, so dass ein bestimmungsgemäßer Betrieb möglich ist.
	\end{itemize}
	\end{normalsize}
\end{frame}

\section{EMV \& EMVU}

\begin{frame}
	\frametitle{EMV \& EMVU}
	\begin{itemize}
		\item Im §7 des AFuG steht etwas zu den Schutzanforderungen und EMV geschrieben
		\item Genaueres dazu findet sich auch im §17 des AFuV
	\end{itemize}
\end{frame}

\section{Störungen}

\begin{frame}
	\frametitle{Was sind Störungen? und was hilft dagegen?}
	\begin{itemize}
		\item Störungen sind unerwünschte Ausstrahlung (wie z.B. Oberwellen oder 								Nebenaussendungen)
		\item Um Störungen zu vermeiden, kann man Filter verwenden
		\item Man kann auch mit Filtern gegen Störungen vorgehen
	\end{itemize}
	\vspace{0.5cm}
	\begin{center}
	\begin{tabular}{|c|c|}
		\hline
		Bis 25 W & über 25 W \\ \hline
		\multicolumn{2}{|l|}{Kurzwellensender:} \\ \hline
		max. +4 dBm & $>$ 40 dB gedämpft	\\ \hline
		\multicolumn{2}{|l|}{VHF-/UHF-Sender:} \\ \hline
		max. -16 dBm & $>$ 60 dB gedämpft \\ \hline
	\end{tabular}
	\end{center}
	
\end{frame}

\begin{frame}
	\frametitle{Wo störts denn?}
	\Large{Aber wo stört es denn nun?}
\end{frame}

\begin{frame}
 	\frametitle{Wo störts denn?}
 	\includegraphics[scale=0.2]{e18/Stoerungen.png}\\
 	\tiny{Quelle: DM7MD} 	
\end{frame}

\begin{frame}
	\frametitle{Was stört uns?}
	\begin{itemize}
		\item Intermodulationen (Übersteuern der Mischstufe durch mehrere, zu starke Signale)
		\item Diese erzeugen dann Phantomsignale
		\item Zustopfeffekte
	\end{itemize}
\end{frame}

\begin{frame}
	\frametitle{Vorbeugende Maßnahmen}
	\begin{itemize}
		\item Sendeanlage so weit weg von Empfangsantennen wie möglich aufbauen
		\item Eine Richtantenne mit geringem vertikalen Öffnungswinkel auf dem Dach ist ein 					guter Anfang 
		\item Anpassen der Sendeleistung
	\end{itemize}
\end{frame}

\section{Personenschutz (EMVU)}

\begin{frame}
	\frametitle{Personenschutz (EMVU)}
	\begin{itemize}
		\item Es gilt zwei Bereiche zu beachten:
			\begin{itemize}
				\item Expositionsbereich 1 (Vom Funker kontrollierbarer Bereich) und 
				\item Expositionsbereich 2 (Öffentlicher Bereich)
			\end{itemize}
		\item Es gibt festgelegte Grenzwerte und Formeln der BNetzA zur Berechnung des 						Sicherheitsabstandes
	\end{itemize}
\end{frame}

\begin{frame}
	\frametitle{Sicherheitsabstand}
	\begin{center}
	\begin{LARGE}
		$r = \frac{\sqrt{30 \cdot P_{EIRP}[W]}}{E[\frac{V}{m}]}$
	\end{LARGE}
	\vspace{0.5cm}
	
	\begin{tabular}{|c|c|}
	\hline
	\multicolumn{2}{|l|}{Grenzwerte für Personenschutz} \\ \hline
	Frequenzbereich & Elektrische Feldstärke \\ \hline
	unter 10 MHz & $E = 87 / \sqrt{f(MHz)} V/m$ \\ \hline
	10 - 40 MHz & $E = 27,5 V/m$ \\ \hline
	40 - 2000 MHz & $E = 1,375 \cdot \sqrt{f(MHz)} V/m$ \\ \hline
	über 2000 MHz & $E = 61 V/m$ \\ \hline
	\end{tabular}
	\end{center}
\end{frame}

\begin{frame}
	\frametitle{Leistungsgewinnfaktoren und Reduzierungsfaktoren}
	\begin{tabular}{|c|c|c|}
		\hline
		\multicolumn{2}{|l|}{Leistungsgewinnfaktoren} & oder in dBi \\ \hline
		Dipol & 1,64 & 2,15 dbi \\ \hline
		$\lambda / 4$ - Strahler & 3,28 & 5,15 dBi \\ \hline
	\end{tabular}
	\vspace{0.5cm}
	
	\begin{tabular}{|c|c|}
		\hline
		Bstriebsart & Reduzierungsfaktor \\ \hline
		SSB & $1 : 6 = 0,167$ \\ \hline
		CW & $1 : 4 = 0,25$ \\ \hline
		FM (RTTY, SSTV) & 1 \\ \hline		
	\end{tabular}
	\begin{itemize}
		\item	Reduzierungsfaktoren sind Effektivwerte der maximalen Feldstärke gemittelt über 		einen 6-Minuten Intervall
	\end{itemize}
\end{frame}

\section{Sicherheit}

\begin{frame}
	\frametitle{Gesetze und Vorschriften, wohin man schaut}
	\begin{itemize}
		\item Selbstgebaute Geräte müssen den VDE-Bestimmungen entsprechen
		\item Für den Bau von Antennenanlagen gelten die Bauverordnungen des jeweiligen Landes
		\item Des Weiteren sind die Blitzschutzbestimmungen zu beachten
	\end{itemize}
\end{frame}

\subsection{elektrische Sicherheit}

\begin{frame}
	\frametitle{elektrische Sicherheit}
	\begin{itemize}
		\item	Sicherheitsanforderungen
		\item	Berührungsschutz
		\item	Schutzkleinspannung \& Funktionskleinspannung
		\item	Schutzisolierung
		\item	Schutztrennung
		\item	Schutz durch Abschalten
	\end{itemize}
\end{frame}

\begin{frame}
	\frametitle{Antennenerdung}
	\begin{itemize}
		\item	Alle leitfähigen Teile der Antennenanlage müssen geerdet werden
	\end{itemize}
	\begin{tabular}{|c|c|}
		\hline
		Werkstoff & Abmessungen oder Art \\ \hline
		Kupfer & $16 mm^2$ , blank oder isoliert \\ \hline
		Aluminium & $25 mm^2$ , isoliert, in Innenräumen auch blank \\ \hline
		Stahl & $50 mm^2$ , verzinkt, z.B. Band $20mm \cdot 2,5mm$ \\ \hline
		\multicolumn{2}{|l|}{Volldraht oder mehrdrähtig, aber nicht feindrähtig} \\
		\multicolumn{2}{|l|}{Kennzeichnung der Erdleiter: grün-gelb} \\ \hline 	
	\end{tabular}
\end{frame}

\begin{frame}
	\frametitle{Blitzschutz}
	\begin{itemize}
		\item	Antennenanlagen müssen gegen Blitze geschützt werden
		\item	Der Mast muss auf dem kürzesten Weg mit einer fachlich korrekt installierten 					Gebäudeblitzschutzanlage verbunden werden
		\item	Als Blitzschutzerder darf jeder ordnungsgemäß verlegter Fundamenterder 							verwendet werden
		\item	Außenleiter von Koaxkabeln (Abschirmung) müssen über einen 										Potentialausgleichsleiter ordnungsgemäß mit Erde verbunden werden 
 	\end{itemize}
 	\begin{center}
 		\includegraphics[scale=0.2]{e18/Funkenstrecke}
 		\footnote{\url{http://de.wikipedia.org/wiki/Datei:Suspension_insulators_with_arcing_horns.jpg}}
 	\end{center}
 
\end{frame}

\subsection{mechanische Sicherheit}

\begin{frame}
	\frametitle{mechanische Sicherheit}
	\begin{itemize}
		\item	Sendeanlage muss mechanisch gesichert werden
		\item	Windlast ist zu beachten
	\end{itemize}
	\begin{center}
		\Large{$F_{A} = p \cdot A$}
	\end{center}	
	\begin{normalsize}
	
	\end{normalsize}
	\begin{itemize}
		\item	Sendeanlagen dürfen im KFZ installiert werden
		\item	Dafür immer Herstellerangaben beachten
		\item	Um Störungen zu vermeiden, Antenne und Antennenkabel möglichst weit weg von 					Fahrzeugelektronik installieren
	\end{itemize}
\end{frame}

\renewcommand{\refname}{Referenzen}

\hypertarget{refs}{}
\textcolor{white}{} \\ %\vspace{} geht nicht
\Large Referenzen/Links
\footnotesize

\begin{thebibliography}{}
    \bibitem{bv14}  Moltrecht B/V 14: \\
                    \url{http://www.darc.de/referate/ajw/ausbildung/darc-online-lehrgang/betriebstechnikvorschriften/kapitel-bv14/}
    \bibitem{e18}   Moltrecht E 18: \\
                    \url{http://www.darc.de/referate/ajw/ausbildung/darc-online-lehrgang/technik-klasse-e/technik-e18/}
    \bibitem{wp}    Wikipedia DE: \\
                    \url{http://de.wikipedia.org/wiki/Blitzschutz}\\ 
                    \url{http://de.wikipedia.org/wiki/Funkenstrecke}\\  
\end{thebibliography} 

% Hier könnte noch eine Kontaktfolie stehen

\end{document}

