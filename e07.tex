% Foliensatz: "AFu-Kurs nach DJ4UF" von DK0TU, Amateurfunkgruppe der TU Berlin
% Lizenz: CC BY-NC-SA 3.0 de (http://creativecommons.org/licenses/by-nc-sa/3.0/de/)
% Autoren: Martin Deutschmann
% Korrekturen: Lars Weiler <dc4lw@darc.de>, Felix Baum <baumfelix@gmail.com>

preamble.dk0tu.tex
\subtitle{Technik Klasse E 07: \\
  Schwingkreise \& Filter \\[2em]}
\date{Stand 18.09.2017}
 \begin{document}

\begin{frame}
    \titlepage
    \vfill
    \begin{center}
        \ccbyncsaeu\\
        {\tiny This work is licensed under the \em{Creative Commons Attribution-NonCommercial-ShareAlike 3.0 License}.}\\[0.5ex]
         \tiny Amateurfunkgruppe der Technische Universität Berlin (AfuTUB), DKØTU
         %\includegraphics[scale=0.5]{img/DK0TU_Logo.pdf}
    \end{center}
\end{frame}


\section*{Schwingungs\-vorgang}
% TODO Tabelle

\begin{frame}
  \frametitle{Schwingungen, wo gibt es denn sowas?}
  \begin{center}
    \begin{figure}
      \includegraphics[width=\textwidth,height=.75\textheight,keepaspectratio]{e07/Schwingkreise.png}
      \caption{Reihen- \& Parallelschwingkreise}
    \end{figure}
  \end{center}
\end{frame}

\begin{frame}
  \frametitle{Reihenschwingkreis}
  \begin{columns}
    \column{0.45\textwidth}{
    \begin{figure}
      \includegraphics[width=1\textwidth,height=.75\textheight,keepaspectratio]{e07/Serirenschw.png}
      \attribcaption{Reihenschwingkreis}{V4711}{http://commons.wikimedia.org/wiki/File:RLC_series_circuit_v1.svg}{\ccbysa}
    \end{figure}
    }
    \column{0.55\textwidth}
    {
    \begin{tabular}{l|llll}
      $f$ & $X_C$ & $X_L$ & $Z_g$ & $I_{ges}$ \\ \hline
      \hline
      $0$ & \only<1>{}\only<2>{$\infty$}\only<3>{$\infty$}  & \only<1>{}\only<2>{$0$}\only<3>{$0$} & \only<1>{}\only<2>{}\only<3>{$\infty$}  & \only<1>{}\only<2>{}\only<3>{$0$} \\
      $\infty$ & \only<1>{}\only<2>{$0$}\only<3>{$0$}   & \only<1>{}\only<2>{$\infty$}\only<3>{$\infty$} & \only<1>{}\only<2>{}\only<3>{$\infty$} & \only<1>{}\only<2>{}\only<3>{$0$}  \\
      $f_{0}$  & \only<1>{}\only<2>{$\frac{1}{2\pi\cdot f \cdot C} =$}\only<3>{$\frac{1}{2\pi\cdot f \cdot C} =$} & \only<1>{}\only<2>{$2\pi\cdot f \cdot L$}\only<3>{$2\pi \cdot f \cdot L$}  & \only<1>{}\only<2>{}\only<3>{Min}  & \only<1>{}\only<2>{}\only<3>{Max} \\
    \end{tabular}
    }
  \end{columns}
\end{frame}

\begin{frame}
  \frametitle{Parallelschwingkreis}
  \begin{columns}
    \column{0.37\textwidth}{
    \begin{figure}
      \includegraphics[width=1\textwidth,height=.75\textheight,keepaspectratio]{e07/Parallelschw.png}
      \caption{Parallelschwingkreis}
    \end{figure}
    }
    \column{0.62\textwidth}
    {
    \begin{tabular}{l|llll}
      $f$ & $X_C$ & $X_L$ & $Z_g$ & $I_{ges}$ \\ \hline
      \hline
      $0$ & \only<1>{}\only<2>{$\infty$}\only<3>{$\infty$}  & \only<1>{}\only<2>{$0$}\only<3>{$0$} & \only<1>{}\only<2>{}\only<3>{$0$}  & \only<1>{}\only<2>{}\only<3>{Max} \\
      $\infty$ & \only<1>{}\only<2>{$0$}\only<3>{$0$}   & \only<1>{}\only<2>{$\infty$}\only<3>{$\infty$} & \only<1>{}\only<2>{}\only<3>{$0$} & \only<1>{}\only<2>{}\only<3>{Max}  \\
      $f_{0}$  & \only<1>{}\only<2>{$\frac{1}{2\pi\cdot f \cdot C} =$}\only<3>{$\frac{1}{2\pi\cdot f \cdot C} =$} & \only<1>{}\only<2>{$2\pi \cdot f \cdot L$}\only<3>{$2\pi \cdot f \cdot L$}  & \only<1>{}\only<2>{}\only<3>{Max}  & \only<1>{}\only<2>{}\only<3>{Min} \\
    \end{tabular}
    }
  \end{columns}
\end{frame}

\begin{frame}
  \frametitle{Aber warum schwingt das denn jetzt?}
  \begin{center}
    \begin{figure}
      \includegraphics[width=.8\textwidth,height=.75\textheight,keepaspectratio]{e07/Schwingkreis.png}
      \attribcaption{Energie in einem LC-Schwingkreis}{X3ntar}{http://commons.wikimedia.org/wiki/File:LC_circuit_4_times_new_version.svg}{\ccpd}
    \end{figure}
  \end{center}
\end{frame}

\begin{frame}
  \begin{center}
    \begin{figure}
      \only<1>{\includegraphics[width=\textwidth,height=.9\textheight,keepaspectratio]{e07/Tuned_circuit_animation_3-0.png}}
      \only<2>{\includegraphics[width=\textwidth,height=.9\textheight,keepaspectratio]{e07/Tuned_circuit_animation_3-1.png}}
      \only<3>{\includegraphics[width=\textwidth,height=.9\textheight,keepaspectratio]{e07/Tuned_circuit_animation_3-2.png}}
      \only<4>{\includegraphics[width=\textwidth,height=.9\textheight,keepaspectratio]{e07/Tuned_circuit_animation_3-3.png}}
      \only<5>{\includegraphics[width=\textwidth,height=.9\textheight,keepaspectratio]{e07/Tuned_circuit_animation_3-4.png}}
      \only<6>{\includegraphics[width=\textwidth,height=.9\textheight,keepaspectratio]{e07/Tuned_circuit_animation_3-5.png}}
      \only<7>{\includegraphics[width=\textwidth,height=.9\textheight,keepaspectratio]{e07/Tuned_circuit_animation_3-6.png}}
      \only<8>{\includegraphics[width=\textwidth,height=.9\textheight,keepaspectratio]{e07/Tuned_circuit_animation_3-7.png}}
      \only<9>{\includegraphics[width=\textwidth,height=.9\textheight,keepaspectratio]{e07/Tuned_circuit_animation_3-8.png}}
      \only<10>{\includegraphics[width=\textwidth,height=.9\textheight,keepaspectratio]{e07/Tuned_circuit_animation_3-9.png}}
      \only<11>{\includegraphics[width=\textwidth,height=.9\textheight,keepaspectratio]{e07/Tuned_circuit_animation_3-10.png}}
      \only<12>{\includegraphics[width=\textwidth,height=.9\textheight,keepaspectratio]{e07/Tuned_circuit_animation_3-11.png}}
      \attribcaption{Energie in einem LC-Schwingkreis}{Chetvorno}{http://en.wikipedia.org/wiki/File:Tuned_circuit_animation_3.gif}{\cczero}
    \end{figure}
  \end{center}
\end{frame}

\begin{frame}
  \frametitle{Schwingung in der Realität?}
  \begin{center}
    \begin{figure}
      \includegraphics[width=.8\textwidth,height=.75\textheight,keepaspectratio]{e07/Damped_oscillation_graph.png}
      \caption{Gedämpfte Schwingung}
    \end{figure}
  \end{center}
  \begin{block}{Übung}
    Wodurch wird die Schwingung gedämpft? \\
    Was müsste getan werden um die Schwingung aufrecht zu erhalten?
  \end{block}
\end{frame}

\begin{frame}
  \frametitle{Resonanzfrequenz}
  \begin{block}{Frequenz mit der sich die Schwingung wiederholt}
    \begin{center}
      \Huge{$ f_{0} = \cfrac{1}{2 \cdot \pi \cdot \sqrt{L \cdot C}} $}
    \end{center}
  \end{block}
  \pause
  Durch Verluste (insbesondere Widerstand) kommt es zur gedämpften Schwingung.
\end{frame}

\section*{Reihen\-schwing\-kreis}
\begin{frame}
  \frametitle{Reihenschwingkreis}
  \begin{columns}
    \column{0.5\textwidth}
    \begin{figure}
      \includegraphics[width=\textwidth,height=.4\textheight,keepaspectratio]{e07/Serirenschw.png}
      \attribcaption{Reihenschwingkreis}{V4711}{http://commons.wikimedia.org/wiki/File:RLC_series_circuit_v1.svg}{\ccbysa}
    \end{figure}
    \column{0.5\textwidth}
    \begin{figure}
      \includegraphics[width=\textwidth,height=.4\textheight,keepaspectratio]{e07/SerirenschwSig.png}
      \attribcaption{Resonanzwiderstand}{Unknown}{https://commons.wikimedia.org/wiki/File:Resonanzwiderstand_serie.svg}{\ccpd}
    \end{figure}
  \end{columns}
  \begin{itemize}
    \item Im Verlauf der Frequenzänderung ändert sich der Gesamtwellenwiderstand $Z$ des Schwingkreises
    \item Der Schwingkreis hat als minimale Impedanz seinen ohmschen Wert, da sich bei der Resonanzfrequenz $f_{0}$ die induktiven und kapazitiven Anteile gegenseitig aufheben
  \end{itemize}
\end{frame}

\section*{Parallel\-schwing\-kreis}
\begin{frame}
  \frametitle{Parallelschwingkreis}
  \begin{center}
    \begin{columns}
      \column{0.5\textwidth}
      \begin{figure}
        \includegraphics[width=\textwidth,height=.4\textheight,keepaspectratio]{e07/Parallelschw.png}
        \attribcaption{Parallelschwingkreis}{Tillmann Walther}{http://commons.wikimedia.org/wiki/File:KondiSpuleWiderstandParallel.svg}{\ccpd}
      \end{figure}
      \column{0.5\textwidth}
      \begin{figure}
        \includegraphics[width=\textwidth,height=.35\textheight,keepaspectratio]{e07/ParallelschwSig.png}
        \attribcaption{Resonanzwiderstand}{Unknown}{http://commons.wikimedia.org/wiki/File:Resonanzwiderstand_parallel.svg}{\ccpd}
      \end{figure}
    \end{columns}
  \end{center}
  \begin{itemize}
    \item Der Parallelschwingkreis verhält sich genau entgegen gesetzt zum Reihenschwingkreis
    \item Dieser zeigt bei niedrigen und hohen Frequenzen das Verhalten eines Leiters
    \item Bei der Resonanzfrequenz hingegen steigt der Wellenwiderstand an, da hier nur noch der ohmsche Widerstand wirkt
  \end{itemize}
\end{frame}



\section*{Filter}
\subsection*{Saugkreis}
\begin{frame}
  \frametitle{Saugkreis}
  \begin{center}
    \begin{columns}
      \column{.5\textwidth}
      \begin{figure}
        \includegraphics[width=\textwidth,height=.45\textheight,keepaspectratio]{e07/Saugkreis.png}
        \attribcaption{Saugkreis}{Herbertweidner}{https://commons.wikimedia.org/wiki/File:Saugkreis.png}{\ccpd}
      \end{figure}
      \column{.5\textwidth}
      \begin{figure}
        \includegraphics[width=\textwidth,height=.45\textheight,keepaspectratio]{e07/SerirenschwSig.png}
        \attribcaption{Resonanzwiderstand}{Unknown}{https://commons.wikimedia.org/wiki/File:Resonanzwiderstand_serie.svg}{\ccpd}
      \end{figure}
    \end{columns}
  \end{center}
  \pause
  \begin{itemize}
    \item vor und nach der Resonanzfrequenz hoher Widerstand
    \item nur Wechselspannungen mit Frequenzen in der Nähe der Resonanzfrequenz werden durchgelassen
    \item Anwendung: Audiotechnik
  \end{itemize}
\end{frame}

\subsection*{Sperrkreis}
\begin{frame}
  \frametitle{Sperrkreis}
  \begin{center}
    \begin{columns}
      \column{.5\textwidth}
      \begin{figure}
        \includegraphics[width=\textwidth,height=.45\textheight,keepaspectratio]{e07/Sperrkreis.png}
        \attribcaption{Sperrkreis}{Herbertweidner}{https://commons.wikimedia.org/wiki/File:Sperrkreis.png}{\ccpd}
      \end{figure}
      \column{.5\textwidth}
      \begin{figure}
        \includegraphics[width=\textwidth,height=.45\textheight,keepaspectratio]{e07/ParallelschwSig.png}
        \attribcaption{Parallelschwingkreis}{Tillmann Walther}{http://commons.wikimedia.org/wiki/File:KondiSpuleWiderstandParallel.svg}{\ccpd}
      \end{figure}
    \end{columns}
  \end{center}
  \pause
  \begin{itemize}
    \item bei der Resonanzfrequenz hoher Widerstand
    \item die Resonanzfrequenz wird gefiltert
    \item Anwendungen: Mehrbandantennen; Filtern von starken Sendern
  \end{itemize}
\end{frame}

\subsection*{Tiefpass}
\begin{frame}
  \frametitle{Tiefpass}
  \begin{center}
    \begin{figure}
      \includegraphics[width=\textwidth,height=.45\textheight,keepaspectratio]{e07/LC-Tiefpass.png}
      \caption{LC-Tiefpass}
    \end{figure}
  \end{center}
  \begin{itemize}
    \item Bei steigender Frequenz steigt der Blindwiderstand $X_L$ und der Blindwiderstand $X_C$ sinkt
    \item Bei sinkender Frequenz hingegen sinkt $X_L$ und $X_C$ steigt
    \item Dadurch werden nur niedrige Frequenzen durchgelassen
  \end{itemize}
\end{frame}

\subsection*{Hochpass}
\begin{frame}
  \frametitle{Hochpass}
  \begin{center}
    \begin{figure}
      \includegraphics[width=\textwidth,height=.45\textheight,keepaspectratio]{e07/LC-Hochpass.png}
      \caption{LC-Hochpass}
    \end{figure}
  \end{center}
  \begin{itemize}
    \item Bei steigender Frequenz steigt der Blindwiderstand $X_L$ und der Blindwiderstand $X_C$ sinkt
    \item Bei sinkender Frequenz hingegen sinkt $X_L$ und $X_C$ steigt
    \item Dadurch werden nur hohe Frequenzen durchgelassen
  \end{itemize}
\end{frame}


\section*{Referenzen}

\renewcommand{\refname}{Referenzen}

\hypertarget{refs}{}
\textcolor{white}{} \\ %\vspace{} geht nicht
\Large Referenzen/Links
\footnotesize

\begin{thebibliography}{}
  \bibitem{e07}   Moltrecht E 07: \\
    \url{https://www.darc.de/der-club/referate/ajw/lehrgang-te/e07/}
  \bibitem{wp}    Wikipedia DE: \\
    \url{http://de.wikipedia.org/wiki/Ohmsches_Gesetz}\\
    \url{http://de.wikipedia.org/wiki/Elektrische_Leistung}\\
    \url{http://de.wikipedia.org/wiki/Elektrische_Energie#Elektrische_Energie_in_einem_elektrischen_Feld}\\

\end{thebibliography}

% Hier könnte noch eine Kontaktfolie stehen

\end{document}

