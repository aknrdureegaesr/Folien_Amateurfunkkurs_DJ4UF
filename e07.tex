% Foliensatz: "AFu-Kurs nach DJ4UF" von DK0TU, Amateurfunkgruppe der TU Berlin
% Lizenz: CC BY-NC-SA 3.0 de (http://creativecommons.org/licenses/by-nc-sa/3.0/de/)
% Autoren: Martin Deutschmann

preamble.dk0tu.tex
\subtitle{Technik Klasse E 07: \\
          Schwingkreise \& Filter \\[2em]}
\date{Stand 20.11.2014}
 \begin{document}

\begin{frame}
    \titlepage
    \vfill
    \begin{center}
        \ccbyncsaeu\\
        {\tiny This work is licensed under the \em{Creative Commons Attribution-NonCommercial-ShareAlike 3.0 License}.}\\[0.5ex]
         \tiny Amateurfunkgruppe der Technische Universität Berlin (AfuTUB), DKØTU
         %\includegraphics[scale=0.5]{img/DK0TU_Logo.pdf}
    \end{center}
\end{frame}


\section*{Schwingungsvorgang}
\begin{frame}
\frametitle{Schwingungen, wo gibt es denn sowas?}
	\begin{center}
		\includegraphics[scale=1.4]{e07/Schwingkreise.png}\\
		Abb.1: Serien- \& Paralellschwingkreise
	\end{center}
\end{frame}

\begin{frame}
\frametitle{Aber warum schwingt das denn jetzt?}
\begin{center}
	\includegraphics[scale=0.4]{e07/Schwingkreis.png}\\
	Abb.2: Energie in einem LC-Schwingkreis \cite{wmde} \\
	\vspace{3mm}
	\begin{itemize}
		\item durch Verluste kommt es zur gedämpfte Schwingung\\
		\item animierte Darstellung (\url{http://en.wikipedia.org/wiki/File:Tuned_circuit_animation_3.gif})
	\end{itemize}
\end{center}
\end{frame}

\begin{frame}
\frametitle{Resonanzfrequenz}
\begin{center}
	\Huge{$ f_{res} = \frac{1}{2 \cdot \pi \cdot \sqrt{L \cdot C}} $}
\end{center}
\end{frame}

\section*{Reihenschwingkreis}
\begin{frame}
\frametitle{Reihenschwingkreis}
\begin{center}
	\begin{minipage}{0.4\textwidth}
	\includegraphics[scale=0.8]{e07/Serirenschw.png}\\
	\tiny{Abb.3: Serienschwingkreis \cite{wmen}}
	\end{minipage}
	\begin{minipage}{0.4\textwidth}
	\includegraphics[scale=0.2]{e07/SerirenschwSig.png}\\
	\tiny{Abb.4: Resonanzwiderstand \cite{wmen}} 
	\end{minipage}
\end{center}
\begin{itemize}
	\item Im Verlauf der Frequenzänderung ändert sich der Gesamtwellenwiderstand Z des Schwingkreises
	\item Der Schwingkreis hat als minimale Impedanz seinen ohmschen Wert, da sich bei der Resonanzfrequenz $f_R$ die induktiven und kapazitiven Anteile gegenseitig aufheben
\end{itemize}
\end{frame}

\section*{Parallelschwingkreis}
\begin{frame}
\frametitle{Parallelschwingkreis}
\begin{center}
	\begin{minipage}{0.4\textwidth}
	\includegraphics[scale=1]{e07/Parallelschw.png}\\
	\tiny{Abb.5: Parallelschwingkreis \cite{wmen}}
	\end{minipage}
	\begin{minipage}{0.4\textwidth}
	\includegraphics[scale=0.2]{e07/ParallelschwSig.png}\\
	\tiny{Abb.6: Resonanzwiderstand \cite{wmen}} 
	\end{minipage}
\end{center}
\begin{itemize}
	\item Der Parallelschwingkreis verhält sich genau entgegen gesetzt zum Reihenschwingkreis
	\item Dieser zeigt bei niedrigen und hohen Frequenzen das Verhalten eines Leiters
	\item Bei der Resonanzfrequenz hingegen steigt der Wellenwiderstand an, da hier nur noch der ohmsche Widerstand wirkt
\end{itemize}
\end{frame}

\section*{Bandreite}
\begin{frame}
\frametitle{Bandbreite}
\begin{center}
	\includegraphics[scale=0.4]{e07/Bandbreite.png}\\
	\tiny{Abb.7: Bandweite \cite{wmen}}
\end{center}
\end{frame}

\section*{Tiefpass}
\begin{frame}
\frametitle{what does the Tiefpass say?}
\begin{center}
	\includegraphics[scale=1.2]{e07/LC-Tiefpass.png}\\
	Abb.8: LC-Tiefpass
\end{center}
\begin{itemize}
	\item Bei steigender Frequenz sinkt der Blindwiderstand $X_L$ und der Blindwiderstand $X_C$ steigt
	\item Bei sinkender Frequenz hingegen steigt $X_L$ und $X_C$ sinkt
	\item Dadurch werden nur niedrige Frequenzen durchgelassen 
\end{itemize}
\end{frame}


\section*{Hochpass}
\begin{frame}
\frametitle{what does the Hochpass say?}
\begin{center}
	\includegraphics[scale=1.2]{e07/LC-Hochpass.png}\\
	Abb.9: LC-Hochpass
\end{center}
\begin{itemize}
	\item Bei steigender Frequenz steigt der Blindwiderstand $X_L$ und der Blindwiderstand $X_C$ sinkt
	\item Bei sinkender Frequenz hingegen sinkt $X_L$ und $X_C$ steigt
	\item Dadurch werden nur hohe Frequenzen durchgelassen 
\end{itemize}
\end{frame}

\renewcommand{\refname}{Referenzen}

\hypertarget{refs}{}
\textcolor{white}{} \\ %\vspace{} geht nicht
\Large Referenzen/Links
\footnotesize

\begin{thebibliography}{}
    \bibitem{e03}   Moltrecht E 07: \\
                    \url{http://www.darc.de/referate/ajw/ausbildung/darc-online-lehrgang/technik-klasse-e/technik-e07/}
    \bibitem{wp}    Wikipedia DE: \\
                    \url{http://de.wikipedia.org/wiki/Ohmsches_Gesetz}\\ 
                    \url{http://de.wikipedia.org/wiki/Elektrische_Leistung}\\ 
                    \url{http://de.wikipedia.org/wiki/lektrische_Energie#Elektrische_Energie_in_einem_elektrischen_Feld}\\ 
    \bibitem{wmde}	Wikimedia DE:\\
    				\url{http://commons.wikimedia.org/wiki/File:LC_circuit_4_times_new_version.svg?uselang=de}\\
   	\bibitem{wmen}	Wikimedia EN:\\
   					\url{http://en.wikipedia.org/wiki/File:Tuned_circuit_animation_3.gif}\\
   					\url{http://commons.wikimedia.org/wiki/File:RLC_series_circuit_v1.svg}\\
   					\url{http://commons.wikimedia.org/wiki/File:Resonanzwiderstand_serie.svg}\\
   					\url{http://commons.wikimedia.org/wiki/File:KondiSpuleWiderstandParallel.svg}\\
   					\url{http://commons.wikimedia.org/wiki/File:Resonanzwiderstand_parallel.svg}\\
   					\url{http://commons.wikimedia.org/wiki/File:Bandlimited_3dB_LP.svg}\\
   					
\end{thebibliography} 

% Hier könnte noch eine Kontaktfolie stehen

\end{document}

