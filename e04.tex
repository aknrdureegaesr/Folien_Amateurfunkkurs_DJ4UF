% Foliensatz: "AFu-Kurs nach DJ4UF" von DK0TU, Amateurfunkgruppe der TU Berlin
% Lizenz: CC BY-NC-SA 3.0 de (http://creativecommons.org/licenses/by-nc-sa/3.0/de/)
% Autoren: Felix Baum <baum@campus.tu-berlin.de>, Lars Weiler <dc4lw@darc.de>

preamble.dk0tu.tex
\subtitle{Technik Klasse E 04: \\
          Der Widerstand und seine Schaltungsarten \\[2em]}
\date{Stand 27.10.2016}
 \begin{document}

\begin{frame}
    \titlepage
    \vfill
    \begin{center}
        \ccbyncsaeu\\
        {\tiny This work is licensed under the \em{Creative Commons Attribution-NonCommercial-ShareAlike 3.0 License}.}\\[0.5ex]
         \tiny Amateurfunkgruppe der Technische Universität Berlin (AfuTUB), DKØTU
         %\includegraphics[scale=0.5]{img/DK0TU_Logo.pdf}
    \end{center}
\end{frame}


%fixme Referenzen/Fußnoten-Systematik vereinheitlichen

\section*{Einleitung}

\begin{frame}
    \frametitle{Einleitung / Widerstand}
    \begin{center}
        \Large{Was ist das?} \\
        \Large{Wie sieht er aus?}
    \end{center}
\end{frame}


\begin{frame}
    \frametitle{Einleitung / Widerstand}


  \begin{center}
    \begin{figure}
      \includegraphics[width=1\textwidth,height=.75\textheight,keepaspectratio]{e04/Widerstaende.jpg}
      \attribcaption{Widerstände}{Honina}{https://commons.wikimedia.org/wiki/File:Widerstände.JPG}{\ccbysa}
    \end{figure}
  \end{center}

 	

\end{frame}

\begin{frame}
    \frametitle{Einleitung / Widerstand}

    \begin{center}
           \begin{figure}
      \includegraphics[width=.3\textwidth,height=.3\textheight,keepaspectratio]{e04/Resistor_symbol_IEC.png}
      \attribcaption{Widerstandssymbol nach IEC}{Markus Kuhn}{https://commons.wikimedia.org/wiki/File:Resistor_symbol_IEC.svg}{\ccpd}
    \end{figure}
    \end{center}

    \begin{center}
         \begin{figure}
      \includegraphics[width=.4\textwidth,height=.4\textheight,keepaspectratio]{e04/R_LTspice.png}
      \caption{aus LTspice, Freeware zur Schaltungssimulation \ExternalLink~\url{http://www.linear.com/designtools/software/\#Spice}}
    \end{figure}
    \end{center}
 	
\end{frame}



\section*{Spezifischer Widerstand}

\begin{frame}
  \frametitle{Leitende Materialien}
  \begin{columns}
    \column{.6\textwidth}
    \begin{tabular}{lr}
      Material & Spezifischer Widerstand\footnotemark $\rho \text{ in } \frac{\Omega\cdot mm^2}{m}$ \\ \hline
      Silber & $1,587 \cdot 10^{-2}$ \\
      Kupfer & $1,721 \cdot 10^{-2}$ \\
      Gold & $2,214 \cdot 10^{-2}$ \\
      Aluminium & $2,65 \cdot 10^{-2}$ \\
      Zinn & $1,15 \cdot 10^{-1}$ \\
      Blei & $2,08 \cdot 10^{-1}$ \\
      Quecksilber & $9,412 \cdot 10^{-1}$ \\
      Germanium & \only<2>{$\leftarrow$ merken \hspace{2pc}} $4,6 \cdot 10^{5}$\\
      Porzellan & \only<2>{$\leftarrow$ \textbf{Isolator} \hspace{2pc}} $1 \cdot 10^{18}$ \\
    \end{tabular}

    \column{.35\textwidth}
    \only<3>{
    \begin{block}{Berechnung des Widerstands}
      $$R = \rho \cdot \frac{\ell}{A}$$
    \end{block}
    }
  \end{columns}
  \footnotetext[1]{\tiny \ExternalLink \url{http://de.wikipedia.org/wiki/Spezifischer_Widerstand}}
\end{frame}

\section*{Widerstandswerte}

\begin{frame}
	\begin{center}
        \begin{figure}
      \includegraphics[width=\textwidth,height=.9\textheight,keepaspectratio]{e04/4-Ringe.png}
      \attribcaption{Farbkodierung von Widerständen mit 4 Ringen}{Screenshot}{https://de.wikipedia.org/wiki/Widerstand_(Bauelement)}{\ccbysa}
    \end{figure}
\end{center}
\end{frame}

\begin{frame}
	\begin{center}
       \begin{figure}
      \includegraphics[width=\textwidth,height=.9\textheight,keepaspectratio]{e04/5-Ringe.png}
      \attribcaption{Farbkodierung von Widerständen mit 5 Ringen}{Screenshot}{https://de.wikipedia.org/wiki/Widerstand_(Bauelement)}{\ccbysa}
    \end{figure}
\end{center}
\end{frame}

\begin{frame}
    \frametitle{SMD Widerstände}
    \begin{center}
        \begin{figure}
      \includegraphics[width=\textwidth,height=.2\textheight,keepaspectratio]{e04/Rsistor_SMD.jpg}
      \attribcaption{SMD Widerstand}{Haragayato}{https://commons.wikimedia.org/wiki/File:Register3.jpg}{\ccbysa}
    \end{figure}
    \end{center}
    \begin{center}
    \begin{tabular}{l||l|l|l|l|l|l|l|l|l|l}\hline
        1.Ziffer & - & 1 &2 & 3 & 4 & 5 & 6 & 7 & 8 & 9 \\ \hline
        2.Ziffer & 0 & 1 &2 & 3 & 4 & 5 & 6 & 7 & 8 & 9 \\ \hline
        3.Ziffer & 0 & 1 &2 & 3 & 4 & 5 & 6 & 7 &  &  \\ \hline
    \end{tabular}
    \end{center}
    Zum Beispiel:
    \begin{center}
    \begin{tabular}{l||l}\hline
	Aufdruck & Widerstandswert \\ \hline
        470 & $47 \cdot 10^{0} \Omega$ \\ \hline
        223 & $22 \cdot 10^{3} \Omega$ \\ \hline
        4R7 & $4,7 \Omega$ \\ \hline
    \end{tabular}
    \end{center}

\end{frame}
\section*{Besondere Widerstandsarten}
\begin{frame}
    \frametitle{Drehpotentiometer}

    \begin{center}
      \begin{figure}
      \includegraphics[width=\textwidth,height=.75\textheight,keepaspectratio]{e04/Potenziometer.jpg}
      \attribcaption{Zwei Trimmpotentiometer und ein Schiebepotentiometer}{Honina}{https://commons.wikimedia.org/wiki/File:Potenziometer.JPG}{\ccbysa}
    \end{figure}
    \end{center}
 	
\end{frame}
\begin{frame}
    \frametitle{Besondere Widerstände}

    \begin{center}
      \begin{figure}
      \includegraphics[width=.6\textwidth,height=.5\textheight,keepaspectratio]{e04/bild-TC106.png}
      \caption{Technik Fragenkatalog Klasse E 2006-09 Frage TC106}
    \end{figure}
 	
 	\begin{tabular}{l||l}\hline
        A & NTC - Negativer Temperaturkoeffizient \\ \hline
        B & PTC - Positiver Temperaturkoeffizient \\ \hline
        C & Lichteinfallgesteuerter Widerstand (Lichtsensor) \\ \hline
        D & Spannungsgesteuerter Widerstand \\ \hline
    \end{tabular}
 	    \end{center}
\end{frame}

\section*{Rechnen mit Widerständen}

\begin{frame}
    \frametitle{Reihenschaltung}
        
    $$R_{gesamt} = R_1 + R_2 + R_3 + ...$$

	\begin{center}
        \begin{figure}
        \includegraphics[width=.4\textwidth,height=.75\textheight,keepaspectratio]{e04/Reihe.png}
        \caption{aus LTspice}
      \end{figure}
    \end{center}
\end{frame}

\begin{frame}
    \frametitle{Parallelschaltung}
        $$\frac{1}{R_{gesamt}} = \frac{1}{R_1} + \frac{1}{R_2} + \frac{1}{R_3} + ...$$
        
	\begin{center}
    \begin{figure}
        \includegraphics[width=\textwidth,height=.75\textheight,keepaspectratio]{e04/Parallel.png}
        \caption{aus LTspice}
      \end{figure}
    \end{center}
    

\end{frame}


\section*{Übung}

\begin{frame}

	\begin{center}
        Widerstandswerte aus Farbcode ablesen
    \end{center}
    
\end{frame}

\begin{frame}
  \frametitle{Übungen}
  \pause
  \begin{center}
    Einfühung Steckbrett
  \end{center}
\end{frame}

\subsection*{Übung 1}
\begin{frame}
  \begin{columns}
    \column{0.4\textwidth}
    \begin{center}
      \includegraphics[width=1\textwidth]{e04/Uebung1_Schaltplan.pdf}
    \end{center}
    \column{0.55\textwidth}
    \begin{alertblock}{Aufgabe 1}
      Baue die Schaltung auf dem Steckbrett auf.\\
      Berechne den Ersatzwiderstand. Miss zur Überprüfung den Gesamtwiderstand.
    \end{alertblock}
    \begin{alertblock}{Aufgabe 2}
      Berechne die Spannung über $R_1$, $R_2$, $R_3$ und $R_4$. Miss zur Überprüfung nach.\\
      Miss und erkläre die Spannungen über $A$---$B$, $A$---$C$ und $A$---$D$.
    \end{alertblock}
   \end{columns}
  \begin{alertblock}{Aufgabe 3}
    Berechne die Ströme durch $R_1$, $R_2$, $R_3$, $R_4$ und den Gesamtstrom. Miss zur Überprüfung nach.
  \end{alertblock}
\end{frame}

\subsection*{Übung 2}
\begin{frame}
  \begin{columns}
    \column{0.4\textwidth}
    \begin{center}
      \includegraphics[width=1\textwidth]{e04/Uebung2_Schaltplan.pdf}
    \end{center}
    \column{0.55\textwidth}
    \begin{alertblock}{Aufgabe 1}
      Baue die Schaltung auf dem Steckbrett auf.
      Berechne den Ersatzwiderstand. Miss zur Überprüfung den Gesamtwiderstand.
    \end{alertblock}
    \begin{alertblock}{Aufgabe 2}
      Berechne die Spannung über $R_1$, $R_2$ und $R_3$. Miss zur Überprüfung nach.\\
      Miss und erkläre die Spannungen über $A$---$B$, $A$---$C$ und $B$---$C$.
    \end{alertblock}
  \end{columns}
  \begin{alertblock}{Aufgabe 3}
    Berechne die Ströme durch $R_1$, $R_2$, $R_3$ und den Gesamtstrom. Miss zur Überprüfung nach.
  \end{alertblock}
\end{frame}

\subsection*{Übung 3}
\begin{frame}
  \begin{alertblock}{Aufgabe 1}
    Die rote Leuchtdiode benötigt einen Strom von $20mA$. Für die Aufgabe sei erstmal angenommen, dass keine Spannung über die Leuchtdiode abfällt. Dimensioniere einen Spannungsteiler so, dass mit der gegebenen Batterie von 6V an der Leuchtdiode eine Spannung von 3V bis 5V anliegt. Gegeben sind die Widerstände in der Größe $4\times100\Omega$, $2\times180\Omega$, $4\times470\Omega$ und $2\times510\Omega$. Berechne zuerst den Spannungsteiler und zeichne einen Schaltplan.
  \end{alertblock}
\end{frame}

\begin{frame}

	\begin{center}
	\begin{figure}
        \includegraphics[width=.5\textwidth]{e04/URI.png}
        \caption{\tiny TU Wien \url{http://www.fet.at/twiki/pub/Homepage/OnlineFetzn/fetzn_ausgabe_maerz-2013_online.pdf}}
	\end{figure}
    \end{center}
    
\end{frame}

\section*{Referenzen}

\begin{frame}
    \frametitle{Referenzen/Links}
    
    \footnotesize
    \begin{itemize}
        \item Moltrecht E 04: \\
              \url{http://www.darc.de/referate/ajw/ausbildung/darc-online-lehrgang/technik-klasse-e/technik-e04/}
    \end{itemize}

\end{frame}

% Hier könnte noch eine Kontaktfolie stehen

\end{document}

