% Foliensatz: "AFu-Kurs nach DJ4UF" von DK0TU, Amateurfunkgruppe der TU Berlin
% Lizenz: CC BY-NC-SA 3.0 de (http://creativecommons.org/licenses/by-nc-sa/3.0/de/)
% Autoren: Felix Baum <baum@campus.tu-berlin.de>
% Korrekturen: Lars Weiler <dc4lw@darc.de>

preamble.dk0tu.tex
\subtitle{Technik Klasse E 04: \\
  Der Widerstand und seine Schaltungsarten \\[2em]}
\date{Stand 18.09.2017}
 \begin{document}

\begin{frame}
    \titlepage
    \vfill
    \begin{center}
        \ccbyncsaeu\\
        {\tiny This work is licensed under the \em{Creative Commons Attribution-NonCommercial-ShareAlike 3.0 License}.}\\[0.5ex]
         \tiny Amateurfunkgruppe der Technische Universität Berlin (AfuTUB), DKØTU
         %\includegraphics[scale=0.5]{img/DK0TU_Logo.pdf}
    \end{center}
\end{frame}


\section*{Einleitung}
\subsection*{Widerstand}

\begin{frame}
  \frametitle{Widerstand}
  \begin{center}
    \Large{Was ist das?} \\
    \Large{Wie sieht er aus?}
  \end{center}
\end{frame}


\begin{frame}
  \frametitle{Bauelement}

  \begin{center}
    \begin{figure}
      \includegraphics[width=1\textwidth,height=.75\textheight,keepaspectratio]{e04/Widerstaende.jpg}
      \caption{Drahtwiderstände \cite{wdst}}
      \label{fig_wdst}
      %\attribcaption{Widerstände}{Honina}{https://commons.wikimedia.org/wiki/File:Widerstände.JPG}{\ccbysa}
    \end{figure}
  \end{center}


\end{frame}

\begin{frame}
  \frametitle{Schaltbild}

  \begin{center}
    \begin{figure}
      \includegraphics[width=.3\textwidth,height=.3\textheight,keepaspectratio]{e04/Resistor_symbol_IEC.png}
      \caption{Schaltsymbol nach IEC \cite{wdst_iec}}
      \label{fig_wdst_iec}
      %\attribcaption{Widerstandssymbol nach IEC}{Markus Kuhn}{https://commons.wikimedia.org/wiki/File:Resistor_symbol_IEC.svg}{\ccpd}
    \end{figure}
  \end{center}

  \begin{center}
    \begin{figure}
      \includegraphics[width=.4\textwidth,height=.4\textheight,keepaspectratio]{e04/R_LTspice.png}
      %\caption{aus LTspice, Freeware zur Schaltungssimulation %\ExternalLink~\url{http://www.linear.com/designtools/software/\#Spice}}
    \end{figure}
  \end{center}

\end{frame}


\section*{Spezifischer Widerstand}

\begin{frame}
  \frametitle{Leitende Materialien}
  \begin{columns}
    \column{.6\textwidth}
    \begin{tabular}{lr}
      Material & Spezifischer Widerstand\footnotemark $\rho$ \\
        & $\text{ in } \frac{\Omega\cdot mm^2}{m}$ \\ \hline
      Silber & $1,587 \cdot 10^{-2}$ \\
      Kupfer & $1,721 \cdot 10^{-2}$ \\
      Gold & $2,214 \cdot 10^{-2}$ \\
      Aluminium & $2,65 \cdot 10^{-2}$ \\
      Zinn & $1,15 \cdot 10^{-1}$ \\
      Blei & $2,08 \cdot 10^{-1}$ \\
      Quecksilber & $9,412 \cdot 10^{-1}$ \\
      Germanium & \only<2>{$\leftarrow$ merken \hspace{2pc}} $4,6 \cdot 10^{5}$\\
      Porzellan & \only<2>{$\leftarrow$ \textbf{Isolator} \hspace{2pc}} $1 \cdot 10^{18}$ \\
    \end{tabular}

    \column{.35\textwidth}
    \only<3>{
    \begin{block}{Berechnung des Widerstands}
      $$R = \rho \cdot \frac{\ell}{A}$$
    \end{block}
    }
  \end{columns}
  \footnotetext[1]{\tiny \ExternalLink \url{http://de.wikipedia.org/wiki/Spezifischer_Widerstand}}
\end{frame}

\section*{Wider\-stands\-werte}

\begin{frame}
  \begin{center}
    \begin{figure}
      \includegraphics[width=\textwidth,height=.9\textheight,keepaspectratio]{e04/4-Ringe.png}
      \caption{Farbcodierung von Widerständen mit 4 Ringen \cite{ringe4}}
      \label{fig_ringe4}
      %\attribcaption{Farbkodierung von Widerständen mit 4 Ringen}{Screenshot}{https://de.wikipedia.org/wiki/Widerstand_(Bauelement)}{\ccbysa}
    \end{figure}
  \end{center}
\end{frame}

\begin{frame}
  \begin{center}
    \begin{figure}
      \includegraphics[width=\textwidth,height=.9\textheight,keepaspectratio]{e04/5-Ringe.png}
      \caption{Farbcodierung von Widerständen mit 5 Ringen \cite{ringe5}}
      \label{fig_ringe5}
      %\attribcaption{Farbkodierung von Widerständen mit 5 Ringen}{Screenshot}{https://de.wikipedia.org/wiki/Widerstand_(Bauelement)}{\ccbysa}
    \end{figure}
  \end{center}
\end{frame}

\begin{frame}
  \frametitle{SMD Widerstände}
  \begin{center}
    \begin{figure}
      \includegraphics[width=\textwidth,height=.2\textheight,keepaspectratio]{e04/Rsistor_SMD.jpg}
      \caption{SMD-Widerstand \cite{smd}}
      \label{fig_smd}
      %\attribcaption{SMD Widerstand}{Haragayato}{https://commons.wikimedia.org/wiki/File:Register3.jpg}{\ccbysa}
    \end{figure}
  \end{center}

  \begin{center}
    \begin{tabular}{l||l|l|l|l|l|l|l|l|l|l}\hline
      1.Ziffer & - & 1 &2 & 3 & 4 & 5 & 6 & 7 & 8 & 9 \\ \hline
      2.Ziffer & 0 & 1 &2 & 3 & 4 & 5 & 6 & 7 & 8 & 9 \\ \hline
      3.Ziffer & 0 & 1 &2 & 3 & 4 & 5 & 6 & 7 &  &  \\ \hline
    \end{tabular}
  \end{center}
  \pause
  \begin{exampleblock}{Beispiel}
    \scriptsize
    \begin{center}
      \begin{tabular}{l|l}
        Aufdruck & Widerstandswert \\ \hline
        470 & $47 \cdot 10^{0} \Omega$ \\
        223 & $22 \cdot 10^{3} \Omega$ \\
        4R7 & $4,7 \Omega$ \\
      \end{tabular}
    \end{center}
  \end{exampleblock}

\end{frame}

\section*{Besondere Widerstandsarten}
\begin{frame}
  \frametitle{Potentiometer}

  \begin{center}
    \begin{figure}
      \includegraphics[width=\textwidth,height=.75\textheight,keepaspectratio]{e04/Potenziometer.jpg}
      \caption{Potentiometer \cite{potentiometer}}
      \label{fig_potentiometer}
      %\attribcaption{Zwei Trimmpotentiometer und ein Schiebepotentiometer}{Honina}{https://commons.wikimedia.org/wiki/File:Potenziometer.JPG}{\ccbysa}
    \end{figure}
  \end{center}

\end{frame}

\begin{frame}
  \frametitle{Besondere Widerstände}

  \begin{center}
    \begin{figure}
      \includegraphics[width=.6\textwidth,height=.5\textheight,keepaspectratio]{e04/bild-TC106.png}
      \caption{Technik Fragenkatalog Klasse E 2006-09 Frage TC106}
    \end{figure}

    \begin{tabular}{l||l}\hline
      A & NTC - Negativer Temperaturkoeffizient \\ \hline
      B & PTC - Positiver Temperaturkoeffizient \\ \hline
      C & Lichteinfallgesteuerter Widerstand (Lichtsensor) \\ \hline
      D & Spannungsgesteuerter Widerstand \\ \hline
    \end{tabular}
  \end{center}
\end{frame}


\section*{Schaltungen}

\begin{frame}
  \frametitle{Reihenschaltung}

  \begin{columns}
    \column{.4\textwidth}
    \begin{center}
      \begin{figure}
        \includegraphics[width=.4\textwidth,height=.75\textheight,keepaspectratio]{e04/Reihe.png}
        \caption{aus LTspice}
      \end{figure}
    \end{center}
    \pause
    \column{.55\textwidth}
    \begin{block}{Berechnung}
      $$R_{gesamt} = R_1 + R_2 + R_3 + ...$$
    \end{block}
  \end{columns}

\end{frame}

\begin{frame}
  \frametitle{Parallelschaltung}
  \begin{columns}
    \column{.4\textwidth}
    \begin{center}
      \begin{figure}
        \includegraphics[width=\textwidth,height=.75\textheight,keepaspectratio]{e04/Parallel.png}
        \caption{aus LTspice}
      \end{figure}
    \end{center}
    \pause
    \column{.55\textwidth}
    \begin{block}{Berechnung}
      $$\frac{1}{R_{gesamt}} = \frac{1}{R_1} + \frac{1}{R_2} + \frac{1}{R_3} + ...$$
    \end{block}
  \end{columns}
\end{frame}

\begin{frame}
  \frametitle{Ersatzwiderstand}
  \begin{columns}
    \column{.47\textwidth}
    \begin{figure}
      \includegraphics[width=.75\textwidth,height=.3\textheight,keepaspectratio]{e04/Ersatzwiderstand1.png}
      \caption{aus dem Fragenkatalog}
    \end{figure}
    \column{.47\textwidth}
    \begin{figure}
      \includegraphics[width=.75\textwidth,height=.3\textheight,keepaspectratio]{e04/Ersatzwiderstand2.png}
      \caption{aus dem Fragenkatalog}
    \end{figure}
  \end{columns}
  \begin{columns}
    \column{.47\textwidth}
    \pause
    \begin{exampleblock}{Berechnung}
      $R_1 + (R_2 \parallel R_3)$ \\[1.5em]
      $\Rightarrow R_1 + \cfrac{1}{\cfrac{1}{R_2} + \cfrac{1}{R_3}}$
    \end{exampleblock}
    \column{.47\textwidth}
    \pause
    \begin{exampleblock}{Berechnung}
      $(R_1 + R_2) \parallel R_3$\\[1.5em]
      $\Rightarrow \cfrac{1}{\cfrac{1}{R_1 + R_2} + \cfrac{1}{R_3}}$
    \end{exampleblock}
  \end{columns}
\end{frame}

\begin{frame}
  \frametitle{Spannungsteiler}
  \begin{columns}
    \column{.4\textwidth}
    \begin{center}
      \begin{figure}
        \includegraphics[width=.6\textwidth,height=.75\textheight,keepaspectratio]{e04/Spannungsteiler.png}
        \caption{aus dem Fragenkatalog}
      \end{figure}
    \end{center}
    \column{.55\textwidth}
    \begin{block}{Berechnung}
      $\cfrac{U}{R_1 + R_2} = \cfrac{U_2}{R_2} = \cfrac{U_1}{R_1}$
    \end{block}
  \end{columns}
\end{frame}

\begin{frame}
	\frametitle{Stromteiler}
	\begin{columns}
		\column{.4\textwidth}
		\begin{center}
			\begin{figure}
				\includegraphics[width=.6\textwidth,height=.75\textheight,keepaspectratio]{e04/Stromteiler.png}
				\caption{Stromteiler \cite{stromteiler}}
				\label{fig_stromteiler}
			\end{figure}
		\end{center}
		\column{.55\textwidth}
		\begin{block}{Berechnung}
			$\cfrac{I_1}{I_2} = \cfrac{R_2}{R_1}$
		\end{block}
	\end{columns}
\end{frame}

%\begin{frame}
%  \begin{alertblock}{Hausaufgabe}
%    Aus Fragenkatalog Klasse E Kapitel 1.3.1 ``Widerstand'' (TC101--TC111) durcharbeiten.\\
%    Aus Fragenkatalog Klasse E Kapitel TD101--TD104, TD108--TD110 berechnen.
%  \end{alertblock}
%\end{frame}

\section*{Referenzen}

\begin{frame}
  \frametitle{Referenzen/Links}

  \footnotesize
  \begin{itemize}
    \item Moltrecht E 04: \\
      \url{https://www.darc.de/der-club/referate/ajw/lehrgang-te/e04/}
      
    \bibitem{wdst}  Abbildung \ref{fig_wdst}: Drahtwiderstände\\
	    \url{https://commons.wikimedia.org/wiki/File:Widerstande.JPG}\\
	\bibitem{wdst_iec}  Abbildung \ref{fig_wdst_iec}: Schaltsymbol nach IEC\\
		\url{https://commons.wikimedia.org/wiki/File:Resistor\_symbol\_IEC.svg}\\
    \bibitem{ringe4}  Abbildung \ref{fig_ringe4}: Farbcodierungstafel\\
		\url{https://de.wikipedia.org/wiki/Widerstand\_(Bauelement)}\\
	\bibitem{ringe5}  Abbildung \ref{fig_ringe5}: Farbcodierungstafel\\
		\url{https://de.wikipedia.org/wiki/Widerstand\_(Bauelement)}\\
    \bibitem{smd}  Abbildung \ref{fig_smd}: SMD-Widerstand\\
		\url{https://commons.wikimedia.org/wiki/File:Register3.jpg}\\
	\bibitem{potentiometer}  Abbildung \ref{fig_potentiometer}: Potentiometer\\
		\url{https://commons.wikimedia.org/wiki/File:Potenziometer.JPG}\\
	\bibitem{stromteiler}  Abbildung \ref{fig_stromteiler}: Stromteiler\\
		\url{https://commons.wikimedia.org/wiki/File:Stromteiler.svg}\\

  \end{itemize}

\end{frame}

% Hier könnte noch eine Kontaktfolie stehen

\end{document}

