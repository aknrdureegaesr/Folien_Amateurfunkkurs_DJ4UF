% Foliensatz: "AFu-Kurs nach DJ4UF" von DK0TU, Amateurfunkgruppe der TU Berlin
% Lizenz: CC BY-NC-SA 3.0 de (http://creativecommons.org/licenses/by-nc-sa/3.0/de/)
% Autoren: Felix Baum <baum@campus.tu-berlin.de>, Lars Weiler <dc4lw@darc.de>

preamble.dk0tu.tex
\subtitle{Technik Klasse E 04: \\
          Der Widerstand und seine Schaltungsarten \\[2em]}
\date{Stand 27.10.2014}
 \begin{document}

\begin{frame}
    \titlepage
    \vfill
    \begin{center}
        \ccbyncsaeu\\
        {\tiny This work is licensed under the \em{Creative Commons Attribution-NonCommercial-ShareAlike 3.0 License}.}\\[0.5ex]
         \tiny Amateurfunkgruppe der Technische Universität Berlin (AfuTUB), DKØTU
         %\includegraphics[scale=0.5]{img/DK0TU_Logo.pdf}
    \end{center}
\end{frame}


%fixme Referenzen/Fußnoten-Systematik vereinheitlichen

\section*{Einleitung}

\begin{frame}
    \frametitle{Einleitung / Widerstand}
    \begin{center}
        \Large{Was ist das?} \\
        \Large{Wie sieht er aus?}
    \end{center}
\end{frame}


\begin{frame}
    \frametitle{Einleitung / Widerstand}

    \begin{center}
        \includegraphics[width=0.8\textwidth]{e04/Widerstaende.jpg}
        \footnote{\tiny By Honina at de.wikipedia. Later version(s) were uploaded by Montauk at de.wikipedia.}
    \end{center}
 	

\end{frame}

\begin{frame}
    \frametitle{Einleitung / Widerstand}

    \begin{center}
        \includegraphics[width=.4\textwidth]{e04/Resistor_symbol_IEC.png}
        \footnote{\tiny By Markus Kuhn (Made in Inkscape from scratch) [Public domain], via Wikimedia Commons}
    \end{center}

    \begin{center}
        \includegraphics[width=.4\textwidth]{e04/R_LTspice.png}
        \footnote{\tiny aus LTspice, Freeware zur Schaltungssimulation \url{http://www.linear.com/designtools/software/\#Spice}}
    \end{center}
 	
\end{frame}

\section*{Besondere Widerstandsarten}
\begin{frame}
    \frametitle{Drehpotentiometer}

    \begin{center}
        \includegraphics[width=.5\textwidth]{e04/Potenziometer.jpg}
        \footnote{\tiny „Potenziometer“ von Honina.Original uploader was Honina at de.wikipedia.Later version(s) were uploaded by Ulfbastel at de.wikipedia. Lizenziert unter Creative Commons Attribution-Share Alike 3.0 über Wikimedia Commons}
    \end{center}
 	
\end{frame}

\section*{Spezifischer Widerstand}

\begin{frame}
    \frametitle{Leitende Materialien}

     \begin{tabular}{lr}
  	Material & Spezifischer Widerstand\footnote{\tiny \url{de.wikipedia.org/wiki/Spezifischer_Widerstand}} \\ \hline
  	Silber & $1,587 \cdot 10^{-2}$ \\
  	Kupfer & $1,721 \cdot 10^{-2}$ \\
  	Gold & $2,214 \cdot 10^{-2}$ \\
  	Aluminium & $2,65 \cdot 10^{-2}$ \\
  	Zinn & $1.15 \cdot 10^{-1}$ \\
  	Blei & $2,08 \cdot 10^{-1}$ \\
  	Quecksilber & $9.412 \cdot 10^{-1}$ \\
  	Germanium & $4.6 \cdot 10^{5}$\\
  	Porzellan & $1 \cdot 10^{18}$ \\
 	\end{tabular}
 	
 	$$R = \rho \cdot \frac{l}{A}$$
 	

\end{frame}

\section*{Widerstandswerte Ablesen}

\begin{frame}
	\begin{center}
        \includegraphics[width=.74\textwidth]{e04/4-Ringe.png}
        \footnote{\tiny By Wikipedia CC-BY-SA-3.0\url{https://de.wikipedia.org/wiki/Widerstand_(Bauelement)}]}
\end{center}
\end{frame}

\begin{frame}
	\begin{center}
        \includegraphics[width=1\textwidth]{e04/5-Ringe.png}
        \footnote{\tiny By Wikipedia CC-BY-SA-3.0\url{https://de.wikipedia.org/wiki/Widerstand_(Bauelement)}]}
\end{center}
\end{frame}

\begin{frame}
    \frametitle{SMD Widerstände}
    \begin{center}
        \includegraphics[width=.3\textwidth]{e04/Rsistor_SMD.jpg}
        \footnote{\tiny By Haragayato CC-BY-SA-3.0 (\url{http://creativecommons.org/licenses/by-sa/3.0/)}], via Wikimedia Commons}
    \end{center}
    
    \begin{center}
    \begin{tabular}{l||l|l|l|l|l|l|l|l|l|l}\hline
        1.Ziffer & - & 1 &2 & 3 & 4 & 5 & 6 & 7 & 8 & 9 \\ \hline
        2.Ziffer & 0 & 1 &2 & 3 & 4 & 5 & 6 & 7 & 8 & 9 \\ \hline
        3.Ziffer & 0 & 1 &2 & 3 & 4 & 5 & 6 & 7 &  &  \\ \hline
    \end{tabular}
    \end{center}
    Zum Beispiel:
    \begin{center}
    \begin{tabular}{l||l}\hline
        470 & $47 \cdot 10^{0} \Omega$ \\ \hline
        223 & $22 \cdot 10^{3} \Omega$ \\ \hline
        4R7 & $4,7 \Omega$ \\ \hline
    \end{tabular}
    \end{center}

\end{frame}

\begin{frame}
    \frametitle{Besondere Widerstände}

    \begin{center}
        \includegraphics[width=.7\textwidth]{e04/bild-TC106.png}
        \footnote{\tiny Technik Fragenkatalog KlasseE 2006-09 Frage TC106}
 	
 	\begin{tabular}{l||l}\hline
        A & NTC - Negativer Temperaturkoeffizient \\ \hline
        B & PTC - Positiver Temperaturkoeffizient \\ \hline
        C & Lichteinfallgesteuerter Widerstand (Lichtsensor) \\ \hline
        D & Spannungsgesteuerter Widerstand \\ \hline
    \end{tabular}
 	    \end{center}
\end{frame}

\section*{Rechnen mit Widerständen}

\begin{frame}
    \frametitle{Reihenschaltung}
        
    $$R_{gesamt} = R_1 + R_2 + R_3 + ...$$

	\begin{center}
        \includegraphics[width=.2\textwidth]{e04/Reihe.png}
        \footnote{\tiny aus LTspice}
    \end{center}
\end{frame}

\begin{frame}
    \frametitle{Parallelschaltung}
        $$\frac{1}{R_{gesamt}} = \frac{1}{R_1} + \frac{1}{R_2} + \frac{1}{R_3} + ...$$
        
	\begin{center}
        \includegraphics[width=.5\textwidth]{e04/Parallel.png}
        \footnote{\tiny aus LTspice}
    \end{center}
    

\end{frame}


\section*{Übung}

\begin{frame}

	\begin{center}
        Widerstandswerte aus Farbcode ablesen
    \end{center}
    
\end{frame}

\begin{frame}
  \frametitle{Übungen}
  \pause
  \begin{center}
    Einfühung Steckbrett
  \end{center}
\end{frame}

\subsection*{Übung 1}
\begin{frame}
  \begin{columns}
    \column{0.4\textwidth}
    \begin{center}
      \includegraphics[width=1\textwidth]{e04/Uebung1_Schaltplan.pdf}
    \end{center}
    \column{0.55\textwidth}
    \begin{alertblock}{Aufgabe 1}
      Baue die Schaltung auf dem Steckbrett auf.\\
      Berechne den Ersatzwiderstand. Miss zur Überprüfung den Gesamtwiderstand.
    \end{alertblock}
    \begin{alertblock}{Aufgabe 2}
      Berechne die Spannung über $R_1$, $R_2$, $R_3$ und $R_4$. Miss zur Überprüfung nach.\\
      Miss und erkläre die Spannungen über $A$---$B$, $A$---$C$ und $A$---$D$.
    \end{alertblock}
   \end{columns}
  \begin{alertblock}{Aufgabe 3}
    Berechne die Ströme durch $R_1$, $R_2$, $R_3$, $R_4$ und den Gesamtstrom. Miss zur Überprüfung nach.
  \end{alertblock}
\end{frame}

\subsection*{Übung 2}
\begin{frame}
  \begin{columns}
    \column{0.4\textwidth}
    \begin{center}
      \includegraphics[width=1\textwidth]{e04/Uebung2_Schaltplan.pdf}
    \end{center}
    \column{0.55\textwidth}
    \begin{alertblock}{Aufgabe 1}
      Baue die Schaltung auf dem Steckbrett auf.
      Berechne den Ersatzwiderstand. Miss zur Überprüfung den Gesamtwiderstand.
    \end{alertblock}
    \begin{alertblock}{Aufgabe 2}
      Berechne die Spannung über $R_1$, $R_2$ und $R_3$. Miss zur Überprüfung nach.\\
      Miss und erkläre die Spannungen über $A$---$B$, $A$---$C$ und $B$---$C$.
    \end{alertblock}
  \end{columns}
  \begin{alertblock}{Aufgabe 3}
    Berechne die Ströme durch $R_1$, $R_2$, $R_3$ und den Gesamtstrom. Miss zur Überprüfung nach.
  \end{alertblock}
\end{frame}

\subsection*{Übung 3}
\begin{frame}
  \begin{alertblock}{Aufgabe 1}
    Die rote Leuchtdiode benötigt einen Strom von $20mA$. Für die Aufgabe sei erstmal angenommen, dass keine Spannung über die Leuchtdiode abfällt. Dimensioniere einen Spannungsteiler so, dass mit der gegebenen Batterie von 6V an der Leuchtdiode eine Spannung von 3V bis 5V anliegt. Gegeben sind die Widerstände in der Größe $4\times100\Omega$, $2\times180\Omega$, $4\times470\Omega$ und $2\times510\Omega$. Berechne zuerst den Spannungsteiler und zeichne einen Schaltplan.
  \end{alertblock}
\end{frame}

\begin{frame}

	\begin{center}
        \includegraphics[width=.8\textwidth]{e04/URI.png}
        \footnote{\tiny TU Wien \url{http://www.fet.at/twiki/pub/Homepage/OnlineFetzn/fetzn_ausgabe_maerz-2013_online.pdf}}
    \end{center}
    
\end{frame}

\section*{Referenzen}

\begin{frame}
    \frametitle{Referenzen/Links}
    
    \footnotesize
    \begin{itemize}
        \item Moltrecht E 04: \\
              \url{http://www.darc.de/referate/ajw/ausbildung/darc-online-lehrgang/technik-klasse-e/technik-e04/}
    \end{itemize}

\end{frame}

% Hier könnte noch eine Kontaktfolie stehen

\end{document}

