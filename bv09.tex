% Foliensatz: "AFu-Kurs nach DJ4UF" von DK0TU, Amateurfunkgruppe der TU Berlin
% Lizenz: CC BY-NC-SA 3.0 de (http://creativecommons.org/licenses/by-nc-sa/3.0/de/)
% Autoren: Sebastian Lange <dl7bst@dk0tu.de>

preamble.dk0tu.tex
\subtitle{Betriebstechnik/Vorschriften 09: \\
          Betriebsarten, Sendearten, Frequenzen \\[2em]}
\date{Stand 11.12.2014}
 \begin{document}

\begin{frame}
    \titlepage
    \vfill
    \begin{center}
        \ccbyncsaeu\\
        {\tiny This work is licensed under the \em{Creative Commons Attribution-NonCommercial-ShareAlike 3.0 License}.}\\[0.5ex]
         \tiny Amateurfunkgruppe der Technische Universität Berlin (AfuTUB), DKØTU
         %\includegraphics[scale=0.5]{img/DK0TU_Logo.pdf}
    \end{center}
\end{frame}


%todo Struktur nach Moltrecht, aber sehr komisch
%     Besser Top-Down: ITU ... AFuV ... HAM Conventions

\section{Einleitung}

\begin{frame}
    \frametitle{Einleitung: International}

    Internationaler Frequenzbereichszuweisungsplan: \emph{ITU
    \footnote{International Telecommunication Union. Gesetzesgrundlagen siehe
    Lektion \texttt{BV06}: ITU, CEPT, AFuV...} Frequency Allocation Table}

    \begin{itemize}
        \item enthält die Frequenzbereichszuweisungen aller Funkdienste in
              verschiedenen Funkregionen der Erde
              \footnote{Regionen siehe Lektion \texttt{BV06}}
    \end{itemize}

\end{frame}

\begin{frame}
    \frametitle{Einleitung: National}

    National geltendes Recht (in DL \texttt{AFuV}):

    \begin{itemize}
        \item Anlage 1 der \texttt{AFuV} regelt Nutzungsbedingungen und Frequenzbereiche
        \item welche Bereiche von wem wie genutzt? \\
              Frequenzen, Bandbreiten, Leistung, ...
        \item folglich: Nicht alle Bereiche der RR (VO Funk) dürfen automatisch
              verwendet werden
        \item Bandgrenzen HF bis 13cm muss man lernen %todo Tabelle
              \footnote{bei vielen Prüfungsfragen hilft einfache Fausformel
                        $MHz = \frac{300}{Meter}$}
    \end{itemize}

\end{frame}

\begin{frame}
    \frametitle{Einleitung: Vereinbarungen der HAMs}

    International:

    \begin{itemize}
        \item IARU \footnote{International Amateur Radio Union} legt
              Nutzungs\emph{empfehlungen} der Funkamateure untereinander fest
    \end{itemize}

    Lokale Absprachen, z.B.:

    \begin{itemize}
        \item Club QRGs
        \item Lokalrunden
        \item Rundsprüche
        \item ...
    \end{itemize}

\end{frame}

\section[Übertragungstechn.]{Übertragungstechnik}

\begin{frame}
    \frametitle{Übertragungstechnik}
        
    \begin{center}
        Modulationsarten, Betriebsarten $\Rightarrow$ Sendearten
    \end{center}

\end{frame}

\begin{frame}
    \frametitle{Modulationsarten}

    Radio Regulations verwendet offizielle Codierung von Aussendungen als
    neunstelligen Code \texttt{BBBBMSIDX}: \\[2em]

    \begin{itemize}
        \item BBBB: vier Stellen Bandbreite
        \item MSI: drei Stellen Sendeart (Modulationsart, Signalart,
                   Informationsgehalt)
        \item DX: ggf. zwei weitere Stellen für Signaleinzelheiten
                  (Detaillierung, Multiplexverfahren)
    \end{itemize}

\end{frame}

\begin{frame}
    \frametitle{Sendearten: Amateurfunk}

    Amateurfunk: drei Kennzeichen für die Modulationsart ausreichend

    %todo enumerate color broken
    \begin{enumerate}
        \item Modulationsart
        \item Signalart, die den Hauptträger moduliert
        \item Art der Informationen
    \end{enumerate}

    %todo Tabellen http://de.wikipedia.org/wiki/Modulationsart

\end{frame}

\begin{frame}
    \frametitle{Sendearten: Schlüsselzeichen}

    \begin{itemize}
        \item Modulationsart des Hauptträgers
    \end{itemize}

    %\begin{center}
    \begin{tabular}{|l|l|}\hline
        A & Amplitudenmodulation \\ \hline
        J & SSB (AM, Seitenband mit unterdrücktem Träger) \\ \hline
        F & Winkelmodulation, Frequenzmodulation \\ \hline
    \end{tabular}
    %\end{center}

    \begin{itemize}
        \item Signalart
    \end{itemize}
    
    %\begin{center}
    \begin{tabular}{|l|l|}\hline
        1 & Einkanaliges quantisiertes Signal ohne Hilfsträger \\ \hline
        2 & Einkanaliges quantisiertes Signal mit einem Hilfsträger \\ \hline
        3 & Einkanaliges Analogsignal \\ \hline
    \end{tabular}
    %\end{center}

    \begin{itemize}
        \item Informationsart
    \end{itemize}
 
    %\begin{center}
    \begin{tabular}{|l|l|}\hline
        A & Morsetelegrafie \\ \hline
        B & Telegrafie für maschinellen Empfang (Teletype) \\ \hline
        C & Fax \\ \hline
        D & Daten, Telemetrie, Fernsteuerung \\ \hline
        E & Telefonie, Rundfunk \\ \hline
        F & Fernsehsignal \\ \hline
    \end{tabular}
    %\end{center}
   
\end{frame}

\begin{frame}
    \frametitle{Sendearten: Beispiele}

    \begin{itemize}
        \item A1A: CW
        \item F2A: CW via FM-Hilfsträger
        \item F3E: FM
        \item J3E: SSB
        \item J2B: RTTY
        \item J2B: Pactor
    \end{itemize}

\end{frame}

\begin{frame}
    \frametitle{Bandbreiten}

    %todo Tabelle hier auch?
    \begin{itemize}
        \item A1A: 0,1... 0,5 kHz 
        \item F1B: 0,5 kHz 
        \item A3E: 5 kHz 
        \item F3E: 12 kHz 
        \item J3E: 2,4 kHz (2,7 kHz)
    \end{itemize}

    Weitere Einschränkungen, z.B. 160m-Band kein FM.

\end{frame}


\section{IARU-Bandplan (Kurzwelle)}

\begin{frame}
    \frametitle{IARU-Bandplan (Kurzwelle)}

    % todo Bandplan vorher drucken
    
    \begin{itemize}
        \item Bei uns Region 1 - remember?
        \item Aufteilung: Vereinbarung der HAMs untereinander mit ''Charakter
              einer Empfehlung''
        \item CW überall erlaubt, aber am Bandanfang exklusiv
        \item auch hier: Frequenzen kann man sich etwas herleiten, aber im
              Prinzip muss man Lernen, sorry
        \item Ansonsten: Lösung ist immer ''Ich schaue im aktuellen HF-Bandplan
              der IARU nach ...'' ;-)
    \end{itemize}

\end{frame}

% todo Fiese Frage?
%  Aus welchem Grund sollten Sie in der Dunkelheit und im Winter auch tagsüber im
%  Bereich von 3500-3510 kHz keine innerdeutschen oder innereuropäischen
%  Telegrafie-QSOs durchführen?
%    * Im IARU-Region-1-Kurzwellenbandplan ist dieser Bereich als "CW DX"
%      ausgewiesen und sollte für interkontinentale Verbindungen freigehalten
%      werden. 

\subsection{Aktivitätszentren}

\begin{frame}
    \frametitle{Aktivitätszentren}

    \begin{itemize}[<+->]
        \item QRP - wozu?
        \item anbei: Verwendung von \texttt{/qrp} nicht offiziell, aber geduldet
        \item Bildübertragung
        \item Notfunk
        \item Frequenzbereichen bevorzugt Funkverkehr in digitalen Betriebsarten 
    \end{itemize}

\end{frame}

\subsection{Seitenbänder}

\begin{frame}
    \frametitle{Seitenbänder}

    $<10 MHz$ LSB \\[1em]
    $>10 MHz$ USB \\[3em]

    Nicht mit deutschen Abkürzungen (USB, OSB) verwechseln!

\end{frame}

\subsection{Betriebstechniken}

\begin{frame}
    \frametitle{Betriebstechniken}

    \begin{itemize}
        \item QRGs werden ''occupied'' - man fragt vorher, ob sie frei sind
        \item für höhere Bänder gibt es Anruffrequenzen, weil diese groß sind und oft mit
              Richtantennen gearbeitet wird
        \begin{itemize}
            \item nach Kontaktaufnahme sofort QSY
        \end{itemize}
        \item Satellitenfunkbetrieb
        \begin{itemize}
            \item Uplink/Downlink-Bereiche
            \item auch mit HFGs frei halten!
            \item auch hier: Bereiche leider Lernstoff
        \end{itemize}
    \end{itemize}

\end{frame}

\section[FRQ-Nutzungsplan]{Frequenznutzungsplan}

\begin{frame}
    \frametitle{Frequenznutzungsplan}

    in Anlage 1 der \texttt{AFuV}\cite{afuv}: \\[1em]

    %todo Frequenznutzungsplan ggf. in die Folien mit einbinden

    Ausführlichen Nutzungsbedingungen und die ausgewiesenen Frequenzbereiche für
    den Amateurfunkdienst.

    \begin{itemize}
        \item maximal zulässigen Sender- bzw. Strahlungsleistungen
        \item erlaubte Frequenzbereiche
        \item aufgegliedert nach Zeugnisklassen
        \item Einzelheiten über Aufteilung und Nutzung im
              Frequenznutzungsplan und im Frequenzbereichszuweisungsplan
    \end{itemize}
   
\end{frame}

\begin{frame}
    \frametitle{Frequenznutzungsplan: Lesehilfen}

    \begin{itemize}[<+->]
        \item Status: Der (S)ekundärfunkdienst hat im Störungsfall gegenüber einem
              (P)rimärfunkdienst eingeschränkte Nutzungsrechte.
        \item ISM: Dieser Frequenzbereich wird für industrielle,
              wissenschaftliche, medizinische, häusliche oder ähnliche Anwendungen mitbenutzt.
        \begin{itemize}
            \item Welche ISM-Dienste kennt ihr?
        \end{itemize}
        \item richtige Anfangs- und Endfrequenzen und Bandbreiten muss man
              leider teilweise lernen
    \end{itemize}

\end{frame}

\begin{frame}
    \frametitle{Frequenznutzungsplan: Fragen}

    ''Funfacts'':
    
    \begin{itemize}
        %\item Bei 135,7-137,8 kHz und 1810-1850 kHz beträgt die zulässige maximale
        %      Bandbreite nur 800 Hz
        \item CB-Funkverkehr darf nur mit speziell für diesen Frequenzbereich
              hergestellten Geräten durchgeführt werden, für die eine
              Konformitätsbewertung oder Zulassung vorliegt.
        \item Betriebsfunk: Nein. Außerhalb des Amateurfunks dürfen nur zugelassene Geräte oder
              konformitätsbewertete Geräte benutzt werden.
    \end{itemize}

\end{frame}

\section{Lernhinweise}

\begin{frame}
    \frametitle{Lernhinweise}

    Die Frequenzbereiche gehören leider mit zu dem stupidesten Lernstoff für die
    Prüfung. Tips:

    \begin{itemize}
        \item Don't Panic! ;-)
        \item bei der Prüfungssimulation am Anfang die Frequenzfragen mit Hilfe
              der Bandpläne beantworten und sacken lassen
        \item viele Fragen sind auch über die ''Faustformel'' zu lösen
        \item Antworten mit ''ich schau in Regelwerken nach'' sind immer gut
    \end{itemize}

\end{frame}

\renewcommand{\refname}{Referenzen}

\begin{frame}
    \frametitle{Referenzen/Links}
    \hypertarget{refs}{}
    \footnotesize

    \begin{thebibliography}{}
        \bibitem{dj4uf} Moltrecht B/V 09: \\
                        \url{http://www.amateurfunkpruefung.de/lehrg/bv09/bv09.html}
        \bibitem{wp}    Wikipedia DE: \\
                        \url{http://de.wikipedia.org/wiki/Frequenzzuteilung} \\
                        \url{http://de.wikipedia.org/wiki/Amateurfunkband} \\
                        \url{http://de.wikipedia.org/wiki/Betriebsart_\%28Amateurfunk\%29} \\
                        \url{http://de.wikipedia.org/wiki/Modulationsart}
        \bibitem{frqs}  Frequenzplan der BRD von der BNetzA: \\
                        \url{http://www.bundesnetzagentur.de/DE/Sachgebiete/Telekommunikation/Unternehmen_Institutionen/Frequenzen/Grundlagen/Frequenzplan/frequenzplan-node.html}
        \bibitem{darc}  Kurzwellenbandpläne der IARU Region 1: \\
                        \url{http://www.darc.de/referate/hf/bandplaene/}
        \bibitem{afuv}  Verordnung zum Gesetz über den Amateurfunk: \\
                        \url{http://www.gesetze-im-internet.de/afuv_2005/}
    \end{thebibliography} 
   
\end{frame}

% Hier könnte noch eine Kontaktfolie stehen

\end{document}

