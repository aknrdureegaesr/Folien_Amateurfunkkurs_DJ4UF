% Foliensatz: "AFu-Kurs nach DJ4UF" von DK0TU, Amateurfunkgruppe der TU Berlin
% Lizenz: CC BY-NC-SA 3.0 de (http://creativecommons.org/licenses/by-nc-sa/3.0/de/)
% Autoren: Sebastian Lange <dl7bst@dk0tu.de>

preamble.dk0tu.tex
\subtitle{Technik Klasse A 11: \\
          Signale \\[2em]}
\date{Stand 11.06.2015}
 \begin{document}

\begin{frame}
    \titlepage
    \vfill
    \begin{center}
        \ccbyncsaeu\\
        {\tiny This work is licensed under the \em{Creative Commons Attribution-NonCommercial-ShareAlike 3.0 License}.}\\[0.5ex]
         \tiny Amateurfunkgruppe der Technische Universität Berlin (AfuTUB), DKØTU
         %\includegraphics[scale=0.5]{img/DK0TU_Logo.pdf}
    \end{center}
\end{frame}


\section{Einleitung}

\begin{frame}
    \frametitle{Einleitung}

    Was sind Signale?

\end{frame}

\begin{frame}
    \frametitle{Einleitung}

    Allgemein: \\[2em]

    Signale sind Zeichen. Ist ihre Bedeutung festgelegt können Nachrichten
    transportiert werden. \\[2em]

    Dazu ein wenig Theorie für den Bereich der elektrischen Nachrichtenübertragung...

\end{frame}

\section{Periodische Signale}

\begin{frame}
    \frametitle{Periodische Signale}

    Nachrichtentechnik: \\[2em]
    
    Grundsätzliche Unterscheidung zwischen
    \textbf{periodischen} und \textbf{nicht-periodischen} Signalen. \\[2em]

    Bekannte periodische sind z.B. Sin/Cos, Rechteck, Dreieck, Sägezahn.

\end{frame}

\begin{frame}
    \frametitle{Periodische Signale}

    \begin{center}
        \includegraphics[width=1\textwidth]{a11/Waveforms_de.png}
        \tiny \hyperlink{refs}{\cite{wc}}
    \end{center}

\end{frame}

\subsection{Sinus}

\begin{frame}
    \frametitle{Sinus}

    Fundamental: Sinus bzw. Cosinus.

    \begin{center}
        \includegraphics[width=1\textwidth]{a11/Sine_cosine_one_period.png}
        \tiny \hyperlink{refs}{\cite{wc}}
    \end{center}

    Was sind Amplitude, Scheitelwert, Spitze-Spitze-Wert, Periodendauer, Frequenz?

\end{frame}

\begin{frame}
    \frametitle{Sinus}

    \begin{block}{Wie hoch sind die Effektivwerte von sin/cos?}
        \only<2>{$\frac{1}{\sqrt{2}} \approx 0,7$}
    \end{block}

\end{frame}

\begin{frame}
    \frametitle{Sinus}

    \begin{block}{(TB608) Der Spitzenwert der häuslichen 230-V-Stromversorgung beträgt?}
        \only<2>{$230V \cdot \sqrt{2} \approx 325V$}
    \end{block}

    \only<2>{\vspace{2em}Bei Leistungsberechnungen mit Wechselströmen daran denken!}

\end{frame}

\begin{frame}
    \frametitle{Sinus}

    \begin{block}{Welche Periodendauer hat europäische Netzspannung?}
        \only<2>{$T = \frac{1}{f} = \frac{1}{50 Hz} \approx 20ms$}
    \end{block}

\end{frame}

\subsubsection{Oszi (Wdh)}
 
\begin{frame}
    \frametitle{Kurze Wiederholung Messtechnik: Oszilloskop}

    % TODO auf entsprechenden Foliensatz referenzieren

    \begin{block}{
        Wie hoch sind Spitze-Spitze-Spannung und Frequenz?
        \begin{center}
            \includegraphics[width=0.5\textwidth]{a11/TB605.png}
            \tiny (TB606)
        \end{center}
    }
        \only<2>{
            $U_{SS} = 40 V$ \\
            $f = \frac{1}{4 \cdot 0,03 \cdot 10^{-6} s} \approx 8,33 \cdot 10^6 Hz = 8,33 MHz$
        }
    \end{block}

\end{frame}
   
\subsubsection{Phasenwinkel}

\begin{frame}
    \frametitle{Zeigerdarstellung}

    \begin{center}
        \includegraphics[width=0.5\textwidth]{a11/Sine_Cosine_Exponential_qtl1.png}
        \tiny \hyperlink{refs}{\cite{wc}}\\
        \normalsize Trigonometrische Zusammenhänge im Einheitskreis.
    \end{center}

    Animierte Darstellungen:
    \begin{itemize}
        \item Wikimedia [\href{http://commons.wikimedia.org/wiki/File:Einheitskreis_mit_Sinus_und_Kosinusfunktion.gif}{Link}]
        \item H. Krauß [\href{http://www.hutschdorf.de/flash/sinus.htm}{Link}]
    \end{itemize}

\end{frame}

\begin{frame}
    \frametitle{Phasenwinkel}

    \begin{block}{
        Wie groß ist folgende Phasenverschiebung?
        \begin{center}
            \includegraphics[width=0.5\textwidth]{a11/TB612.png}
            \tiny (TB61)
        \end{center}
    }
        \only<2>{$\Delta\varphi = \frac{\pi}{4} = 45 ^{\circ}$}
    \end{block}


\end{frame}

\subsection{Weitere Signale}

\begin{frame}
    \frametitle{Ableitung weiterer Signale}

    Alle möglichen period. Signale lassen sich aus Überlagerung von
    Sinussignalen ableiten:

    \begin{itemize}
        \item \textbf{Grundfrequenz}
        \item beliebige \textbf{ganzzahlige Vielfache} der Grundfrequenz
    \end{itemize}

\end{frame}

\begin{frame}
\frametitle{Oberwellen}

    \begin{block}{Welche Frequenz hat die erste Harmonische von $430.200 MHz$?}
        \only<2-3>{Die erste Harmonische ist die Frequenz selbst. \\
                 Die erste Oberwelle bzw. zweite Harmonische liegt bei $860.40 MHz$}
    \end{block}

    \only<3>{\vspace{3em}\Large Merke: Harmonische = Oberwelle + 1}

\end{frame}

\begin{frame}
    \frametitle{Sägezahn}

    Grundwelle + alle Harmonischen reduzierter Amplitude:

    % TODO Grafik und Formel?

    \begin{center}
        \includegraphics[width=0.5\textwidth]{a11/TB705.png}
        \tiny (TB705)
    \end{center}

    \begin{itemize}
        \item 1. Harmonische, $\frac{1}{1}$ Amplitude (Bsp. $2kHz$)
        \item 2. Harmonische, $\frac{1}{2}$ Amplitude (Bsp. \only<1>{\textbf{?}}
                                                            \only<2>{$4 kHz$})
        \item 3. Harmonische, $\frac{1}{3}$ Amplitude
        \item ...
    \end{itemize}

\end{frame}

\begin{frame}
    \frametitle{Rechteck}

    Grundwelle + ungeradzahligen Harmonische:

    % TODO Grafik und Formel?

    \begin{center}
        \includegraphics[width=0.5\textwidth]{a11/TB706.png}
        \tiny (TB706)
    \end{center}

     \begin{itemize}
        \item 1. Harmonische, $\frac{1}{1}$ Amplitude (Bsp. $2kHz$)
        \item 3. Harmonische, $\frac{1}{3}$ Amplitude (Bsp. \only<1>{\textbf{?}}
                                                            \only<2>{$6 kHz$})
        \item 5. Harmonische, $\frac{1}{5}$ Amplitude
        \item ...
    \end{itemize}
   

\end{frame}

\section{Nicht-periodische Signale}

\begin{frame}
    \frametitle{Nicht-periodische Signale}

    Nicht wiederkehrende Signale beliebiger Form, z.B. Impulse.

    \begin{center}
        \includegraphics[width=0.5\textwidth]{a11/TB702a.png}
        \tiny (TB702a)
    \end{center}

    Messung Impulsdauer oder Pulsbreite bei ca. 50\% der Amplitude.
    
    Wie breit ist das Signal im Beispiel?\only<2>{\footnote{Antwort: $0,2 ms$}}

\end{frame}

\begin{frame}
    \frametitle{Nicht-periodische Signale}

    Zur Messung von Impulsantworten oder Übergangsfunktionen (Sprungantwort)
    spielen Dirac-Impuls und Heaviside-Funktion eine große Rolle. \\[2em]

    \begin{center}
        \includegraphics[width=0.5\textwidth]{a11/High_accuracy_settling_time_measurements_figure_1.png}
        \tiny \hyperlink{refs}{\cite{wc}}
    \end{center}

    Praxis: \textbf{Sprungantwort aus Rechteck} - überführbar zu Impulsantwort.

\end{frame}

%%%%%%%% TODO ab hier nochmal ordentlich überarbeiten

\section{Modulation}

\begin{frame}
    \frametitle{Modulation}

    Modulation: ''Aufprägen'' von Signalen auf einen periodischen Träger durch
    Mischung/Multiplikation.\footnote{siehe Technik Klasse E} % TODO Welches Kapitel?
    \\[2em]

    Im einfachen reellen Fall betrachtet:

    \begin{equation*}
        u = \hat{u} \cdot sin(\omega t + \varphi)
    \end{equation*}

    % TODO Parameter erläutern
    \vspace{2em}
    Welche Parameter lassen sich modulieren?

\end{frame}

\subsection{Amplitude}

\begin{frame}
    \frametitle{Amplitude}

    Mit einer niedrigeren Frequenz wird eine Hüllkurve auf den Träger geprägt
    und muss zur Demodulation lediglich gleichgerichtet werden. \\[2em]

    Animation: Wikimedia [\href{http://commons.wikimedia.org/wiki/File:Amfm3-en-de.gif}{Link}]
    \\[2em]

    Mischung sorgt immer für Spiegelfrequenzen.

\end{frame}

\begin{frame}
    \frametitle{Amplitude}

    Arten:

    \begin{itemize}
        \item AM mit Träger ($> 2 \cdot$ NF-Bandbreite)
        \item AM ohne Träger ($> 2 \cdot$ NF-Bandbreite)
        \item SSB: LSB/USB ($\approx$ NF-Bandbreite)
    \end{itemize}

    % TODO Skizzen

    \vspace{2em} Woraus ergeben sich die Bandbreiten?

\end{frame}

\subsection{Frequenz}

\begin{frame}
    \frametitle{Frequenz}


\end{frame}

\subsection{Phase}

\begin{frame}
    \frametitle{Phase}

\end{frame}

\renewcommand{\refname}{Referenzen}

\hypertarget{refs}{}
\textcolor{white}{} \\ %\vspace{} geht nicht
\Large Referenzen/Links
\footnotesize

\begin{thebibliography}{}
    \bibitem{darc}  DARC Online-Lehrgang Klasse A:
                    \url{http://www.darc.de/referate/ajw/ausbildung/darc-online-lehrgang/technik-klasse-a/technik-a11/}
    \bibitem{wc}    Wikimedia Commons: \\
                    \url{http://commons.wikimedia.org/wiki/File:Waveforms_de.svg}\\
                    \url{http://commons.wikimedia.org/wiki/File:Sine_cosine_one_period.svg}\\
                    \url{http://commons.wikimedia.org/wiki/File:Sine_Cosine_Exponential_qtl1.svg}\\
                    \url{http://commons.wikimedia.org/wiki/File:High_accuracy_settling_time_measurements_figure_1.png}
\end{thebibliography} 

% Hier könnte noch eine Kontaktfolie stehen

\end{document}

