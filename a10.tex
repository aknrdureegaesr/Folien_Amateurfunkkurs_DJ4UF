% Foliensatz: "AFu-Kurs nach DJ4UF" von DK0TU, Amateurfunkgruppe der TU Berlin
% Lizenz: CC BY-NC-SA 3.0 de (http://creativecommons.org/licenses/by-nc-sa/3.0/de/)
% Autoren: Martin Deutschmann
% Korrekturen: Lars Weiler <dc4lw@darc.de>, Sebastian Lange <dl7bst@dk0tu.de>

preamble.dk0tu.tex
\subtitle{Technik Klasse A 10: \\
  HF-Leitungen \& Kabel \\[2em]}
\date{Stand 08.06.2017}
 \begin{document}

\begin{frame}
    \titlepage
    \vfill
    \begin{center}
        \ccbyncsaeu\\
        {\tiny This work is licensed under the \em{Creative Commons Attribution-NonCommercial-ShareAlike 3.0 License}.}\\[0.5ex]
         \tiny Amateurfunkgruppe der Technische Universität Berlin (AfuTUB), DKØTU
         %\includegraphics[scale=0.5]{img/DK0TU_Logo.pdf}
    \end{center}
\end{frame}


\section{HF-Leitung}
\begin{frame}
  \frametitle{Hochfrequenzleitungen}
  \begin{center}
    \begin{figure}
      \includegraphics[width=\textwidth,height=.25\textheight,keepaspectratio]{a10/parallel.png}
      \attribcaption{Paralleldrahtleitung}{SpinningSpark, verändert durch Inductiveload und Wdwd}{https://commons.wikimedia.org/wiki/File:Twin-lead_cable_dimension.svg}{\ccbysa}
    \end{figure}
    \begin{figure}
      \includegraphics[width=\textwidth,height=.35\textheight,keepaspectratio]{a10/coax.png}
      \attribcaption{Koaxialkabel Schnittmodell}{Tkgd2007, verändert durch Fleshgrinder}{https://commons.wikimedia.org/wiki/File:Coaxial_cable_cutaway_new.svg}{\ccby}
    \end{figure}
  \end{center}
\end{frame}

\begin{frame}
  \frametitle{Hochfrequenzleitungen}
  \begin{center}
    \begin{figure}
      \includegraphics[width=\textwidth,height=.6\textheight,keepaspectratio]{a10/hohl.jpg}
      \attribcaption{Elliptischer Hohlleiter für den Frequenzbereich von 3,8 bis 5,8 GHz. Zwischen dem gewellten Kupferrohr und dem schwarzen Mantel befindet sich eine klebrige Schicht, welche die Rillen des Kupferrohrs ausfüllt und somit eine bessere Biegsamkeit bewirkt. An der Luft härtet diese klebrige Schicht aus und bewirkt so einen gewissen Selbstheilungseffekt bei kleineren Defekten.}{Averse}{https://commons.wikimedia.org/wiki/File:Elli_holl.jpg}{\ccbysa}
    \end{figure}
  \end{center}
\end{frame}

\section{Wellen"-widerstand}
\begin{frame}
  \frametitle{Wellenwiderstand}
  \begin{figure}
    \includegraphics[width=\textwidth,height=.30\textheight,keepaspectratio]{a10/wellenesb.png}
    \caption{Ersatzschaltbild}
  \end{figure}
  \begin{figure}
    \includegraphics[width=\textwidth,height=.30\textheight,keepaspectratio]{a10/wellenesbex.png}
    \caption{Genaues Ersatzschaltbild eines Koxialkabels}
  \end{figure}
\end{frame}

\begin{frame}
  \frametitle{Wellenwiderstand}
  \begin{itemize}
      \begin{block}{Wellenwiderstand}
        $Z_W = \sqrt{\cfrac{L'}{C'}}$
      \end{block}
    \item Paralleldrahtleitungen: $Z_W = 150 \Omega \text{ bis } 600 \Omega$
    \item Koaxialleitungen: $Z_W = 50 \Omega \text{ bis } 95 \Omega$
    \item Der Wellenwiderstand entspricht dem Abschlusswiderstand einer Leitung, bei dem keine stehenden Wellen auftreten.
  \end{itemize}
\end{frame}

\section{Verkürzungs"-faktor}
\begin{frame}
  \frametitle{Verkürzungsfaktor}
  \begin{itemize}
    \item das Dielektrikum verlangsamt die Ausbreitungsgeschwindigkeit im Kabel: \\
      $v = \cfrac{1}{\sqrt{L' C'}}$
    \item durch geringere Ausbreitungsgeschwindigkeit verkürzt sich die
      Wellenlänge auf der Leitung: \\ $k = \cfrac{v}{c}$
  \end{itemize}
\end{frame}

\begin{frame}
  \frametitle{Typische Verkürzungsfaktoren}
  \begin{center}
    \begin{tabular}{l|l}
      \textbf{Kabeltyp} & \textbf{Verkürzungsfaktor} \\ \hline
      Koaxialkabel, normal & $k = 0,66$ \\
      Koaxialkabel mit Luftisolation & $k = 0,85$ \\
      offene $600 \Omega$ Speiseleitung & $k = 0,98$ \\
      Flachleitung mit $300 \Omega$ & $k = 0,82$ \\
    \end{tabular}
  \end{center}
\end{frame}

\section[Skin-Effekt]{Skin-Effekt (Wdh.)}
\begin{frame}
  \frametitle{Der Skin-Effekt (Wiederholung aus A02)}
  \begin{itemize}
    \item tritt bei höherfrequenter Wechselspannung auf
    \item verdrängt Elektronen aus dem Leitungsinneren an die Leiteroberfläche \\
      $\rightarrow$ Widerstand im Leiter steigt
  \end{itemize}
\end{frame}

\begin{frame}
  \frametitle{Ursachen des Skin-Effektes}
  \begin{center}
    \begin{figure}
      \includegraphics[width=.4\textwidth,height=.35\textheight,keepaspectratio]{a02/Skineffect.png}
      \attribcaption{Überlagerung von Wechsel- und Wirbelströmen}{Biezl}{https://commons.wikimedia.org/wiki/File:Skineffect_reason.svg}{\ccpd}
    \end{figure}
    \begin{itemize}
      \item Ursache des Skin-Effektes ist das magnetische Feld
      \item Es erzeugt Wirbelströme im Innern des Leiters
      \item Diese sind dem Erzeugerstrom entgegengerichtet
      \item Das wechselnde Magnetfeld erzeugt im Leiter eine höhere Gegenspannung als am Rand
    \end{itemize}
  \end{center}
\end{frame}

\begin{frame}
  \frametitle{Folgen \& Gegenmaßnahmen}
  \textbf{Folgen:}
  \begin{itemize}
    \item Der Leiterquerschnitt sinkt
    \item Die Impedanz steigt
  \end{itemize}
  \textbf{Gegenmaßnahmen}
  \begin{itemize}
    \item Verwendung von Hohlleitern
    \item Mehrere voneinander isolierte Drähte nutzen
    \item Oberfläche versilbern
  \end{itemize}
\end{frame}

\begin{frame}
    \textbf{TC314}  \textbf{Welche Folgen hat der Skin-Effekt?}\\
     \only<2>{Der Strom fließt bei hohen Frequenzen nur noch in der Oberfläche des Leiters. Mit sinkendem stromdurchflossenen Querschnitt steigt daher der effektive Widerstand des Leiters.}
\end{frame}

\section{D\"ampfung}
\begin{frame}
  \frametitle{Die Dämpfung}
  \begin{itemize}
    \item Gibt den Leistungsverlust über das Kabel an
    \item Hängt vom Verlustwiderstand und dem Dielektrikum ab
    \item Wird meist in dB pro $100m$ angegeben
      \begin{Large}
      \item $n = \sqrt{\cfrac{f_{hoch}}{f_{niedrig}}}$
      \end{Large}
  \end{itemize}
\end{frame}

\begin{frame}
  \frametitle{Kabeldämpfung Beispiel 1}

  \begin{exampleblock}{RG 213/U hat bei $100MHz$ eine Dämpfung von $6,7dB$.
    Wie groß ist die Dämpfung bei $145MHz$?}
    \only<1>{Hinweis: $n = \sqrt{\cfrac{f_{hoch}}{f_{niedrig}}}$
    \vspace{2em}}
    \only<2>{$n = \sqrt{\cfrac{f_2}{f_1}} = \sqrt{\cfrac{145}{100}} = \sqrt{1,45} = 1,2$ \\[1em]
    Bei $145MHz$ ist die Dämpfung also: $1,2 \cdot 6,7dB = 8dB$
    }
  \end{exampleblock}

\end{frame}

\begin{frame}
  \frametitle{Kabeldämpfung Beispiel 2}

  % FIXME Musterloesungen mit rein! Dämpfungsdiagramm
  \begin{exampleblock}{Löse mit Hilfe des Dämpfungsdiagramms aus der Formelsammlung:}
    \begin{minipage}{0.3\textwidth}
      \begin{itemize}
        \item \textbf{RG58}\\
        \item 15 m\\
        \item 28 MHz
      \end{itemize}
    \end{minipage}
    \begin{minipage}{0.3\textwidth}
      \begin{itemize}
        \item \textbf{Aircell7}\\
        \item 15 m\\
        \item 28 MHz
      \end{itemize}
    \end{minipage}
    \begin{minipage}{0.3\textwidth}
      \begin{itemize}
        \item \textbf{RG174}\\
        \item 15 m\\
        \item 28 MHz
      \end{itemize}
    \end{minipage}
  \end{exampleblock}
\end{frame}

\section[SWR]{Steh"-wellen"-verh\"altnis (Wdh.)}
\begin{frame}
  \frametitle{Stehwellenverh\"altnis (Wiederholung)}
  \begin{itemize}
    \item ist ein Maß für die Anpassung
      \begin{Large}
        $SWR = s = \cfrac{U_{max}}{U_{min}}$
      \end{Large}
    \item hängt vom Verhältnis Abschlusswiderstand $R_a$ zu Wellenwiderstand $Z_W$ ab \\
      \vspace{2mm}
      \begin{Large}
        $SWR = s = \cfrac{U_{max}}{U_{min}} = \frac{Z}{R_a} \text{ für } R_a \geq Z$
        \vspace{2mm}
        $SWR = s = \cfrac{U_{max}}{U_{min}} = \frac{R_a}{Z} \text{ für } Z \geq R_a$
      \end{Large}
      \vspace{2mm}
    \item ist das Verhältnis von vorlaufender zu zurücklaufender Welle
    \item \url{http://commons.wikimedia.org/wiki/File:Stehwelle_(Animation).gif}
  \end{itemize}
\end{frame}

\section{Lecherleitung}
\begin{frame}
  \frametitle{Lecherleitung}
  \begin{center}
    \begin{figure}
      \includegraphics[width=\textwidth,height=.2\textheight,keepaspectratio]{a10/Lecher_wires_and_oscillator_1932.png}
      \attribcaption{Lecherleitung Vorführapparat mit Kurzschlussschieber und Lampe für den Resonanzpunkt}{D. L. Barr}{https://commons.wikimedia.org/wiki/File:Lecher_wires_and_oscillator_1932.png}{\ccpd}
    \end{figure}
  \end{center}
  \begin{itemize}
    \item Ist ein Sonderfall einer Transformationsleitung mit einem Abschlusswiderstand von $0 \Omega$ oder $\infty \Omega$
    \item Gibt man HF-Signal auf Doppelleitung, mit $R_a = 0 \Omega$ wird die gesamte Energie reflektiert
    \item Dadurch entstehen Auslöschungen und Anhebungen
    \item Wellenwiderstand kehrt sich alle $\lambda /4$ um
    \item Lässt man das Leitungsende offen, kehren sich alle Verhältnisse um
    \item Dieser Effekt tritt auch bei einer $\lambda /2$ Leitung auf
  \end{itemize}
\end{frame}

\begin{frame}
  \frametitle{Lecherleitung}

  Zusammenfassung:

  \begin{itemize}
    \item $\lambda /4$ Leitung kehrt Impedanzverhältnisse um (niederohmig $\leftrightarrow$ hochohmig), wirkt wie Schwingkreis
    \item $\lambda /2$ Leitung transformiert 1:1, wirkt auch wie ein Schwingkreis
  \end{itemize}

\end{frame}

\begin{frame}
  \begin{tabular}{l||p{.8\textwidth}}\hline
    \textbf{TH325} & \textbf{Eine Lecherleitung besteht aus zwei parallelen Leitern. Wovon ist ihre Resonanzfrequenz wesentlich abhängig? Sie ist abhängig}\\ \hline\hline
    A & vom verwendeten Abschlusswiderstand. \\ \hline
    B \only<2>\checkmark & von der Leitungslänge \\ \hline
    C & vom Wellenwiderstand der beiden parallelen Leiter. \\ \hline
    D & vom Leerlauf-Kurzschlussverhalten. \\ \hline
  \end{tabular}
\end{frame}

\begin{frame}
\begin{center}
   
 \includegraphics[width=\textwidth,height=.5\textheight,keepaspectratio]{a10/th326.png}\\
     \textbf{TH326} \textbf{Was zeigt diese Darstellung?}\\ 
    \only<2>{ Sie zeigt die Strom- und Spannungsverteilung an einer offenen $\lambda/4$-Lecherleitung. Sie wirkt als Reihenschwingkreis. }
\end{center}
\end{frame}

\section{Trans"-formations"-leitungen}
\begin{frame}
  \frametitle{Transformationsleitungen}
  \begin{itemize}
    \item Transformationsleitungen dienen der Anpassung von Antenne zum Sender zur \dots
      \begin{itemize}
        \item Anpassung des Sender-Widerstandes an die HF-Leitung.
        \item Anpassung der Antennenimpedanz an das Kabel.
      \end{itemize}
    \item $R_i = Z_w = Z_{Antenne}$
  \end{itemize}
\end{frame}

\begin{frame}
  \begin{center}
    \begin{figure}
      \includegraphics[width=\textwidth,height=.8\textheight,keepaspectratio]{a10/800px-EingangswiderstandAusgangswiderstandA.png}
      \attribcaption{Die Widerstände bzw. Impedanzen am Eingang und Ausgang von elektrischen Geräten}{Frank Murmann}{https://commons.wikimedia.org/wiki/File:EingangswiderstandAusgangswiderstandA.svg}{\ccpd}
    \end{figure}
  \end{center}
\end{frame}


\begin{frame}
  \frametitle{Prinzip der Transformationsleitung}
  \begin{center}
    \begin{figure}
      \includegraphics[width=\textwidth,height=.25\textheight,keepaspectratio]{a10/800px-EingangswiderstandAusgangswiderstandA.png}
      \attribcaption{Die Widerstände bzw. Impedanzen am Eingang und Ausgang von elektrischen Geräten}{Frank Murmann}{https://commons.wikimedia.org/wiki/File:EingangswiderstandAusgangswiderstandA.svg}{\ccpd}
    \end{figure}
  \end{center}
  \begin{itemize}
    \item Eine $\lambda /4$-Leitung kann Widerstände transformieren
    \item Aber nur in einer begrenzten Bandbreite
    \item Leitung wirkt als Transformator
    \item Eine solche Leitung bestimmter Länge wird auch als abgestimmte Speiseleitung bezeichnet
    \item Mit ihrem Wellenwiderstand abgeschlossenen Leitungen zur Vermeidung von Stehwellen, nennt man unabgestimmte Speiseleitung
  \end{itemize}
\end{frame}

\begin{frame}

  Will man zwei Impedanzen $Z_E$ \& $Z_A$ mit einem Viertelwellentransformator
  anpassen, so muss die Transformationsleitung folgende Werte besitzen:

  \begin{itemize}
    \item Wellenwiderstand:\\
      \vspace{1em}
      \begin{Large}
        $Z = \sqrt{Z_E \cdot Z_A}$
      \end{Large}
      \vspace{1em}
    \item Länge: \\
      \vspace{1em}
      \begin{Large}
        $\ell = (2n - 1) \cdot \frac{\lambda}{4} \cdot k$
      \end{Large}
      \vspace{1em}
    \item Bei Koaxialkabeln sieht das Ganze wie folgt aus:
      \vspace{1em}
      \begin{align*}
        Z &= \frac{60 \Omega}{\sqrt{\varepsilon_r}} \cdot ln(\frac{D}{d}) \\
        &= \frac{138 \Omega}{\sqrt{\varepsilon_r}} \cdot lg(\frac{D}{d})
      \end{align*}
  \end{itemize}
\end{frame}

\section{Symmetrierung}
\begin{frame}
  \frametitle{Symmetrierung}
  \begin{columns}
    \column{.4\textwidth}
    \begin{figure}
      \includegraphics[width=\textwidth,height=.8\textheight,keepaspectratio]{a10/balun.jpg}
      \attribcaption{Balun 4:1, 13 Windungen auf T200A/2 Ringkern}{Giorgio Brida}{https://commons.wikimedia.org/wiki/File:T200A2.jpg}{\ccby}
    \end{figure}
    \column{.6\textwidth}
    \begin{itemize}
      \item Wird bei Verbindungen zwischen symmetrischen und unsymmetrischen Punkten verwendet
      \item Koaxialkabel ist unsymmetrisch
      \item Paralleldraht ist symmetrisch
      \item Alle Dipole sind symmetrisch
      \item Alle Antennen, die gegen Erde erregt werden sind unsymmetrisch
      \item Ohne Symmetrierung entstehen Mantelwellen
    \end{itemize}
  \end{columns}
\end{frame}

\begin{frame}
  \frametitle{Balun}
  \begin{columns}
    \column{.4\textwidth}
    \begin{center}
      \begin{figure}
        \includegraphics[height=.6\textheight,width=\textwidth,keepaspectratio]{a10/balun.png}
        \attribcaption{Spartrafo als Balun}{Wolfmankurd}{https://commons.wikimedia.org/wiki/File:Cdbalun2.svg}{\ccbysa}
      \end{figure}
    \end{center}
    \column{.6\textwidth}
    \begin{itemize}
      \item Balun kann symmetrieren und gleichzeitig die Impedanz anpassen
      \item Wird der Eingang an halber Windungszahl des Ausganges angeschlossen, erhält man einen 1:4 Übertrager
    \end{itemize}
  \end{columns}
\end{frame}

\begin{frame}
  \frametitle{Die $\lambda /2$-Umwegleitung}
  \begin{columns}
    \column{.4\textwidth}
    \begin{figure}
      \includegraphics[height=.7\textheight,width=\textwidth,keepaspectratio]{a10/Umwegleitung.jpg}
      \attribcaption{$\lambda /2$-Umwegleitung}{Charly Whisky}{https://commons.wikimedia.org/wiki/File:Balun(semirigid).jpg}{\ccbysa}
    \end{figure}
    \column{.6\textwidth}
    \begin{itemize}
      \item An der Einspeisestelle teilt sich der Strom je zur Hälfte auf
      \item Eine Hälfte geht direkt zur Antenne, die andere in die Umwegleitung
      \item Nach dem ohmschen Gesetz verdoppelt sich dadurch der Widerstand
      \item Bei $50 \Omega$ ergeben sich also $100 \Omega$
      \item Die Umwegleitung stellt den Widerstand auf der anderen Seite nochmal mit $100 \Omega$ zur Verfügung
      \item Somit ergeben sich insgesamt $200 \Omega$ Impedanz an der Antenne
    \end{itemize}
  \end{columns}
\end{frame}

\section{Topfkreis}
\begin{frame}
  \frametitle{Der Topfkreis}
  \begin{columns}
    \column{.35\textwidth}
    \begin{figure}
      \includegraphics[width=\textwidth,height=.25\textheight,keepaspectratio]{a10/500px-Topfkreis.png}
      \attribcaption{Topfkreis}{Mik81}{https://commons.wikimedia.org/wiki/File:Topfkreis.svg}{\ccpd}
    \end{figure}
    \begin{figure}
      \includegraphics[width=\textwidth,height=.25\textheight,keepaspectratio]{a10/Topfkreis-ESB.png}
      \caption{Ersatzschaltbild}
    \end{figure}
    \column{.65\textwidth}
    \begin{itemize}
      \item Im UHF-Bereich werden Aluminium- oder versilberte Messingbecher als Leiter genutzt
      \item Sie besitzen einen Mittelleiter, wodurch sie wie eine Koaxleitung wirken
      \item Sie lassen sich wie auch die Lecherleitung durch einen Kurzschlussschieber abstimmen
      \item Dieser Schwingkreis ist vollkommen abgestimmt und von außen nicht beeinflussbar
      \item Versilbert man die Innenflächen, lassen sich die Verluste minimieren
      \item Dadurch werden die elektrischen Eigenschaften für hohe Frequenzen verbessert
    \end{itemize}
  \end{columns}
\end{frame}

\begin{frame}
  \begin{exampleblock}{Hausaufgabe}
    Kapitel 1.8.3 Übertragungsleitungen mit Fragen TH301--TH331.\\
    Kapitel 1.8.4 Anpassung, Transformation und Symmetrierung mit Fragen TH401--TH423.\\
  \end{exampleblock}
\end{frame}

\renewcommand{\refname}{Referenzen}

\hypertarget{refs}{}
\textcolor{white}{} \\ %\vspace{} geht nicht
\Large Referenzen/Links
\footnotesize

\begin{thebibliography}{}
  \bibitem{darc}  DARC Online-Lehrgang Lektion A08:
    \url{https://www.darc.de/der-club/referate/ajw/lehrgang-ta/a10/}
  \bibitem{bna}   Fragenkatalog Bundesnetzagentur Technik Klasse A:
    \mbox{\url{https://www.bundesnetzagentur.de/SharedDocs/Downloads/DE/Sachgebiete/Telekommunikation/Unternehmen_Institutionen/Frequenzen/Amateurfunk/Fragenkatalog/TechnikFragenkatalogKlasseAf252rId9014pdf.pdf?__blob=publicationFile&v=3}}
\end{thebibliography}

% Hier könnte noch eine Kontaktfolie stehen

\end{document}

