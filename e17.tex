% Foliensatz: "AFu-Kurs nach DJ4UF" von DK0TU, Amateurfunkgruppe der TU Berlin
% Lizenz: CC BY-NC-SA 3.0 de (http://creativecommons.org/licenses/by-nc-sa/3.0/de/)
% Autoren: Felix Baum DB4UM <baum@campus.tu-berlin.de>

preamble.dk0tu.tex
\subtitle{Technik E 17: \\
          Messtechnik \\[2em]}
\date{Stand 26.1.2015}
 \begin{document}

\begin{frame}
    \titlepage
    \vfill
    \begin{center}
        \ccbyncsaeu\\
        {\tiny This work is licensed under the \em{Creative Commons Attribution-NonCommercial-ShareAlike 3.0 License}.}\\[0.5ex]
         \tiny Amateurfunkgruppe der Technische Universität Berlin (AfuTUB), DKØTU
         %\includegraphics[scale=0.5]{img/DK0TU_Logo.pdf}
    \end{center}
\end{frame}


%fixme Referenzen/Fußnoten-Systematik vereinheitlichen

\section*{Einleitung}

\begin{frame}
    \frametitle{Messgeräte}
    \begin{itemize}
		\item Was kann alles beim Funken gemessen werden?
    \end{itemize}
\end{frame}
  
\section*{Analog}

\begin{frame}
    \frametitle{Drehspulenmessgerät (Antik)}
	\begin{minipage}{0.4\textwidth}
	    \includegraphics[width=.95\textwidth]{e17/drehspulenMess.png}\\
	\end{minipage}
	        \footnote{\tiny \url{https://commons.wikimedia.org/wiki/File:Moving_coil_instrument_principle.png}}
		\hspace{0.5cm}
	\begin{minipage}{0.4\textwidth}	
	\begin{itemize} \tiny
		\item \tiny 1) Weicheisenkern der Drehspule
		\item 2) Permanentmagnet
		\item 3) Polschuh zur Bündelung des Magnetfeldes
		\item 4) Skala
		\item 5) Hilfsspiegel zur genauen Ablesung
		\item 6) Rückstellfeder
		\item 7) Drehspule
		\item 8) Drehspule in Nulllage
		\item 9) Drehspule bei Maximalausschlag
		\item 10) Joch der Spule
		\item 11) Stellschraube für Nullpunktseinstellung
		\item 12) Zeiger
		\item 13) Zeiger in Nulllage
		\item 14) Zeiger bei Maximalausschlag
	\end{itemize}
	\end{minipage}
\end{frame}

\begin{frame}
    \frametitle{Prüfungsfrage}
    \begin{center}
        \includegraphics[width=.95\textwidth]{e17/messbereich.png}
        \footnote{\tiny Fragenkatalog Bundesnetzargentur Klasse E}
	\end{center}
\end{frame}

\section*{Digital}

\begin{frame}
    \frametitle{Digitales Multimeter}
    \begin{center}
        \includegraphics[width=.35\textwidth]{e17/digitalmultimeter.jpg}
	\end{center}
\end{frame}

\begin{frame}
    \frametitle{Was wo anschließen?}
	\begin{minipage}{0.4\textwidth}
        \includegraphics[width=1\textwidth]{e17/digitalmultimeterMess.jpg}
	\end{minipage}
	\begin{minipage}{0.4\textwidth}	
	\begin{itemize}
		\item Was kann alles gemessen werden?
		\item Wo anschließen zum Strom messen?
		\item Wo anschließen zum Spannung messen?
		\item Welcher Messbereich?
	\end{itemize}
	\end{minipage}
\end{frame}

\begin{frame}
    \frametitle{Wie sollte gemessen werden?}
        \includegraphics[width=1\textwidth]{e17/stromSpannung.png}
        \footnote{\tiny Fragenkatalog Bundesnetzargentur Klasse E}
\end{frame}

\begin{frame}
    \frametitle{Messfehler}
    \begin{center}
    		\begin{itemize}
				\item Nullpunktsabweichung
				\item Empfindlichkeitsabweichung
				\item Linearitätsabweichung
    		\end{itemize}
        \includegraphics[width=1\textwidth]{e17/werMisstMisst.png}
        \footnote{\tiny \url{https://commons.wikimedia.org/wiki/File:AMT_Fehler.svg}}
	\end{center}
\end{frame}

\begin{frame}
    \frametitle{Prüfungsfrage}
    \begin{center}
    \begin{tabular}{l||l}\hline
        TJ102 & Die Auflösung eines Messinstrumentes entspricht... \\  \hline\hline
         A  & der Genauigkeit des Instrumentes \\
         " " & in Bezug auf den tatsächlichen Wert. \\ \hline
         B & der Genauigkeit des Instrumentes. \\ \hline
         C & der kleinsten Einteilung der Anzeige. \\\hline
         D & dem Vollausschlag der Instrumentenanzeige. \\\hline
    \end{tabular}
 	\end{center}
\end{frame}

\begin{frame}
    \frametitle{Prüfungsfrage}
    \begin{center}
    \begin{tabular}{l||l}\hline
        TJ102 & Die Auflösung eines Messinstrumentes entspricht... \\ \hline\hline
         " " & der Genauigkeit des Instrumentes \\
         " " & in Bezug auf den tatsächlichen Wert. \\ \hline
         " " & der Genauigkeit des Instrumentes. \\ \hline
         X & der kleinsten Einteilung der Anzeige. \\\hline
         " " & dem Vollausschlag der Instrumentenanzeige. \\\hline
    \end{tabular}
 	\end{center}
\end{frame}

\section*{Oszilloscope}

\begin{frame}
    \frametitle{Oszilloscope}
    \begin{center}
        \includegraphics[width=1\textwidth]{e17/osziModern.jpg}
        \footnote{\tiny \url{https://commons.wikimedia.org/wiki/File:Modernes_Speicheroszilloskop.jpg}}
	\end{center}
\end{frame}

\begin{frame}
    \frametitle{Oszilloscope}
    \begin{center}
        \includegraphics[width=.8\textwidth]{e17/WTPCOscilloscopeBeschreiben.jpg}
        \footnote{\tiny \url{https://en.wikipedia.org/wiki/File:WTPC_Oscilloscope-1.jpg}}
	\end{center}
\end{frame}

\begin{frame}
    \frametitle{Oszilloscope - Ablesen}
    \begin{center}
        $100mV / Div$ und $0.1ms / Div$
        \includegraphics[width=.8\textwidth]{e17/OsziTon.png}
        \footnote{\tiny \url{https://commons.wikimedia.org/wiki/File:Oszi_Ton.svg}}
	\end{center}
\end{frame}

\begin{frame}
    \frametitle{Spannungen}
    \begin{center}
    \begin{itemize}
			\item PEP - Peak to Peak
			\item RMS - Effektivwert
			\item $u_{Spitze} = sqrt{2} \cdot U_{eff}$ \\
			\item Aufgabe: Berechnet Spitzen-Spitzen Spannung vom Netzstrom 
    \end{itemize}
	\end{center}
\end{frame}

\section*{Dipmeter}

\begin{frame}
    \frametitle{Dipmeter}
    \begin{center}
        \includegraphics[width=0.55\textwidth]{e17/Dipmeter.jpg}
        \footnote{\tiny \url{https://commons.wikimedia.org/wiki/File:Dipmeter_and_its_probe_coils.jpg}}
	\end{center}
\end{frame}

\section*{SWR-Meter}

\begin{frame}
    \frametitle{Stehwellenmessgerät}
    \begin{center}
        \includegraphics[width=1\textwidth]{e17/RS_SWR.jpg}
        \footnote{\tiny \url{https://commons.wikimedia.org/wiki/File:RS_SWR.jpg}}
	\end{center}
\end{frame}

\begin{frame}
    \frametitle{Interner Aufbau}
    \begin{center}
    $$s = \frac{U_{max}}{U_{min}} = \frac{U_v + U_r}{U_v - U_r}$$
        \includegraphics[width=.8\textwidth]{e17/SWRMeterInnen.png}
        \footnote{\tiny \url{https://commons.wikimedia.org/wiki/File:SWR_Meter.svg}}
	\end{center}
\end{frame}

\begin{frame}
    \frametitle{VSWR-Rechnen}
    \begin{center}
    $$s = \frac{U_{max}}{U_{min}} = \frac{U_v + U_r}{U_v - U_r}$$
    $$s = \frac{1V + 0.5V}{1V - 0.5V} = \frac{1.5V}{0.5V} = 3$$
	$$s = \frac{1V + 0V}{1V - 0V} = \frac{1V}{1V} = 1$$
	\end{center}
\end{frame}

\begin{frame}
    \frametitle{Prüfungsfrage}
    \begin{center}
    \begin{tabular}{l||l}\hline
        TH401&Eine Antenne hat ein Stehwellenverhältnis (VSWR)\\
         " "  &von 3. Wie viel Prozent der Sendeleistung\\
         " "  &wird von der Antenne abgestrahlt, wenn sonst\\
         " "  &keine Verluste auftreten?\\ \hline\hline
         A & $50 \%$ \\\hline
         B & $29 \%$ \\\hline
         C & $25 \%$ \\ \hline
         D & $75 \%$\\\hline
    \end{tabular}
 	\end{center}
\end{frame}

\begin{frame}
    \frametitle{Prüfungsfrage}
    \begin{center}
    \begin{tabular}{l||l}\hline
        TH401&Eine Antenne hat ein Stehwellenverhältnis (VSWR)\\
         " "  &von 3. Wie viel Prozent der Sendeleistung\\
         " "  &wird von der Antenne abgestrahlt, wenn sonst\\
         " "  &keine Verluste auftreten?\\ \hline\hline
         " " & $50 \%$ \\\hline
         " " & $29 \%$ \\\hline
         " " & $25 \%$ \\ \hline
         X & $75 \%$\\\hline
    \end{tabular}
    $50 \%$ der Spannung sind $25 \%$ der Leistung.
 	\end{center}
\end{frame}

\begin{frame}
    \frametitle{Wo das SWR-Meter einschleifen?}
        \includegraphics[width=.97\textwidth]{e17/SWROrt.png}
        \footnote{\tiny Fragenkatalog Bundesnetzargentur Klasse E}
\end{frame}

\begin{frame}
    \frametitle{Wo das SWR-Meter einschleifen?}
        \includegraphics[width=.97\textwidth]{e17/SWROrt.png}
        \footnote{\tiny Fragenkatalog Bundesnetzargentur Klasse E}
    Am Besten in 1, da alles dahinter zur Antenne bzw. Antennenanlage gehört.
\end{frame}


\section*{Dummy Load}

\begin{frame}
    \frametitle{Messen}
    \begin{center}
        \includegraphics[width=.99\textwidth]{e17/DummyLoad.jpg}
        \footnote{\tiny \url{https://commons.wikimedia.org/wiki/File:Dummy_load.jpg}}
	\end{center}
\end{frame}

\begin{frame}
    \frametitle{Zu Beachten}
    \begin{itemize}
		\item Aufbau als Widerstandsdekade
		\item Lastwiderstände
		\item Außer bei besonderem Genau $50 \Omega$ Realer Widerstand
		\item Am besten Metalloxid Widerstände
		\item Große Kühlkörper
		\item selbstbau: \url{http://der-bastelbunker.blogspot.de/2011/04/qro-dummy-load-von-kw-bis-vhf-fur-1.html}
    \end{itemize}
\end{frame}

\begin{frame}
    \frametitle{Dummy in Nauen}
	\begin{minipage}{0.4\textwidth}
	\begin{center}
	    \includegraphics[width=.9\textwidth]{e17/DummyNauen.jpg}
	\end{center}
	\end{minipage}
	        \footnote{\tiny Bild DB4UM}
		\hspace{0.5cm}
	\begin{minipage}{0.4\textwidth}	
	\begin{itemize}
		\item Dummyload in Nauen
		\item Auch als Wattmeter nutzbar
		\item Temperaturanstieg pro Zeiteinheit
	\end{itemize}
	\end{minipage}
\end{frame}

\section*{Referenzen}

\begin{frame}
    \frametitle{Referenzen/Links}
    
    \footnotesize
    \begin{itemize}
        \item Moltrecht E 17: \\
              \url{http://www.darc.de/referate/ajw/ausbildung/darc-online-lehrgang/technik-klasse-e/technik-e17/}
         \item SWR-Meter Selbstbau \\
         	\url{http://www.nogaqrp.org/projects/NOGAwatt/nogawattwithnewschematic2.pdf}
         \item Dummy Selbstbau \\
         	\url{http://der-bastelbunker.blogspot.de/2011/04/qro-dummy-load-von-kw-bis-vhf-fur-1.html}
    \end{itemize}

\end{frame}

% Hier könnte noch eine Kontaktfolie stehen

\end{document}

