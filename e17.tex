% Foliensatz: "AFu-Kurs nach DJ4UF" von DK0TU, Amateurfunkgruppe der TU Berlin
% Lizenz: CC BY-NC-SA 3.0 de (http://creativecommons.org/licenses/by-nc-sa/3.0/de/)
% Autor: Felix Baum DB4UM <baum@campus.tu-berlin.de>
% Korrekturen: Lars Weiler <dc4lw@darc.de>

preamble.dk0tu.tex
\subtitle{Technik E 17: \\
  Messtechnik \\[2em]}
\date{Stand 18.09.2017}
 \begin{document}

\begin{frame}
    \titlepage
    \vfill
    \begin{center}
        \ccbyncsaeu\\
        {\tiny This work is licensed under the \em{Creative Commons Attribution-NonCommercial-ShareAlike 3.0 License}.}\\[0.5ex]
         \tiny Amateurfunkgruppe der Technische Universität Berlin (AfuTUB), DKØTU
         %\includegraphics[scale=0.5]{img/DK0TU_Logo.pdf}
    \end{center}
\end{frame}


\section*{Einleitung}

\begin{frame}
  \frametitle{Messgeräte}
  \begin{itemize}
    \item Was kann alles beim Funken gemessen werden?
  \end{itemize}
\end{frame}

\section*{Analog}

\begin{frame}
  \frametitle{Drehspulenmessgerät (Antik)}
  \begin{minipage}{0.3\textwidth}
    \begin{figure}
      \includegraphics[width=.95\textwidth,height=.75\textheight,keepaspectratio]{e17/drehspulenMess.png}
      \attribcaption{Technische Ausführung eines Drehspulinstruments}{Søren Peo Pedersen}{https://commons.wikimedia.org/wiki/File:Moving_coil_instrument_principle.png}{\ccbysa}
    \end{figure}
  \end{minipage}
  %\hspace{0.5cm}
  \begin{minipage}{0.65\textwidth}
    \begin{enumerate} \footnotesize
      \item  Weicheisenkern der Drehspule
      \item  Permanentmagnet
      \item  Polschuh zur Bündelung des Magnetfeldes
      \item  Skala
      \item  Hilfsspiegel zur genauen Ablesung
      \item  Rückstellfeder
      \item  Drehspule
      \item  Drehspule in Nulllage
      \item  Drehspule bei Maximalausschlag
      \item  Joch der Spule
      \item  Stellschraube für Nullpunkteinstellung
      \item  Zeiger
      \item  Zeiger in Nulllage
      \item  Zeiger bei Maximalausschlag
    \end{enumerate}
  \end{minipage}
\end{frame}

\begin{frame}
  \frametitle{Prüfungsfrage}
  \begin{center}
    \begin{figure}
      \includegraphics[width=.95\textwidth,height=.75\textheight,keepaspectratio]{e17/messbereich.png}
      \caption{Fragenkatalog BundesNetzAgentur Klasse E}
    \end{figure}
  \end{center}
\end{frame}

\section*{Digital}

\begin{frame}
  \frametitle{Digitales Multimeter}
  \begin{center}
    \begin{figure}
      \includegraphics[width=.35\textwidth,height=.75\textheight,keepaspectratio]{e17/digitalmultimeter.jpg}
      \attribcaption{Digitales Multimeter}{MichaelHaeckel}{https://commons.wikimedia.org/wiki/File:Digitalmultimeter.jpg}{\ccpd}
    \end{figure}
  \end{center}
\end{frame}

\begin{frame}
  \frametitle{Was wo anschließen?}
  \begin{minipage}{0.4\textwidth}
    \begin{figure}
      \includegraphics[width=1\textwidth,height=.75\textheight,keepaspectratio]{e17/digitalmultimeterMess.jpg}
      \attribcaption{Digitales Multimeter}{MichaelHaeckel}{https://commons.wikimedia.org/wiki/File:Digitalmultimeter.jpg}{\ccpd}
    \end{figure}
  \end{minipage}
  \begin{minipage}{0.55\textwidth}
    \begin{itemize}
      \item Was kann alles gemessen werden?
      \item Wo anschließen zum Strom messen?
      \item Wo anschließen zum Spannung messen?
      \item Welcher Messbereich?
    \end{itemize}
  \end{minipage}
\end{frame}

\begin{frame}
  \frametitle{Wie sollte gemessen werden?}
  \begin{figure}
    \includegraphics[width=1\textwidth,height=.75\textheight,keepaspectratio]{e17/stromSpannung.png}
    \caption{Fragenkatalog BundesNetzAgentur Klasse E}
  \end{figure}
\end{frame}

\begin{frame}
  \frametitle{Messfehler}
  \begin{center}
    \begin{tabular}{ccc}
      Nullpunktabweichung & Empfindlichkeitsabweichung & Linearitätsabweichung
    \end{tabular}
    \begin{figure}
      \includegraphics[width=1\textwidth,height=.6\textheight]{e17/werMisstMisst.png}
      \attribcaption{Nullpunktsabweichung, Empfindlichkeitsabweichung, Linearitätsabweichung}{Saure}{https://commons.wikimedia.org/wiki/File:AMT_Fehler.svg}{\ccbysa}
    \end{figure}
  \end{center}
\end{frame}

%\begin{frame}
%  \frametitle{Prüfungsfrage}
%  \begin{tabular}{l||p{.8\textwidth}}\hline
%    \textbf{TJ102} & \textbf{Die Auflösung eines Messinstrumentes entspricht\ldots} \\  \hline\hline
%    A  & der Genauigkeit des Instrumentes in Bezug auf den tatsächlichen Wert. \\ \hline
%    B & der Genauigkeit des Instrumentes. \\ \hline
%    C \only<2>\checkmark & der kleinsten Einteilung der Anzeige. \\\hline
%    D & dem Vollausschlag der Instrumentenanzeige. \\\hline
%  \end{tabular}
%\end{frame}

\section*{Oszilloskop}

\begin{frame}
  \frametitle{Oszilloskop}
  \begin{center}
    \begin{figure}
      \includegraphics[width=1\textwidth,height=.75\textheight,keepaspectratio]{e17/osziModern.jpg}
      \attribcaption{Modernes Speicheroszilloskop}{Björn Heller}{https://commons.wikimedia.org/wiki/File:Modernes_Speicheroszilloskop.jpg}{\ccbysa}
    \end{figure}
  \end{center}
\end{frame}

\begin{frame}
  \frametitle{Oszilloskop}
  \begin{center}
    \begin{figure}
      \includegraphics[width=.8\textwidth,height=.75\textheight,keepaspectratio]{e17/WTPCOscilloscopeBeschreiben.jpg}
      \attribcaption{Bedienungselemente eines Oszilloskops}{Brian S. Elliott}{https://en.wikipedia.org/wiki/File:WTPC_Oscilloscope-1.jpg}{\ccbysa}
    \end{figure}
  \end{center}
\end{frame}

\begin{frame}
  \frametitle{Oszilloskop -- Ablesen}
  \begin{center}
    $100mV / Div$ und $0.1ms / Div$\\
    \begin{figure}
      \includegraphics[width=.8\textwidth,height=.65\textheight,keepaspectratio]{e17/OsziTon.png}
      \attribcaption{Schematisches Oszilloskopbild eines Tons}{Klaus-Dieter Keller}{https://commons.wikimedia.org/wiki/File:Oszi_Ton.svg}{\ccbysa}
    \end{figure}
  \end{center}
\end{frame}

\begin{frame}
  \frametitle{Spannungen}
  \begin{center}
    \begin{itemize}
      \item PEP -- Peak to Peak
      \item RMS -- Effektivwert
      \item $u_{Spitze} = \sqrt{2} \cdot U_{eff}$ \\
    \end{itemize}
  \end{center}
  \begin{exampleblock}{Aufgabe}
    Berechnet Spitzen-Spitzen Spannung vom Netzstrom
  \end{exampleblock}
\end{frame}

\section*{Dipmeter}

\begin{frame}
  \frametitle{Dipmeter}
  \begin{center}
    \begin{figure}
      \includegraphics[width=0.55\textwidth,height=.75\textheight,keepaspectratio]{e17/Dipmeter.jpg}
      \attribcaption{Dipmeter}{(Japanische Zeichen nicht darstellbar)}{https://commons.wikimedia.org/wiki/File:Dipmeter_and_its_probe_coils.jpg}{\ccbysa}
    \end{figure}
  \end{center}
\end{frame}

\section*{SWR-Meter}

\begin{frame}
  \frametitle{Stehwellenmessgerät}
  \begin{center}
    \begin{figure}
      \includegraphics[width=1\textwidth,height=.75\textheight,keepaspectratio]{e17/RS_SWR.jpg}
      \attribcaption{SWR-Meter}{HBD20}{https://commons.wikimedia.org/wiki/File:RS_SWR.jpg}{\ccbysa}
    \end{figure}
  \end{center}
\end{frame}

\begin{frame}
  \frametitle{Interner Aufbau}
  \begin{block}{Stehwellenverhältnsis VSWR Berechnung}
    $$s = \cfrac{U_{max}}{U_{min}} = \cfrac{U_v + U_r}{U_v - U_r}$$
  \end{block}
  \begin{center}
    \begin{figure}
      \includegraphics[width=.8\textwidth,height=.5\textheight,keepaspectratio]{e17/SWRMeterInnen.png}
      \attribcaption{Interner Aufbau SWR-Meter}{Inductiveload}{https://commons.wikimedia.org/wiki/File:SWR_Meter.svg}{\ccpd}
    \end{figure}
  \end{center}
\end{frame}

\begin{frame}
  \frametitle{VSWR-Rechnen}
  \begin{exampleblock}{Beispiel}
    \begin{center}
      $$s = \frac{U_{max}}{U_{min}} = \frac{U_v + U_r}{U_v - U_r}$$
      $$s = \frac{1V + 0.5V}{1V - 0.5V} = \frac{1.5V}{0.5V} = 3$$
      $$s = \frac{1V + 0V}{1V - 0V} = \frac{1V}{1V} = 1$$
    \end{center}
  \end{exampleblock}
\end{frame}


%\begin{frame}
%  \frametitle{Prüfungsfrage}
%  \begin{center}
%    \begin{tabular}{l||p{.8\textwidth}}\hline
%      \textbf{TH401} & \textbf{Eine Antenne hat ein Stehwellenverhältnis (VSWR) von 3. Wie viel Prozent der Sendeleistung wird von der Antenne abgestrahlt, wenn sonst keine Verluste auftreten?}\\ \hline\hline
%      A & $50 \%$ \\\hline
%      B & $29 \%$ \\\hline
%      C & $25 \%$ \\ \hline
%      D \only<2>\checkmark & $75 \%$\\\hline
%    \end{tabular}\\[1.5em]
%    \only<2>{$50 \%$ der Spannung sind $25 \%$ der Leistung.}
%  \end{center}
%\end{frame}

\begin{frame}
  \frametitle{Wo das SWR-Meter einschleifen?}
  \begin{figure}
    \includegraphics[width=.97\textwidth,height=.4\textheight,keepaspectratio]{e17/SWROrt.png}
    \caption{Fragenkatalog BundesNetzAgentur Klasse E}
  \end{figure}
  \pause
  Am Besten in 1, da alles dahinter zur Antenne bzw. Antennenanlage gehört.
\end{frame}


\section*{Dummy Load}

\begin{frame}
  \frametitle{Messen}
  \begin{center}
    \begin{figure}
      \includegraphics[width=.99\textwidth,height=.75\textheight,keepaspectratio]{e17/DummyLoad.jpg}
      \attribcaption{Dummy Load}{Karl-Martin Skontorp}{https://commons.wikimedia.org/wiki/File:Dummy_load.jpg}{\ccby}
    \end{figure}
  \end{center}
\end{frame}

\begin{frame}
  \frametitle{Zu Beachten}
  \begin{itemize}
    \item Aufbau als Widerstandsdekade
    \item Genau $50 \Omega$ Realer Widerstand
    \item Lastwiderstände
    \item Am besten Metalloxid-Widerstände
    \item Große Kühlkörper
    \item Selbstbau: \url{http://der-bastelbunker.blogspot.de/2011/04/qro-dummy-load-von-kw-bis-vhf-fur-1.html}
  \end{itemize}
\end{frame}

\begin{frame}
  \frametitle{Dummy Load in der Sendeanlage Nauen}
  \begin{minipage}{0.4\textwidth}
    \begin{center}
      \begin{figure}
        \includegraphics[width=.9\textwidth,height=.8\textheight,keepaspectratio]{e17/DummyNauen.jpg}
        \attribcaption{Dummy Load in Nauen}{DB4UM}{http://www.dk0tu.de/users/DB4UM/}{}
      \end{figure}
    \end{center}
  \end{minipage}
  %\hspace{0.5cm}
  \begin{minipage}{0.55\textwidth}
    \begin{itemize}
      \item Dummyload in Nauen
      \item Auch als Wattmeter nutzbar
      \item Temperaturanstieg pro Zeiteinheit
    \end{itemize}
  \end{minipage}
\end{frame}

\section*{Referenzen}

\begin{frame}
  \frametitle{Referenzen/Links}

  \footnotesize
  \begin{itemize}
    \item Moltrecht E 17: \\
      \url{https://www.darc.de/der-club/referate/ajw/lehrgang-te/e17/}
    \item SWR-Meter Selbstbau \\
      \url{http://www.nogaqrp.org/projects/NOGAwatt/nogawattwithnewschematic2.pdf}
    \item Dummy Selbstbau \\
      \url{http://der-bastelbunker.blogspot.de/2011/04/qro-dummy-load-von-kw-bis-vhf-fur-1.html}
  \end{itemize}

\end{frame}

% Hier könnte noch eine Kontaktfolie stehen

\end{document}

