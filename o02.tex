% Foliensatz: "AFu-Kurs nach DJ4UF" von DK0TU, Amateurfunkgruppe der TU Berlin
% Lizenz: CC BY-NC-SA 3.0 de (http://creativecommons.org/licenses/by-nc-sa/3.0/de/)
% Autoren: Lars Weiler <dc4lw@darc.de>

preamble.dk0tu.tex
\usepackage{amssymb}
\subtitle{Organisatorisches 02\\ Abschluss Klasse A}
\date{Stand 20.03.2017}
 \begin{document}

\begin{frame}
    \titlepage
    \vfill
    \begin{center}
        \ccbyncsaeu\\
        {\tiny This work is licensed under the \em{Creative Commons Attribution-NonCommercial-ShareAlike 3.0 License}.}\\[0.5ex]
         \tiny Amateurfunkgruppe der Technische Universität Berlin (AfuTUB), DKØTU
         %\includegraphics[scale=0.5]{img/DK0TU_Logo.pdf}
    \end{center}
\end{frame}


\section{Rückblick}
\begin{frame}
  \frametitle{Rückblick Klasse A}
  \begin{center}
    {\Huge Was haben wir gelernt?}
  \end{center}
\end{frame}

\begin{frame}
  \frametitle{Technik}
  \begin{itemize}
    \item Vertiefung der gelernten Technik von Klasse E
    \item mehr Rechnen mit noch mehr Formeln
    \item Vertiefung von Bauelementen
    \item weitere Filter
    \item Funktionsweise von Sender und Empfänger
    \item Wellen und De-/Modulation
    \item Abschirmung und Sicherheit
  \end{itemize}
\end{frame}

\begin{frame}
  \frametitle{Lernen}
  \begin{itemize}
    \item Selbstständiges Lernen
    \item Gruppenarbeit 
    \item Prüfungsvorbereitung 
  \end{itemize}
\end{frame}

\begin{frame}
  \frametitle{Kursfahrt}
  \begin{center}
    \begin{figure}
      \includegraphics[width=1\textwidth,height=.7\textheight,keepaspectratio]{o01/P1090003.JPG}
      \attribcaption{Antenne im Gegenlicht}{DO7TWI (?)}{}{}
    \end{figure}
  \end{center}
\end{frame}

\begin{frame}
  \frametitle{Dank}
  Mein Dank geht an alle Ausbildenden
  \begin{itemize}
    \item Loria
    \item Rolf DO1RBC
    \item Andreas DJ3EI
    \item Jörg DC7BWK
    \item Stefan DD6SW
  \end{itemize}
  \pause
  \vspace{3em}
  {\Large Und euch allen für die Mithilfe bei der Kursfahrt, bei der Fehlersuche, bei unklaren Erklärungen und bei offenen Fragen!}
\end{frame}


\section{\ldots ab hier weiter}
\begin{frame}
  \begin{center}
    {\Huge Wie geht es nun weiter?}
  \end{center}
\end{frame}

\section{Aktivitäten}
\subsection{Chaoswelle}
\begin{frame}
  \frametitle{Chaoswelle-Aktivitäten}
  \begin{itemize}
    \item Bei diversen CCC-Veranstaltungen dabei
      \begin{itemize}
        \item Congress
        \item Camp 
        \item EasterHegg 
      \end{itemize}
    \item Eigene Field-Days organisieren
    \item Teilnahme am IARU- und WAG-Contest 
  \end{itemize}
\end{frame}

\begin{frame}
  \frametitle{Antennenanlage in Schöppingen}
  \begin{center}
    \begin{figure}
      \includegraphics[width=1\textwidth,height=.7\paperheight,keepaspectratio]{o01/IMG_2912.jpg}
      \attribcaption{Antennenanlage in Schöppingen}{DC4LW}{}{}
    \end{figure}
  \end{center}
\end{frame}

\subsection{Berlin}
\begin{frame}
  \frametitle{Regelmäßiges Treffen}
  \begin{itemize}
    \item \only<1>{Schnittchen und Sektempfang}\only<2>{Manöverkritik und Austausch} nach der Prüfung am \only<1>{27.\,März}\only<2>{3.\,April}
    \item dann, jeden Montag ab 19:30 Uhr im CCCB -- offen für alle Funkamateure und Interessierte
    \item Basteln, z.\,B. Antennen
    \item Röhrenfunkgerät wieder aufrüsten
    \item Clubstation (DAØCCC) aufbauen
    \item Vorträge und eigene Projekte vorstellen
    \item im Sommer im Park funken
  \end{itemize}
\end{frame}

\section{DARC e.V.}
\begin{frame}
  \frametitle{DARC e.V. Mitgliedschaft}
  \begin{itemize}
    \item Amateurfunk in Deutschland unterstützen
    \item QSL-Service nutzen
    \item Auf mehr Informationen der Webseite zugreifen
    \item Monatszeitschrift CQ\,DL
    \item Kostenfreie Gastmitgliedschaft für 6 Monate \url{http://www.darc.de/einsteiger/gast-im-darc/}
  \end{itemize}
  \vspace{1.5em}
  $\rightarrow$ \url{http://www.darc.de/darc-info/mitglied-werden/}
\end{frame}

\begin{frame}
  \frametitle{\checkmark}
  \begin{center}
    {\Huge Viel Spaß mit dem neuen Hobby!}\\[.5em]
    {\Huge Viel Erfolg bei der Prüfung!}\\[.5em]
    Und lasst uns in Kontakt bleiben.
  \end{center}
\end{frame}

% Hier könnte noch eine Kontaktfolie stehen

\end{document}

