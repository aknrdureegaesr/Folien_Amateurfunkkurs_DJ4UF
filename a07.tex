% Foliensatz: "AFu-Kurs nach DJ4UF" von DK0TU, Amateurfunkgruppe der TU Berlin
% Lizenz: CC BY-NC-SA 3.0 de (http://creativecommons.org/licenses/by-nc-sa/3.0/de/)
% Autoren: Felix Baum DB4UM <baum@campus.tu-berlin.de>
% Korrekturen: Lars Weiler DC4LW <dc4lw@darc.de>, Sebastian Lange <dl7bst@dk0tu.de>

preamble.dk0tu.tex
\subtitle{Technik A07: \\
  Oszillator und Hochfrequenzverstärker \\[2em]}
\date{Stand 30.01.2017}
 \begin{document}

\begin{frame}
    \titlepage
    \vfill
    \begin{center}
        \ccbyncsaeu\\
        {\tiny This work is licensed under the \em{Creative Commons Attribution-NonCommercial-ShareAlike 3.0 License}.}\\[0.5ex]
         \tiny Amateurfunkgruppe der Technische Universität Berlin (AfuTUB), DKØTU
         %\includegraphics[scale=0.5]{img/DK0TU_Logo.pdf}
    \end{center}
\end{frame}


\section*{Wiederholung}

\begin{frame}
  \frametitle{Verstärker Wiederholung}
  \begin{center}
    \large Um was für eine Transistorschaltung handelt es sich?
    \begin{figure}
      \includegraphics[width=0.7\textwidth,height=.6\textheight,keepaspectratio]{a07/Transistor_Verstaerker_emitter.png}
      \attribcaption{Transistorschaltung}{BNetzA}{https://www.bundesnetzagentur.de/amateurfunk/}{}
    \end{figure}
  \end{center}
\end{frame}

\begin{frame}
  \frametitle{Verstärker Wiederholung}
  \begin{center}
    \large Emitterschaltung, da der Emitter auf dem gemeinsamen Potential liegt.\\
    Phasendrehung von $180^{\circ}$\\
    \begin{figure}
      \includegraphics[width=0.7\textwidth,height=.6\textheight,keepaspectratio]{a07/Transistor_Verstaerker_emitter.png}
      \attribcaption{Transistorschaltung}{BNetzA}{https://www.bundesnetzagentur.de/amateurfunk/}{}
    \end{figure}
  \end{center}
\end{frame}

\section*{Verstärkung vs Bandbreite}

\begin{frame}
  \frametitle{Verstärkungsbandbreiteprodukt}
  \begin{center}
    \begin{figure}
      \includegraphics[width=0.7\textwidth,height=.5\textheight,keepaspectratio]{a07/Closed_loop_gain.png}
      \attribcaption{Mit sinkender Verstärkung vergrößert sich die Bandbreite. Unter Bandbreite versteht man den Bereich konstanter Verstärkung.}{Herbertweidner}{https://commons.wikimedia.org/wiki/File:Closed_loop_gain.png}{\ccpd}
    \end{figure}
    \begin{itemize}
      \item Aufgrund von Kapazitäten im Transistor geringer Welchselstromwiderstand und somit geringe Verstärkung
      \item Mit sinkender Verstärkung vergrößert sich die Bandbreite
    \end{itemize}
  \end{center}
\end{frame}

\begin{frame}
  \frametitle{Ersatzschaltbild MOSFET mit Kapazitäten}
  \begin{center}
    \begin{figure}
      \includegraphics[width=0.7\textwidth,height=.6\textheight,keepaspectratio]{a07/HF_Ersatzschaltbild.png}
      \attribcaption{Ersatzschaltbild eines HF Feldeffekttransistors}{Hubi, geändert von Markus Matern}{https://commons.wikimedia.org/wiki/File:HF-Ersatzschaltbild.svg}{\ccbysa}
    \end{figure}
    Probleme mit Kapazitäten im MOSFET (Gate und Drain/Source, wie Kapazitätsdioden)
  \end{center}
\end{frame}

\begin{frame}
  \frametitle{Breitbandverstärker}
  \begin{center}
    \begin{figure}
      \includegraphics[width=0.7\textwidth]{a07/Breitbandverstarker.png}
      \attribcaption{Breitbandverstärker mit Transistor}{Herbertweidner}{https://commons.wikimedia.org/wiki/File:Breitbandverstärker.GIF}{\ccpd}
    \end{figure}
    Ausgangslast bei HF auch kaum noch Impedanz
  \end{center}
\end{frame}

\begin{frame}
  \frametitle{Selektiver Verstärker\,/\,Schmalbandverstärker}
  \begin{center}
    \begin{figure}
      \includegraphics[width=0.7\textwidth,height=.6\textheight,keepaspectratio]{a07/Selektiver_Verstarker.png}
      \attribcaption{Selektiver Verstärker mit Transistor}{Herbertweidner}{https://commons.wikimedia.org/wiki/File:Selektiver_Verstärker.GIF}{\ccpd}
    \end{figure}
    Ausgangslast Teil des Schwingkreises, somit bei Sperrfrequenz hoher Widerstand
  \end{center}
\end{frame}

\section*{Oszillator}

\begin{frame}
  \frametitle{Rückgekoppelte Systeme / Schwingbedingungen}
  \begin{center}
    \begin{figure}
      \includegraphics[width=1\textwidth,height=.6\textheight,keepaspectratio]{a07/Lawine.jpg}
      \attribcaption{Lawine}{Ben-Zin}{https://commons.wikimedia.org/wiki/File:Lawine.jpg}{\ccpd}
    \end{figure}
    Eine Mitkopplung des Schnees $\rightarrow$ Ein wenig Schnee beginnt und reißt immer mehr mit
  \end{center}
\end{frame}

\begin{frame}
  \frametitle{Rückgekoppelte Systeme / Schwingbedingungen}
  \begin{center}
    \begin{figure}
      \includegraphics[width=1\textwidth,height=.6\textheight,keepaspectratio]{a07/Oszillator_Anschwingen.png}
      \attribcaption{Oszillogramm: Exponentieller Amplitudenanstieg eines schwach rückgekoppelten Oszillators. Nach 40$\mu s$ setzt die Amplitudenbegrenzung wegen der nichtlinearen Kennlinie ein}{Herbertweidner}{https://commons.wikimedia.org/wiki/File:Oszillator_Anschwingen.png}{\ccpd}
    \end{figure}
    Anschwingen eines Oszillators
  \end{center}
\end{frame}

\begin{frame}
  \frametitle{Rückgekoppelte Systeme / Schwingbedingungen}
  \begin{center}
    \begin{figure}
      \includegraphics[width=1\textwidth,height=.45\textheight,keepaspectratio]{a07/Gegenkopplung.png}
      \attribcaption{Blockschaltbild eines Rückkopplungskreises mit Gegenkopplung}{WolfgangS}{https://commons.wikimedia.org/wiki/File:Gegenkopplung.png}{\ccpd}
    \end{figure}
  \end{center}
  \begin{itemize}
    \item Für Mitkopplung muss Signal phasengleich sein
    \item Für Gegenkopplung muss Signal um $n \cdot 180^{\circ}$ verschoben sein
    \item Rückkopplung muss Verluste ausgleichen
    \item Zum Anschwingen Rückkopplung größer
  \end{itemize}
\end{frame}

\begin{frame}
  \begin{tabular}{l||p{.8\textwidth}}\hline
    \textbf{TTD609} & \textbf{Welche Bedingungen müssen zur Erzeugung ungedämpfter Schwingungen in Oszillatoren erfüllt sein?} \\ \hline\hline
    A & Die Schleifenverstärkung des Signalwegs im Oszillator muss kleiner als 1 sein, und das entstehende Oszillatorsignal darf auf dem Rückkopplungsweg nicht in der Phase gedreht werden. \\ \hline
    B & Die Schleifenverstärkung des Signalwegs im Oszillator muss größer als 1 sein, und das Ausgangssignal muss über den Rückkopplungsweg in der Phase so gedreht werden, dass es gegenphasig zum Ausgangspunkt zurückgeführt wird. \\ \hline
    C \only<2>\checkmark & Das an einem Schaltungspunkt betrachtete Oszillatorsignal muss auf dem Signalweg im Oszillator so verstärkt und phasengedreht werden, dass es wieder gleichphasig und mit mindestens der gleichen Amplitude zum selben Punkt zurückgekoppelt wird. \\ \hline
    D & Die Grenzfrequenz des verwendeten Verstärkerelements muss mindestens der Schwingfrequenz des Oszillators entsprechen, und das entstehende Eingangssignal muss über den Rückkopplungsweg wieder gegenphasig zum Eingang zurückgeführt werden. \\ \hline
  \end{tabular}
\end{frame}


\begin{frame}
  \frametitle{Meißner}
  \begin{center}
    \begin{figure}
      \includegraphics[width=0.67\textwidth,height=.5\textheight,keepaspectratio]{a07/Meissner_oszi.png}
      \attribcaption{Meißner Oszillator}{Appaloosa}{https://commons.wikimedia.org/wiki/File:Meissner_oszi.svg}{\ccbysa}
    \end{figure}
    \begin{itemize}
      \item Benannt nach Alexander Meißner, der 1913 patentierte
      \item Rückkopplung über Transformator
      \item $180^{\circ} \text{Transistor} + 180^{\circ} \text{Spule} = 360^{\circ}$ verschoben
    \end{itemize}
  \end{center}
\end{frame}

\begin{frame}
  \frametitle{Hartley}
  \begin{center}
    \begin{figure}
      \includegraphics[width=0.67\textwidth,height=.5\textheight,keepaspectratio]{a07/Hartley_osc.png}
      \attribcaption{Hartley Oszillator}{Omegatron, verändert durch ArdWar}{https://commons.wikimedia.org/wiki/File:Hartley_osc.svg}{\ccbysa}
    \end{figure}
    \begin{itemize}
      \item Benannt nach Ralph Hartley, der 1920 patentierte
      \item Rückkopplung über Spule, die wie Trafo wirkt
      \item Spannung am Gate bewirkt Strom aus Source
    \end{itemize}
  \end{center}
\end{frame}

\begin{frame}
  \frametitle{Colpitts}
  \begin{columns}
    \column{.35\textwidth}
    \begin{center}
      \begin{figure}
        \includegraphics[width=\textwidth,height=.7\textheight,keepaspectratio]{a07/Cc_colp2.png}
        \attribcaption{Colpitts Oszillator in Kollektorschaltung}{Radomir Sebek}{https://commons.wikimedia.org/wiki/File:Cc_colp2.png}{\ccbysa}
      \end{figure}
    \end{center}
    \column{.55\textwidth}
    \begin{itemize}
      \item Benannt nach Edwin H. Colpitts, der 1918 patentierte
      \item Rückkopplung über Kondensator
      \item Keine Phasenverschiebung da Kollektorschaltung
    \end{itemize}
  \end{columns}
\end{frame}

\begin{frame}
  \frametitle{Colpitts Beispiel}
  \begin{center}
    \begin{figure}
      \includegraphics[width=1\textwidth,height=.7\textheight,keepaspectratio]{a07/NPN_Colpitts_oscillator_collector_coil.png} \\[1em]
      \attribcaption{Colpitts in Basisschaltung}{Msiddalingaiah, verändert durch Redhatter}{https://commons.wikimedia.org/wiki/File:NPN_Colpitts_oscillator_collector_coil.svg}{\ccpd}
    \end{figure}
  \end{center}
\end{frame}

\begin{frame}
  \frametitle{Zusammenfassung Dreipunkt-Schaltungen}
  \begin{itemize}
    \item Alle Oszillatoren möglich als Basis-, Kollektor- oder Emitter-Schaltung
    \item Benannt nach Erfinder und unterschiedliche Rückkopplungen
    \item Colpitts sehr verbreitet da simple Spule
  \end{itemize}
\end{frame}

\begin{frame}
  \frametitle{Quarzoszillator}
  \begin{columns}
    \column[c]{.5\textwidth}
    \begin{figure}
      \includegraphics[width=\textwidth,height=.7\textheight,keepaspectratio]{a07/PIERCE_CRYSTAL_OSCILLATOR.jpg}
      \attribcaption{Quarzoszillator in Basis und Kollektorschaltung}{Almazi}{https://commons.wikimedia.org/wiki/File:PIERCE_CRYSTAL_OSCILLATOR.jpg}{\ccbysa}
    \end{figure}
    \column{.45\textwidth}
    \begin{itemize}
      \item Quarzoszillator in Basis und Kollektorschaltung
    \end{itemize}
  \end{columns}
\end{frame}

\begin{frame}
  \frametitle{Quarzoszillator Besonderheiten}
  \begin{columns}
    \column{.3\textwidth}
    \begin{itemize}
      \item Sehr frequenzstabil
      \item Betrieb in Oberschwingung mit Sperrkreis möglich
      \item Oberschwingungen sind ein Vielfaches der Grundfrequenz des Quarzes
    \end{itemize}
    \column{.75\textwidth}
    \begin{figure}
      \includegraphics[width=\textwidth,height=.7\textheight,keepaspectratio]{a07/TD606_quarz_Oberschwingung.png}
      \attribcaption{Oberschwingung (TD606)}{BNetzA}{https://www.bundesnetzagentur.de/amateurfunk/}{}
    \end{figure}
  \end{columns}
\end{frame}

\begin{frame}
  \begin{exampleblock}{Hausaufgabe}
    Prüfungsfragen Kapitel 1.4.6 Oszillator TD601 -- TD615
  \end{exampleblock}
\end{frame}

\begin{frame}
  \begin{center}
    \Huge Pause
  \end{center}
\end{frame}

\section*{HF-Leistungs\-verstärker}

\begin{frame}
  \frametitle{Blockschaltbild Verstärkung}
  \begin{center}
    \begin{figure}
      \includegraphics[width=1\textwidth,height=.5\textheight,keepaspectratio]{a07/paBsb.png}
      \caption{Verstärkung in einzelnen Stufen}
    \end{figure}
    \begin{itemize}
      \item Verstärkung der Leistung in Stufen
      \item Meist höchstens $10dB$ Verstärkung in den Treiberstufen
    \end{itemize}
  \end{center}
\end{frame}

\begin{frame}
  \frametitle{Wirkungsgrad}
  \begin{center}
    \begin{figure}
      \includegraphics[width=1\textwidth,height=.5\textheight,keepaspectratio]{a07/paBsb.png}
      \caption{TRX Stufen}
    \end{figure}
    \begin{block}{Wirkungsgrad}
      $\eta = \cfrac{P_{Ausgang}}{P_{Versorgung}}$
    \end{block}
  \end{center}
\end{frame}

\begin{frame}
  \frametitle{Betriebsart Transistor}
  \begin{columns}[c]
    \column[c]{5cm}
    \begin{center}
      \begin{figure}
        \includegraphics[width=1\textwidth,height=.7\textheight,keepaspectratio]{a07/TD419.png}
        \attribcaption{Betriebsarten eines Transistors (TD419)}{BNetzA}{https://www.bundesnetzagentur.de/amateurfunk/}{}
      \end{figure}
    \end{center}
    \column{5cm} \Large
    \begin{enumerate}
      \item $P_1$: C-Betrieb
      \item $P_2$: B-Betrieb
      \item $P_3$: AB-Betrieb
      \item $P_4$: A-Betrieb
    \end{enumerate}
  \end{columns}
\end{frame}

\begin{frame}
  \frametitle{Betriebsart Röhre}
  \begin{center}
    \begin{figure}
      \includegraphics[width=0.54\textwidth,height=.7\textheight,keepaspectratio]{a07/Ia_Ug_Kennlinie_ECC40.png}
      \attribcaption{Kennlinie mit Arbeitspunkten bei der Röhre ECC40}{Poc}{https://commons.wikimedia.org/wiki/File:Ia-Ug-Kennlinie-ECC40.png}{\ccpd}
    \end{figure}
  \end{center}
\end{frame}

\begin{frame}
  \frametitle{A-Betrieb}
  \begin{columns}[c]
    \column[c]{.5\textwidth}
    \begin{center}
      \begin{figure}
        \includegraphics[width=1\textwidth,height=.8\textheight,keepaspectratio]{a07/Electronic_Amplifier_Class_A.png}
        \attribcaption{Verstärker in A-Betrieb}{GRAHAMUK, verändert von Yves-Laurent Allaert}{https://commons.wikimedia.org/wiki/File:Electronic_Amplifier_Class_A.png}{\ccbysa}
      \end{figure}
    \end{center}
    \column{.45\textwidth} \large
    \begin{block}{A-Betrieb}
      \begin{enumerate}
        \item Beide Halbwellen werden verstärkt
        \item Hoher Verluststrom
        \item Kaum Signalverzerrung
        \item Einfacher Aufbau
        \item Um $40\%$ Wirkungsgrad
      \end{enumerate}
    \end{block}
  \end{columns}
\end{frame}

\begin{frame}
  \frametitle{B-Betrieb}
  \begin{columns}[c]
    \column[c]{.5\textwidth}
    \begin{center}
      \begin{figure}
        \includegraphics[width=1\textwidth,height=.8\textheight,keepaspectratio]{a07/Electronic_Amplifier_Class_B_fixed.png}
        \attribcaption{Verstärker in B-Betrieb}{Nitram cero}{https://commons.wikimedia.org/wiki/File:Electronic_Amplifier_Class_B_fixed.png}{\ccbysa}
      \end{figure}
    \end{center}
    \column{.45\textwidth}
    \begin{block}{B-Betrieb}
      \begin{enumerate}
        \item Nur die obere Halbwelle wird verstärkt
        \item Geringer Verluststrom
        \item Signalverzerrung
        \item Einfacher Aufbau
        \item Bis $80\%$ Wirkungsgrad
      \end{enumerate}
    \end{block}
  \end{columns}
\end{frame}

\begin{frame}
  \frametitle{AB-Betrieb}
  \begin{columns}[c]
    \column[c]{.5\textwidth}
    \begin{center}
      \begin{figure}
        \includegraphics[width=1\textwidth,height=.8\textheight,keepaspectratio]{a07/Electronic_Amplifier_Push-pull.png}
        \attribcaption{Verstärker in AB-Betrieb (Push-pull amplifier)}{Lakkasuo}{https://commons.wikimedia.org/wiki/File:Electronic_Amplifier_Push-pull.svg}{\ccbysa}
      \end{figure}
    \end{center}
    \column{.45\textwidth} \large
    \begin{block}{AB-Betrieb}
      \begin{enumerate}
        \item Ein Transistor pro Halbwelle
        \item Akzeptabler Verluststrom
        \item Minimale Signalverzerrung
        \item Komplizierter Aufbau
        \item Bis $75\%$ Wirkungsgrad
      \end{enumerate}
    \end{block}
  \end{columns}
\end{frame}

\begin{frame}
  \frametitle{C-Betrieb}
  \begin{columns}[c]
    \column[c]{.5\textwidth}
    \begin{center}
      \begin{figure}
        \includegraphics[width=1\textwidth,height=.8\textheight,keepaspectratio]{a07/electronic_amplifier_class_c.png}
        \attribcaption{Verstärker in C-Betrieb}{GRAHAMUK, verändert durch Yves-Laurent Allaert}{https://commons.wikimedia.org/wiki/File:Electronic_Amplifier_Class_C.png}{\ccbysa}
      \end{figure}
    \end{center}
    \column{.45\textwidth} \large
    \begin{block}{C-Betrieb}
      \begin{enumerate}
        \item Nur Signalspitze wird verstärkt
        \item Quasi kein Verluststrom
        \item Starke Signalverzerrung
        \item Einfacher Aufbau
        \item Bis $87.5\%$ Wirkungsgrad
      \end{enumerate}
    \end{block}
  \end{columns}
\end{frame}

% Uebungsaufgabe raus -> Skript
% \begin{frame}
%   \begin{tabular}{l||p{.8\textwidth}}\hline
%     \textbf{TD419} &
%     \begin{tabular}{c}
%       \includegraphics[width=\textwidth,height=.4\textheight,keepaspectratio]{a06/td419.png}
%     \end{tabular}
%     \begin{tabular}{p{.55\textwidth}}
%       \textbf{Das folgende Bild zeigt eine idealisierte Steuerkennlinie eines Transistors mit vier eingezeichneten Arbeitspunkten $P_1$ bis $P_4$. Welcher Arbeitspunkt ist welcher Verstärkerbetriebsart zuzuordnen?}
%     \end{tabular}
%     \\ \hline\hline
%     A \only<2>\checkmark & $P_1$ entspricht C-Betrieb, $P_2$ entspricht B-Betrieb, $P_3$ entspricht AB-Betrieb, $P_4$ entspricht A-Betrieb. \\ \hline
%     B & $P_2$ entspricht C-Betrieb, $P_3$ entspricht B-Betrieb, $P_4$ entspricht A-Betrieb, $P_1$ ist kein geeigneter Verstärkerarbeitspunkt. \\ \hline
%     C & $P_2$ entspricht A-Betrieb, $P_3$ entspricht B-Betrieb, $P_4$ entspricht C-Betrieb, $P_1$ ist kein geeigneter Verstärkerarbeitspunkt. \\ \hline
%     D & $P_1$ entspricht A-Betrieb, $P_2$ entspricht AB-Betrieb, $P_3$ entspricht B-Betrieb, $P_4$ entspricht C-Betrieb. \\ \hline
%   \end{tabular}
% \end{frame}

%\begin{frame}
%  \frametitle{Betriebsarten}
%  Auflösung zur Frage TD419
%  \begin{itemize}
%    \item Aus Zeiten mit Elektronenröhren als Verstärker
%    \item Gilt auch für einfache Transistorschaltungen
%    \item Häufig im Gegentaktbetrieb für beide Halbwellen
%  \end{itemize}
%  \begin{description}
%    \item[A-Betrieb] Im linearen Bereich der Verstärkungskennlinie $\rightarrow$ beide Halbwellen werden etwas verstärkt
%    \item[B-Betrieb] Nur positive Halbwellen werden verstärkt $\rightarrow$ hohe Verstärkung
%    \item[AB-Betrieb] Verstärkung für beide Halbwellen $\rightarrow$ kleine Signale wie im A-Betrieb, große wie im B-Betrieb
%    \item[C-Betrieb] Um den 0-Punkt herum $\rightarrow$ Wird bei FM-Verstärkung verwendet
%  \end{description}
%\end{frame}

\begin{frame}
  \frametitle{HF-Verstärkerschaltung}
  \begin{center}
    \begin{figure}
      \includegraphics[width=0.85\textwidth,height=.65\textheight,keepaspectratio]{a07/TG237.png}
      \attribcaption{HF-Verstärkerschaltung (TG237--TG240)}{BNetzA}{https://www.bundesnetzagentur.de/amateurfunk/}{}
    \end{figure}
  \end{center}
  Breitband HF-Verstärker aus 2 Stufen
\end{frame}

\begin{frame}
  \frametitle{HF-Verstärkerschaltung Fragen}
  \begin{columns}
    \column{.4\textwidth}
    \begin{figure}
      \includegraphics[width=0.7\textwidth,height=.5\textheight,keepaspectratio]{a07/TG237.png}
      \attribcaption{HF-Verstärkerschaltung (TG237--TG240)}{BNetzA}{https://www.bundesnetzagentur.de/amateurfunk/}{}
    \end{figure}
    \column{.6\textwidth}
    \begin{center}
      \begin{exampleblock}{TG238}
        Ist die Schaltung um den 2N3866 eine Basis, Emitter oder Kollektor Schaltung? Und wozu dient der Transformator?
      \end{exampleblock}
    \end{center}
  \end{columns}
  \pause
  \begin{exampleblock}{Antwort}
    Es handelt sich um eine Emitterschaltung. Der Transformator dient der Anpassung des Ausgangswiderstandes an den Eingang der folgenden Schaltung.
  \end{exampleblock}
\end{frame}

\begin{frame}
  \frametitle{HF-Verstärkerschaltung Fragen}
  \begin{columns}
    \column{.4\textwidth}
    \begin{figure}
      \includegraphics[width=0.7\textwidth,height=.5\textheight,keepaspectratio]{a07/TG237.png}
      \attribcaption{HF-Verstärkerschaltung (TG237--TG240)}{BNetzA}{https://www.bundesnetzagentur.de/amateurfunk/}{}
    \end{figure}
    \column{.6\textwidth}
    \begin{center}
      \begin{exampleblock}{TG239}
        Warum sind oft zwei Kondensatoren parallel gegen Masse geschaltet?
      \end{exampleblock}
    \end{center}
  \end{columns}
  \pause
  \begin{exampleblock}{Antwort}
    Der Kondensator mit der geringen Kapazität dient zum Abblocken der hohen und der Kondensator mit der hohen Kapazität zum Abblocken der niedrigen Frequenzen.
  \end{exampleblock}
\end{frame}

\begin{frame}
  \frametitle{FM-Verstärkerschaltung}
  \begin{center}
    \begin{figure}
      \includegraphics[width=1\textwidth,height=.6\textheight,keepaspectratio]{a07/TG222.png}
      \attribcaption{FM-Verstärkerschaltung (TG222--TG225)}{BNetzA}{https://www.bundesnetzagentur.de/amateurfunk/}{}
    \end{figure}
  \end{center}
  $2m$ FM-Endstufe
\end{frame}

\begin{frame}
  \frametitle{FM-Verstärkerschaltung Fragen}
  \begin{columns}
    \column{.5\textwidth}
    \begin{figure}
      \includegraphics[width=1\textwidth,height=.5\textheight,keepaspectratio]{a07/TG222.png}
      \attribcaption{FM-Verstärkerschaltung (TG222--TG225)}{BNetzA}{https://www.bundesnetzagentur.de/amateurfunk/}{}
    \end{figure}
    \column{.5\textwidth}
    \begin{center}
      \begin{exampleblock}{TG224}
        Welchem Zweck dient die Anzapfung an $L_1$ in der folgenden Schaltung?
      \end{exampleblock}
    \end{center}
  \end{columns}
  \pause
  \begin{exampleblock}{Antwort}
    Sie dient zur Anpassung der Eingangsimpedanz der Stufe.
  \end{exampleblock}
\end{frame}

\begin{frame}
  \frametitle{FM-Verstärkerschaltung Fragen}
  \begin{columns}
    \column{.5\textwidth}
    \begin{figure}
      \includegraphics[width=1\textwidth,height=.5\textheight,keepaspectratio]{a07/TG222.png}
      \attribcaption{FM-Verstärkerschaltung (TG222--TG225)}{BNetzA}{https://www.bundesnetzagentur.de/amateurfunk/}{}
    \end{figure}
    \column{.5\textwidth}
    \begin{center}
      \begin{exampleblock}{TG225}
        Welchem Zweck dient $C_2$ in der Schaltung?
      \end{exampleblock}
    \end{center}
  \end{columns}
  \pause
  \begin{exampleblock}{Antwort}
    Zur Festlegung der HF-Kopplung \\
    \emph{Merke: Bei Fragen mit Kondensatoren immer die HF-Antwort}
  \end{exampleblock}
\end{frame}

\begin{frame}
  \frametitle{HF-Verstärker mit Röhren}

  \begin{center}
    \begin{figure}
      \includegraphics[width=1\textwidth,height=.55\textheight,keepaspectratio]{a07/TG313.png}
      \attribcaption{HF-Verstärkerschaltung mit Röhren (TG313--TG318)}{BNetzA}{https://www.bundesnetzagentur.de/amateurfunk/}{}
    \end{figure}
  \end{center}
  Röhrenendstufe mit Pi-Filter ($C_1$, $C_2$, $L_1$) am Ausgang zur Anpassung an die Antenne
\end{frame}

\begin{frame}
  \frametitle{Röhrenverstärker abstimmen}
  \begin{columns}
    \column{.5\textwidth}
    \begin{figure}
      \includegraphics[width=0.65\textwidth,height=.5\textheight,keepaspectratio]{a07/TG313.png}
      \attribcaption{HF-Verstärkerschaltung mit Röhren (TG313--TG318)}{BNetzA}{https://www.bundesnetzagentur.de/amateurfunk/}{}
    \end{figure}
    \column{.5\textwidth}
    \begin{center}
      \begin{exampleblock}{TG315}
        Welche Bedeutung und Funktion haben $C_1$, $C_2$ und $L_1$? Wie sind die Bedienknöpfe der beiden Kondensatoren an einer Endstufe wahrscheinlich beschriftet?
      \end{exampleblock}
    \end{center}
  \end{columns}
  \pause
  \begin{exampleblock}{Antwort}
    An dem Drehknopf für $C_1$ steht $C_{Plate}$ oder ``Plate'', an dem für $C_2$ steht $C_{Load}$ oder ``Load''. Die drei Bauelemente $C_1$, $C_2$ und $L_1$ bilden zusammen einen so genannten Pi-Tankkreis zur Anpassung der Ausgangsimpedanz der Röhre an die Antennenimpedanz.
  \end{exampleblock}
\end{frame}

\begin{frame}
  \frametitle{Röhrenverstärker abstimmen}
  \begin{columns}
    \column{.5\textwidth}
    \begin{figure}
      \includegraphics[width=0.6\textwidth,height=.5\textheight,keepaspectratio]{a07/TG313.png}
      \attribcaption{HF-Verstärkerschaltung mit Röhren (TG313--TG318)}{BNetzA}{https://www.bundesnetzagentur.de/amateurfunk/}{}
    \end{figure}
    \column{.5\textwidth}
    \begin{center}
      \begin{exampleblock}{TG316}
        Wie wird die folgende Endstufe richtig auf die Sendefrequenz abgestimmt?
      \end{exampleblock}
    \end{center}
  \end{columns}
  \pause
  \begin{exampleblock}{Antwort}
    Zum Abstimmen $C_1$ und $C_2$ auf maximale Kapazität stellen. $C_1$ auf Dip im Anodenstrom (Resonanz) stellen, dann mit $C_2$ einen etwas höheren Anodenstrom einstellen (Leistung auskoppeln). Vorgang mit $C_1$ und $C_2$ wechselweise mehrmals wiederholen bis die maximale Ausgangsleistung erreicht ist. Nach dem Abstimmvorgang sollte ein Dip von etwa $10\%$ verbleiben.
  \end{exampleblock}
\end{frame}

\begin{frame}
  \begin{center}
    \begin{figure}
      \includegraphics[width=\textwidth,height=.8\textheight,keepaspectratio]{a07/Heathkit.jpg}
      \attribcaption{Röhrenverstärker}{DC4LW}{http://dc4lw.de}{\ccpd}
    \end{figure}
  \end{center}
\end{frame}

\section*{Leistungs\-angaben}

\begin{frame}
  \frametitle{Senderleistung}
  \begin{tabular}{l||p{.8\textwidth}}\hline
    \textbf{TB901} & \textbf{Die Ausgangsleistung eines Senders ist} \\ \hline\hline
    A \only<2>\checkmark & die unmittelbar nach dem Senderausgang messbare Leistung, bevor sie Zusatzgeräte (z.B. Anpassgeräte) durchläuft. \\ \hline
    B & die unmittelbar nach dem Senderausgang gemessene Differenz aus vorlaufender und rücklaufender Leistung. \\ \hline
    C & die unmittelbar nach den erforderlichen Zusatzgeräten (z.\,B. Anpassgeräte) messbare Leistung. \\ \hline
    D & die unmittelbar nach dem Senderausgang gemessene Summe aus vorlaufender und rücklaufender Leistung. \\ \hline
  \end{tabular}
\end{frame}

\begin{frame}
  \frametitle{Spitzenleistung}
  \begin{block}{Spitzenleistung (engl. peak envelope power, PEP)}
    PEP bezeichnet die mittlere hochfrequente Leistung am Ausgang einer Sendeendstufe, während das modulierende Signal seinen Spitzenwert hat.\\
    Wird meist bei SSB angegeben.
  \end{block}
\end{frame}

\begin{frame}
  \frametitle{Strahlungsleistung}
  \begin{block}{ERP}
    Leistung aus der Antenne im Vergleich zu Dipol
  \end{block}
  \begin{block}{EIRP}
    Leistung aus der Antenne im Vergleich zu Isotroper Kugelstrahler
  \end{block}
\end{frame}

\begin{frame}
  \frametitle{Mittlere Leistung}
  \begin{block}{Mittlere Leistung}
    Durchschnittliche Leistung, die ein Sender unter normalen Betriebsbedingungen während eines Zeitintervalls als HF-Leistung abgibt.
  \end{block}
\end{frame}

\begin{frame}
  \frametitle{Signalverzerrung}
  \begin{center}
    \begin{figure}
      \includegraphics[width=1\textwidth,height=.5\textheight,keepaspectratio]{a07/splatter.jpg}
      \attribcaption{Splatter}{Quelle unbekannt}{}{}
    \end{figure}
    Zu starke Verstärkung führt zu unlinearer Verstärkung, also Verzerrung des Signals und Splatter\\
    $10dB$ pro Dekade. Normales SSB-Signal $3kHz$, dieses $9kHz$
  \end{center}
\end{frame}

\renewcommand{\refname}{Referenzen}

\hypertarget{refs}{}
\textcolor{white}{} \\ %\vspace{} geht nicht
\Large Referenzen/Links
\footnotesize

\begin{thebibliography}{}
  \bibitem{darc}  DARC Online-Lehrgang Lektion A07:
    \url{https://www.darc.de/der-club/referate/ajw/lehrgang-ta/a07/}
  \bibitem{wp}    Wikipedia - Die freie Enzyklopädie:
    \url{https://de.wikipedia.org/wiki/Gleichrichter}
  \bibitem{bna}   Fragenkatalog Bundesnetzagentur Technik Klasse A:
    \url{https://www.bundesnetzagentur.de/SharedDocs/Downloads/DE/Sachgebiete/Telekommunikation/Unternehmen_Institutionen/Frequenzen/Amateurfunk/Fragenkatalog/TechnikFragenkatalogKlasseAf252rId9014pdf.pdf?__blob=publicationFile&v=3}
  \bibitem{fi}    Freie Inhalte (DK0TU):
    \url{http://www.dk0tu.de/Projekte/Freie_Inhalte/}
\end{thebibliography}

% Hier könnte noch eine Kontaktfolie stehen

\end{document}

