% Foliensatz: "AFu-Kurs nach DJ4UF" von DK0TU, Amateurfunkgruppe der TU Berlin
% Lizenz: CC BY-NC-SA 3.0 de (http://creativecommons.org/licenses/by-nc-sa/3.0/de/)
% Autoren:
% Martin Deutschmann
% Felix Baum <baum@campus.tu-berlin.de>
% Korrekturen:
% Lars Weiler <dc4lw@darc.de>

preamble.dk0tu.tex
\subtitle{Technik Klasse E 14: \\
  Modulation \& Demodulation \\[2em]}
\date{Stand 18.09.2017}
 \begin{document}

\begin{frame}
    \titlepage
    \vfill
    \begin{center}
        \ccbyncsaeu\\
        {\tiny This work is licensed under the \em{Creative Commons Attribution-NonCommercial-ShareAlike 3.0 License}.}\\[0.5ex]
         \tiny Amateurfunkgruppe der Technische Universität Berlin (AfuTUB), DKØTU
         %\includegraphics[scale=0.5]{img/DK0TU_Logo.pdf}
    \end{center}
\end{frame}


\section*{Prinzip der Nachrichtenübertragung}
\begin{frame}
  \frametitle{Prinzip der Nachrichtenübertragung}
  \begin{center}
    \begin{figure}
      \includegraphics[width=\textwidth,height=.45\textheight,keepaspectratio]{e14/Sender.png}\\
      \caption{Sender}
    \end{figure}
  \end{center}
  \begin{itemize}
    \item Nachrichtentechnik ist in drahtgebundene und drahtlose eingeteilt
    \item Sender muss NF-Signal in HF-Signal umwandeln
    \item Dieser Vorgang wird Modulation genannt
    \item Demodulation macht aus einem HF-Signal ein NF-Signal
  \end{itemize}
\end{frame}

\begin{frame}
  \frametitle{Sendearten (Wiederholung)}
  \begin{tiny}
    \begin{minipage}{\textwidth}

      \begin{minipage}{0.4\textwidth}
        \begin{tabular}{|l|l|}
          \hline
          \multicolumn{2}{|c|}{\textbf{1. Symbol: Modulation des Haupträgers}}\\
          \hline
          A & Zweiseitenband AM \\ \hline
          C & Restseitenband AM  \\ \hline
          F & Frequenzmodulation  \\
          G & Phasenmodulation\\ \hline
          J & Einseitenband AM, unterdrückter Träger\\ \hline
        \end{tabular}
      \end{minipage}
      \hspace{1cm}
      \begin{minipage}{0.4\textwidth}
        \begin{tabular}{|l|l|}
          \hline
          \multicolumn{2}{|c|}{\textbf{2. Symbol: Signalmodulation}}\\
          \hline
          " " & Einkanal mit quantisierter oder \\
          1  & digitaler Information ohne \\
          " " & Modulation des Hilfsträgers \\ \hline
          " " & Einkanal mit quantisierter oder\\
          2  & digitaler Information mittels \\
          " " & eines modulierten Hilfsträgers \\ \hline
          3  & Einkanal mit analoger Modulation  \\ \hline
        \end{tabular}
      \end{minipage}
    \end{minipage}
    \vspace{0.5cm}
    \begin{minipage}{0.4\textwidth}
      \begin{tabular}{|l|l|}
        \hline
        \multicolumn{2}{|c|}{\textbf{3. Symbol: Art der auszusendenden Information}}\\
        \hline
        A & Tastung durch Morsetelegrafie\\ \hline
        B & Fernschreiben \\ \hline
        C & Faksimile(Bildübertragung)\\ \hline
        D & Datenübertragung, Fernsteuerung\\ \hline
        E & Sprechfunk \\ \hline
        F & Fernsehen (Video)\\ \hline
      \end{tabular}
    \end{minipage}
  \end{tiny}
  \vspace{0.5cm}
  \begin{itemize}
    \item Es gibt eine große Vielfalt von Sendearten
    \item zum Beispiel: J3E, A1A, F3E, C3F
  \end{itemize}
\end{frame}

%\begin{frame}
%  \begin{tabular}{l||p{.8\textwidth}}\hline
%    \textbf{TD501} & \textbf{Durch Modulation...}\\ \hline\hline
%      A \only<2>\checkmark & werden Informationen auf einen oder mehrere Träger übertragen. \\ \hline
%      B & werden einem oder mehreren Trägern Informationen entnommen. \\ \hline
%      C & werden Sprach- und CW-Signale kombiniert. \\ \hline
%      D & werden dem Signal NF-Komponenten entnommen. \\ \hline
%    \end{tabular}
%\end{frame}

\begin{frame}
  \frametitle{Frequenzen eines Geigentons}
  \begin{center}
    \begin{figure}
      \includegraphics[height=0.75\textheight,width=\textwidth,keepaspectratio]{e14/geigenton.png}
      \attribcaption{Amplitude spectrum of an audio sample of a violin, open string}{Clngre derivative work: Michael Lenz}{https://commons.wikimedia.org/wiki/File:Geigentonspektrum.svg}{\ccbysa}
    \end{figure}
  \end{center}
\end{frame}

\begin{frame}
  \frametitle{Frequenzen bei Sprache}
  \begin{center}
    \begin{figure}
      \includegraphics[height=0.75\textheight,width=\textwidth,keepaspectratio]{e14/spracheFrequenzen.png}
      \caption{Frequenzssprektum bei 10sek Sprachaufnahme mit Audacity}
    \end{figure}
  \end{center}
\end{frame}

\section[Arten]{Modulationsarten}
\begin{frame}
  \frametitle{Modulationsarten}
  \begin{center}
    \begin{figure}
      \includegraphics[width=\textwidth,height=.5\textheight,keepaspectratio]{e14/modulationen.jpg}\\
      \attribcaption{verschiedene Arten der Modulation}{Berserkerus}{https://commons.wikimedia.org/wiki/File:Amfm3-en-de.gif}{\ccbysa}
    \end{figure}

    \begin{itemize}
      \item Ursprünglich waren nur Amplitudenmodulation und Frequenzmodulation im Einsatz
      \item Phasenmodulation ist im Digitalfunk zu finden
      \item der eingefügte Träger muss sinusförmig sein
    \end{itemize}
  \end{center}
\end{frame}

\subsection[AM]{Amplitudenmodulation}
\begin{frame}
  \frametitle{Amplitudenmodulation}
  \begin{minipage}{0.3\textwidth}
    \begin{figure}
      \includegraphics[width=1\textwidth,height=.75\textheight,keepaspectratio]{e14/AM1.png}
      \attribcaption{Beispiel zur Amplitudenmodulation (m=0,5)}{Erico Billich}{https://commons.wikimedia.org/wiki/File:Amplitudenmodulation3.png}{\ccbysa}
    \end{figure}
  \end{minipage}
  \hspace{0.5cm}
  \begin{minipage}{0.5\textwidth}
    \begin{itemize}
      \item wird fast nur noch im Rundfunk angewendet
      \item Information steckt in der Amplitude
      \item Überlegen: Wie verändert sich das AM Signal bei höheren/niedrigeren Frequenzen
    \end{itemize}
  \end{minipage}
\end{frame}

\begin{frame}
  \frametitle{Modulationsgrad}
  \begin{center}
    \begin{figure}
      \includegraphics[width=\textwidth,height=.75\textheight,keepaspectratio]{e14/TE103.png}
      \caption{Modulationsgrad bei AM}
    \end{figure}

    \hspace{0.5cm}

    \begin{block}{Modulationsgrad}
      $m = \cfrac{\hat{U}_{mod}}{\hat{U}_T}$
    \end{block}
  \end{center}
\end{frame}

\begin{frame}
  \frametitle{Bandbreite bei AM}
  \begin{center}
    \begin{figure}
      \includegraphics[width=.8\textwidth,height=.4\textheight,keepaspectratio]{e14/Bandbreite.png}
      \attribcaption{Bandbreite AM mit Frequenzspektrum}{Appaloosa}{https://commons.wikimedia.org/wiki/File:Am_spek1.png}{\ccbysa}
    \end{figure}
  \end{center}
  Bandbreite ist die Differenz zwischen der höchsten und der niedrigsten Frequenz des HF-Signals
  \begin{block}{Bandbreite}
    {$B = 2 \cdot f_{mod~max}$}
  \end{block}
\end{frame}

\begin{frame}
  \frametitle{Leistung bei AM}
  \begin{figure}
    \includegraphics[width=\textwidth,height=.75\textheight,keepaspectratio]{e14/AMP.png}
    \caption{Leistung bei AM}
  \end{figure}
\end{frame}

\subsection[DSBSC]{Trägerunterdrückung}
\begin{frame}
  \frametitle{Trägerunterdrückung (DSBSC)}
  \begin{itemize}
    \item Double side band suppressed carrier (DSBSC)
    \item Um Leistung zu sparen, kann der Träger unterdrückt werden
    \item Dies erfordert mehr Aufwand im Empfänger
    \item Nach dem Umwandeln in ein HF-Signal wird der Träger herausgefiltert
    \item Die beiden Seitenbänder bleiben trotzdem als HF-Signal vorhanden
    \item Wird auch Doppelseitenbandmodulation genannt
    \item Im Amateurfunk nicht zulässig
  \end{itemize}
\end{frame}

\subsection[SSB]{Einseitenbandmodulation}
\begin{frame}
  \frametitle{Einseitenbandmodulation SSB}
  \begin{small}
    \begin{itemize}
      \item Um weitere Leistung zu sparen, kann man zusätzlich noch eines der Seitenbänder unterdrücken
      \item Dadurch wird die benötigte Bandbreite halbiert
      \item dies ist möglich, da beide Seitenbänder die gleiche Information beinhalten
      \item Im Amateurfunk wird unterhalb von 10 MHz das LSB (Lower Side Band) und oberhalb von 10 MHz das USB (Upper Side Band) genutzt
      \item Die Bandbreite eines SSB-Signals ist identisch mit der Bandbreite des NF-Signals, also etwas geringer als die Hälfte der Bandbreite von AM.
    \end{itemize}
  \end{small}
\end{frame}

\begin{frame}
  \begin{center}
    \begin{figure}
      \includegraphics[width=1\textwidth,height=.9\textheight,keepaspectratio]{e16/Ssb-de.png}
      \attribcaption{SSB}{DB1BMN}{https://commons.wikimedia.org/wiki/File:Ssb-de.png}{\ccpd}
    \end{figure}
  \end{center}
\end{frame}

\subsection[FM]{Frequenzmodulation}
\begin{frame}
  \frametitle{Frequenzmodulation}
  \begin{center}
    \begin{figure}
      \includegraphics[width=\textwidth,height=.4\textheight,keepaspectratio]{e14/modulationen.jpg}
      \attribcaption{verschiedene Arten der Modulation}{Berserkerus}{https://commons.wikimedia.org/wiki/File:Amfm3-en-de.gif}{\ccbysa}
    \end{figure}
    \begin{itemize}
      \item Wird im VHF\,/\,UHF Bereich angewandt
      \item Vor allem bei mobilem Funkbetrieb
      \item Findet auch bei Packet-Radio Anwendung
      \item Information steckt in der Frequenz
      \item Amplitude bleibt konstant
    \end{itemize}
  \end{center}
\end{frame}

\begin{frame}
  \frametitle{Hub bei FM}
  \begin{center}
    \begin{figure}
      \includegraphics[width=\textwidth,height=.75\textheight,keepaspectratio]{e14/Hub.png}
      \caption{Hub}  % FIXME Autor und Lizenz
    \end{figure}
  \end{center}
\end{frame}

\begin{frame}
  \frametitle{Modulationsindex}
  Verhältnis von Frequenzhub zu Modulationsfrequenz
  \begin{block}{Modulationsindex}
    $m = \cfrac{\Delta f_T}{f_{mod}}$
  \end{block}
  \begin{description}
    \item[$m\textless2$] Schmalband-FM (NFM)
    \item[$m\geq$2] Breitband-FM (WFM)
  \end{description}
  \begin{exampleblock}{Modulationsindex}
    Amateurfunk: $m = \frac{3kHz}{3kHz} = 1$\\
    UKW-Hörfunk (mono): $m = \frac{75kHz}{15kHz} = 5$\\
  \end{exampleblock}
\end{frame}

\begin{frame}
  \frametitle{Bandbreite bei FM}
  \begin{itemize}
    \item FM erzeugt Seitenbänder
    \item Im Amateurfunk wird ein geringer Hub verwendet, der die höchste vorkommende Niederfrequenz nicht überschreitet
  \end{itemize}
  \begin{block}{Ungefähre Bandbreite (Carson Bandbreite)}
    $B = 2 \cdot (\Delta f_T + f_{mod~max})$
  \end{block}
  \begin{exampleblock}{Bandbreite}
    Amateurfunk: $B =2 \cdot (3kHz + 3kHz) = 12kHz$\\
    UKW-Hörfunk (mono): $B = 2 \cdot (75kHz + 15kHz) = 180kHz$
  \end{exampleblock}
\end{frame}

\begin{frame}
  \begin{center}
    \begin{figure}
      \includegraphics[width=1\textwidth,height=.85\textheight,keepaspectratio]{e16/Dpx-fm-radio.png}
      \attribcaption{A DPX graph of an FM-radio transmission. It's like a spectrum, but the colour shows how many times a certain level has been reached at a certain frequency.}{Casandro~commonswiki}{https://commons.wikimedia.org/wiki/File:Dpx-fm-radio.png}{\ccbysa}
    \end{figure}
  \end{center}
\end{frame}


\begin{frame}
  \frametitle{FM Vor- \& Nachteile}
  \textbf{\Large{Vorteile}}
  \begin{itemize}
    \item Störungsicher, da die Amplitude konstant bleibt
  \end{itemize}
  \vspace{1cm}
  \textbf{\Large{Nachteile}}
  \begin{itemize}
    \item benötigt mehr Bandbreite
    \item nur der stärkste Sender kann empfangen werden
  \end{itemize}
\end{frame}

%\begin{frame}
%  \begin{tabular}{l||p{.8\textwidth}}\hline
%    \textbf{TE202} & \textbf{FM hat gegenüber SSB den Vorteil der}\\ \hline\hline
%    A & geringeren Anforderungen an die Bandbreite. \\ \hline
%    B & größeren Entfernungsüberbrückung. \\ \hline
%    C \only<2>\checkmark & geringeren Beeinflussung durch Störquellen. \\ \hline
%    D & besseren Kreisgüte. \\ \hline
%  \end{tabular}
%\end{frame}

%\begin{frame}
%  \begin{tabular}{l||p{.8\textwidth}}\hline
%    \textbf{TE203} & \textbf{Ein zu großer Hub eines FM-Senders führt dazu,}\\ \hline\hline
%    A & dass Verzerrungen auf Grund gegenseitiger Auslöschung der Seitenbänder auftreten. \\ \hline
%    B \only<2>\checkmark & dass die HF-Bandbreite zu groß wird. \\ \hline
%    C & dass Verzerrungen auf Grund unerwünschter Unterdrückung der Trägerfrequenz auftreten. \\ \hline
%    D & dass die Sendeendstufe übersteuert wird. \\ \hline
%  \end{tabular}
%\end{frame}

%\begin{frame}
%  \begin{tabular}{l||p{.8\textwidth}}\hline
%    \textbf{TE204} & \textbf{Größerer Frequenzhub führt bei einem FM-Sender zu}\\ \hline\hline
%    A & einer Reduktion der Amplituden der Seitenbänder. \\ \hline
%    B & einer Erhöhung der Amplitude der Trägerfrequenz. \\ \hline
%    C & einer Erhöhung der Senderausgangsleistung. \\ \hline
%    D \only<2>\checkmark & einer größeren HF-Bandbreite. \\ \hline
%  \end{tabular}
%\end{frame}

\section*{Referenzen}
\begin{frame}
  \frametitle{Referenzen/Links}

  \footnotesize
  \begin{itemize}
    \item Moltrecht E 14: \\
      \url{https://www.darc.de/der-club/referate/ajw/lehrgang-te/e14/}
    \item Fragenkatalog der BNetzA
      \url{https://www.dk0tu.de/Kurse/AFu-Lizenz/docs/TechnikFragenkatalogKlasseE.2006-09.pdf}
  \end{itemize}

\end{frame}


% Hier könnte noch eine Kontaktfolie stehen

\end{document}

