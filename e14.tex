% Foliensatz: "AFu-Kurs nach DJ4UF" von DK0TU, Amateurfunkgruppe der TU Berlin
% Lizenz: CC BY-NC-SA 3.0 de (http://creativecommons.org/licenses/by-nc-sa/3.0/de/)
% Autoren: Martin Deutschmann

preamble.dk0tu.tex
\subtitle{Technik Klasse E 14: \\
          Modulation \& Demodulation \\[2em]}
\date{Stand 08.01.2015}
 \begin{document}

\begin{frame}
    \titlepage
    \vfill
    \begin{center}
        \ccbyncsaeu\\
        {\tiny This work is licensed under the \em{Creative Commons Attribution-NonCommercial-ShareAlike 3.0 License}.}\\[0.5ex]
         \tiny Amateurfunkgruppe der Technische Universität Berlin (AfuTUB), DKØTU
         %\includegraphics[scale=0.5]{img/DK0TU_Logo.pdf}
    \end{center}
\end{frame}


%fixme Referenzen/Fußnoten-Systematik vereinheitlichen

\section*{Prinzip der Nachrichtenübertragung}
\begin{frame}
\frametitle{Prinzip der Nachrichtenübertragung}
\begin{center}
\includegraphics[scale=1.]{e14/Sender.png}\\
Abb. 1: Sender\\
\end{center}
\begin{itemize}
	\item Nachrichtentechnik in drahtgebundene und drahtlose eingeteilt
	\item Sender muss NF-Signal in HF-Signal umwandeln
	\item Dieser Vorgang wird Modulation genannt
	\item Demodulation macht aus einem HF-Signal ein NF-Signal
\end{itemize}
\end{frame}

\begin{frame}
\frametitle{Sendearten}
\begin{tiny}
\begin{minipage}{\textwidth}

\begin{minipage}{0.4\textwidth}
	\begin{tabular}{|l|l|}
	\hline
		\multicolumn{2}{|c|}{\textbf{1. Symbol: Modulation des Haupträgers}}\\
		\hline
		A & Zweiseitenband AM \\ \hline
		C & Restseitenband AM  \\ \hline
		F & Frequenzmodulation  \\ 
		G & Phasenmodulation\\ \hline
		J & Einseitenband AM, unterdrückter Träger\\ \hline		
	\end{tabular}
\end{minipage}
\hspace{1cm}
\begin{minipage}{0.4\textwidth}
	\begin{tabular}{|l|l|}
	\hline
		\multicolumn{2}{|c|}{\textbf{2. Symbol: Signalmodulation}}\\
		\hline
		" " & Einkanal mit quantisierter oder \\
		 1  & digitaler Information ohne \\
		" " & Modulation des Hilfsträgers \\ \hline		
		" " & Einkanal mit quantisierter oder\\
		 2  & digitaler Information mittels \\
		" " & eines modulierten Hilfsträgers \\ \hline	
		 3  & Einkanal mit analoger Modulation  \\ \hline	
	\end{tabular}
\end{minipage}
\end{minipage}
\vspace{0.5cm}
\begin{minipage}{0.4\textwidth}
	\begin{tabular}{|l|l|}
	\hline
		\multicolumn{2}{|c|}{\textbf{3. Symbol: Art der auszusendenden Information}}\\
		\hline
		A & Tastung durch Morsetelegrafie\\ \hline
		B & Fernschreiben \\ \hline
	    C & Faksimile(Bildübertragung)\\ \hline		
		D & Datenübertragung, Fernsteuerung\\ \hline
		E & Sprechfunk \\ \hline
		F & Fernsehen (Video)\\ \hline	
	\end{tabular}
\end{minipage}
\end{tiny}
\vspace{0.5cm}
\begin{itemize}
	\item Es gibt eine große Vielfalt von Sendearten
	\item zum Beispiel: J3E, A1A, F3E, C3F
\end{itemize}
\end{frame}

\begin{frame}
	\begin{small}	
	\begin{tabular}{|l|l|l|}
	\hline
		\multicolumn{3}{|c|}{\textbf{TD501: Durch Modulation...}}\\
		\hline
		A & werden Informationen auf einen & ??? \\
		" " &  oder mehrere Träger übertragen. & " " \\ \hline
		B & werden einem oder mehreren & ??? \\ 
		" " & Trägern Informationen entnommen.  & " " \\ \hline
		C & werden Sprach- und CW-Signale kombiniert.  & ??? \\ \hline
		D & werden dem Signal NF-Komponenten entnommen. & ??? \\ \hline 	
	\end{tabular}
	\end{small}
\end{frame}

\begin{frame}
	\begin{small}	
	\begin{tabular}{|l|l|l|}
	\hline
		\multicolumn{3}{|c|}{\textbf{TD501: Durch Modulation...}}\\
		\hline
		A & werden Informationen auf einen & Richtig \\
		" " &  oder mehrere Träger übertragen. & " " \\ \hline
		B & werden einem oder mehreren & ??? \\ 
		" " & Trägern Informationen entnommen.  & " " \\ \hline
		C & werden Sprach- und CW-Signale kombiniert.  & ??? \\ \hline
		D & werden dem Signal NF-Komponenten entnommen. & ??? \\ \hline 	
	\end{tabular}
	\end{small}
\end{frame}

\section*{Modulationsarten}
\begin{frame}
\frametitle{Modulationsarten}
\begin{center}
\includegraphics[scale=0.8]{e14/modulationen.jpg}\\
	Abb. 2: verschiedene Arten der Modulation
	\footnote{\url{http://upload.wikimedia.org/wikipedia/commons/a/a4/Amfm3-en-de.gif}}\\
	
\begin{itemize}
	\item grundsätzlich sind nur Amplitudenmodulation und Frequenzmodulation im Einsatz
	\item der eingefügte Träger muss sinusförmig sein
\end{itemize}
\end{center}
\end{frame}

\section{Amplitudenmodulation}
\begin{frame}
\frametitle{Amplitudenmodulation}
\begin{minipage}{0.4\textwidth}
\includegraphics[scale=0.24]{e14/AM1.png}\\
\end{minipage}
\footnote{\url{http://upload.wikimedia.org/wikipedia/commons/b/b8/Amplitudenmodulation3.png}}
\hspace{0.5cm}
\begin{minipage}{0.4\textwidth}	
	\begin{itemize}
		\item wird nur noch im Rundfunk angewendet
		\item Information steckt in der Amplitude
		\item \url{http://upload.wikimedia.org/wikipedia/commons/6/62/Am2_spec.gif}
	\end{itemize}
\end{minipage}
\end{frame}

\section*{Modulationsgrad}
\begin{frame}
\frametitle{Modulationsgrad}
\begin{center}
\includegraphics[scale=0.8]{e14/TE103.png}\\
Abb. 3: Modulationsgrade bei AM

\hspace{0.5cm}

\begin{itemize}
	\item	\LARGE{$m = \frac{\hat{U_{mod}}}{\hat{U_T}}$}
\end{itemize}
\end{center}
\end{frame}

\section*{Bandbreite bei AM}
\begin{frame}
\frametitle{Bandbreite bei AM}
\begin{center}
\includegraphics[scale=0.4]{e14/Bandbreite.png}\\
\footnote{\url{http://upload.wikimedia.org/wikipedia/commons/b/b9/Am_spek1.png}}
	\begin{itemize}
		\item Bandbreite ist die Differenz zwischen der höchsten und der niedrigsten Frequenz des HF-Signals
		\item Formel: \LARGE{$b_{AM} = 2 \cdot f_{NF,max}$}
	\end{itemize}
\end{center}
\end{frame}

\section*{Leistung bei AM}
\begin{frame}
\frametitle{Leistung bei AM}
\includegraphics[scale=0.8]{e14/AMP.png}
\end{frame}

\section{Trägerunterdrückung (DSB)}
\begin{frame}
\frametitle{Trägerunterdrückung (DSB)}
\begin{itemize}
	\item Um Leistung zu sparen, kann der Träger unterdrückt werden
	\item Dies erfordert mehr Aufwand im Empfänger
	\item Nach dem umwandeln in ein HF-Signal wird der Träger herausgefiltert
	\item Die beiden Seitenbänder bleiben trotzdem als HF-Signal vorhanden
	\item Wird auch Doppelseitenbandmodulation genannt
	\item Im Amateurfunk nicht zulässig
\end{itemize}
\end{frame}

\section*{Einseitenbandmodulation SSB}
\begin{frame}
\frametitle{Einseitenbandmodulation SSB}
\begin{small}
\begin{itemize}
	\item \url{http://upload.wikimedia.org/wikipedia/commons/4/4e/Ssb_spec.gif}
	\item Um weitere Leistung zu sparen, kann man zusätzlich noch eines der Seitenbänder unterdrücken
	\item Dadurch wird die benötigte Bandbreite halbiert
	\item dies ist möglich, da beide Seitenbänder die gleiche Information beinhalten
	\item Im Amateurfunk wird unterhalb von 10 MHz das LSB und oberhalb von 10 MHz das USB genutzt	
	\item Die Bandbreite eines SSB-Signals ist identisch mit der Bandbreite des NF-Signals, also etwas geringer als die Hälfte der Bandbreite von AM.
	\item Bandbreite: $b_{SSB} = f_{NF,max} - f_{NF,min}$
	\item da $f_{NF,min}$ wesentlich kleiner als $f_{NF,max}$ gilt vereinfacht:
	\item $b_{SSB} = f_{NF,max}$
\end{itemize}
\end{small}
\end{frame}

\section*{Frequenzmodulation}
\begin{frame}
\frametitle{Frequenzmodulation}
	\begin{center}
\includegraphics[scale=0.8]{e14/modulationen.jpg}\\
	Abb. 2: verschiedene Arten der Modulation
	\footnote{\url{http://upload.wikimedia.org/wikipedia/commons/a/a4/Amfm3-en-de.gif}}\\
	\begin{itemize}
		\item Wird im VHF / UHF Bereich angewandt
		\item Vor allem bei mobilem Funkbetrieb
		\item Findet auch bei Packet-Radio Anwendung
		\item Information steckt in der Frequenz
		\item Amplitude bleibt konstant
	\end{itemize}
	\end{center}
\end{frame}

\begin{frame}
\frametitle{Hub bei FM}
\begin{center}
\includegraphics[scale=0.6]{e14/Hub.png}
\end{center}
\end{frame}

\section*{Bandbreite bei FM}
\begin{frame}
\frametitle{Bandbreite bei FM}
\begin{itemize}
	\item FM erzeugt Seitenbänder
	\item Im Amateurfunk wird ein geringer Hub verwendet, der die höchste vorkommende Niederfrequenz nicht überschreitet
	\item Dadurch gilt folgende Formel:
	\item \LARGE{$b_{FM} = 2 \cdot (\Delta f + f_{NF,max})$}
\end{itemize}
\end{frame}


\section*{FM pro \& contra}
\begin{frame}
\frametitle{FM Vor- \& Nachteile}
\textbf{\Large{Vorteile}}
\begin{itemize}
	\item Störungsicher, da die Amplitude konstant bleibt 
\end{itemize}
\vspace{1cm}
\textbf{\Large{Nachteile}}
\begin{itemize}
	\item benötigt mehr Bandbreite
	\item nur der stärkste Sender kann empfangen werden
\end{itemize}
\end{frame}


\section*{Referenzen}
\begin{frame}
    \frametitle{Referenzen/Links}
    
    \footnotesize
    \begin{itemize}
        \item Moltrecht E 14 : \\
              \url{http://www.darc.de/referate/ajw/ausbildung/darc-online-lehrgang/technik-klasse-e/technik-e14/}
		\item Fragenkatalog der BNetzA
			\url{http://www.dk0tu.de/Kurse/AFu-Lizenz/docs/TechnikFragenkatalogKlasseE.2006-09.pdf}
    \end{itemize}

\end{frame}

% Hier könnte noch eine Kontaktfolie stehen

\end{document}

