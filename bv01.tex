% Foliensatz: "AFu-Kurs nach DJ4UF" von DK0TU, Amateurfunkgruppe der TU Berlin
% Lizenz: CC BY-NC-SA 3.0 de (http://creativecommons.org/licenses/by-nc-sa/3.0/de/)
% Autoren: Sebastian Lange <dl7bst@dk0tu.de>
% Korrekturen: Lars Weiler <dc4lw@darc.de>

preamble.dk0tu.tex
\subtitle{Betriebstechnik/Vorschriften 01: \\
  Welche Rechte/Pflichten hat ein Funkamateur? \\[2em]}
\date{Stand 18.09.2017}
 \begin{document}

\begin{frame}
    \titlepage
    \vfill
    \begin{center}
        \ccbyncsaeu\\
        {\tiny This work is licensed under the \em{Creative Commons Attribution-NonCommercial-ShareAlike 3.0 License}.}\\[0.5ex]
         \tiny Amateurfunkgruppe der Technische Universität Berlin (AfuTUB), DKØTU
         %\includegraphics[scale=0.5]{img/DK0TU_Logo.pdf}
    \end{center}
\end{frame}


\section{Einleitung}

\subsection{Amateurfunk damals}

\begin{frame}
  \frametitle{Geschichte / HAM -- ein saftiger Schinken?}

  \begin{center}
    \begin{figure}
      \includegraphics[height=0.3\textheight,width=.5\textwidth,keepaspectratio]{bv01/International_amateur_radio_symbol.png}
      \attribcaption{International Amateur Radio Symbol}{Denelson83}{http://commons.wikimedia.org/wiki/File:International_amateur_radio_symbol.svg}{\ccpd}
    \end{figure}
  \end{center}

  \begin{itemize}
    \item Begriff wird für Funkamateure bereits in Pionierzeiten seit Beginn
      des 20.\,Jahrhunders verwendet
    \item Bänder und Rufzeichen damals noch völlig unreglementiert
    \item alles andere unbestätigte Mythen
      \begin{itemize}
        \item z.B. bekannteste Legende der drei Amateure \textbf{H}yman,
          \textbf{A}lmay und \textbf{M}urray
        \item es gibt verschiedene Quellen \hyperlink{refs}{\cite{ham}},
          die das mal genauer recherchiert haben
      \end{itemize}
  \end{itemize}

\end{frame}

\begin{frame}
  \frametitle{Geschichte / Manifestierung}

  Fakt ist:

  \begin{itemize}
    \item von Beginn an Interessenkonflikte zwischen kommerziellen Stationen
      und Amateuren
    \item Gewerbe wollte zuerst vorrangig Langwelle
    \item \textbf{HAM}s erforschten alles drüber und erweckten neue
      Begehrlichkeiten
    \item durch stetige Arbeit der Amateurfunkverbände gab es dann in
      vielen Bändern ``Häppchen'' für die Hobbyfunker
  \end{itemize}

\end{frame}

\begin{frame}
  \frametitle{Geschichte / Es ist noch nicht vorbei}

  \begin{center}
    \begin{figure}
      \includegraphics[height=0.3\textheight,width=.5\textwidth,keepaspectratio]{bv01/DARC_Logo.pdf}
      \caption{Logo des Deutschen Amateur-Radio-Club e.V.}
    \end{figure}
  \end{center}

  \begin{itemize}
    \item Frequenzen heute wertvoller denn je
    \item in Gremien weltweit sitzt ``unsere Lobby''
    \item uns eint kein kommerzielles Interesse, sondern der
      \emph{HAM Spirit} \hyperlink{refs}{\cite{wp}}
  \end{itemize}

\end{frame}

\subsection{Amateurfunk heute}

\begin{frame}
  \frametitle{Amateurfunk -- längst überholtes Hobby?}

  Natürlich ist es kein Problem oder Kostenfaktor mehr mit Leuten auf
  anderen Kontinenten zu kommunizieren. Aber:

  \begin{itemize}
    \item Technik in all ihren Facetten experimentieren
    \item weltweit Gleichgesinnte kennenlernen
    \item egal ob neue Innovation oder gesellschaftliche Bedeutung -- es beginnt
      immer mit dem Verstehen der technischen Grundlagen
    \item globale autarke P2P-Verbindungen kann eben \textbf{nicht} jeder
    \item direkte Kommunikation ohne Provider weltweit und darüber heraus
  \end{itemize}

\end{frame}

\section{AFu-Zeugnis}

\begin{frame}
  \frametitle{Das Amateurfunkzeugnis}

  Novice Licence (Klasse E)

  \begin{itemize}
    \item einfacherer Technikprüfungsteil
    \item erlaubte Bänder und Leistungen eingeschränkt
  \end{itemize}

  Advanced Licence (Klasse A)

  \begin{itemize}
    \item Technikfragenkatalog umfangreicher und anspruchsvoller
    \item alle Bänder und Leistungsgrenzen im gesetzlichen Umfang
      \pause
    \item ``die Lizenz zum Löten''
  \end{itemize}

\end{frame}

\subsection{Prüfung}

\begin{frame}
  \frametitle{AFu-Zeugnis / Prüfung}

  schriftliche Prüfung bei der Bundesnetzagentur (BNetzA) mit öffentlichen
  Fragenkatalogen der Bestandteile:

  \begin{itemize}
    \item Technik Klasse E oder A\footnote{bei Upgrade auf Klasse A wird nur
      der Technik-Teil geprüft}
    \item Betriebstechnik
    \item Vorschriften
  \end{itemize}

  \vspace{2em}

  Morseprüfung freiwillig und in DL \textbf{nicht} für das AFu-Zeugnis benötigt

\end{frame}

\begin{frame}
  \frametitle{AFu-Zeugnis / Prüfung}

  Bedingungen:

  \begin{itemize}
    \item \textbf{keine} Beschränkung des Alters oder der Staatsangehörigkeit
    \item Wohnsitz muss in DL sein
  \end{itemize}

  Mit bestandener Prüfung\footnote{mehr Infos: \scriptsize
  \url{http://www.dk0tu.de/Kurse/AFu-Lizenz/FAQ\#prfung}} kann die
  Amateurfunkzulassung beantragt werden.

\end{frame}

\section{AFu-Zulassung}

\begin{frame}
  \frametitle{Die Amateurfunkzulassung}

  Mit der Amateurfunkzulassung wird ein \textbf{personengebundenes(!)}
  Rufzeichen zugewiesen:

  \begin{itemize}
    \item erst ab Zuteilung darf Amateurfunkbetrieb ausgeübt werden
    \item kein Anspruch auf bestimmtes Call
    \item Rufzeichen werden ein Jahr gesperrt\footnote{z.B. bei
      vorübergehender Abmeldung oder nach Tod}
    \item Änderungen der Anschrift der ortsfesten Amateurfunkstelle
      unverzüglich anzeigen
  \end{itemize}

\end{frame}

\section{Rechtliche Eingrenzung}

\subsection{``Funkamateur''}

\begin{frame}
  \frametitle{``Funkamateur''}

  Begriff \emph{Funkamateur} im Sinne des Gesetzes:

  \begin{itemize}
    \item persönliche Neigung
    \item keine gewerblich-wirtschaftlichen Interessen
    \item darf selbst gefertigte oder umgebaute Sendeanlagen auf Amateurfunkfrequenzen betreiben
  \end{itemize}

\end{frame}

\subsection{Verbote}

\begin{frame}
  \frametitle{Verbote}

  nicht erlaubt ist

  \begin{itemize}
    \item Senden ohne Zulassung (Rufzeichen)
    \item gewerblich-wirtschaftliche Zwecke
    \item Übermitteln von Nachrichten an Dritte
    \item Nutzung von Frequenzen außerhalb des zugelassenen Nutzungsbereichs
    \item Funkverkehr mit Nicht-Amateurfunkstellen\footnote{außer in Notfällen}
    \item Übermittlung von verschlüsselten Nachrichten zum Zwecke der Verschleierung
    \item religiöse oder politische Inhalte
    \item rundfunkähnliche Aussendungen
  \end{itemize}

\end{frame}

\subsection{Strafen}

\begin{frame}
  \frametitle{Strafen}

  Die Bundesnetzagentur kann anordnen:

  \begin{itemize}
    \item Einschränkung des Betriebs oder Außerbetriebnahme der
      Amateurfunkstelle\footnote{Betrieb einer Amateurfunkstelle ohne
      AFu-Zulassung ist eine Ordnungswidrigkeit}
    \item Geldbuße
      \begin{itemize}
        \item bei Nichtzahlung ggf. Maßnahmen nach Verwaltungs-Vollstreckungsgesetz
      \end{itemize}
    \item Entzug der \emph{Zulassung zur Teilnahme am Amateurfunkdienst}
  \end{itemize}

  \vspace{1em}

  Beachte: Amateurfunkzeugnis und Sendeanlage\footnote{wenn keine Straftat
  damit verübt wurde} können nicht entzogen werden

\end{frame}

\begin{frame}

  \begin{alertblock}{Hausaufgabe (fakultativ)}
    Amateurfunkgesetz (AFuG, 13 Paragraphen auf 4 DIN-A4-Seiten) und
    Amateurfunkverordnung (AFuV, 20 Paragraphen + Anlagen auf 11
    DIN-A4-Seiten) einmal querlesen.
  \end{alertblock}

\end{frame}

\renewcommand{\refname}{Referenzen}

\begin{frame}
  \frametitle{Referenzen/Links}
  \hypertarget{refs}{}
  \footnotesize

  \begin{thebibliography}{}
    \bibitem{dj4uf} Moltrecht B/V 01: \\
      \url{https://www.darc.de/der-club/referate/ajw/lehrgang-bv/bv01/}
    \bibitem{wp}    Wikipedia DE: \\
      \url{http://de.wikipedia.org/wiki/Ham_Spirit} \\
    \bibitem{ham}   Recherchen zum Begriff ''HAM'':
      \url{http://www.indyradioclub.org/wordham.pdf} \\
      \url{http://www.w7eca.org/FILES/y_ham.doc} \\
      \url{http://www.hotarc.org/files/origin_of_HAM_by_KN4AQ.pdf}
  \end{thebibliography}

\end{frame}

% Hier könnte noch eine Kontaktfolie stehen

\end{document}

