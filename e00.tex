% Foliensatz: "AFu-Kurs nach DJ4UF" von DK0TU, Amateurfunkgruppe der TU Berlin
% Lizenz: CC BY-NC-SA 3.0 de (http://creativecommons.org/licenses/by-nc-sa/3.0/de/)
% Autoren: Sebastian Lange <dl7bst@dk0tu.de>

preamble.dk0tu.tex
\subtitle{Technik Klasse E 00: \\
          Curriculum \& Organisatorisches \\[2em]}
\date{Stand 18.09.2017}
 \begin{document}

\begin{frame}
    \titlepage
    \vfill
    \begin{center}
        \ccbyncsaeu\\
        {\tiny This work is licensed under the \em{Creative Commons Attribution-NonCommercial-ShareAlike 3.0 License}.}\\[0.5ex]
         \tiny Amateurfunkgruppe der Technische Universität Berlin (AfuTUB), DKØTU
         %\includegraphics[scale=0.5]{img/DK0TU_Logo.pdf}
    \end{center}
\end{frame}


% TODO weitere Infos ergänzen? Was ist Afu (CQ-Talk) in die Präsi mit einflechten?

\section{Vorweg}

\begin{frame}

    \begin{columns}[c]
        \column[c]{3.5cm}
        \begin{center}
            \includegraphics[height=1\textheight]{e00/watthertzmorse.jpg}
            \tiny \hyperlink{refs}{\cite{meph}}
        \end{center}
        \column{6.5cm}
            \only<1>{Vorweg: Kennt ihr die drei?}
            \only<2-4>{
                \begin{itemize}
                    \item James Watt (1736-1819), \\ schottischer Erfinder
                    \item Heinrich Hertz (1857-1894), \\ deutscher Physiker
                    \item Samuel F. B. Morse (1791-1872), \\ US-amerikanischer Erfinder
                \end{itemize}
            }
            \only<3-4>{\bigskip Telegrafie vor der Frequenz? \\[1em]}
            \only<4>{$\rightarrow$ klar, Telegrafie lange Zeit leitungsgebunden.}
    \end{columns}

\end{frame}

\section{Überblick}

\begin{frame}
    \frametitle{Überblick}

    Dieser \textbf{Grundlagenkurs (Klasse E)} geht vom Abiturwissen
    \emph{Sekundarstufe II} aus (vgl. Curriculum\hyperlink{refs}{\cite{curr}}).

    \vspace{2em}

    \includedia{e00/Zeitstrahl}
    \includegraphics[width=1\textwidth]{e00/Zeitstrahl_diatmp}\footnote{Für
    Hochschulen empfohlener Zeitplan \tiny (mit rein zufälliger Farbwahl \emph{HI})}

\end{frame}

\section{E vs. A}

\begin{frame}
    \frametitle{Klasse E vs. A}

    \textbf{Betriebstechnik und Vorschriften} sind äquivalent -- Klasse A geht
    viel tiefer in die Technik. \\[2em]

    Dafür "`winken"' als Zielprämie:

    \begin{itemize}
        \item mehr benutzbare TX-Frequenzen (alle AFu-Bänder)
        \item weitaus höhere Sendeleistungen bis zu $750W$
    \end{itemize}

\end{frame}

\section{Material}

\subsection[DARC-Lehrgang]{DARC Online-Lehrgang}

\begin{frame}
    \frametitle{DARC Online-Lehrgang}

    Wesentliche Materialgrundlage ist der deutschsprachige
    \emph{Amateurfunklehrgang\hyperlink{refs}{\cite{darc}} des
    DARC\footnote{Deutscher Amateur-Radio-Club}}.

    \begin{columns}[c]
        \column[c]{5cm}
        \begin{center}
            \includegraphics[width=0.7\textwidth]{e00/Amateurfunklehrgang-Betriebstechnik-und-Vorschriften.jpg}
            \tiny \hyperlink{refs}{\cite{darcv}}
        \end{center}
        \column{5cm}
            Inhaltlich entspricht dieser den von DJ4UF geschrieben Büchern - im
            Amaterfunk bekannt als ''Der Moltrecht''\hyperlink{refs}{\cite{dj4uf}}
    \end{columns}

\end{frame}


\begin{frame}
    \frametitle{DARC Online-Lehrgang}

    Wesentliche Materialgrundlage ist der deutschsprachige
    \emph{Amateurfunklehrgang\hyperlink{refs}{\cite{darc}} des
    DARC\footnote{Deutscher Amateur-Radio-Club}}.

    \begin{columns}[c]
        \column[c]{5cm}
        \begin{center}
            \includegraphics[width=0.7\textwidth]{e00/Amateurfunklehrgang-Technik-fuer-das-Amateurfunkzeugnis-Klasse-E.jpg}
            \tiny \hyperlink{refs}{\cite{darcv}}
        \end{center}
        \column{5cm}
            Mit Prüfungsziel \emph{Klasse A} kann auf das Klasse-E-Buch
            verzichtet werden.
    \end{columns}

\end{frame}

\subsection{Fragenkataloge}

\begin{frame}
    \frametitle{Fragenkataloge}

    Dreh- und Angelpunkt aller Kurse: Der offizielle Fragenkatalog der
    Bundesnetzagentur\footnote{als Print z.B. direkt von der BNetzA oder als PDF im WWW}.

    \begin{columns}[c]
        \column[c]{3cm}
        \begin{center}
            \includegraphics[height=0.6\textheight]{e00/Fragenkatalog-Klasse-A-und-E-Betriebliche-Kenntnisse-und-Kenntnisse-von-Vorschriften.jpg}
            \tiny \hyperlink{refs}{\cite{darcv}}
        \end{center}
        \column{7cm}
        \begin{itemize}
            \item digital verfügbar in verschiedenen Übungsprogrammen und
            Prüfungssimulatoren
            \item Empfehlung Offline-Tool: \emph{AFUTrainer}\hyperlink{refs}{\cite{afut}}
            \item Empfehlung Browser-Tool: \emph{AfuP}\hyperlink{refs}{\cite{afup}}
            \item auch kostenpflichtige Apps verfügbar
        \end{itemize}
    \end{columns}

\end{frame}

\begin{frame}
    \frametitle{Fragenkataloge}

    \begin{center}
        \includegraphics[height=0.7\textheight]{e00/Fragenkatalog-Klasse-E-Technische-Kenntnisse.jpg}
        \tiny \hyperlink{refs}{\cite{darcv}}
    \end{center}

    Auch dieser braucht für das Ziel \emph{Klasse A} nicht beachtet zu werden.

\end{frame}

\subsubsection{Formelsammlung}

\begin{frame}
    \frametitle{Formelsammlung}

    Wichtigster Auszug aus dem offiziellen Fragenkatalog der \emph{BNetzA} ist
    die Formelsammlung im Anhang. \\[3em]

    Auch wenn man sonst papierlos unterwegs ist:
    \textbf{Ausdrucken}\footnote{S.131-138 (PDF-Seiten 133-140)} lohnt sich!

\end{frame}

\section{Curriculum}

\subsection{Aufbau}

\begin{frame}
    \frametitle{Curriculum / Abhängigkeitsgraph}

    \includedia{e00/Abhaengigkeitsgraph}
    \includegraphics[width=1\textwidth]{e00/Abhaengigkeitsgraph_diatmp}

\end{frame}

\begin{frame}
    \frametitle{Curriculum / Lehreinheiten}

    Die genaue \textbf{Aufteilung der Lektionen in 12 bis 13 Lehreinheiten}
    ändert sich immer mal wieder etwas. Der Arbeitsstand kann auf der Website
    von DK0TU\hyperlink{refs}{\cite{curr}} nachgeschlagen werden. \\[2em]

    Dort verlinkt sind zu jeden Thema:

    \begin{itemize}
        \item zusätzliche Anmerkungen\footnote{die noch nicht den Weg in die
              Folien gefunden haben}
        \item entsprechendes Moltrechtkapitel
        \item Foliensatz als PDF
        \item kurze Vorbereitungsaufgaben oder Notizen
    \end{itemize}

\end{frame}

\subsection{Praxis}

\begin{frame}
    \frametitle{Praxis}

    Der Lehrstoff wird immer mit einem Praxisthema aus dem Bereich des Afu
    verbunden. Grundsätzlicher Ablauf:

    \begin{itemize}
        \item Betriebstechnik/Vorschriften
        \item Technik Klasse E
        \item Praxis
        \item[+] einmalig Kleingruppen-Zusatztermin\footnote{Pflicht für ECTS} in der
        Funkstation (max. 6 Leute)
    \end{itemize}

\end{frame}

\section{Weitere Quellen}

\subsection{Bibliothek}
\begin{frame}
    \frametitle{Weitere Quellen / Bibliothek}

    Neben dem naheliegenden Kauf in einer Buchhandlung, haben viele Bibliotheken
    wie die \emph{UB der TU Berlin}\hyperlink{refs}{\cite{ub}} einiges zum Thema
    Amateurfunk anzubieten.

\end{frame}

\subsection{WWW}

\begin{frame}
    \frametitle{Weitere Quellen / WWW}

    ''Das Internet ist ein großer Misthaufen, in dem man allerdings auch kleine
    Schätze und Perlen finden kann.''\hyperlink{refs}{\cite{quote}} \\[1em]
    \flushright --Joseph Weizenbaum \\[4em]

    \center Wo soll man da nur anfangen?

\end{frame}

\begin{frame}
    \frametitle{WWW}

    Für den Amateurfunkkurs haben wir bereits ein wenig
    Material\hyperlink{refs}{\cite{mat}} zusammengestellt. Für die schnelle
    Suche ist die Wikipedia\hyperlink{refs}{\cite{wp}} ein guter Ausgang. \\[2em]

    Wir haben auch weiterführende freie Inhalte\hyperlink{refs}{\cite{fi}}
    verlinkt. Und ansonsten wie gewohnt die \$SUCHMASCHINE eurer Wahl.

\end{frame}

\section{Fragen}

\begin{frame}
    \frametitle{Last-but-not-least}

    Disclaimer: Der Kurs findet sich in stetiger Entwicklung.

    \bigskip
    \pause

    Seid ihr bei uns angemeldet? Sonst keine Infos.

    \bigskip
    \pause

    Fragen?

\end{frame}

\renewcommand{\refname}{Referenzen}

\hypertarget{refs}{}
\textcolor{white}{} \\ %\vspace{} geht nicht
\Large Referenzen/Links
\footnotesize

\begin{thebibliography}{}
    \bibitem{meph}  \emph{Originalquelle existiert nicht mehr}
    \bibitem{darc}  DARC Online-Lehrgang Klasse E:
                    \url{http://www.darc.de/referate/ajw/ausbildung/darc-online-lehrgang/technik-klasse-e/}
    \bibitem{dj4uf} Amateurfunklehrgang Betriebstechnik und Vorschriften (E. Moltecht): \\
                    ISBN 978-3-88180-803-3 \\
                    Amateurfunklehrgang Technik Klasse E (E. Moltecht): \\
                    ISBN 978-3-88180-364-9
    \bibitem{curr}  Curriculum DK0TU Amateurfunkkurs: \\
                    \url{https://www.dk0tu.de/Kurse/AFu-Lizenz/Curriculum/}
    \bibitem{darcv} DARC Verlag:
                    \url{http://darcverlag.de/Amateurfunklehrgang-Technik-fuer-das-Amateurfunkzeugnis-Klasse-E}
    \bibitem{mat}   Material und Dokumente für den Kurs:
                    \url{https://www.dk0tu.de/Kurse/AFu-Lizenz#material}
    \bibitem{afut}  AFUTrainer von DM1OLI:
                    \url{http://www.oliver-saal.de/software/afutrainer/}
    \bibitem{afup}  Prüfungen zum Amateurfunkzeugnis vom Ortsverband A36:
                    \url{http://www.afup.a36.de/pruefungen/pruefungen.html}
    \bibitem{ub}    Universitätsbibliothek der TU Berlin:
                    \url{http://www.ub.tu-berlin.de}
    \bibitem{quote} Joseph Weizenbaum, aus Wikiquote:
                    \url{http://de.wikiquote.org/wiki/Joseph_Weizenbaum}
    \bibitem{wp}    Wikipedia - Die freie Enzyklopädie:
                    \url{http://www.wikipedia.org/}
    \bibitem{fi}    Freie Inhalte (DK0TU):
                    \url{https://www.dk0tu.de/Projekte/Freie_Inhalte/}
\end{thebibliography} 

% Hier könnte noch eine Kontaktfolie stehen

\end{document}

