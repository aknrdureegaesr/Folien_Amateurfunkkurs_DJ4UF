% Foliensatz: "AFu-Kurs nach DJ4UF" von DK0TU, Amateurfunkgruppe der TU Berlin
% Lizenz: CC BY-NC-SA 3.0 de (http://creativecommons.org/licenses/by-nc-sa/3.0/de/)
% Autoren: Sebastian Lange <dl7bst@dk0tu.de>
% Korrekturen: Lars Weiler <dc4lw@darc.de>

preamble.dk0tu.tex
\subtitle{Technik Klasse E 08: \\
  Elektromagnetisches Feld \\[2em]}
\date{Stand 18.09.2017}
 \begin{document}

\begin{frame}
    \titlepage
    \vfill
    \begin{center}
        \ccbyncsaeu\\
        {\tiny This work is licensed under the \em{Creative Commons Attribution-NonCommercial-ShareAlike 3.0 License}.}\\[0.5ex]
         \tiny Amateurfunkgruppe der Technische Universität Berlin (AfuTUB), DKØTU
         %\includegraphics[scale=0.5]{img/DK0TU_Logo.pdf}
    \end{center}
\end{frame}


\section*{Elektrisches Feld}
\begin{frame}
  \frametitle{Das elektrische Feld}
  \begin{center}
    \begin{figure}
      \includegraphics[width=\textwidth,height=.6\textheight,keepaspectratio]{e08/PlattenkondensatorFeld.png}
      \attribcaption{Elektrisches Feld zwischen zwei leitenden Platten}{wdwd}{https://commons.wikimedia.org/wiki/File:PlattenkondensatorFeld.svg}{\ccby}
    \end{figure}
    \begin{itemize}
      \item wird durch Spannung erzeugt
      \item ist homogen zwischen zwei parallelen Platten
    \end{itemize}
  \end{center}
\end{frame}

\begin{frame}
  \begin{block}{Elektrisches Feld}
    \begin{center}
      \huge$E = \cfrac{U}{d} \text{ in } [\frac{V}{m}]$\\
    \end{center}
  \end{block}
\end{frame}

% von e06 übernommen
\begin{frame}
  \frametitle{Das magnetische Feld}

  \begin{center}
    \begin{minipage}{0.45\textwidth}
      \begin{center}
        \begin{figure}
          \includegraphics[width=\textwidth,height=.3\textheight,keepaspectratio]{e08/RechteHand.png}
          \attribcaption{Magnetisches Feld um einen Leiter}{Smial}{https://commons.wikimedia.org/wiki/File:RechteHand.png}{\ccbysa}
        \end{figure}
      \end{center}
    \end{minipage}
    \begin{minipage}{0.45\textwidth}
      \begin{center}
        \begin{figure}
          \includegraphics[width=\textwidth,height=.3\textheight,keepaspectratio]{e08/VFPt_cylindrical_coil_real.png}\\
          \attribcaption{magnetisches Feld in einer Spule}{Geek3}{https://commons.wikimedia.org/wiki/File:VFPt_cylindrical_coil_real.svg}{\ccbysa}
        \end{figure}
      \end{center}
    \end{minipage}

    \bigskip

    \begin{itemize}
      \item um jeden stromdurchflossenen Leiter baut sich ein konzentrisches, magnetisches Feld auf
      \item magnetische Felder summieren sich in einer Spule
      \item magnetische Felder in einer Spule sind homogen
      \item wird mit zunehmendem Strom stärker
      \item nimmt mit zunehmendem Abstand ab
    \end{itemize}

  \end{center}
\end{frame}

\section*{Elektro\-magnetisches Feld}

\begin{frame}
  \frametitle{Das elektromagnetische Feld}

  \begin{center}
    \begin{figure}
      \only<1>{\includegraphics[width=\textwidth,height=.45\textheight,keepaspectratio]{e08/Dipolentstehung-0.png}}
      \only<2>{\includegraphics[width=\textwidth,height=.45\textheight,keepaspectratio]{e08/Dipolentstehung-1.png}}
      \only<3>{\includegraphics[width=\textwidth,height=.45\textheight,keepaspectratio]{e08/Dipolentstehung-2.png}}
      \only<4>{\includegraphics[width=\textwidth,height=.45\textheight,keepaspectratio]{e08/Dipolentstehung-3.png}}
      \only<5>{\includegraphics[width=\textwidth,height=.45\textheight,keepaspectratio]{e08/Dipolentstehung-4.png}}
      \only<6>{\includegraphics[width=\textwidth,height=.45\textheight,keepaspectratio]{e08/Dipolentstehung-5.png}}
      \only<7>{\includegraphics[width=\textwidth,height=.45\textheight,keepaspectratio]{e08/Dipolentstehung-6.png}}
      \only<8>{\includegraphics[width=\textwidth,height=.45\textheight,keepaspectratio]{e08/Dipolentstehung-7.png}}
      \only<9>{\includegraphics[width=\textwidth,height=.45\textheight,keepaspectratio]{e08/Dipolentstehung-8.png}}
      \only<10>{\includegraphics[width=\textwidth,height=.45\textheight,keepaspectratio]{e08/Dipolentstehung-9.png}}
      \only<11>{\includegraphics[width=\textwidth,height=.45\textheight,keepaspectratio]{e08/Dipolentstehung-10.png}}
      \only<12>{\includegraphics[width=\textwidth,height=.45\textheight,keepaspectratio]{e08/Dipolentstehung-11.png}}
      \attribcaption{Dipolentstehung}{Averse}{https://commons.wikimedia.org/wiki/File:Dipolentstehung.gif}{\ccbysa}
    \end{figure}
  \end{center}

  \begin{itemize}
    \item elektromagnetisches Feld bildet sich durch ein sich änderndes
      elektrisches und ein sich änderndes magnetisches Feld
    \item zieht man die Kondensatorplatten auseinander und streckt die Spule,
      erhält man eine Dipolantenne
  \end{itemize}

\end{frame}

\begin{frame}
  \centering
  \begin{figure}
    \includegraphics[width=\textwidth,height=1.5\textheight,keepaspectratio,angle=270]{e08/AntSchwingkreis-crop.pdf}\\[1em]
    \caption{Elektromagnetisches Feld im geschlossenen und offenen Schwingreis}
  \end{figure}
\end{frame}

\section*{Polarisation}

\begin{frame}
  \frametitle{Polarisation}

  \begin{center}
    \begin{minipage}{0.45\textwidth}
      \begin{figure}
        \only<1>{\includegraphics[width=\textwidth,height=.5\textheight,keepaspectratio]{e08/Dipole_xmting_antenna_animation_4_408x318x150ms-0.png}}
        \only<2>{\includegraphics[width=\textwidth,height=.5\textheight,keepaspectratio]{e08/Dipole_xmting_antenna_animation_4_408x318x150ms-1.png}}
        \only<3>{\includegraphics[width=\textwidth,height=.5\textheight,keepaspectratio]{e08/Dipole_xmting_antenna_animation_4_408x318x150ms-2.png}}
        \only<4>{\includegraphics[width=\textwidth,height=.5\textheight,keepaspectratio]{e08/Dipole_xmting_antenna_animation_4_408x318x150ms-3.png}}
        \only<5>{\includegraphics[width=\textwidth,height=.5\textheight,keepaspectratio]{e08/Dipole_xmting_antenna_animation_4_408x318x150ms-4.png}}
        \only<6>{\includegraphics[width=\textwidth,height=.5\textheight,keepaspectratio]{e08/Dipole_xmting_antenna_animation_4_408x318x150ms-5.png}}
        \only<7>{\includegraphics[width=\textwidth,height=.5\textheight,keepaspectratio]{e08/Dipole_xmting_antenna_animation_4_408x318x150ms-6.png}}
        \only<8>{\includegraphics[width=\textwidth,height=.5\textheight,keepaspectratio]{e08/Dipole_xmting_antenna_animation_4_408x318x150ms-7.png}}
        \attribcaption{Elektromagnetisches Feld einer Antenne}{Chetvorno}{https://commons.wikimedia.org/wiki/File:Dipole_xmting_antenna_animation_4_408x318x150ms.gif}{\cczero}
      \end{figure}
    \end{minipage}
    \begin{minipage}{0.45\textwidth}
      \begin{figure}
        \includegraphics[width=\textwidth,height=.8\textheight,keepaspectratio]{e08/Onde_electromagnetique.png}
        \attribcaption{Polarisation einer Antenne}{SuperManu}{https://commons.wikimedia.org/wiki/File:Onde_electromagnetique.svg}{\ccbysa}
      \end{figure}
    \end{minipage}\\[1.5em]

    \begin{itemize}
      \item E-Feld bestimmt die Richtung der Polarisation
      \item magnetisches Feld ist um $90^\circ$ gedreht
    \end{itemize}

  \end{center}

\end{frame}

\section*{Wellenlänge}

\begin{frame}
  \frametitle{Wellenlänge}

  \begin{center}
    \begin{block}{Wellenlänge $\lambda$}
      \begin{center}
        {\Huge$ \lambda [m] = \cfrac{c}{f [Hz]} \Rightarrow
        c = f \cdot \lambda $} \\[1em]
        mit $c$: Lichtgeschwindigkeit $299\,792\,458 \frac{m}{s} \approx
        300\cdot10^6\frac{m}{s}$
      \end{center}
    \end{block}

    \begin{normalsize}
      \begin{itemize}
        \item elektromagnetische Wellen breiten sich im Freiraum fast mit Lichtgeschwindigkeit aus
        \item Beispiel: in einer Sekunde bewegt sich eine elektromagnetische Welle mehr als sieben mal um die Erde
      \end{itemize}
    \end{normalsize}
  \end{center}
\end{frame}

\begin{frame}
  \begin{center}
    \begin{Large}
      \begin{tabular}{|l|l|l|}
        \hline
        Frequenz & Wellenlänge & Abkürzung\\
        \hline \hline
        3 - 30 kHz     & $10^{4}m$    & VLF \\ \hline
        \textbf{30 - 300 kHz}   & \textbf{$10^{3}m$}      & \textbf{LF}  \\ \hline
        \textbf{300 - 3000 kHz} & \textbf{$10^{2}m$}     & \textbf{MF}  \\ \hline
        \textbf{3 - 30 MHz}     & \textbf{$10^{1}m$}      & \textbf{HF}  \\ \hline
        \textbf{30 - 300 MHz}   & \textbf{$10^{0}m$}          & \textbf{VHF} \\ \hline
        \textbf{300 - 3000 MHz} & \textbf{$10^{-1}m$}      & \textbf{UHF} \\ \hline
        \textbf{3 - 30 GHz}     & \textbf{$10^{-2}m$}     & \textbf{SHF} \\ \hline
        30 - 300 GHz   & $10^{-3}m$     & EHF \\ \hline
        300 - 3000 GHz & $10^{-4}m$ & " " \\ \hline
      \end{tabular}\\[1em]
      {\scriptsize Tabelle 1: Wellenbereiche (fett: Bereiche des Amateurfunks)}
    \end{Large}
  \end{center}
\end{frame}

%\section*{Übung}
%
%\begin{frame}
%  \begin{center}
%        \begin{tabular}{l||p{.8\textwidth}}\hline
%            \textbf{TB403} & \textbf{Wenn Strom durch einen gestreckten Leiter fließt, entsteht ein \ldots} \\\hline\hline
%            A & elektrisches Feld aus konzentrischen Kreisen um den Leiter. \\ \hline
%            B \only<2>\checkmark & Magnetfeld aus konzentrischen Kreisen um den Leiter. \\ \hline
%            C & homogenes Magnetfeld um den Leiter. \\ \hline
%            D & homogenes elektrisches Feld um den Leiter. \\ \hline
%      \end{tabular}
%  \end{center}
%\end{frame}
%
%\begin{frame}
%  \begin{center}
%        \begin{tabular}{l||p{.8\textwidth}}\hline
%           \textbf{TB301} & \textbf{Welche Einheit wird für die elektrische Feldstärke verwendet?} \\\hline\hline
%           A & Watt pro Quadratmeter ($W/m^2$) \\ \hline
%           B & Ampere pro Meter ($A/m$) \\ \hline
%           C & Henry pro Meter ($H/m$) \\ \hline
%           D \only<2>\checkmark & Volt pro Meter ($V/m$) \\ \hline
%      \end{tabular}
%    \end{center}
%\end{frame}
%
%\begin{frame}
%  \begin{center}
%        \begin{tabular}{l||p{.8\textwidth}}\hline
%           \textbf{TB401} & \textbf{Welche Einheit wird für die magnetische Feldstärke verwendet?} \\\hline\hline
%           A & Watt pro Quadratmeter ($W/m^2$) \\ \hline
%           B \only<2>\checkmark & Ampere pro Meter ($A/m$) \\ \hline
%           C & Henry pro Meter ($H/m$) \\ \hline
%           D & Volt pro Meter ($V/m$) \\ \hline
%      \end{tabular}
%  \end{center}
%\end{frame}
%
%\begin{frame}
%    \begin{center}
%        \begin{tabular}{l||p{.8\textwidth}}\hline
%            \textbf{TB501} & \textbf{Wodurch entsteht ein elektromagnetisches Feld?
%                                     Ein elektromagnetisches Feld entsteht,} \\\hline\hline
%            A \only<2>\checkmark & wenn ein zeitlich schnell veränderlicher
%                                   Strom durch einen elektrischen Leiter
%                                   fließt, dessen Länge mindestens 1/100 der Wellenlänge ist. \\ \hline
%            B & wenn durch einen elektrischen Leiter, dessen Länge mindestens
%                1/100 der Wellenlänge ist, ein konstanter Strom fließt. \\ \hline
%            C & wenn sich elektrische Ladungen in einem Leiter befinden, dessen
%                Länge mindestens 1/100 der Wellenlänge ist. \\ \hline
%            D & wenn an einem elektrischen Leiter, dessen Länge mindestens
%                1/100 der Wellenlänge ist, eine konstante Spannung angelegt wird. \\ \hline
%      \end{tabular}
%  \end{center}
%\end{frame}
%
%\begin{frame}
%  \begin{center}
%        \begin{tabular}{l||p{.8\textwidth}}\hline
%            \textbf{TB504} & \textbf{Der Winkel zwischen den elektrischen und magnetischen Feldkomponenten eines elektromagnetischen Feldes beträgt im Fernfeld}\\ \hline\hline
%            A & $45^\circ$. \\ \hline
%            B \only<2>\checkmark & $90^\circ$. \\ \hline
%            C & $180^\circ$. \\ \hline
%            D & $360^\circ$. \\ \hline
%      \end{tabular}
%  \end{center}
%\end{frame}
%
%\begin{frame}
%  \begin{center}
%        \begin{tabular}{l||p{.8\textwidth}}\hline
%            \textbf{TI201} & \textbf{Die Ausbreitungsgeschwindigkeit freier elektromagnetischer Wellen beträgt etwa \ldots}\\ \hline\hline
%            A & $3\,000\,000\,km/s$. \\ \hline
%            B & $30\,000\,km/s$. \\ \hline
%            C \only<2>\checkmark & $300\,000\,km/s$. \\ \hline
%            D & $3\,000\,km/s$. \\ \hline
%      \end{tabular}
%    \end{center}
%\end{frame}
%
%\begin{frame}
%  \begin{center}
%        \begin{tabular}{l||p{.8\textwidth}}\hline
%            \textbf{TB602} & \textbf{Welcher Wellenlänge $\lambda$ entspricht die Frequenz 1,84MHz?} \\ \hline\hline
%            A & 16,3m \\ \hline
%            B \only<2>\checkmark & 163m \\ \hline
%            C & 0,613m \\ \hline
%            D & 61,3m \\ \hline
%      \end{tabular}
%  \end{center}
%\end{frame}
%
%\begin{frame}
%  \begin{center}
%       \begin{tabular}{l||p{.8\textwidth}}\hline
%         \textbf{TB604} & \textbf{Eine Wellenlänge von 2,06m entspricht einer Frequenz von~\ldots} \\ \hline\hline
%           A & 135,754 MHz \\ \hline
%           B & 148,927 MHz \\ \hline
%           C & 150,247 MHz \\ \hline
%           D \only<2>\checkmark & 145,631 MHz \\ \hline
%      \end{tabular}
%  \end{center}
%\end{frame}
%
%\begin{frame}
%  \begin{center}
%       \begin{tabular}{l||p{.8\textwidth}}\hline
%         \textbf{TB609} & \textbf{Das 70-cm-Band befindet sich im \ldots} \\ \hline\hline
%           A & VHF-Bereich. \\ \hline
%           B \only<2>\checkmark & UHF-Bereich. \\ \hline
%           C & SHF-Bereich. \\ \hline
%           D & EHF-Bereich. \\ \hline
%        \end{tabular}
%  \end{center}
%\end{frame}

\section*{Referenzen}
\begin{frame}
  \frametitle{Referenzen/Links}

  \footnotesize
  \begin{itemize}
    \item Moltrecht E 08 : \\
      \url{https://www.darc.de/der-club/referate/ajw/lehrgang-te/e08/}
  \end{itemize}

\end{frame}

% Hier könnte noch eine Kontaktfolie stehen

\end{document}

