% Foliensatz: "AFu-Kurs nach DJ4UF" von DK0TU, Amateurfunkgruppe der TU Berlin
% Lizenz: CC BY-NC-SA 3.0 de (http://creativecommons.org/licenses/by-nc-sa/3.0/de/)
% Autoren: Martin Deutschmann <martin.deutschmann@campus.tu-berlin.de>
% Korrekturen: Lars Weiler <dc4lw@darc.de>, Sebastian Lange <dl7bst@dk0tu.de>

preamble.dk0tu.tex
\subtitle{Technik Klasse A 06: \\
  Transistor \& Verstärker \\[2em]}
\date{Stand 04.05.2016}
 \begin{document}

\begin{frame}
    \titlepage
    \vfill
    \begin{center}
        \ccbyncsaeu\\
        {\tiny This work is licensed under the \em{Creative Commons Attribution-NonCommercial-ShareAlike 3.0 License}.}\\[0.5ex]
         \tiny Amateurfunkgruppe der Technische Universität Berlin (AfuTUB), DKØTU
         %\includegraphics[scale=0.5]{img/DK0TU_Logo.pdf}
    \end{center}
\end{frame}


\section*{Aufbau}

\begin{frame}
  \frametitle{Bipolarer Transistor}
  \begin{columns}
    \column{.4\textwidth}
    \begin{figure}
      \includegraphics[width=\textwidth,height=.5\textheight,keepaspectratio]{a06/NPN_hlb.png}
      \caption{Schichten eines NPN-Transistors}
    \end{figure}
    \column{.4\textwidth}
    \begin{figure}
      \includegraphics[width=\textwidth,height=.5\textheight,keepaspectratio]{a06/PNP_hlb.png}
      \caption{Schichten eines PNP-Transistors}
    \end{figure}
  \end{columns}
  \begin{center}
    \begin{itemize}
      \item Transistoren bestehen aus drei Halbleiterschichten
      \item Anschlüsse: Basis (B), Kollektor (C), Emitter (E)
    \end{itemize}
  \end{center}
\end{frame}

\begin{frame}
  \frametitle{Ersatzschaltbild}
  \begin{columns}
    \column{.4\textwidth}
    \begin{figure}
      \includegraphics[width=\textwidth,height=.5\textheight,keepaspectratio]{a06/NPN_esb.png}
      \caption{ESB eines NPN-Transistors}
    \end{figure}
    \column{.4\textwidth}
    \begin{figure}
      \includegraphics[width=\textwidth,height=.5\textheight,keepaspectratio]{e13/PNP_esb.png}
      \caption{ESB eines PNP-Transistors}
    \end{figure}
  \end{columns}
  \begin{itemize}
    \item Basis-Emitterübergang muss in Durchlassrichtung gepolt sein
    \item Basis braucht ein um etwa 0,6 V höheres Potential als der Emitter
  \end{itemize}
\end{frame}


\section*{FET}
\begin{frame}
  \frametitle{Feldeffekt Transistor (FET)}
  \begin{center}
    \begin{figure}
      \includegraphics[width=\textwidth,height=.6\textheight,keepaspectratio]{a06/FET-aufbau.png}
      \attribcaption{Eingefärbtes Schema eines n-Kanal Sperrschicht-FET (JFET)}{Arne Nordmann (norro)}{https://commons.wikimedia.org/wiki/File:Scheme_of_n-junction_field-effect_transistor_de.svg}{\ccbysa}
    \end{figure}
    \begin{itemize}
      \item Gate steuert Kanalbreite durch Spannung
      \item Je dünner der Kanal, desto höher ist der Kanalwiderstand
    \end{itemize}
  \end{center}
\end{frame}

\begin{frame}
  \begin{center}
    \begin{figure}
      \includegraphics[width=\textwidth,height=.85\textheight,keepaspectratio]{a06/Isolierschicht-FET-intern.png}
      \attribcaption{Eingefärbter Technologiequerschnitt eines Metall-Oxid-Halbleiters FET (MOSFET)}{Arne Nordmann (norro)}{http://commons.wikimedia.org/wiki/File:Scheme_of_metal_oxide_semiconductor_field-effect_transistor.svg}{\ccbysa}
    \end{figure}
  \end{center}
\end{frame}

\begin{frame}
  \frametitle{Weitere FET-Arten}
  \begin{center}
    \begin{figure}
      \includegraphics[width=\textwidth,height=.8\textheight,keepaspectratio]{a06/FET-overview.png}
      \attribcaption{Übersicht von Feldeffekttransistoren mit zugehörigen Schaltzeichen (DIN)}{Cepheiden}{https://commons.wikimedia.org/wiki/File:FET-Typen_(mit_Schaltbildern).svg?uselang=de}{\ccpd}
    \end{figure}
  \end{center}
\end{frame}

\begin{frame}
  \begin{center}
    \begin{figure}
      \includegraphics[width=\textwidth,height=.85\textheight,keepaspectratio]{a06/MOSFET-Symbole.png}
      \attribcaption{Schaltsymbole von MISFET-Transistoren (metal insulator semiconductor)}{Arne Nordmann (norro)}{https://commons.wikimedia.org/wiki/File:MISFET-Transistor_Symbole.svg?uselang=de}{\ccbysa}
    \end{figure}
  \end{center}
\end{frame}

\begin{frame}
  \begin{center}
    \begin{figure}
      \includegraphics[width=\textwidth,height=.6\textheight,keepaspectratio]{a06/Dual-Gate-MOSFET.png}
      \caption{Schaltzeichen eines Dual-Gate-MOSFETs}
    \end{figure}
  \end{center}
  \begin{itemize}
    \item Besitzt zwei Gateanschlüsse
    \item Wird für Mischerschaltungen genutzt
  \end{itemize}
\end{frame}

\begin{frame}
  \begin{tabular}{p{2.5pc}||c||p{2.5pc}||c}\hline
    \textbf{TC604} & \multicolumn{3}{p{.8\textwidth}}{\textbf{Welcher der folgenden Transistoren ist ein selbstleitender P"~Kanal MOSFET}} \\ \hline\hline
    A & \includegraphics[width=.35\textwidth,height=.35\textheight,keepaspectratio]{a06/tc604c.png} &
    B & \includegraphics[width=.35\textwidth,height=.35\textheight,keepaspectratio]{a06/tc604d.png} \\ \hline
    C \only<2>\checkmark & \includegraphics[width=.35\textwidth,height=.35\textheight,keepaspectratio]{a06/tc604a.png} &
    D & \includegraphics[width=.35\textwidth,height=.35\textheight,keepaspectratio]{a06/tc604b.png} \\ \hline
  \end{tabular}
\end{frame}

\begin{frame}
  \begin{tabular}{p{2.5pc}||c||p{2.5pc}||c}\hline
    \textbf{TC605} & \multicolumn{3}{p{.8\textwidth}}{\textbf{Welcher der folgenden Transistoren ist ein selbstsperrender N"~Kanal MOSFET}} \\ \hline\hline
    A \only<2>\checkmark & \includegraphics[width=.35\textwidth,height=.35\textheight,keepaspectratio]{a06/tc604c.png} &
    B & \includegraphics[width=.35\textwidth,height=.35\textheight,keepaspectratio]{a06/tc604d.png} \\ \hline
    C & \includegraphics[width=.35\textwidth,height=.35\textheight,keepaspectratio]{a06/tc604a.png} &
    D & \includegraphics[width=.35\textwidth,height=.35\textheight,keepaspectratio]{a06/tc604b.png} \\ \hline
  \end{tabular}
\end{frame}


\section*{Bipolar\-transistor}
\begin{frame}
  \frametitle{Anwendungen von Bipolartransistoren}
  \begin{columns}
    \column{.4\textwidth}
    \begin{center}
      \begin{figure}
        \includegraphics[width=\textwidth,height=.7\textheight,keepaspectratio]{a06/Transistor-Schalter.png}
        \attribcaption{Transistor als Schalter}{BNetzA}{https://www.bundesnetzagentur.de/amateurfunk/}{}
      \end{figure}
    \end{center}
    \column{.4\textwidth}
    \begin{center}
      \begin{figure}
        \includegraphics[width=\textwidth,height=.7\textheight,keepaspectratio]{a06/Transistor-Verstaerker.png}
        \attribcaption{Transistor als Verstärker}{BNetzA}{https://www.bundesnetzagentur.de/amateurfunk/}{}
      \end{figure}
    \end{center}
  \end{columns}
  Transistoren können als \textbf{Schalter} oder als \textbf{Verstärker} genutzt werden
\end{frame}

\section*{Anwendungen}
\subsection*{Transistor als Schalter}
\begin{frame}
  \frametitle{Transistor als Schalter}
  \begin{columns}
    \column{.4\textwidth}
    \begin{center}
      \begin{figure}
        \includegraphics[width=\textwidth,height=.85\textheight,keepaspectratio]{a06/Transistor-Schalter.png}
        \attribcaption{Transistor als Schalter mit Belastung gegen Masse}{BNetzA}{https://www.bundesnetzagentur.de/amateurfunk/}{}
      \end{figure}
    \end{center}
    \column{.5\textwidth}
    \begin{itemize}
      \item $U_e = 0 V \rightarrow$ Transistor sperrt\\ $\rightarrow U_a = U_B$\\ $\Rightarrow$ Eingang 0, Ausgang 1
      \item $U_e \gg 0,6 V \rightarrow$ Transistor leitet\\ $\rightarrow U_a \approx 0.1 V$\\ $\Rightarrow$ Eingang 1, Ausgang 0
      \item Der Transistor erfüllt hier die Funktion eines \textbf{Inverters}
    \end{itemize}
  \end{columns}
\end{frame}

\begin{frame}
  \frametitle{Transistor mit Koppelkondensator als Schalter}
  \begin{columns}
    \column{.4\textwidth}
    \begin{center}
      \begin{figure}
        \includegraphics[width=\textwidth,height=.85\textheight,keepaspectratio]{a06/Transistor-Schalter+C.png}
        \attribcaption{Transistor mit Koppelkondensator als Schalter}{BNetzA}{https://www.bundesnetzagentur.de/amateurfunk/}{}
      \end{figure}
    \end{center}
    \column{.4\textwidth}
    \begin{itemize}
      \item Der Kondensator blockt die Gleichspannung und bildet den Mittelwert der Wechselspannung
    \end{itemize}
  \end{columns}
\end{frame}

\begin{frame}
  \begin{tabular}{p{2.5pc}||l||p{2.5pc}||l}\hline
    \textbf{TD431} & \multicolumn{3}{p{.8\textwidth}}{
    \begin{tabular}{c}
      \includegraphics[width=\textwidth,height=.45\textheight,keepaspectratio]{a06/td431.png}
    \end{tabular}
    \begin{tabular}{p{.55\textwidth}}
      \textbf{An den Eingang dieser Schaltung wird das folgende Signal gelegt. Welches ist ein mögliches Ausgangssignal $U_A$?}
    \end{tabular}
    } \\ \hline\hline
    A & \includegraphics[width=.25\textwidth,height=.25\textheight,keepaspectratio]{a06/td431d.png} &
    B & \includegraphics[width=.25\textwidth,height=.25\textheight,keepaspectratio]{a06/td431c.png} \\ \hline
    C & \includegraphics[width=.25\textwidth,height=.25\textheight,keepaspectratio]{a06/td431b.png} &
    D \only<2>\checkmark & \includegraphics[width=.25\textwidth,height=.25\textheight,keepaspectratio]{a06/td431a.png} \\ \hline
  \end{tabular}
\end{frame}

\begin{frame}
  \begin{tabular}{p{2.5pc}||l||p{2.5pc}||l}\hline
    \textbf{TD432} & \multicolumn{3}{p{.8\textwidth}}{
    \begin{tabular}{c}
      \includegraphics[width=\textwidth,height=.45\textheight,keepaspectratio]{a06/td432.png}
    \end{tabular}
    \begin{tabular}{p{.55\textwidth}}
      \textbf{An den Eingang dieser Schaltung wird das folgende Signal gelegt. Welches ist ein mögliches Ausgangssignal $U_A$?}
    \end{tabular}
    } \\ \hline\hline
    A & \includegraphics[width=.25\textwidth,height=.25\textheight,keepaspectratio]{a06/td431d.png} &
    B & \includegraphics[width=.25\textwidth,height=.25\textheight,keepaspectratio]{a06/td431c.png} \\ \hline
    C \only<2>\checkmark & \includegraphics[width=.25\textwidth,height=.25\textheight,keepaspectratio]{a06/td431b.png} &
    D & \includegraphics[width=.25\textwidth,height=.25\textheight,keepaspectratio]{a06/td431a.png} \\ \hline
  \end{tabular}
\end{frame}


\begin{frame}
  \frametitle{Transistor als Schalter mit Überspannungsschutzdiode}
  \begin{columns}
    \column{.4\textwidth}
    \begin{center}
      \begin{figure}
        \includegraphics[width=\textwidth,height=.85\textheight,keepaspectratio]{a06/Transistor-Schalter+L.png}
        \attribcaption{Transistor als Schalter einer indkutiven Last}{BNetzA}{https://www.bundesnetzagentur.de/amateurfunk/}{}
      \end{figure}
    \end{center}
    \column{.5\textwidth}
    \begin{itemize}
      \item Durch plötzliches Abschalten baut sich eine hohe Induktionsspannung auf
      \item Diese kann den Transistor zerstören
      \item Um das zu verhindern wird eine Diode parallel zur Spule eingebaut
      \item Diese führt die Induktionsspannung an dem Transistor vorbei ab
    \end{itemize}
  \end{columns}
\end{frame}

% \begin{frame}
%   \begin{tabular}{p{2.5pc}||c||p{2.5pc}||c}\hline
%     \textbf{TC528} & \multicolumn{3}{p{.8\textwidth}}{\textbf{In welcher der folgenden Schaltungen ist die Diode zur Spannungsbegrenzung einer Schaltstufe richtig eingesetzt?}} \\ \hline\hline
%     A \only<2>\checkmark & \includegraphics[width=.35\textwidth,height=.35\textheight,keepaspectratio]{a06/tc528a.png} &
%     B & \includegraphics[width=.35\textwidth,height=.35\textheight,keepaspectratio]{a06/tc528d.png} \\ \hline
%     C & \includegraphics[width=.35\textwidth,height=.35\textheight,keepaspectratio]{a06/tc528c.png} &
%     D & \includegraphics[width=.35\textwidth,height=.35\textheight,keepaspectratio]{a06/tc528b.png} \\ \hline
%   \end{tabular}
% \end{frame}

\subsection*{Verstärker}
\begin{frame}
  \frametitle{Verstärker}
  \begin{block}{Definition eines Verstärkers nach Captain Obvious}
    \begin{Large}
      Es ist nur dann eine Verstärkung, wenn die Leistung am Ausgang größer ist, als die am Eingang
    \end{Large}
  \end{block}
\end{frame}

\begin{frame}
  \frametitle{Transistor als Verstärker in Emitterschaltung}
  \begin{columns}
    \column{.4\textwidth}
    \centering
    \begin{figure}
      \includegraphics[width=\textwidth,height=.85\textheight,keepaspectratio]{a06/Transistor-Verstaerker.png}
      \attribcaption{Transistor als Spannungsverstärker}{BNetzA}{https://www.bundesnetzagentur.de/amateurfunk/}{}
    \end{figure}
    \only<1>{
    \column{.5\textwidth}
    \begin{itemize}
      \item Benötigt einen richtig eingestellten Arbeitspunkt, um vernünftig zu funktionieren
      \item Dazu wird die Basis"=Emitter"=Spannung $U_{BE}$ auf einen definierten Wert größer $0,6 V$ gesetzt
    \end{itemize}
    }
    \only<2>{
    \column{.5\textwidth}
    \begin{itemize}
      \item Dem durch den Arbeitspunkt eingestellten Ruhestrom überlagert sich die eingekoppelte Wechselspannung
      \item Dadurch wird der Kollektorstrom je nach Eingangsgröße größer oder kleiner
      \item Da der Bipolartransistor nur den Strom verstärkt, wird die Stromverstärkung mittels eines Widerstandes $R_1$ in eine Spannungsverstärkung umgewandelt
    \end{itemize}
    }
    \only<3>{
    \column{.5\textwidth}
    \begin{itemize}
      \item Ändert man die Eingangsspannung $U_e$ der Schaltung, verändert sich auch der Basisstrom
      \item Dies ruft eine Veränderung des Kollektorstromes $I_C$ hervor, die um die Stromverstärkung des Transistors größer ist als der Basisstrom
      \item Die Spannung über $R_1$ verhält sich bei Änderungen genauso wie der Kollektorstrom $I_C$
      \item Dadurch sinkt die Kollektorspannung $U_{CE}$ des Transistors
    \end{itemize}
    }
    \only<4>{
    \column{.5\textwidth}
    \begin{itemize}
      \item Die Kondensatoren $C_2$ und $C_3$ entkoppeln das Eingangs- und das Ausgangssignal
      \item Dadurch wird der Gleich\-spannungs\-anteil entfernt, um das Signal beispielsweise einer weiteren Verstärkerstufe zuzuführen
    \end{itemize}
    }
  \end{columns}
\end{frame}

\begin{frame}
  \begin{tabular}{p{2.5pc}||l||p{2.5pc}||l}\hline
    \textbf{TC626} & \multicolumn{3}{p{.8\textwidth}}{
    \begin{tabular}{c}
      \includegraphics[width=\textwidth,height=.45\textheight,keepaspectratio]{a06/tc626.png}
    \end{tabular}
    \begin{tabular}{p{.48\textwidth}}
      \textbf{Folgendes Signal $U_E$ wurde auf den Eingang folgender Schaltung gegeben. In welcher Antwort sind alle dargestellten Signale phasenrichtig zugeordnet?}
    \end{tabular}
    } \\ \hline\hline
    A & \includegraphics[width=.25\textwidth,height=.25\textheight,keepaspectratio]{a06/tc626c.png} &
    B \only<2>\checkmark & \includegraphics[width=.25\textwidth,height=.25\textheight,keepaspectratio]{a06/tc626a.png} \\ \hline
    C & \includegraphics[width=.25\textwidth,height=.25\textheight,keepaspectratio]{a06/tc626d.png} &
    D & \includegraphics[width=.25\textwidth,height=.25\textheight,keepaspectratio]{a06/tc626b.png} \\ \hline
  \end{tabular}
\end{frame}

\begin{frame}
  \frametitle{Verstärkungen eines Transistors bei Wechselstrom}
  \centering
  \begin{block}{Spannungsverstärkung}
    $v_U = \cfrac{\Delta U_{CE}}{\Delta U_{BE}}$
  \end{block}

  \begin{block}{Stromverstärkung}
    $v_I = \beta = \cfrac{\Delta I_C}{\Delta I_B}$
  \end{block}

  \begin{block}{Leistungsverstärkung}
    $v_P = v_U \cdot v_I$
  \end{block}
\end{frame}


\subsection*{Basisvorspannung}
\begin{frame}
  \frametitle{Generieren der Basisvorspannung}
  \begin{columns}
    \column{.5\textwidth}
    \begin{center}
      \begin{figure}
        \includegraphics[width=\textwidth,height=.7\textheight,keepaspectratio]{a06/Transistor-Verstaerker.png}
        \attribcaption{Transistor als Spannungsverstärker}{BNetzA}{https://www.bundesnetzagentur.de/amateurfunk/}{}
      \end{figure}
    \end{center}
    \column{.5\textwidth}
    \begin{center}
      \begin{figure}
        \includegraphics[width=\textwidth,height=.7\textheight,keepaspectratio]{a06/Transistor-Verstaerker-C.png}
        \attribcaption{Andere Möglichkeit die Basisvorspannung zu erzeugen}{BNetzA}{https://www.bundesnetzagentur.de/amateurfunk/}{}
        \label{fig:spannungsteiler}
      \end{figure}
    \end{center}
  \end{columns}
\end{frame}

\begin{frame}
  \frametitle{Basisvorspannung, warum eigentlich?}
  \begin{itemize}
    \item Will man eine Wechselspannung verstärken, muss man den Mittelpunkt des Signals in den Aussteuerbereich verschieben
    \item Da sonst die obere oder untere Halbwelle nicht originalgetreu wiedergegeben wird
    \item Dazu überlagert man die Wechselspannung mit einer mittleren Gleichspannung
    \item Das Erzeugen der mittleren Gleichspannung nennt man einstellen des Arbeitspunktes oder der Basisvorspannung
  \end{itemize}
\end{frame}

\begin{frame}
  \begin{figure}
    \includegraphics[width=\textwidth,height=\textheight,keepaspectratio]{a06/Arbeitspunkt.png}
    \attribcaption{Arbeitspunkt eines Transistorverstärkers}{Honina}{https://commons.wikimedia.org/wiki/File:Arbeitspunkt.PNG}{\ccbysa}
  \end{figure}
\end{frame}

\begin{frame}
  \frametitle{Einstellen des Arbeitspunktes}
  \begin{columns}
    \column{.5\textwidth}
    \begin{figure}
      \includegraphics[width=\textwidth,height=.7\textheight,keepaspectratio]{a06/Kombiniertes_Kennlinienfeld_Transistor_2.pdf}
      \attribcaption{Kombiniertes Kennlinienfeld eines Transistors}{Biezl}{https://commons.wikimedia.org/wiki/File:Kombiniertes_Kennlinienfeld_Transistor_2.svg}{\ccpd}
    \end{figure}
    \column{.5\textwidth}
    \begin{itemize}
        \only<1>{
      \item Will man den Arbeitspunkt eines Transistors mit der Vorspannung einstellen, muss man zwei Spannungen dimensionieren
      \item Zum einen muss die Basisspannung eingestellt werden und zum anderen die Kollektorspannung
      \item Die Kollektorspannung bestimmen wir, indem wir einen Spannungsteiler über $R_1$ und dem Kollektor"=Emitter"=Widerstand $R_{CE}$ (Lastwiderstand) berechnen
        }
        \only<2>{
      \item Für eine symmetrische Aussteuerung nehmen wir an, dass sowohl über $R_1$ als auch über $R_{CE}$  $\frac{V_{DD}}{2}$ anliegen
      \item Danach wählt man einen Kollektorstrom den der Transistor verkraften kann
      \item Dann berechnet man mittels Ohmschen Gesetz den Widerstand $R_1$
        }
        \only<3>{
      \item Um die Basisvorspannung zu erzeugen, gibt es zwei Möglichkeiten:
        \begin{itemize}
          \item mit einem Widerstand
          \item mit einem Spannungsteiler (siehe Abbildung~\autoref{fig:spannungsteiler})
        \end{itemize}
        }
    \end{itemize}
  \end{columns}
\end{frame}

\begin{frame}
  \frametitle{Arbeitspunktstabilisierung durch Stromgegenkopplung}
  \begin{columns}
    \column{.4\textwidth}
    \begin{center}
      \begin{figure}
        \includegraphics[width=\textwidth,height=.85\textheight,keepaspectratio]{a06/Transistor-Verstaerker-APstab1.png}
        \attribcaption{Transistor als Spannungsverstärker}{BNetzA}{https://www.bundesnetzagentur.de/amateurfunk/}{}
      \end{figure}
    \end{center}
    \column{.5\textwidth}
    % $U_2$ = konstant
    % $I_C$ steigt $\rightarrow I_E$ steigt $\rightarrow U_{R3}$ steigt $\rightarrow U_{R2} = U_{R3} + U_{BE} \rightarrow U_{BE}$ sinkt $\rightarrow $Transistor sperrt$ \rightarrow I_C sinkt$
    \begin{itemize}
      \item Der Transistor ist nicht temperaturstabil, was zum Anstieg von $I_B$ und damit $I_C$ und $I_E$ führt
      \item $U_{R3}$ steigt und $U_{BE}$ sinkt
      \item Bei kleinerer $U_{BE}$ sinkt $I_B$ und damit $I_C$ und $I_E$
      \item Die Spannung $U_{R3}$ wird wieder kleiner und $U_{BE}$ wieder größer
    \end{itemize}
  \end{columns}
\end{frame}

\begin{frame}
  \begin{itemize}
    \item Querstrom $I_{quer}$ durch $R_2$ wird 3 bis 10 mal so groß wie der
      Basisstrom dimensioniert: \\

      $I_{quer} = 3 ... 10 \cdot I_B$

    \item Kondensator am Emitter überbrückt Wechselstromsignale
      \begin{itemize}
        \item dadurch werden Verstärkungsverluste bei höheren Frequenzen aufgehoben
        \item baut man am Emitter keinen Kondensator ein, fällt die Verstärkung des
          Transistors stark ab \& liegt dann bei dem Verhältnis aus Kollektor- zu
          Emitterwiderstand
      \end{itemize}
  \end{itemize}
\end{frame}

\begin{frame}
  \frametitle{Arbeitspunktstabilisierung durch Spannungsgegenkopplung}
  \begin{columns}
    \column{.5\textwidth}
    \begin{center}
      \begin{figure}
        \includegraphics[width=\textwidth,height=.7\textheight,keepaspectratio]{a06/Transistor-Verstaerker-APstab2a.png}
        \attribcaption{Eine Möglichkeit zur Spannungsgegenkopplung}{BNetzA}{https://www.bundesnetzagentur.de/amateurfunk/}{}
      \end{figure}
    \end{center}
    \column{.5\textwidth}
    \begin{center}
      \begin{figure}
        \includegraphics[width=\textwidth,height=.7\textheight,keepaspectratio]{a06/Transistor-Verstaerker-APstab2b.png}
        \attribcaption{Eine andere Möglichkeit zur Spannungsgegenkopplung}{BNetzA}{https://www.bundesnetzagentur.de/amateurfunk/}{}
      \end{figure}
    \end{center}
  \end{columns}
\end{frame}

\begin{frame}
  \begin{columns}
    \column{.4\textwidth}
    \begin{center}
      \begin{figure}
        \includegraphics[width=\textwidth,height=.85\textheight,keepaspectratio]{a06/Transistor-Verstaerker-APstab2a.png}
        \attribcaption{Transistor als Spannungsverstärker}{BNetzA}{https://www.bundesnetzagentur.de/amateurfunk/}{}
      \end{figure}
    \end{center}
    \column{.5\textwidth}
    \begin{itemize}
      \item $R_1$ stabilisiert gegen thermischen Einfluss
      \item $I_C$ steigt $\rightarrow U_{CE}$ sinkt $\rightarrow U_{BE}$ sinkt
        $\rightarrow$ Transistor ``sperrt'' $\rightarrow I_C$ sinkt wieder
    \end{itemize}
  \end{columns}
\end{frame}

\subsection*{Grundschaltungen}
\begin{frame}
  \frametitle{Grundschaltungen des Transistors}
  \begin{small}
    \begin{tabular}{|c|c|c|c|}
      \hline
      " " & Emitterschaltung & Kollektorschaltung & Basisschaltung \\ \hline
      " "     & \includegraphics[width=.3\textwidth,height=.3\textheight,keepaspectratio]{a06/Emitterschaltung.png}
      & \includegraphics[width=.3\textwidth,height=.3\textheight,keepaspectratio]{a06/Kollektorschaltung.png}
      & \includegraphics[width=.3\textwidth,height=.3\textheight,keepaspectratio]{a06/Basisschaltung.png} \\ \hline
      $r_e$ & mittel z.B. $1k\Omega$ & klein z.B. $50\Omega$ & groß z.B. $100k\Omega$ \\ \hline
      $r_a$ & mittel z.B. $10k\Omega$ & groß z.B. $100k\Omega$ & klein z.B. $50\Omega$ \\ \hline
      $v_i$ & groß z.B. $100$ & $<1$ z.B. $0,9$ & groß z.B. $100$ \\ \hline
      $v_u$ & groß z.B. $100$ & groß z.B. $100$ & $<1$ z.B. $0,99$ \\ \hline
      $v_p$ & sehr groß z.B. $1k\Omega$ & groß z.B. $100$ & groß z.B. $100$ \\ \hline
      $\varphi_u$ & gegenphasig $180^{\circ}$ & gleichphasig $0^{\circ}$ & gleichphasig $0^{\circ}$ \\ \hline
    \end{tabular}
  \end{small}
  \begin{figure}
    \attribcaption{Transistorgrundschaltungen}{Frank Murmann}{https://commons.wikimedia.org/wiki/File:Transistorgrundschaltungen.svg}{\ccpd}
  \end{figure}
\end{frame}

\begin{frame}
  \begin{exampleblock}{Fakultative Hausaufgabe}
    Prüfungsfragen TC618--TC625, TF321--TF324
  \end{exampleblock}
\end{frame}

\section*{Integierte Schaltung}
\begin{frame}
  \frametitle{Integierte Schaltung}
  Heutzutage komplexe Schaltungen auf einem Halbleiterkristall
  \begin{columns}
    \column{.5\textwidth}
    \begin{center}
      \begin{figure}
        \includegraphics[width=\textwidth,height=.6\textheight,keepaspectratio]{a06/IC.jpg}
        \attribcaption{Ic auf Platine}{Luestling}{https://commons.wikimedia.org/wiki/File:Chips_3_bg_102602.jpg}{\ccpd}
      \end{figure}
    \end{center}
    \column{.6\textwidth}
    \begin{center}
      \begin{figure}
        \includegraphics[width=\textwidth,height=.6\textheight,keepaspectratio]{a06/IC2.jpg}
        \attribcaption{Offener IC Intel 8742}{Ioan Sameli}{https://en.wikipedia.org/wiki/File:Intel_8742_153056995.jpg}{\ccbysa}
      \end{figure}
    \end{center}
  \end{columns}
\end{frame}

\section*{Operations\-verstärker}

\begin{frame}
  \frametitle{Operationsverstärker}
  \begin{center}
    \begin{figure}
      \includegraphics[width=\textwidth,height=.85\textheight,keepaspectratio]{a06/OPV-intern.png}
      \attribcaption{Innerer Aufbau eines OPVs}{Daniel Braun, bearbeitet von Cepheiden}{https://commons.wikimedia.org/wiki/File:OpAmpTransistorLevel_Colored_DE.svg}{\ccby}
    \end{figure}
  \end{center}
\end{frame}

\begin{frame}
  \frametitle{Invertierender Verstärker}
  \begin{columns}
    \column{.4\textwidth}
    \begin{center}
      \begin{figure}
        \includegraphics[width=\textwidth,height=.85\textheight,keepaspectratio]{a06/OPV-Inverter.png}
        \attribcaption{OPV als invertierender Verstärker}{Quelle unbekannt}{}{}
      \end{figure}
    \end{center}
    \column{.5\textwidth}
    \centering
    \begin{block}{Invertierender Verstärker}
      $v_u = -\cfrac{U_A}{U_E} = -\cfrac{R_2}{R_1}$
    \end{block}
  \end{columns}
\end{frame}

\begin{frame}
  \frametitle{Nicht-invertierender Verstärker}
  \begin{columns}
    \column{.4\textwidth}
    \begin{center}
      \begin{figure}
        \includegraphics[width=\textwidth,height=.85\textheight,keepaspectratio]{a06/OPV-nonInverter.png}
        \attribcaption{OPV als nichtinvertierender Verstärker}{Quelle unbekannt}{}{}
      \end{figure}
    \end{center}
    \column{.5\textwidth}
    \centering
    \begin{block}{Nicht-invertierender Verstärker}
      $v_u = \cfrac{U_A}{U_E} = 1+\cfrac{R_2}{R_1}$
    \end{block}
    Wird häufig als Impedanzwandler eingesetzt
  \end{columns}
\end{frame}

\begin{frame}
  \frametitle{Impedanzwandler}
  \begin{columns}
    \column{.4\textwidth}
    \begin{center}
      \begin{figure}
        \includegraphics[width=\textwidth,height=.85\textheight,keepaspectratio]{a06/Impedanzwandler.png}
        \attribcaption{OPV als Imepdanzwandler}{Quelle unbekannt}{}{}
      \end{figure}
    \end{center}
    \column{.5\textwidth}
    \begin{itemize}
      \item Besitzt sehr großen Eingangswiderstand und sehr kleinen Ausgangswiderstand
      \item Ist quasi ein nicht-invertierender Verstärker mit $R_2 = 0 \Omega$ und $R_1 = \infty\Omega$
    \end{itemize}
  \end{columns}
\end{frame}


\begin{frame}
  \begin{exampleblock}{Fakultative Hausaufgabe}
    Prüfungsfragen TC711--TC717
  \end{exampleblock}
\end{frame}

\section*{Elektronenröhre}

\begin{frame}
  \begin{center}
    \begin{figure}
      \includegraphics[width=\textwidth,height=.5\textheight,keepaspectratio]{a06/Roehren.jpg}
      \attribcaption{Verschiedene Elektronenröhren}{Stefan Riepl (Quark48)}{https://commons.wikimedia.org/wiki/File:Elektronenroehren-auswahl.jpg}{\ccbysa}
    \end{figure}
  \end{center}
  \begin{itemize}
    \item Arbeiten mit hohen Spannungen und geringen Strömen
    \item Ist ein gepoltes Bauteil
  \end{itemize}
\end{frame}

\begin{frame}
  \frametitle{Die Röhre}
  \begin{columns}
    \column{.3\textwidth}
    \begin{figure}
      \includegraphics[width=\textwidth,height=.7\textheight,keepaspectratio]{a06/ERohre.png}
      \attribcaption{Symbol einer Triode}{RokerHRO, geändert von Fgli}{https://commons.wikimedia.org/wiki/File:Triode-Symbol_de.svg}{\ccpd}
    \end{figure}
    \column{.7\textwidth}
    \begin{small}
      \begin{itemize}
        \item Heizung löst Elektronen aus Kathode
        \item Elektronen werden Richtung Anode beschleunigt
        \item Gitter verändert elektrisches Feld
        \item Gitterspannung steuert Anodenstrom
      \end{itemize}
    \end{small}
    \begin{center}
      \begin{figure}
        \includegraphics[scale=0.4 ]{a06/Triode.jpg}
        \attribcaption{Strahltriode aus altem Fernseher}{Ulfbastel}{https://de.wikipedia.org/wiki/Datei:Strahltriode.jpg}{\ccpd}
      \end{figure}
    \end{center}
  \end{columns}
\end{frame}

% \begin{frame}
%   \begin{tabular}{l||p{.8\textwidth}}\hline
%     \textbf{TC718} & \textbf{Worauf beruht die Verstärkerwirkung von Elektronenröhren?} \\ \hline\hline
%     A & Die Heizspannung steuert das elektrische Feld an der Katode und damit den Anodenstrom. \\ \hline
%     B & Die Anodenspannung steuert das magnetische Feld an der Anode und damit den Anodenstrom. \\ \hline
%     C \only<2>\checkmark & Das von der Gitterspannung hervorgerufene elektrische Feld steuert den Anodenstrom. \\ \hline
%     D & Die Katodenvorspannung steuert das magnetische Feld an der Katode und damit den Gitterstrom. \\ \hline
%   \end{tabular}
% \end{frame}

\renewcommand{\refname}{Referenzen}

\hypertarget{refs}{}
\textcolor{white}{} \\ %\vspace{} geht nicht
\Large Referenzen/Links
\footnotesize

\begin{thebibliography}{}
  \bibitem{a06}   Moltrecht A 06: \\
    \url{https://www.darc.de/der-club/referate/ajw/lehrgang-ta/a06/}

  \bibitem{yu}    Youtube Video “How Does a Transistor Work?”:
    \url{https://www.youtube.com/watch?v=IcrBqCFLHIY}
\end{thebibliography}

% Hier könnte noch eine Kontaktfolie stehen

\end{document}

