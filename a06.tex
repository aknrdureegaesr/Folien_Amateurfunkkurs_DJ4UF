% Foliensatz: "AFu-Kurs nach DJ4UF" von DK0TU, Amateurfunkgruppe der TU Berlin
% Lizenz: CC BY-NC-SA 3.0 de (http://creativecommons.org/licenses/by-nc-sa/3.0/de/)
% Autoren: Martin Deutschmann <martin.deutschmann@campus.tu-berlin.de>

preamble.dk0tu.tex
\subtitle{Technik Klasse A 06: \\
          Transistor \& Verstärker \\[2em]}
\date{Stand \today}
 \begin{document}

\begin{frame}
    \titlepage
    \vfill
    \begin{center}
        \ccbyncsaeu\\
        {\tiny This work is licensed under the \em{Creative Commons Attribution-NonCommercial-ShareAlike 3.0 License}.}\\[0.5ex]
         \tiny Amateurfunkgruppe der Technische Universität Berlin (AfuTUB), DKØTU
         %\includegraphics[scale=0.5]{img/DK0TU_Logo.pdf}
    \end{center}
\end{frame}


%fixme Referenzen/Fußnoten-Systematik vereinheitlichen

\section*{Bipolarer Transistor}
\begin{frame}
\frametitle{Bipolarer Transistor}
\begin{minipage}{0.4\textwidth}
	\includegraphics[scale=0.2]{a06/NPN_hlb.png}\\
	Abb. 1: Schichten eines NPN-Transistors
\end{minipage}
\hspace{0.5cm}
\begin{minipage}{0.4\textwidth}
	\includegraphics[scale=0.2 ]{a06/PNP_hlb.png}\\
	Abb. 2: Schichten eines PNP-Transistors
\end{minipage}
\vspace{0.5cm}
\begin{center}
\begin{itemize}
	\item Transistoren bestehen aus drei Halbleiterschichten
	\item Anschlüsse: Basis (B), Kollektor (C), Emitter (E)
\end{itemize}
\end{center}
\end{frame}

\begin{frame}
\frametitle{Ersatzschaltbild}

\begin{minipage}{0.4\textwidth}
	\includegraphics[scale=1.4]{a06/NPN_esb.png}\\
	Abb. 3: ESB eines NPN-Transistors
\end{minipage}
\hspace{0.5cm}
\begin{minipage}{0.4\textwidth}
	\includegraphics[scale=1.4]{e13/PNP_esb.png}\\
	Abb. 4: ESB eines PNP-Transistors
\end{minipage}
\vspace{2mm}
\begin{itemize}
	\item	Basis-Emitterübergang muss in Durchlassrichtung gepolt sein
	\item	Basis braucht ein um etwa 0,6 V höheres Potential als der Emitter
\end{itemize}

\end{frame}


\section*{Feldeffekt Transistor (FET)}
\begin{frame}
\frametitle{Feldeffekt Transistor (FET)}
\begin{center}
	\includegraphics[scale=0.2]{a06/FET-aufbau.png}\\
	Abb. 5: Aufbau eines J-FET \cite{wp}	
\begin{itemize}
	\item Gate steuert Kanalbreite durch Spannung
	\item Je dünner der Kanal, desto höher ist der Kanalwiderstand
\end{itemize}
\end{center}
\end{frame}

\begin{frame}
\frametitle{Feldeffekt Transistor (FET)}
\begin{center}
	\includegraphics[scale=0.35]{a06/FET-overview.png}\\
	Abb. 6: Übersicht über FETs \cite{wp}
\end{center}
\end{frame}

\section{Verstärker}
\begin{frame}
\frametitle{Verstärker}
\begin{center}
\begin{Large}
\begin{itemize}
	\item \textbf{Es ist nur dann eine Verstärkung, wenn die Leistung am Ausgang größer ist, als die am Eingang}

\end{itemize}
\end{Large}
\end{center}
\end{frame}

\section*{Der Isolierschicht-FET (IG-FET, MOSFET)}
\begin{frame}
	\begin{center}
		\includegraphics[scale=0.4 ]{a06/Isolierschicht-FET-intern.png}\\
		ABB.7: MOSFET in Planartechnologie \cite{wmde}
	\end{center}
\end{frame}

\begin{frame}
	\begin{center}
		\includegraphics[scale=0.3 ]{a06/MOSFET-Symbole.png}\\
		ABB.8: Schaltzeichen des MOSFETS \cite{wmde}
	\end{center}
\end{frame}

\begin{frame}
	\begin{center}
		\includegraphics[scale=3 ]{a06/Dual-Gate-MOSFET.png}\\
		ABB.9: Schaltzeichen eines Dual-Gate-MOSFETs
	\end{center}
	\begin{itemize}
		\item	Besitzt zwei Gateanschlüsse
		\item	Wird für Mischerschaltungen genutzt
	\end{itemize}
\end{frame}

\section*{Transistor als Schalter}
\begin{frame}
	\begin{minipage}{0.4\textwidth}
		\begin{center}
			\includegraphics[scale=0.8]{a06/Transistor-Schalter.png}\\
		\end{center}	
	\end{minipage}
	\hspace{3mm}
	\begin{minipage}{0.4\textwidth}
		\begin{center}
			\includegraphics[scale=0.8]{a06/Transistor-Verstaerker.png}\\
		\end{center}	
	\end{minipage}\\
	\vspace{3mm}
	Abb.10: Verwendungsmöglichkeiten von Transistoren \cite{bnetza}
	\begin{itemize}
		\item	Transistoren können als Schalter und als Verstärker genutzt werden
	\end{itemize}
\end{frame}

\begin{frame}
	\frametitle{Transistor als Schalter}
	\begin{center}
			\includegraphics[scale=1.2]{a06/Transistor-Schalter.png}\\
			Abb.11: Transistor als Schalter \cite{bnetza}
		\end{center}
		\begin{itemize}
			\item	$U_e = 0 V \rightarrow$ Transistor sperrt $\rightarrow U_a = U_B$
			\item	$U_e \gg 0,6 V \rightarrow$ Transistor leitet $\rightarrow U_a \approx 0.1 V$
			\item	Der Transistor erfüllt hier die Funktion eines Inverters
		\end{itemize}
\end{frame}

\begin{frame}
	\frametitle{Transistor mit Koppelkondensator als Schalter}
	\begin{center}
			\includegraphics[scale=0.5]{a06/Transistor-Schalter+C.png}\\
			Abb.12: Transistor mit Koppelkondensator als Schalter \cite{bnetza}
		\end{center}
		\begin{itemize}
			\item	Ausgangssignal ist ähnlich dem aus Abb. XX
			\item	Der Kondensator blockt die Gleichspannung und bildet den Mittelwert der Wechselspannung
		\end{itemize}
\end{frame}

\begin{frame}
	\frametitle{Transistor mit Koppelkondensator als Schalter}
	\begin{center}
		\includegraphics[scale=0.5]{a06/Transistor-Schalter+L.png}\\
		Abb.13: Transistor als Schalter einer Induktiven Last \cite{bnetza}
	\end{center}
	\begin{itemize}
		\item	Durch plötzliches Abschalten baut sich eine hohe Induktionsspannung auf
		\item	Diese kann den Transistor zerstören
		\item	Um das zu verhindern wird eine Diode wie in Abb. XX zu sehen parallel zum Transistor eingebaut
		\item	Diese lässt führt die Induktionsspannung an dem Transistor vorbei ab
	\end{itemize}
\end{frame}

\section*{Transistor als Verstärker}
\begin{frame}
	\begin{center}
		\includegraphics[scale=1.2]{a06/Transistor-Verstaerker.png}\\
		Abb.14: Transistor als Spannungsverstärker \cite{bnetza}
	\end{center}
	\begin{itemize}
		\item	Benötigt einen richtig eingestellten Arbeitspunkt um vernünftig zu funktionieren
		\item	Dazu wird die Basis-Emitter-Spannung $U_{BE}$ auf einen definierten Wert größer $0,6 V$ gesetzt
	\end{itemize}
\end{frame}

\begin{frame}
	\frametitle{ein paar Details zum Transistor als Spannungsverstärker}
	\begin{itemize}
		\item	Dem durch den Arbeitspunkt eingestellten Ruhestrom überlagert sich die eingekoppelte Wechselspannung
		\item	Dadurch wird der Kollektorstrom je nach Eingangsgröße größer oder kleiner
		\item	Die Kondensatoren $C_2$ und $C_3$ entkoppeln das Eingangssignal und das Ausgangssignal
		\item	Da der Bipolartransistor nur den Strom verstärkt, wird die Stromverstärkung mittels eines Widerstandes $R_1$ in eine Spannungsverstärkung umgewandelt
	\end{itemize}
\end{frame}

\begin{frame}
	\frametitle{Funktionsweise}
	\begin{itemize}
		\item	Ändert man die Eingangsspannung $U_e$ der Schaltung, verändert sich auch der Basisstrom
		\item	Dies ruft eine Veränderung des Kollektorstromes $I_C$ hervor, die um die Stromverstärkung des Transistors größer ist als der Basisstrom
		\item	Die Spannung über $R_1$ verhält sich bei änderungen genauso wie der Kollektorstrom $I_C$
		\item	Dadurch sinkt die Kollektorspannung $U_{CE}$ des Transistors
		\item	Der Gleichspannungsanteil dieser Mischspannung wird dann noch durch $C_3$ ausgekoppelt 
	\end{itemize}
\end{frame}

\begin{frame}
	\frametitle{Eine kleine Übung dazu}
	\begin{center}
		\includegraphics[scale=0.6]{a06/Transistor-Diagramm.png}\\
		Abb.15: Eingangssignal eines Verstärkers \cite{bnetza}
	\end{center}
	\begin{itemize}
		\item	Überlegt euch, wie $I_B$, $I_C$, $U_{CE}$ und $U_a$ in Phase und Amplitude aussehen 
	\end{itemize}
\end{frame}

\begin{frame}
	\frametitle{Eine kleine Übung dazu - Auflösung}
	\begin{center}
		\includegraphics[scale=0.6]{a06/Transistor-Diagramm-compl.png}\\
		Abb.16: Signale eines Verstärkers \cite{bnetza}
	\end{center}
\end{frame}

\begin{frame}
	\frametitle{Verstärkungen eines Transistors}
	\begin{itemize}
		\item	Spannungsverstärkung:\\
		\hspace{3mm}
		\begin{center}
				$v_u = \frac{\Delta U_{CE}}{\Delta U_{BE}}$
		\end{center}
		\hspace{3mm}	
		\item	Stromverstärkung:	\\
		\hspace{3mm}
		\begin{center}
				$v_i = \frac{\Delta I_C}{\Delta I_C}$
		\end{center}		
		\hspace{3mm}		
		\item	Leistungsverstärkung:	\\
		\hspace{3mm}
		\begin{center}
				$v_p = v_u \cdot v_i$
		\end{center}		
	\end{itemize}
\end{frame}

\section*{Generieren der Basisvorspannung}
\begin{frame}
	\begin{center}
		\includegraphics[scale=0.8]{a06/Transistor-Verstaerker.png}
		\vspace{3mm}
		\includegraphics[scale=0.8]{a06/Transistor-Verstaerker-C.png}\\
		Abb.17: Möglichkeiten die Basisvospannung erzeugen 
	\end{center}
\end{frame}

\begin{frame}
	\frametitle{Basisvorspannung, warum eigentlich?}
	\begin{itemize}
		\item	Will man eine Wechselspannung verstärken, muss man den Mittelpunkt des Signals in den Aussteuerbereich verschieben
		\item	Da sonst die obere oder untere Halbwelle nicht originalgetreu wiedergegeben wird
		\item	Dazu überlagert man die Wechselspannung mit einer mittleren Gleichspannung
		\item	Das Erzeugen der mittleren Gleichspannung nennt man einstellen des Arbeitspunktes oder der Basisvorspannung 
	\end{itemize}
\end{frame}

\begin{frame}
	\frametitle{Einstellen des Arbeitspunktes}
	\begin{itemize}
		\item	Will man den Arbeitspunkt eines Transistors mit per Vorspannung einstellen, muss man zwei Spannungen dimensionieren
		\item	Zum einen muss die Basisspannung eingestellt werden und zum anderen die Kollektorspannung
		\item	Die Kollektorspannung bestimmen wir, indem wir einen Spannungsteiler über $R_1$ und dem Kollektor-Emitter-Widerstand $R_{CE}$ berechnen
		\item	Für eine symmetrische Aussteuerung nehmen wir an, das sowohl über $R_1$ als auch über $R_{CE}$  $\frac{V_{DD}}{2}$ anliegen
		\item	Danach wählt man einen Kollektorstrom den der Transistor verkraften kann
		\item	Dann berechnen man mittels ohm'schen Gesetz den Lastwiderstand $R_1$
	\end{itemize}
\end{frame}

\begin{frame}
	\begin{itemize}
		\item	Um die Basisvorspannung zu erzeugen, gibt es zwei Möglichkeiten
		\item	Zum einen die Basisvorspannungserzeugung mit einem Widerstand oder mit einem Spannungsteiler siehe Abb.17
	\end{itemize}
\end{frame}

\section*{Arbeitspunktstabilisierung durch Stromgegekopplung}
\begin{frame}
	\begin{center}
		\includegraphics[scale=1.2]{a06/Transistor-Verstaerker-APstab1.png}\\
		Abb.18: Transistor als Spannungsverstärker \cite{bnetza}
	\end{center}
	\begin{itemize}
		\item	$U_2$ = konstant
		\item	$I_C$ steigt $\rightarrow I_E$ steigt $\rightarrow U_{R3}$ steigt $\rightarrow U_{R2} = U_{R3} + U_{BE} \rightarrow U_{BE}$ sinkt $\rightarrow $Transistor sperrt$ \rightarrow I_C sinkt$
	\end{itemize}
\end{frame}

\begin{frame}
	\begin{itemize}
		\item	Der Kondensator am Emitter überbrückt Wechselstromsignale
		\item	Baut man am Emitter keinen Kondensator ein, fällt die Verstärkung des Transistors stark ab
		\item	Sie liegt dann bei dem Verhältnis aus Kollektor- zu Emitterwiderstand
		\item	Querstrom $I_{quer}$ durch $R_2$ wird 3 bis 10 mal so groß wie der Basisstrom dimensioniert
		\item	$I_{quer} = 3 ... 10 \cdot I_B$
	\end{itemize}
\end{frame}

\section*{Arbeitspunktstabilisierung durch Spannungsgegenkopplung}
\begin{frame}
	\begin{center}
		\includegraphics[scale=0.8]{a06/Transistor-Verstaerker-APstab2a.png}
		\vspace{3mm}
		\includegraphics[scale=0.8]{a06/Transistor-Verstaerker-APstab2b.png}\\
		Abb.19: Möglichkeiten zur Arbeitspunktstabilisierung via Spannungsgegenkopplung 
	\end{center}
\end{frame}

\begin{frame}
	\begin{center}
		\includegraphics[scale=1.2]{a06/Transistor-Verstaerker-APstab2a.png}\\
		Abb.20: Transistor als Spannungsverstärker \cite{bnetza}
	\end{center}
	\begin{itemize}
		\item	$I_C$ steigt $\rightarrow U_{CE}$ sinkt $\rightarrow U_{BE}$ sinkt Transistor sperrt$ \rightarrow I_C sinkt$
	\end{itemize}
\end{frame}

\section*{Grundschaltungen des Transistors}
\begin{frame}
	\begin{small}
	\begin{tabular}{|c|c|c|c|}
	\hline
	" " &Emitterschaltung & Kollektorschaltung & Basisschaltung \\ \hline
	" " &\includegraphics[scale=0.4]{a06/Emitterschaltung.png} & \includegraphics[scale=0.4]{a06/Kollektorschaltung.png} & \includegraphics[scale=0.4]{a06/Basisschaltung.png} \\ \hline
	$r_e$ & mittel z.B. $1k\Omega$ & klein z.B. $50\Omega$ & groß z.B. $100k\Omega$ \\ \hline
	$r_a$ & mittel z.B. $10k\Omega$ & groß z.B. $100k\Omega$ & klein z.B. $50\Omega$ \\ \hline
	$v_i$ & groß z.B. $100$ & <1 z.B. $0,9$ & groß z.B. $100$ \\ \hline
	$v_u$ & groß z.B. $100$ & groß z.B. $100$ & <1 z.B. $0,99$ \\ \hline
	$v_p$ & sehr groß z.B. $1k\Omega$ & groß z.B. $100$ & groß z.B. $100$ \\ \hline
	$\varphi_u$ & gegenphasig $180^{\circ}$ & gleichphasig $0^{\circ}$ & gleichphasig $0^{\circ}$ \\ \hline
	\end{tabular}
	\end{small}
\end{frame}

\section*{Integierte Schaltung}
\begin{frame}
\frametitle{Integierte Schaltung}
\begin{minipage}{0.3\textwidth}
	\includegraphics[scale=0.15]{a06/IC.jpg}\\
	Abb. 21: IC auf einer Platine \cite{wp}
\end{minipage}
\hspace{0.5cm}
\begin{minipage}{0.5\textwidth}
	\begin{itemize}
		\item Heutzutage komplexe Schaltungen auf einem Halbleiterkristall
	\end{itemize}
\end{minipage}\\
\vspace{0.5cm}
\begin{center}
\includegraphics[scale=0.4 ]{a06/IC2.jpg}\\
	Abb. 22: Offener IC \cite{wpen}
\end{center}
\end{frame}

\section*{Operationsverstärker}


\begin{frame}
\frametitle{Operationsverstärker}
\begin{center}
	\includegraphics[scale=0.35]{a06/OPV-intern.png}\\
	Abb. 23: Innerer Aufbau eines OPVs \cite{wp}
\end{center}
\end{frame}

\begin{frame}
\frametitle{Invertierender Verstärker}
	\begin{center}
	\includegraphics[scale=1]{a06/OPV-Inverter.png}\\
	Abb.24: OPV als invertierender Verstärker
	\end{center}
	\begin{itemize}
		\item	$v_u = -\frac{U_A}{U_E} = \frac{R_2}{R_1}$
	\end{itemize}
\end{frame}

\begin{frame}
\frametitle{Nichtinvertierender Verstärker}
	\begin{center}
	\includegraphics[scale=1]{a06/OPV-nonInverter.png}\\
	Abb.25: OPV als nicht-invertierender Verstärker
	\end{center}
	\begin{itemize}
		\item	$v_u = \frac{U_A}{U_E} = 1+\frac{R_2}{R_1}$
		\hspace{3mm}
		\item	Wird häufig als Impedanzwandler eingesetzt
	\end{itemize}
\end{frame}

\begin{frame}
\frametitle{Impedanzwandler}
	\begin{center}
	\includegraphics[scale=1]{a06/Impedanzwandler.png}\\
	Abb.26: OPV als Impedanzwandler
	\end{center}
	\begin{itemize}
		\item	Besitzt sehr großen Eingangswiderstand und sehr kleinen Ausgangswiderstand
		\item	Ist quasi ein nicht-invertierender Verstärker mit $R_2 = 0 \Omega$ und $R_1 = \infty$
	\end{itemize}
\end{frame}


\section*{Die Elektronenröhre}

\begin{frame}
	\begin{center}
		\includegraphics[scale=0.35]{a06/Roehren.jpg}\\
		Abb.27: Verschiedene Elektronenröhren \cite{wmde}
	\end{center}
	\begin{itemize}
		\item	Arbeiten mit hohen Spannungen und geringen Strömen
		\item	Ist ein gepoltes Bauteil
	\end{itemize}
\end{frame}

\begin{frame}
\frametitle{Die Röhre}
\begin{minipage}{0.3\textwidth}
	\includegraphics[scale=0.35]{a06/ERohre.png}\\
	Abb.28: Symbol einer Triode \cite{wp}
\end{minipage}
\hspace{0.5cm}
\begin{minipage}{0.5\textwidth}
\begin{small}
	\begin{itemize}
		\item Heizung löst Elektronen aus Kathode
		\item Elektronen werden Richtung Anode beschleunigt
		\item Gitter verändert elektrisches Feld
		\item Gitterspannung steuert Anodenstrom
	\end{itemize}
	\end{small}
\end{minipage}\\
%\vspace{0.1cm}
\begin{center}
\includegraphics[scale=0.4 ]{a06/Triode.jpg}\\
	Abb.29: Triode aus alten Fabfernsehern \cite{wp}
\end{center}
\end{frame}


\renewcommand{\refname}{Referenzen}

\hypertarget{refs}{}
\textcolor{white}{} \\ %\vspace{} geht nicht
\Large Referenzen/Links
\footnotesize

\begin{thebibliography}{}
    \bibitem{e03}   Moltrecht A 06: \\
                    \url{http://www.darc.de/referate/ajw/ausbildung/darc-online-lehrgang/technik-klasse-a/technik-a06/}
                    
	\bibitem{bnetza}	Fragenkatalog Technik Klasse A der Bundesnetzagentur:\\
		\url{http://snh.rp-online.de/download/technika.pdf}                    
                    
    \bibitem{wp}    Wikipedia DE: \\
    	\url{http://de.wikipedia.org/wiki/Datei:Scheme_of_n-junction_field-effect_transistor_de.svg}
    	\url{http://de.wikipedia.org/wiki/Datei:FET-Typen_(mit_Schaltbildern).svg}
        \url{http://de.wikipedia.org/wiki/Datei:Op-amp_symbol.svg}
        \url{http://de.wikipedia.org/wiki/Datei:Normsymbol_OPV.svg}      
        \url{http://de.wikipedia.org/wiki/Datei:OpAmpTransistorLevel_Colored_DE.svg}  
        \url{http://de.wikipedia.org/wiki/Datei:Chips_3_bg_102602.jpg}
        \url{http://de.wikipedia.org/wiki/Datei:Triode-Symbol_de.svg}
        \url{http://de.wikipedia.org/wiki/Datei:Strahltriode.jpg}
                    
    \bibitem{wpen}	Wikipedia EN:\\
    	\url{http://en.wikipedia.org/wiki/File:Intel_8742_153056995.jpg}
    	
    \bibitem{wmde}	Wikimedia DE:\\
    	\url{http://commons.wikimedia.org/wiki/File:Scheme_of_metal_oxide_semiconductor_field-effect_transistor.svg?uselang=de}
    	\url{http://commons.wikimedia.org/wiki/File:MISFET-Transistor_Symbole.svg?uselang=de}
    	\url{http://commons.wikimedia.org/wiki/File:Elektronenroehren-auswahl.jpg}
    	
    \bibitem{wmen} Wikimedia EN:\\
    	\url{http://commons.wikimedia.org/wiki/File:Transistorgrundschaltungen.svg}\
    				
\end{thebibliography} 

% Hier könnte noch eine Kontaktfolie stehen

\end{document}

