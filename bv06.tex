% Foliensatz: "AFu-Kurs nach DJ4UF" von DK0TU, Amateurfunkgruppe der TU Berlin
% Lizenz: CC BY-NC-SA 3.0 de (http://creativecommons.org/licenses/by-nc-sa/3.0/de/)
% Autoren: Sebastian Lange <dl7bst@dk0tu.de>

preamble.dk0tu.tex
\subtitle{Betriebstechnik/Vorschriften 06: \\
          Rufzeichen - Landeskenner \\[2em]}
\date{Stand 20.11.2014}
 \begin{document}

\begin{frame}
    \titlepage
    \vfill
    \begin{center}
        \ccbyncsaeu\\
        {\tiny This work is licensed under the \em{Creative Commons Attribution-NonCommercial-ShareAlike 3.0 License}.}\\[0.5ex]
         \tiny Amateurfunkgruppe der Technische Universität Berlin (AfuTUB), DKØTU
         %\includegraphics[scale=0.5]{img/DK0TU_Logo.pdf}
    \end{center}
\end{frame}


%fixme Referenzen/Fußnoten-Systematik vereinheitlichen

\section{Einleitung}

\begin{frame}
    \frametitle{Einleitung / Disclaimer}

    \begin{center}
        \Large{Das Kapitel ist stark zusammengekürzt, da dieser Kurs eher
        praktischer Natur sein soll.}
    \end{center}

    \normalsize

    \begin{itemize}
        \item Es werden nur Landeskenner behandelt, die in Prüfungsfragen eine Rolle spielen.
        \item Zur Vollständigkeit lest bitte selbstständig die Lektion
              \texttt{B/V 06} im \emph{Moltrecht}!
    \end{itemize}

\end{frame}


\begin{frame}
    \frametitle{Einleitung / Rufzeichen}

    \begin{center}
        \Large{Wozu gibt es Rufzeichen?} \\[1em]
        \Large{Welche kennt ihr bereits und wie unterscheiden sie sich? ($\Rightarrow$ Tafel)}
    \end{center}

\end{frame}

\begin{frame}
    \frametitle{Einleitung / Landeskenner}

    \begin{center}
        \Large{Wurde vermutlich schon erkannt.}
    \end{center}

\end{frame}

\section{Rufzeichen}

\begin{frame}
    \frametitle{Rufzeichen}

    In \textbf{DL} personengebunden. Kann aber in anderen Staaten auch an die Station
    gebunden sein. \\[1em]

    Mindestens drei, normalerweise bis sechs Zeichen. Kann aber auch länger
    sein.\\[1em]
    
    Aufbau:

    \begin{itemize}
        \item 2-3 Zeichen \textbf{Präfix}
        \begin{itemize}
            \item 1-2 Zeichen Landeskenner
            \item 1 Ziffer (z.B. Lizenzklasse, Region, ...)
        \end{itemize}
        \item $>$1 Zeichen \textbf{Suffix}
    \end{itemize}

\end{frame}

\section{Landeskenner}

\begin{frame}
    \frametitle{Landeskenner}

    \begin{center}
        Deutschland: D\textbf{A}-D\textbf{R} \\[1em]
    \end{center}

    Vorsicht! \\
    DS-DT: Südkorea \\
    DU-DZ: Philippinen \\[2em]

    Nachschlagewerke für Landeskenner:

    \begin{itemize}
        \item DXCC-Liste\footnote{\url{http://dl3vtl.darc.de/Landeskenner.htm}}
              (Landeskennerliste) der ITU
        \item Amateurfunkhandbücher
        \item Rufzeichenlisten
        \item Logging Software
    \end{itemize}

\end{frame}

\subsection{Europäische Landeskenner}

\begin{frame}
    \frametitle{Europäische Landeskenner}

    % todo Liste als Text in die Folien übertragen
    Liste der Prüfungsrelevanten Länder von DJ4UF
    \footnote{\scriptsize\url{http://www.amateurfunkpruefung.de/lehrg/bv06/lk_europa.gif}}... \\[2em]
    
    Don't Panic! ''Eselsbrücken'' helfen weiter, z.B.:
   
    \footnotesize
    \begin{itemize}
        \item HV: Heiliger Vater
        \item PA: Pay Bas
        \item LA: Lachse
        \item HB: Helvetia Bern
        \item EA: España
        \item EI: Eire
        \item HA: Hungary
        \item SP: Soz. Polen
        \item YL: Young Lettin
    \end{itemize}

\end{frame}

\subsection{Außereuropäische Landeskenner}

\begin{frame}
    \frametitle{Außereuropäische Landeskenner: \\ RR Funkregionen}

    %todo Karte der Regionen

    \begin{itemize}
        \item \textbf{Region 1}: Europa, Afrika, Vorderasien (ohne Iran), Russland, Georgien,
              Armenien, Aserbaidschan, Kasachstan, Turkmenistan, Usbekistan,
              Tadschikistan, Kirgisistan, Mongolei
        \item \textbf{Region 2}: Nordund Südamerika, Karibik, Grönland, Hawaii
        \item \textbf{Region 3}: Australien, Neuseeland, Ozeanien und Asien ohne die unter
              Region 1 genannten Länder Asiens.
    \end{itemize}

\end{frame}

\begin{frame}
    \frametitle{Außereuropäische Landeskenner: Region 1}

    Liste der Prüfungsrelevanten Länder von DJ4UF:

    \begin{center}
        \includegraphics[width=0.8\textwidth]{bv06/lk_region1.png}
        \footnote{\tiny \url{http://www.amateurfunkpruefung.de/lehrg/bv06/lk_region1.gif}}
    \end{center}

\end{frame}

\begin{frame}
    \frametitle{Außereuropäische Landeskenner: Region 2}

    Liste der Prüfungsrelevanten Länder von DJ4UF:

    \begin{center}
        \includegraphics[width=0.8\textwidth]{bv06/lk_region2.png}
        \footnote{\tiny \url{http://www.amateurfunkpruefung.de/lehrg/bv06/lk_region2.gif}}
    \end{center}

\end{frame}

\begin{frame}
    \frametitle{Außereuropäische Landeskenner: Region 3}

    Liste der Prüfungsrelevanten Länder von DJ4UF:

    \begin{center}
        \includegraphics[width=0.8\textwidth]{bv06/lk_region3.png}
        \footnote{\tiny \url{http://www.amateurfunkpruefung.de/lehrg/bv06/lk_region3.gif}}
    \end{center}

\end{frame}

\begin{frame}
    \frametitle{Außereuropäische Landeskenner: ''Eselsbrücken''}

    Von ihm auch noch einige Merkhilfen:

    \footnotesize
    \begin{itemize}
        \item TF: Tiefer Frost
        \item VK: Viele Kängurus
        \item OA: Obere Anden
        \item ZL: Zealand
        \item PY: Pyranha
    \end{itemize}

    Diverse ausführliche Präfixlisten\footnote{\url{http://ac6v.com/prefixes.htm}}
    (viel genauer unterteilt als nur Landeskenner) gibt es online.

\end{frame}

\section{T/R 61-01}

\begin{frame}
    \frametitle{CEPT-Empfehlung T/R 61-01}

    Wenn die Empfehlung vom Staat anerkannt wird, darf man mit einem
    zusätzlichen Präfix vorübergehend Amateurfunk ausüben. Beispiele:

    \begin{itemize}
        \item \textbf{F/}DL7BST
        \item \textbf{DL/}HB9ZZ
        \item ...
    \end{itemize}

\end{frame}

\section{Zusatz-Kennzeichnungen}

\begin{frame}
    \frametitle{Zusatz-Kennzeichnungen}

    Es ist möglich, aber nicht zwingend notwendig, folgende
    Zusatz-Kennzeichnungen beim Funkbetrieb zu verwenden:

    \begin{itemize}
        \item \textbf{/mm} \footnote{maritime mobile} offene See (kein Binnengewässer!)
        \item \textbf{/am} \footnote{aeronautic mobile} Luftfahrzeug
        \item \textbf{/m}  \footnote{mobile} bewegliche Amateurfunkstellen, z.B. KFZ, Fahrrad, Boot (Binnengewässer)
        \item \textbf{/p}  \footnote{portable} vorübergehender ortsfester Funkbetrieb
    \end{itemize}

    Eine Sondergenehmigung ist für all diese Betriebsarten nicht notwendig. Auch
    laut \emph{StVO} ist Funken am Steuer nicht verboten.

\end{frame}

\section{Deutsche Rufzeichen}

\begin{frame}
    \frametitle{Deutsche Rufzeichen}

    Geregelt im \textbf{Rufzeichenplan gemäß § 10 Abs. 3 AFuV}. \\[1em]

    In diesem finden sich alle zugeteilten Rufzeichen in Verbindung mit dem
    Namen des Inhabers und die Standorte von Relaisfunkstellen und Funkbaken.
    \\[1em]

    Inzwischen auch online\footnote{\scriptsize\url{http://ans.bundesnetzagentur.de/Amateurfunk/Rufzeichen.aspx}}
    abrufbar.

    %todo Aufteilung DL Rufzeichen siehe Moltrecht.

    Ausführliche Liste der deutschen Rufzeichenblöcke in der Wikipedia
    \footnote{\url{http://de.wikipedia.org/wiki/Amateurfunkrufzeichen}}. \\[1em]

    Einzige Prüfungsfrage: Zuordnung des Präfix \textbf{DO}.

\end{frame}

\section{Lernhinweise}

\begin{frame}
    \frametitle{Lernhinweise}

    \begin{itemize}
        \item zeitnah die Moltrecht-Lektion \texttt{B/V 06} durcharbeiten
        \item in der Prüfung geht es vorrangig um die europäischen und einige
              außereuropäische Landeskenner, die reduziert weden können
        \item Eselsbrücken
        \item Ausschlussverfahren
        \item ... noch geschicktere Ausschlussverfahren
        \item nicht fertig machen lassen und in der Woche vor der Prüfung
              nochmal alles mit dem Simulator durchklicken
    \end{itemize}

\end{frame}

\section{Referenzen}

\begin{frame}
    \frametitle{Referenzen/Links}
    
    \footnotesize
    \begin{itemize}
        \item Moltrecht B/V 06: \\
              \url{http://www.amateurfunkpruefung.de/lehrg/bv06/bv06.html}
        \item Wikipedia DE: \\
              \url{http://de.wikipedia.org/wiki/Amateurfunkrufzeichen} \\
              \url{http://de.wikipedia.org/wiki/Regelungen_im_Amateurfunkdienst}
    \end{itemize}

\end{frame}

% Hier könnte noch eine Kontaktfolie stehen

\end{document}

