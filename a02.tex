% Foliensatz: "AFu-Kurs nach DJ4UF" von DK0TU, Amateurfunkgruppe der TU Berlin
% Lizenz: CC BY-NC-SA 3.0 de (http://creativecommons.org/licenses/by-nc-sa/3.0/de/)
% Autoren: Martin Deutschmann <martin.deutschmann@campus.tu-berlin.de>
% Korrekturen: Lars Weiler <dc4lw@darc.de>, Sebastian Lange <dl7bst@dk0tu.de>

preamble.dk0tu.tex
\subtitle{Technik Klasse A 02: \\
  Der Widerstand und seine Schaltungsarten \\[2em]}
\date{Stand 09.01.2017}
 \begin{document}

\begin{frame}
    \titlepage
    \vfill
    \begin{center}
        \ccbyncsaeu\\
        {\tiny This work is licensed under the \em{Creative Commons Attribution-NonCommercial-ShareAlike 3.0 License}.}\\[0.5ex]
         \tiny Amateurfunkgruppe der Technische Universität Berlin (AfuTUB), DKØTU
         %\includegraphics[scale=0.5]{img/DK0TU_Logo.pdf}
    \end{center}
\end{frame}


%fixme Referenzen/Fußnoten-Systematik vereinheitlichen

\section{Einleitung}

\begin{frame}
  \frametitle{Einleitung / Widerstand}
  \begin{center}
    \begin{figure}
      \includegraphics[width=.7\textwidth,height=.75\textheight,keepaspectratio]{e04/URI.png}
      \attribcaption{Strom, Spannung und Widerstand}{TU Wien}{http://www.fet.at/twiki/pub/Homepage/OnlineFetzn/fetzn_ausgabe_maerz-2013_online.pdf}{}
    \end{figure}
  \end{center}
\end{frame}


\begin{frame}
  \frametitle{Einleitung / Widerstand}

  \begin{center}
    \begin{figure}
      \includegraphics[width=0.8\textwidth,height=.75\textheight,keepaspectratio]{e04/Widerstaende.jpg}
      \attribcaption{Widerstände}{Honina}{https://commons.wikimedia.org/wiki/File:Resistors-photo.JPG}{\ccbysa}
    \end{figure}
  \end{center}

  % TODO SMD-Bauteile ergaenzen

\end{frame}

\begin{frame}
  \frametitle{Einleitung / Widerstand}

  \begin{center}
    \begin{figure}
      \includegraphics[width=.4\textwidth,height=.2\textheight,keepaspectratio]{e04/Resistor_symbol_IEC.png}
      \attribcaption{Schaltzeichen für elektrischen Widerstand nach DIN EN 60617}{Markus Kuhn}{https://commons.wikimedia.org/wiki/File:Resistor_symbol_IEC.svg}{\ccpd}
    \end{figure}
  \end{center}

  \begin{center}
    \begin{figure}
      \includegraphics[width=.4\textwidth,height=.4\textheight,keepaspectratio]{e04/R_LTspice.png}
      \caption{aus LTspice}
    \end{figure}
  \end{center}
\end{frame}

\begin{frame}
  \frametitle{Aufbau von Widerständen}
  \textbf{Kohleschichtwiderstände:}
  \pause
  \begin{itemize}
    \item Günstig in der Herstellung
    \item Hohe Toleranzen
  \end{itemize}
  \pause
  \textbf{Metallschichtwiderstände:}
  \pause
  \begin{itemize}
    \item Sehr präzise
  \end{itemize}
  \pause
  \textbf{Metalloxidwiderstände:}
  \pause
  \begin{itemize}
    \item HF tauglich
  \end{itemize}
  \pause
  \textbf{Drahtwiderstände:}
  \pause
  \begin{itemize}
    \item Leistungswiderstände für NF
  \end{itemize}
\end{frame}

\section{Spezifischer Widerstand}

\begin{frame}
  \frametitle{Leitende Materialien}
  \begin{columns}
    \column{.6\textwidth}
    \begin{tabular}{llr}
      Material & \multicolumn{2}{r}{Spezifischer Widerstand $\rho$ in $\frac{\Omega \cdot mm^2}{m}$} \\ \hline
      Silber & & $1,587 \cdot 10^{-2}$ \\
      Kupfer & & $1,721 \cdot 10^{-2}$ \\
      Gold & & $2,214 \cdot 10^{-2}$ \\
      Aluminium & & $2,65 \cdot 10^{-2}$ \\
      Zinn & & $1,15 \cdot 10^{-1}$ \\
      Blei & & $2,08 \cdot 10^{-1}$ \\
      Quecksilber & & $9,412 \cdot 10^{-1}$ \\
      Germanium & \only<2>{$\leftarrow$ \textbf{Halbleiter}} & $4,6 \cdot 10^{5}$\\
      Porzellan & \only<2>{$\leftarrow$ \textbf{Isolator}} & $1 \cdot 10^{18}$ \\
    \end{tabular}

    \column{.35\textwidth}
    \only<3>{
    \begin{block}{Berechnung des Widerstands}
      $$R = \rho \cdot \frac{\ell}{A}$$
    \end{block}
    }
  \end{columns}
\end{frame}

\section{Skin-Effekt}

\begin{frame}
  \frametitle{der Skin-Effekt}
  \begin{itemize}
    \item Tritt bei höherfrequenter Wechselspannung auf
    \item Verdrängt die Elektronen aus dem Leitungsinneren an die Leiteroberfläche
    \item Dadurch steigt der Widerstand im Leiter
  \end{itemize}
\end{frame}

\begin{frame}
  \frametitle{Ursachen des Skin-Effektes}
  \begin{columns}
    \column{.4\textwidth}
    \begin{center}
      \begin{figure}
        \includegraphics[width=\textwidth,height=.75\textheight,keepaspectratio]{a02/Skineffect.png}\\
        \attribcaption{Überlagerung von Wechsel- und Wirbelströmen}{Biezl}{https://commons.wikimedia.org/wiki/File:Skineffect_reason.svg}{\ccpd}
      \end{figure}
    \end{center}
    \column{.4\textwidth}
    \begin{itemize}
      \item Ursache des Skin-Effektes ist das magnetische Feld
      \item Es erzeugt Wirbelströme im Innern des Leiters
      \item Diese sind dem Erzeugerstrom entgegengerichtet
      \item Das wechselnde Magnetfeld erzeugt im Leiter eine höhere Gegenspannung als am Rand
    \end{itemize}
  \end{columns}
\end{frame}

\begin{frame}
  \frametitle{Folgen \& Gegenmaßnahmen}
  \textbf{Folgen:}
  \begin{itemize}
    \item Der Leiterquerschnitt sinkt
    \item Die Impedanz steigt
  \end{itemize}
  \textbf{Gegenmaßnahmen}
  \begin{itemize}
    \item Verwendung von Hohlleitern
    \item Mehrere voneinander isolierte Drähte nutzen
    \item Oberfläche versilbern
  \end{itemize}
\end{frame}

\section{Ohmsches Gesetz}

\begin{frame}
  \frametitle{Was ist das ohmsche Gesetz?}
  \begin{itemize}
    \item Das ohmsche Gesetz ist folgendes:
  \end{itemize}
  \begin{center}
    \begin{figure}
      \includegraphics[width=.5\textwidth,height=.5\textheight,keepaspectratio]{e03/Ohm_law_triangle.png}
      \attribcaption{Ohmsches Dreieck}{Eirik}{https://commons.wikimedia.org/wiki/File:Ohm\%27s_law_triangle.PNG}{\ccpd}
    \end{figure}
  \end{center}
  \begin{itemize}
    \item Aber was sagt uns das nun?
  \end{itemize}
\end{frame}

\begin{frame}
  \frametitle{Das ohmsche Gesetz}
  \begin{itemize}
    \item Das ohmsche Gesetz gibt uns die Abhängigkeiten zwischen Spannung, Strom \& ohmschen Widerstand an
    \item Dadurch wissen wir, dass sich zum Beispiel der Strom an einem konstanten Widerstand proportional zur Spannung ändert
  \end{itemize}
\end{frame}

\begin{frame}
  \frametitle{Der Innenwiderstand}
  \begin{itemize}
    \item Oftmals bemerken wir einen Spannungsabfall zwischen einer Maschine im Leerlauf und der gleichen Maschine bei Belastung
    \item Dies führen wir auf den Innenwiderstand der Maschine zurück
  \end{itemize}
  \begin{center}
    \begin{figure}
      \includegraphics[width=\textwidth,height=.5\textheight,keepaspectratio]{e03/Innenwiderstand.png}
      \caption{Innenwiderstand einer Batterie}
    \end{figure}
  \end{center}
\end{frame}

\begin{frame}
  \frametitle{der Innenwiderstand}
  \begin{itemize}
    \item Um den Innenwiderstand zu ermitteln nutzen wir wieder das ohmsche Gesetz
    \item Dabei gilt es zu beachten, dass diesmal die Differenzen der Spannungen und des Stromes zwischen dem Leerlauf und dem belasteten Fall verrechnet werden
    \item Es gilt:
  \end{itemize}
  \begin{block}{Innenwiderstand}
    \begin{center}
      $R_{innen} = \cfrac{\Delta U}{\Delta I}$
    \end{center}
  \end{block}
  \begin{itemize}
    \item Um den Wert nicht zu sehr zu verfälschen sollten \textbf{Spannungsquellen einen niedrigen} und \textbf{Stromquellen einen hohen Innenwiderstand} besitzen
  \end{itemize}
\end{frame}

% FIXME + Leistungsanpassung, Strom-/Spannungsanpassung Innenwiderzustand zu Lastwiderstand

\section{Rechnen mit Widerständen}

\subsection{Reihenschaltung}
\begin{frame}
  \frametitle{Reihenschaltung}

  \begin{columns}
    \column{.4\textwidth}
    \begin{center}
      \begin{figure}
        \includegraphics[width=.4\textwidth,height=.5\textheight,keepaspectratio]{e04/Reihe.png}
        \caption{aus LTspice}
      \end{figure}
    \end{center}
    \pause
    \column{.55\textwidth}
    \begin{block}{Berechnung}
      $$R_{gesamt} = R_1 + R_2 + R_3 + ...$$
    \end{block}
  \end{columns}

\end{frame}

\subsection{Parallelschaltung}
\begin{frame}
  \frametitle{Parallelschaltung}
  \begin{columns}
    \column{.4\textwidth}
    \begin{center}
      \begin{figure}
        \includegraphics[width=.8\textwidth,height=.5\textheight,keepaspectratio]{e04/Parallel.png}
        \caption{aus LTspice}
      \end{figure}
    \end{center}
    \pause
    \column{.55\textwidth}
    \begin{block}{Berechnung}
      $$\frac{1}{R_{gesamt}} = \frac{1}{R_1} + \frac{1}{R_2} + \frac{1}{R_3} + ...$$
    \end{block}
  \end{columns}
\end{frame}

\subsection{Ersatzwiderstand}
\begin{frame}
  \frametitle{Ersatzwiderstand}
  \begin{columns}
    \column{.47\textwidth}
    \begin{figure}
      \includegraphics[width=1\textwidth,height=.2\textheight,keepaspectratio]{e04/Ersatzwiderstand1.png}
      \caption{Ersatzwiderstand}
    \end{figure}
    \column{.47\textwidth}
    \begin{figure}
      \includegraphics[width=1\textwidth,height=.2\textheight,keepaspectratio]{e04/Ersatzwiderstand2.png}
      \caption{Ersatzwiderstand}
    \end{figure}
  \end{columns}
  \begin{columns}
    \column{.47\textwidth}
    \pause
    \begin{exampleblock}{Berechnung}
      $R_1 + (R_2 \parallel R_3)$ \\[1.5em]
      $\Rightarrow R_1 + \cfrac{1}{\cfrac{1}{R_2} + \cfrac{1}{R_3}}$
    \end{exampleblock}
    \column{.47\textwidth}
    \pause
    \begin{exampleblock}{Berechnung}
      $(R_1 + R_2) \parallel R_3$\\[1.5em]
      $\Rightarrow \cfrac{1}{\cfrac{1}{R_1 + R_2} + \cfrac{1}{R_3}}$
    \end{exampleblock}
  \end{columns}
\end{frame}

\subsection{Spannungsteiler}
\begin{frame}
  \frametitle{Spannungsteiler}
  \begin{columns}
    \column{.45\textwidth}
    \begin{figure}
      \includegraphics[width=\textwidth,height=.55\textheight,keepaspectratio]{a02/spannungsteiler-unbelastet.png}
      \attribcaption{Unbelasteter Spannungsteiler}{Biezl}{https://commons.wikimedia.org/wiki/File:Einfacher-unbelasteter-Spannungsteiler.svg}{\ccpd}
    \end{figure}
    \column{.45\textwidth}
    \begin{figure}
      \includegraphics[width=\textwidth,height=.55\textheight,keepaspectratio]{a02/spannungsteiler-belastet.png}\\
      \attribcaption{Belasteter Spannungsteiler}{Biezl}{https://commons.wikimedia.org/wiki/File:Einfacher-Spannungsteiler.svg}{\ccpd}
    \end{figure}
  \end{columns}
  \begin{itemize}
    \item $U_2$ ist beim belasteten Spannungsteiler kleiner als beim unbelasteten Spannungsteiler
  \end{itemize}
\end{frame}

\begin{frame}
  \frametitle{Rechnen beim Spannungsteiler}
  \textbf{beim unbelasteten Spannungsteiler:}
  \begin{itemize}
    \item $U_2$ ist die Spannung über $R_2$
  \end{itemize}
  \begin{block}{unbelasteter Spannungsteiler}
    \begin{center}
      $$\frac{U_2}{U} = \frac{R_2}{R_{gesamt}}$$
    \end{center}
  \end{block}

  \textbf{beim belasteten Spannungsteiler:}
  \begin{itemize}
    \item $U_2$ ist die Spannung über den beiden parallelen Widerstände $R_2$ und $R_L$
  \end{itemize}
  \begin{block}{belasteter Spannungsteiler}
    \begin{center}
      $$\frac{U_2}{U} = \frac{R_{2}||R_{L}}{R_{gesamt}}$$
    \end{center}
  \end{block}
\end{frame}

\section{Übung}

\begin{frame}
  \frametitle{Übungsaufgaben}

  Als Teil des Praxisskriptes im Anschluss.

\end{frame}


\section{Referenzen}
\begin{small}
  \begin{thebibliography}{}
      \setbeamertemplate{bibliography item}[online]
    \bibitem{darc}  Moltrecht A 02: \\
      \url{https://www.darc.de/der-club/referate/ajw/lehrgang-ta/a02/}

    \bibitem{wp}    Wikipedia DE: \\
      \url{http://de.wikipedia.org/wiki/Spannungsteiler}\\
      \url{http://de.wikipedia.org/wiki/Ohmsches_Gesetz}\\
      \url{http://de.wikipedia.org/wiki/Elektrische_Leistung}\\
      \url{http://de.wikipedia.org/wiki/Elektrische_Energie#Elektrische_Energie_in_einem_elektrischen_Feld}\\
  \end{thebibliography}
\end{small}

% Hier könnte noch eine Kontaktfolie stehen

\end{document}

