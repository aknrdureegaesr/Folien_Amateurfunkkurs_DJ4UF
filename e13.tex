% Foliensatz: "AFu-Kurs nach DJ4UF" von DK0TU, Amateurfunkgruppe der TU Berlin
% Lizenz: CC BY-NC-SA 3.0 de (http://creativecommons.org/licenses/by-nc-sa/3.0/de/)
% Autoren:Felix Baum <baum@campus.tu-berlin.de>, Martin Deutschmann,

preamble.dk0tu.tex
\subtitle{Technik Klasse E 13: \\
          Transistor \& Verstärker \\[2em]}
\date{Stand 25.11.2015}
 \begin{document}

\begin{frame}
    \titlepage
    \vfill
    \begin{center}
        \ccbyncsaeu\\
        {\tiny This work is licensed under the \em{Creative Commons Attribution-NonCommercial-ShareAlike 3.0 License}.}\\[0.5ex]
         \tiny Amateurfunkgruppe der Technische Universität Berlin (AfuTUB), DKØTU
         %\includegraphics[scale=0.5]{img/DK0TU_Logo.pdf}
    \end{center}
\end{frame}


%fixme Referenzen/Fußnoten-Systematik vereinheitlichen
\begin{frame}
\frametitle{Was macht ein Transistor?}
   	\begin{itemize}
	
	 \item verstärken (analog)
         \item schalten (digital)
         \item Video: \url{https://www.youtube.com/watch?v=IcrBqCFLHIY}
  	\end{itemize}	
\end{frame}
\begin{frame}
  \frametitle{Transistoren}
  \begin{figure}
    \includegraphics[width=\textwidth,height=.65\textheight,keepaspectratio]{e13/Transistors-white.jpg}
    \attribcaption{Verschiedene Transistor-Bauformen}{Benedikt.Seidl}{https://commons.wikimedia.org/wiki/File:Transistors-white.jpg}{\ccpd}
  \end{figure}
\end{frame}
\section*{Bipolarer Transistor}
\begin{frame}
\frametitle{Bipolarer Transistor}
\begin{minipage}{0.4\textwidth}
	\begin{figure}
      \includegraphics[width=\textwidth,height=.4\textheight,keepaspectratio]{e13/NPN_hlb.png}\\
      \caption{Schichten eines \hbox{NPN-Transistors}}
    \end{figure}
\end{minipage}
\hspace{0.5cm}
\begin{minipage}{0.4\textwidth}
	 \begin{figure}
      \includegraphics[width=\textwidth,height=.4\textheight,keepaspectratio]{e13/PNP_hlb.png}\\
      \caption{Schichten eines \hbox{PNP-Transistors}}
    \end{figure}
\end{minipage}
\vspace{0.5cm}
\begin{center}
\begin{itemize}
	\item Transistoren bestehen aus drei Halbleiterschichten
	\item Anschlüsse: Basis (B), Kollektor (C), Emitter (E)
\end{itemize}

\end{center}
\end{frame}

\begin{frame}
\frametitle{Ersatzschaltbild}

\begin{minipage}{0.4\textwidth}
	\begin{figure}
      \includegraphics[width=\textwidth,height=.4\textheight,keepaspectratio]{e13/NPN_esb.png}
      \caption{ESB eines \hbox{NPN-Transistors}}
    \end{figure}
\end{minipage}
\hspace{0.5cm}
\begin{minipage}{0.4\textwidth}
	 \begin{figure}
      \includegraphics[width=\textwidth,height=.4\textheight,keepaspectratio]{e13/PNP_esb.png}
      \caption{ESB eines \hbox{PNP-Transistors}}
    \end{figure}
\end{minipage}

\begin{center}
\begin{block}{Funktionsweise}
Der Basisstrom steuert den Kollektorstrom
$$I_{CE} = I_{BE} \cdot \beta$$
\end{block}
\end{center}
\end{frame}

\begin{frame}
  \frametitle{Bipolartransistor Verständnis}
  \begin{minipage}{0.55\textwidth}
    \begin{center}
      \only<1>{
      \begin{figure}
        \includegraphics[scale=0.3]{e13/Transistor_animation1.png}
        \attribcaption{Animation eines alternativen Transistormodells (geschlossen)}{Stefan Riepl}{https://commons.wikimedia.org/wiki/File:Transistor_animation.gif}{\ccbysa}
      \end{figure}
      }
      \only<2>{
      \begin{figure}
        \includegraphics[scale=0.3]{e13/Transistor_animation2.png}
        \attribcaption{Animation eines alternativen Transistormodells (offen)}{Stefan Riepl}{https://commons.wikimedia.org/wiki/File:Transistor_animation.gif}{\ccbysa}
      \end{figure}
      }
    \end{center}
  \end{minipage}
  \begin{minipage}{0.4\textwidth}
    \begin{center}
      \begin{block}{Funktionsweise}
        Der Basisstrom steuert den Kollektorstrom
        $$I_{CE} = I_{BE} \cdot \beta$$
        $\beta = $ Verstärkungsfaktor \\ (Modellabhängig $\beta = 10-900$)
      \end{block}
    \end{center}
  \end{minipage}
\end{frame}


\begin{frame}
\frametitle{Ersatzschaltbild}

\begin{minipage}{0.4\textwidth}
	 \begin{figure}
      \includegraphics[width=\textwidth,height=.5\textheight,keepaspectratio]{e13/NPN.png}
      \caption{NPN Transistor -- \textbf{N}icht \textbf{P}feil \textbf{N}ach Platte}
    \end{figure}
\end{minipage}
\hspace{0.5cm}
\begin{minipage}{0.4\textwidth}
	 \begin{figure}
      \includegraphics[width=\textwidth,height=.5\textheight,keepaspectratio]{e13/PNP.png}
      \caption{PNP Transistor -- \textbf{P}feil \textbf{N}ach \textbf{P}latte}
    \end{figure}
\end{minipage}

\begin{center}
\begin{block}{Funktionsweise}
NPN Transistor braucht $ U_{CE}=+0.7V$ zum schalten\\
PNP Transistor braucht $ U_{CE}=-0.7V$ zum schalten\\
Beide sind Stromgesteuert $I_{CE} = I_{BE} \cdot \beta$
\end{block}
\end{center}
\end{frame}

\section*{Feldeffekt Transistor (FET)}
\begin{frame}
\frametitle{Metall Oxid Semiconductor Feldeffekt Transistor (MOSFET)}
\begin{center}
	 \begin{figure}
      \includegraphics[width=\textwidth,height=.65\textheight,keepaspectratio]{e13/N-Kanal-MOSFET.png}
      \attribcaption{MOSFET in Planartechnologie}{PNG-Version: Markus A. Hennig, SVG-Umsetzung: Cepheiden}{https://commons.wikimedia.org/wiki/File:N-Kanal-MOSFET_(Schema).svg}{\ccbysa}
    \end{figure}
\end{center}
\end{frame}

\begin{frame}
\frametitle{Metall Oxid Semiconductor Feldeffekt Transistor (MOSFET)}
\begin{center}
	\begin{figure}
      \includegraphics[width=\textwidth,height=.4\textheight,keepaspectratio]{e13/N-Ch_Enh_Labelled.png}
      \caption{N-Kanal beschriftet}
    \end{figure}
\begin{block}{Funktionsweise}
\begin{itemize}
	\item Gate steuert Kanalbreite durch Spannung $U_{GS}$
	\item Je größer $U_{GS}$ desto größer $I_{DS}$, da $R_{DS}$ kleiner wird
\end{itemize}
\end{block}
\end{center}
\end{frame}

\begin{frame}
\frametitle{Feldeffekt Transistor (FET)}
\begin{center}
	\begin{figure}
      \includegraphics[width=\textwidth,height=.65\textheight,keepaspectratio]{e13/FET-overview.png}
      \attribcaption{Übersicht über FETs}{Cepheiden}{https://commons.wikimedia.org/wiki/File:FET-Typen_(mit_Schaltbildern).svg}{\ccpd}
    \end{figure}
\end{center}
\end{frame}

\section{Verstärker}
\begin{frame}
\frametitle{Verstärker}
\begin{center}
\begin{block}{Definition eines Verstärkers nach Captain Obvious}
  \begin{Large}
    Es ist nur dann eine Verstärkung, wenn die Leistung am Ausgang größer ist, als die am Eingang
  \end{Large}
\end{block}
\end{center}
\end{frame}

\section*{Operations\-verstärker}
\begin{frame}
\frametitle{Operationsverstärker}
\begin{minipage}{0.4\textwidth}
	 \begin{figure}
      \includegraphics[width=\textwidth,height=.5\textheight,keepaspectratio]{e13/OPV.png}
      \attribcaption{OPV nach ANSI}{Omegatron}{https://commons.wikimedia.org/wiki/File:Op-amp_symbol.svg}{\ccbysa}
    \end{figure}
\end{minipage}
\hspace{0.5cm}
\begin{minipage}{0.4\textwidth}
	\begin{figure}
      \includegraphics[width=\textwidth,height=.5\textheight,keepaspectratio]{e13/OPV-ger.png}
      \attribcaption{OPV nach DIN}{Daniel Braun}{https://commons.wikimedia.org/wiki/File:Normsymbol_OPV.svg}{\ccpd}
    \end{figure}
\end{minipage}
\begin{center}
\begin{itemize}
      \item Transistoren bestehen aus drei Halbleiterschichten
      \item Anschlüsse für Versorgungsspannung
      \item Anschlüsse: Basis (B), Kollektor (C), Emitter (E)
      \item diverse Schaltungsarten für unterschiedliche Funktionsweise möglich
\end{itemize}
\end{center}
\end{frame}

\begin{frame}
\frametitle{Operationsverstärker}
\begin{center}
	\begin{figure}
      \includegraphics[width=\textwidth,height=.65\textheight,keepaspectratio]{e13/OPV-intern.png}
      \attribcaption{Innerer Aufbau eines OPVs}{Daniel Braun derivative work: Cepheiden}{https://commons.wikimedia.org/wiki/File:OpAmpTransistorLevel_Colored_DE.svg}{\ccbysa}
    \end{figure}
\end{center}
\end{frame}

\section*{Integierte Schaltung}
\begin{frame}
\frametitle{Integierte Schaltung}
\begin{minipage}{0.3\textwidth}
	 \begin{figure}
      \includegraphics[width=\textwidth,height=.6\textheight,keepaspectratio]{e13/IC.jpg}
      \attribcaption{IC auf einer Platine}{Luestling}{https://commons.wikimedia.org/wiki/File:Chips_3_bg_102602.jpg}{\ccpd}
    \end{figure}
\end{minipage}
\begin{minipage}{0.5\textwidth}
	\begin{itemize}
		\item Heutzutage komplexe Schaltungen auf einem Halbleiterkristall
	\end{itemize}

\vspace{0.5cm}
\begin{center}
\begin{figure}
        \includegraphics[width=\textwidth,height=.6\textheight,keepaspectratio]{e13/IC2.jpg}
        \attribcaption{Offener IC}{Ioan Sameli}{https://commons.wikimedia.org/wiki/File:Intel_8742_153056995.jpg}{\ccbysa}
      \end{figure}
\end{center}
\end{minipage}
\end{frame}

\section*{Die Röhre}
\begin{frame}
\frametitle{Die Röhre}
\begin{minipage}{0.3\textwidth}
	 \begin{figure}
      \includegraphics[width=\textwidth,height=.5\textheight,keepaspectratio]{e13/ERohre.png}
      \attribcaption{Symbol einer Triode}{RokerHRO, geändert von Fgli}{https://commons.wikimedia.org/wiki/File:Triode-Symbol_de.svg}{\ccpd}
    \end{figure}
\end{minipage}
\hspace{0.5cm}
\begin{minipage}{0.5\textwidth}
\begin{small}
	\begin{itemize}
		\item Heizung löst Elektronen aus Kathode
		\item Elektronen werden Richtung Anode beschleunigt
		\item Gitter verändert elektrisches Feld
		\item Gitterspannung steuert Anodenstrom
	\end{itemize}
	\end{small}
%\vspace{0.1cm}
\begin{center}
 \begin{figure}
        \includegraphics[width=\textwidth,height=.3\textheight,keepaspectratio]{e13/Triode.jpg}
        \attribcaption{Triode aus alten Fabfernsehern}{Ulfbastel}{https://commons.wikimedia.org/wiki/File:Strahltriode.jpg}{\ccby}
      \end{figure}
\end{center}
\end{minipage}
\end{frame}

\section*{Referenzen}
\begin{frame}
    \frametitle{Referenzen/Links}
    
    \footnotesize
    \begin{itemize}
        \item Moltrecht E 08 : \\
              \url{http://www.darc.de/referate/ajw/ausbildung/darc-online-lehrgang/technik-klasse-e/technik-e08/}
		\item Elektronik Kompendium:
			\url{http://www.elektronik-kompendium.de/sites/bau/index.htm}
    \end{itemize}

\end{frame}

% Hier könnte noch eine Kontaktfolie stehen

\end{document}

