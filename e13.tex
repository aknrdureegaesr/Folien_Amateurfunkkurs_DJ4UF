% Foliensatz: "AFu-Kurs nach DJ4UF" von DK0TU, Amateurfunkgruppe der TU Berlin
% Lizenz: CC BY-NC-SA 3.0 de (http://creativecommons.org/licenses/by-nc-sa/3.0/de/)
% Autoren:Felix Baum <baum@campus.tu-berlin.de>, Martin Deutschmann,

preamble.dk0tu.tex
\subtitle{Technik Klasse E 13: \\
          Transistor \& Verstärker \\[2em]}
\date{Stand 08.12.2014}
 \begin{document}

\begin{frame}
    \titlepage
    \vfill
    \begin{center}
        \ccbyncsaeu\\
        {\tiny This work is licensed under the \em{Creative Commons Attribution-NonCommercial-ShareAlike 3.0 License}.}\\[0.5ex]
         \tiny Amateurfunkgruppe der Technische Universität Berlin (AfuTUB), DKØTU
         %\includegraphics[scale=0.5]{img/DK0TU_Logo.pdf}
    \end{center}
\end{frame}


%fixme Referenzen/Fußnoten-Systematik vereinheitlichen

\section*{Bipolarer Transistor}
\begin{frame}
\frametitle{Bipolarer Transistor}
\begin{minipage}{0.4\textwidth}
	\includegraphics[scale=0.2]{e13/NPN_hlb.png}\\
	Abb. 1: Schichten eines NPN-Transistors
\end{minipage}
\hspace{0.5cm}
\begin{minipage}{0.4\textwidth}
	\includegraphics[scale=0.2 ]{e13/PNP_hlb.png}\\
	Abb. 2: Schichten eines PNP-Transistors
\end{minipage}
\vspace{0.5cm}
\begin{center}
\begin{itemize}
	\item Transistoren bestehen aus drei Halbleiterschichten
	\item Anschlüsse: Basis (B), Kollektor (C), Emitter (E)
\end{itemize}

\end{center}
\end{frame}

\begin{frame}
\frametitle{Ersatzschaltbild}

\begin{minipage}{0.4\textwidth}
	\includegraphics[scale=1.4]{e13/NPN_esb.png}\\
	Abb. 3: ESB eines NPN-Transistors
\end{minipage}
\hspace{0.5cm}
\begin{minipage}{0.4\textwidth}
	\includegraphics[scale=1.4]{e13/PNP_esb.png}\\
	Abb. 4: ESB eines PNP-Transistors
\end{minipage}

\begin{center}
\begin{block}{Funktionsweise}
Der Basisstrom steuert den Kollektorstrom
$$I_{CE} = I_{BE} \cdot \beta$$
\end{block}
\end{center}
\end{frame}

\begin{frame}
\frametitle{Bipolartransistor Verständnis}
\begin{minipage}{0.55\textwidth}
\begin{center}
	\only<1>{\includegraphics[scale=0.3]{e13/Transistor_animation1.png}\\
	Abb. 5: Bipolartransistor geschlossen}
    \only<2>{\includegraphics[scale=0.3]{e13/Transistor_animation2.png}\\
	Abb. 5: Bipolartransistor offen}
    \end{center}
    \end{minipage}
    \begin{minipage}{0.4\textwidth}
    \begin{center}
    \begin{block}{Funktionsweise}
        Der Basisstrom steuert den Kollektorstrom
        $$I_{CE} = I_{BE} \cdot \beta$$
        $\beta = $ Verstärkungsfaktor \\ (Modellabhängig $\beta = 10-900$)
    \end{block}
    \end{center}
    \end{minipage}
	\footnote{\url{https://commons.wikimedia.org/wiki/File:Transistor_animation.gif}}\\
\end{frame}


\begin{frame}
\frametitle{Ersatzschaltbild}

\begin{minipage}{0.4\textwidth}
	\includegraphics[scale=1.4]{e13/NPN.png}\\
	NPN Transistor\\ \textbf{P}feil \textbf{N}ach \textbf{P}latte
\end{minipage}
\hspace{0.5cm}
\begin{minipage}{0.4\textwidth}
	\includegraphics[scale=1.4]{e13/PNP.png}\\
	PNP Transistor\\ \textbf{N}icht \textbf{P}feil \textbf{N}ach Platte
\end{minipage}

\begin{center}
\begin{block}{Funktionsweise}
NPN Transistor braucht $ U_{CE}=+0.7V$ zum schalten\\
PNP Transistor braucht $ U_{CE}=-0.7V$ zum schalten\\
Beide sind Stromgesteuert $I_{CE} = I_{BE} \cdot \beta$
\end{block}
\end{center}
\end{frame}

\section*{Feldeffekt Transistor (FET)}
\begin{frame}
\frametitle{Metall Oxid Semiconductor Feldeffekt Transistor (MOSFET)}
\begin{center}
	\includegraphics[scale=0.2]{e13/N-Kanal-MOSFET.png}\\
	Abb. 5: Mosfet in Planartechnologie
	\footnote{\url{https://commons.wikimedia.org/wiki/File:N-Kanal-MOSFET_\%28Schema\%29.svg}}\\
\end{center}
\end{frame}

\begin{frame}
\frametitle{Metall Oxid Semiconductor Feldeffekt Transistor (MOSFET)}
\begin{center}
	\includegraphics[scale=0.2]{e13/N-Ch_Enh_Labelled.png}\\
\begin{block}{Funktionsweise}
\begin{itemize}
	\item Gate steuert Kanalbreite durch Spannung $U_{GS}$
	\item Je größer $U_{GS}$ desto größer $I_{DS}$, da $R_{DS}$ kleiner wird
\end{itemize}
\end{block}
\end{center}
\end{frame}

\begin{frame}
\frametitle{Feldeffekt Transistor (FET)}
\begin{center}
	\includegraphics[scale=0.35]{e13/FET-overview.png}\\
	Abb. 6: Übersicht über FETs
	\footnote{\url{http://de.wikipedia.org/wiki/Datei:FET-Typen_(mit_Schaltbildern).svg}}\\
\end{center}
\end{frame}

\section{Verstärker}
\begin{frame}
\frametitle{Verstärker}
\begin{center}
\begin{block}{Definition eines Verstärkers nach Captain Obvious}
  \begin{Large}
    Es ist nur dann eine Verstärkung, wenn die Leistung am Ausgang größer ist, als die am Eingang
  \end{Large}
\end{block}
\end{center}
\end{frame}

\section*{Operationsverstärker}
\begin{frame}
\frametitle{Operationsverstärker}
\begin{minipage}{0.4\textwidth}
	\includegraphics[scale=0.6]{e13/OPV.png}\\
	Abb. 7: OPV nach ANSI
\end{minipage}
\footnote{\url{http://de.wikipedia.org/wiki/Datei:Op-amp_symbol.svg}}
\hspace{0.5cm}
\begin{minipage}{0.4\textwidth}
	\includegraphics[scale=0.2 ]{e13/OPV-ger.png}\\
	Abb. 8: OPV nach DIN
\end{minipage}
\footnote{\url{http://de.wikipedia.org/wiki/Datei:Normsymbol_OPV.svg}}
\vspace{0.5cm}
\begin{center}
\begin{itemize}
	\item Transistoren bestehen aus drei Halbleiterschichten
	\item Anschlüsse: Basis (B), Kollektor (C), Emitter (E)
\end{itemize}
\end{center}
\end{frame}

\begin{frame}
\frametitle{Operationsverstärker}
\begin{center}
	\includegraphics[scale=0.35]{e13/OPV-intern.png}\\
	Abb. 9: Innerer Aufbau eines OPVs
	\footnote{\url{http://de.wikipedia.org/wiki/Datei:OpAmpTransistorLevel_Colored_DE.svg}}
\end{center}
\end{frame}

\section*{Integierte Schaltung}
\begin{frame}
\frametitle{Integierte Schaltung}
\begin{minipage}{0.3\textwidth}
	\includegraphics[scale=0.15]{e13/IC.jpg}\\
	Abb. 10: IC auf einer Platine
\end{minipage}
\footnote{\url{http://de.wikipedia.org/wiki/Datei:Chips_3_bg_102602.jpg}}
\hspace{0.5cm}
\begin{minipage}{0.5\textwidth}
	\begin{itemize}
		\item Heutzutage komplexe Schaltungen auf einem Halbleiterkristall
	\end{itemize}
\end{minipage}\\
\vspace{0.5cm}
\begin{center}
\includegraphics[scale=0.4 ]{e13/IC2.jpg}\\
	Abb. 11: Offener IC
	\footnote{\url{http://en.wikipedia.org/wiki/File:Intel_8742_153056995.jpg}}
\end{center}
\end{frame}

\section*{Die Röhre}
\begin{frame}
\frametitle{Die Röhre}
\begin{minipage}{0.3\textwidth}
	\includegraphics[scale=0.35]{e13/ERohre.png}\\
	Abb. 12: Symbol einer Triode
\end{minipage}
\footnote{\url{http://de.wikipedia.org/wiki/Datei:Triode-Symbol_de.svg}}
\hspace{0.5cm}
\begin{minipage}{0.5\textwidth}
\begin{small}
	\begin{itemize}
		\item Heizung löst Elektronen aus Kathode
		\item Elektronen werden Richtung Anode beschleunigt
		\item Gitter verändert elektrisches Feld
		\item Gitterspannung steuert Anodenstrom
	\end{itemize}
	\end{small}
\end{minipage}\\
%\vspace{0.1cm}
\begin{center}
\includegraphics[scale=0.4 ]{e13/Triode.jpg}\\
	Abb. 13: Triode aus alten Fabfernsehern
	\footnote{\url{http://de.wikipedia.org/wiki/Datei:Strahltriode.jpg}}
\end{center}
\end{frame}

\section*{Referenzen}
\begin{frame}
    \frametitle{Referenzen/Links}
    
    \footnotesize
    \begin{itemize}
        \item Moltrecht E 08 : \\
              \url{http://www.darc.de/referate/ajw/ausbildung/darc-online-lehrgang/technik-klasse-e/technik-e08/}
		\item Elektronik Kompendium:
			\url{http://www.elektronik-kompendium.de/sites/bau/index.htm}
    \end{itemize}

\end{frame}

% Hier könnte noch eine Kontaktfolie stehen

\end{document}

