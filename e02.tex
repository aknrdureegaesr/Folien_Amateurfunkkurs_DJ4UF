% Foliensatz: "AFu-Kurs nach DJ4UF" von DK0TU, Amateurfunkgruppe der TU Berlin
% Lizenz: CC BY-NC-SA 3.0 de (http://creativecommons.org/licenses/by-nc-sa/3.0/de/)
% Autoren: Felix Baum DB4UM <baum@campus.tu-berlin.de>

preamble.dk0tu.tex
\subtitle{Technik 02: \\
          Spannung und Strom, Wechselspannung \\[2em]}
\date{Stand 23.2.2015}
 \begin{document}

\begin{frame}
    \titlepage
    \vfill
    \begin{center}
        \ccbyncsaeu\\
        {\tiny This work is licensed under the \em{Creative Commons Attribution-NonCommercial-ShareAlike 3.0 License}.}\\[0.5ex]
         \tiny Amateurfunkgruppe der Technische Universität Berlin (AfuTUB), DKØTU
         %\includegraphics[scale=0.5]{img/DK0TU_Logo.pdf}
    \end{center}
\end{frame}


%fixme Referenzen/Fußnoten-Systematik vereinheitlichen

\section*{Einleitung}

\begin{frame}
    \frametitle{Strom - Spannung}
    \begin{itemize}
		\item Was ist Spannung/Strom (elektrische)?
		\item Woher bekommt man Spannung/Strom?
    \end{itemize}
\end{frame}

\begin{frame}
    \frametitle{Spannung}
    \begin{center}
    Elektrische Spannung wird erzeugt durch die Trennung von Ladungen
		\includegraphics[width=.8\textwidth]{e02/ladung.png}
        \footnote{\tiny \url{https://commons.wikimedia.org/wiki/File:VFPt_charges_plus_minus_thumb.svg}}
  	\end{center}
\end{frame}


\section*{Spannung}

\begin{frame}
    \frametitle{Spannung}
    \begin{center}
        Elektrische Spannung $U$ gemessen in [$V$] Volt
	\end{center}
\end{frame}

\begin{frame}
    \frametitle{Batterie}
    \begin{center}
    Schaltungszeichen einer Batterie / Batteriezelle\\
        \includegraphics[width=.99\textwidth]{e02/batterieEagle.png}
	\end{center}
\end{frame}

\begin{frame}
    \frametitle{9V Batterie}
    \begin{center}
    Schaltungszeichen einer 9V Batterie\\
    Pro Zelle $1.8V$\\
    Werden bei Reihenschaltung addiert\\
        \includegraphics[width=.99\textwidth]{e02/9vBatEagle.png}
	\end{center}
\end{frame}

\begin{frame}
    \frametitle{Spannung bestimmen}
    \begin{center}
        \includegraphics[width=.85\textwidth]{e02/Spannung.png}
	\end{center}
\end{frame}

\begin{frame}
    \frametitle{Digitales Multimeter}
    \begin{center}
        \includegraphics[width=.35\textwidth]{e02/digitalmultimeter.jpg}
	\end{center}
\end{frame}

\begin{frame}
    \frametitle{Was wo anschließen?}
	\begin{minipage}{0.4\textwidth}
        \includegraphics[width=1\textwidth]{e02/digitalmultimeterMess.jpg}
	\end{minipage}
	\begin{minipage}{0.4\textwidth}	
	\begin{itemize}
		\item Was kann alles gemessen werden?
		\item Wo anschließen zum Strom messen?
		\item Wo anschließen zum Spannung messen?
		\item Welcher Messbereich?
	\end{itemize}
	\end{minipage}
\end{frame}

\begin{frame}
    \frametitle{Spannung bestimmen}
    \begin{center}
    Was Zeigt das Messgertät an beim Messen von $A$ nach $C$?\\
    (Abb. Ähnlich)
        \includegraphics[width=.75\textwidth]{e02/SpannungMess.png}
	\end{center}
\end{frame}

\begin{frame}
    \frametitle{Potential}
    \begin{center}
    Spannung ist der Potentialunterschied an 2 Punkten\\
    Ein Multimeter misst den Potentialunterschied\\
    Nullpotential ist ein theoretisches Modell.
	\end{center}
\end{frame}

\section*{Strom}

\begin{frame}
    \frametitle{Strom}
    \begin{center}
        Elektrischer Strom ist die gerichtete Bewegung von Ladungsträgern.\\
        Elektischer Strom $I$ in [$A$] Ampare
	\end{center}
\end{frame}

\begin{frame}
    \frametitle{Physikalische vs. Technische Stromrichtung}
	\begin{center}
		Nur Elektronen können sich bewegen
	    \includegraphics[width=.8\textwidth]{e02/Current_notation.png}\\
	    \footnote{\tiny \url{https://commons.wikimedia.org/wiki/File:Current_notation.svg}}
	 \end{center}
\end{frame}

\begin{frame}
    \frametitle{Strom Messen}
    \begin{center}
    	Strom messen in Reihe
        \includegraphics[width=.8\textwidth]{e02/reiheAmpare.png}
	\end{center}
\end{frame}

\begin{frame}
    \frametitle{Wie sollte gemessen werden?}
        \includegraphics[width=1\textwidth]{e02/stromSpannung.png}
        \footnote{\tiny Fragenkatalog Bundesnetzargentur Klasse E}
\end{frame}

\section*{Ladung}

\begin{frame}
    \frametitle{Ladung}
    \begin{center}
    		$$Q = I \cdot t$$
        \includegraphics[width=.8\textwidth]{e02/ladung.png}
        \footnote{\tiny \url{https://commons.wikimedia.org/wiki/File:VFPt_charges_plus_minus_thumb.svg}}
	\end{center}
\end{frame}

\begin{frame}
    \frametitle{Prüfungsfrage}
    \begin{center}
    \begin{tabular}{l||l}\hline
        TB205 & Wie lange könnte man mit einem voll geladenen\\
        " " & Akku mit 55 Ah einen Amateurfunk-Empfänger\\ 
        " " & betreiben, der einen Strom von\\ 
        " " & 0,8 Ampere aufnimmt?\\ \hline\hline
         A & 68 Stunden und 75 Minuten \\ \hline
         B & Genau 44 Stunden \\ \hline
         C & 6 Stunden 52 min und 30 s \\\hline
         D & 68 Stunden und 45 Minuten \\\hline
    \end{tabular}
 	\end{center}
\end{frame}

\begin{frame}
    \frametitle{Prüfungsfrage}
    \begin{center}
    \begin{tabular}{l||l}\hline
        TB205 & Wie lange könnte man mit einem voll geladenen\\
        " " & Akku mit 55 Ah einen Amateurfunk-Empfänger\\ 
        " " & betreiben, der einen Strom von\\ 
        " " & 0,8 Ampere aufnimmt?\\ \hline\hline
         " " & 68 Stunden und 75 Minuten \\ \hline
         " " & Genau 44 Stunden \\ \hline
         " " & 6 Stunden 52 min und 30 s \\\hline
         X & 68 Stunden und 45 Minuten \\\hline
    \end{tabular}
        \begin{align} 
			t = \frac{Q}{I} = \frac{55Ah}{0.8A} = 68.75
		\end{align}
 	\end{center}
 	
\end{frame}

\section*{Wechselspannung}

\begin{frame}
    \frametitle{Wechselspannungen}
	\begin{minipage}{0.4\textwidth}
        \includegraphics[width=.8\textwidth]{e02/Wechselspannungsformen.png}
	\end{minipage}
	\footnote{\tiny \url{https://commons.wikimedia.org/wiki/File:Wechselspannungsformen.svg}}
	\begin{minipage}{0.4\textwidth}	
	\begin{itemize}
		\item Verschiedene Formen von Wechselspannung
		\item Beim Morsen bestenfalls ein Sinus
		\item Ganz verschiedene Frequenzen
		\item Stromnetz im Hause: $50Hz$ 
		\item $70cm$ AFu Band $435.000.000Hz$
	\end{itemize}
	\end{minipage}
\end{frame}

\begin{frame}
    \frametitle{Effektivwert vs Spitzen Wert}
    \begin{align} 
		\int \! sin^2(w t) \, \mathrm{d}t & = \frac{t}{2} - \frac{1}{4w}sin^2(2wt) + c \\
		U_{eff}^2 \cdot \int_0^T \! sin^2(w t) \, \mathrm{d}t & = \frac{T}{2} \cdot \frac{u_{spitze}^2}{T} \\
		U_{eff} & = \frac{1}{\sqrt{2}} u_{spitze}
	\end{align}
	    \begin{center}
	    \includegraphics[width=.69\textwidth]{e02/EffektivwertSinus.png}\\
	    \footnote{\tiny \url{https://de.wikipedia.org/wiki/Datei:Effektivwert-Sinus.svg}}
	 	\end{center}
\end{frame}

\begin{frame}
    \frametitle{Effektivwert vs Spitzen Wert}
    \begin{align} 
		U_{eff} & = \frac{1}{\sqrt{2}} u_{spitze}
	\end{align}
	    \begin{center}
	    Aufgabe: Berechnet den Spitzenwert des "Hausstoms" mit $U_{eff} = 230V$
	 	\end{center}
\end{frame}

\section*{Referenzen}

\begin{frame}
    \frametitle{Referenzen/Links}
    
    \footnotesize
    \begin{itemize}
        \item Moltrecht E 02: \\
              \url{http://www.darc.de/referate/ajw/ausbildung/darc-online-lehrgang/technik-klasse-e/technik-e02/}
    \end{itemize}

\end{frame}

% Hier könnte noch eine Kontaktfolie stehen

\end{document}

