% Foliensatz: "AFu-Kurs nach DJ4UF" von DK0TU, Amateurfunkgruppe der TU Berlin
% Lizenz: CC BY-NC-SA 3.0 de (http://creativecommons.org/licenses/by-nc-sa/3.0/de/)
% Autoren: Felix Baum DB4UM <baum@campus.tu-berlin.de>, Lars Weiler <dc4lw@darc.de>
% Überarbeitung: Hendrik Boerma

preamble.dk0tu.tex
\subtitle{Technik 02: \\
          Spannung und Strom, Wechselspannung \\[2em]}
\date{Stand 27.10.2016}
 \begin{document}

\begin{frame}
    \titlepage
    \vfill
    \begin{center}
        \ccbyncsaeu\\
        {\tiny This work is licensed under the \em{Creative Commons Attribution-NonCommercial-ShareAlike 3.0 License}.}\\[0.5ex]
         \tiny Amateurfunkgruppe der Technische Universität Berlin (AfuTUB), DKØTU
         %\includegraphics[scale=0.5]{img/DK0TU_Logo.pdf}
    \end{center}
\end{frame}


%fixme Referenzen/Fußnoten-Systematik vereinheitlichen

\section*{Einleitung}

\begin{frame}
    \frametitle{Strom -- Spannung}
    \begin{itemize}
		\item Was ist Spannung/Strom (elektrische)?
		\item Woher bekommt man Spannung/Strom?
    \end{itemize}
\end{frame}

\begin{frame}
    \frametitle{Spannung}
    \begin{center}
      \begin{figure}
 	    \includegraphics[width=.6\textwidth,height=.75\textheight,keepaspectratio]{e02/ladung.png}      
	    \attribcaption{Elektrische Spannung wird durch die Trennung von Ladungen erzeugt}{Geek3}{https://commons.wikimedia.org/wiki/File:VFPt_charges_plus_minus_thumb.svg}{\ccbysa}     
      \end{figure}    
    \end{center}
\end{frame}


\section*{Spannung}

\begin{frame}
    \frametitle{Spannung}
    \begin{center}
        Elektrische Spannung $U$ wird in [$V$] Volt angegeben
    \end{center}
\end{frame}

\begin{frame}
    \frametitle{Batterie}
    \begin{center}
 \begin{figure}
	\includegraphics[width=.7\textwidth]{e02/batterieEagle.png}
	 \caption{Schaltungszeichen einer Batterie}
\end{figure}
    \end{center}
\end{frame}

\begin{frame}
    \frametitle{9V Batterie}
    \begin{center}
	 \begin{figure}
	\includegraphics[width=.99\textwidth]{e02/9vBatEagle.png}
    	\caption{Schaltungszeichen einer 9V Batterie. Pro Zelle $1,8V$. Werden bei Reihenschaltung addiert.}
    	\end{figure}
    \end{center}
\end{frame}

\begin{frame}
  \frametitle{Spannung bestimmen}
  \begin{center}
    \begin{figure}
      \includegraphics[width=.5\textwidth,height=.75\textheight,keepaspectratio]{e02/Spannung.png}
      \caption{Spannungsquellen in einem Netzwerk mit möglichen Messpunkten}
    \end{figure}
  \end{center}
\end{frame}

\begin{frame}
  \frametitle{Digitales Multimeter}
  \begin{center}
    \begin{figure}
      \includegraphics[width=.25\textwidth,height=.75\textheight,keepaspectratio]{e02/digitalmultimeter.jpg}
      \attribcaption{Digitales Multimeter}{MichaelHaeckel}{https://commons.wikimedia.org/wiki/File:Digitalmultimeter.jpg}{\ccpd}
    \end{figure}
  \end{center}
\end{frame}
\begin{frame}
  \frametitle{Was wo anschließen?}
  \begin{minipage}{0.4\textwidth}
    \begin{figure}
      \includegraphics[width=1\textwidth,height=.75\textheight,keepaspectratio]{e02/digitalmultimeterMess.jpg}
      \attribcaption{Digitales Multimeter}{MichaelHaeckel (modifiziert)}{https://commons.wikimedia.org/wiki/File:Digitalmultimeter.jpg}{\ccpd}
    \end{figure}
  \end{minipage}
  \begin{minipage}{0.4\textwidth}
    \begin{itemize}
      \item Was kann alles gemessen werden?
      \item Wo zum Strom messen anschließen?
      \item Wo zum Spannung messen anschließen?
      \item Welcher Messbereich?
    \end{itemize}
  \end{minipage}
\end{frame}

\begin{frame}
    \frametitle{Spannung bestimmen}
    \begin{center}
  \begin{figure}
      \includegraphics[width=.4\textwidth,height=.75\textheight,keepaspectratio]{e02/SpannungMess.png}
      \caption{Frage Welchen Betrag der Spannung zeigt das Messgerät beim Messen von $A$ nach $C$ an?}
    \end{figure}-
    \end{center}
\end{frame}

\begin{frame}
    \frametitle{Potential}
    \begin{block}{}
      Spannung ist der Potentialunterschied an zwei Punkten.
    \end{block}
    \begin{block}{}
      Ein Multimeter misst den Potentialunterschied.
    \end{block}
    \begin{block}{}
      Nullpotential ist ein theoretisches Modell.
    \end{block}
\end{frame}

\section*{Strom}

\begin{frame}
    \frametitle{Strom}
    \begin{center}
    \begin{block}{}
        Elektrischer Strom ist die gerichtete Bewegung von Ladungsträgern.
    \end{block}
    \begin{block}{}
        Elektrischer Strom $I$ wird in [$A$] Ampere angegeben
    \end{block}
    \end{center}
\end{frame}

\begin{frame}
    \frametitle{Physikalische vs. Technische Stromrichtung}
  		\begin{center}
   		 \begin{figure}
      			\includegraphics[width=.7\textwidth,height=.75\textheight,keepaspectratio]{e02/Current_notation.png}
     		 	\attribcaption{Nur Elektronen können sich bewegen}{User:Flekstro (Conventional\_Current.png by User:Romtobbi)}{https://commons.wikimedia.org/wiki/File:Current_notation.svg}{\ccby}
 			\end{figure}
       		\end{center}
\end{frame}

\begin{frame}
    \frametitle{Strom Messen}
    \begin{center}
    \begin{figure}
      \includegraphics[width=.5\textwidth,height=.75\textheight,keepaspectratio]{e02/reiheAmpare.png}
      \caption{Strom wird in Reihe gemessen}
    \end{figure}
  \end{center}
\end{frame}

\begin{frame}
    \frametitle{Wie sollte gemessen werden?}
         \begin{figure}
    \includegraphics[width=\textwidth,height=.75\textheight,keepaspectratio]{e02/stromSpannung.png}
    \caption{Fragenkatalog Bundesnetzagentur Klasse E TJ201}
  \end{figure}
\end{frame}

\section*{Ladung}

\begin{frame}
    \frametitle{Ladung}
     \begin{center}
    \begin{figure}
      \includegraphics[width=\textwidth,height=.75\textheight,keepaspectratio]{e02/ladung.png}
      \attribcaption{$Q = I \cdot t$}{Geek3}{https://commons.wikimedia.org/wiki/File:VFPt_charges_plus_minus_thumb.svg}{\ccbysa}
    \end{figure}
  \end{center}
\end{frame}

\begin{frame}
    \frametitle{Prüfungsfrage}
    \begin{center}
    \begin{tabular}{l||l}\hline
        TB205 & Wie lange könnte man mit einem voll geladenen\\
        " " & Akku mit 55 Ah einen Amateurfunk-Empfänger\\ 
        " " & betreiben, der einen Strom von\\ 
        " " & 0,8 Ampere aufnimmt?\\ \hline\hline
         A & 68 Stunden und 75 Minuten \\ \hline
         B & Genau 44 Stunden \\ \hline
         C & 6 Stunden 52 min und 30 s \\\hline
         D & 68 Stunden und 45 Minuten \\\hline
    \end{tabular}
 	\end{center}
\end{frame}

\begin{frame}
    \frametitle{Prüfungsfrage}
    \begin{center}
    \begin{tabular}{l||l}\hline
        TB205 & Wie lange könnte man mit einem voll geladenen\\
        " " & Akku mit 55 Ah einen Amateurfunk-Empfänger\\ 
        " " & betreiben, der einen Strom von\\ 
        " " & 0,8 Ampere aufnimmt?\\ \hline\hline
         " " & 68 Stunden und 75 Minuten \\ \hline
         " " & Genau 44 Stunden \\ \hline
         " " & 6 Stunden 52 min und 30 s \\\hline
         X & 68 Stunden und 45 Minuten \\\hline
    \end{tabular}
      \begin{align} 
	t = \frac{Q}{I} = \frac{55 Ah}{0,8 A} = 68,75 h = 68 h 45 min
      \end{align}
 	\end{center}
 	
\end{frame}

\section*{Wechsel\-spannung}
\begin{frame}
  \frametitle{Wechselspannung und Frequenz}
  \begin{minipage}{0.4\textwidth}
    \begin{figure}
      \includegraphics[width=.8\textwidth]{e02/Wechselspannung.png}
      \attribcaption{Wechselspannung}{Saure}{https://commons.wikimedia.org/wiki/File:Sinusspannung.svg}{\ccbysa}
      \end{figure}
  \end{minipage}
  \begin{minipage}{0.4\textwidth}
    \begin{itemize}
      \item 1.) Amplitude $u_{spitze}$ 
      \item 2.) Spitze-Spitze-Wert $U_{SS}$
      \item 3.) Effektivwert $U_{eff}$
      \item 4.) Periodendauer $T$ in $s$
      \item Frequenz $f=\frac{1}{T}$ in $Hz = \frac{1}{s}$
    \end{itemize}
  \end{minipage}
\end{frame}
\begin{frame}
  \frametitle{Frage zur Frequenz}
  \begin{minipage}{0.6\textwidth}
    \begin{tabular}{l||l}\hline
        TB610& Welche Frequenz hat die\\
        " " &  in diesem Oszillogramm\\
        " " & dargestellte Spannung?\\ \hline\hline
         A & $833,3 kHz$ \\ \hline
         B & $83,3kHz$\\ \hline
         C & $8,3MHz$ \\\hline
         D & $83,3MHz$ \\\hline
      \end{tabular}
  \end{minipage}
  \begin{minipage}{0.3\textwidth}
    \begin{figure}
       \includegraphics[width=.9\textwidth]{e02/TB610}{}
     \end{figure}
    \end{minipage}
\end{frame}
\begin{frame}
  \frametitle{Frage zur Frequenz}
  \begin{minipage}{0.6\textwidth}
    \begin{tabular}{l||l}\hline
        TB610& Welche Frequenz hat die\\
        " " &  in diesem Oszillogramm\\
        " " & dargestellte Spannung?\\ \hline\hline
         " " & $833,3 kHz$ \\ \hline
         X & $83,3kHz$\\ \hline
         " "& $8,3MHz$ \\\hline
         " " & $83,3MHz$ \\\hline
      \end{tabular}
  \end{minipage}
  \begin{minipage}{0.3\textwidth}
    \begin{figure}
       \includegraphics[width=.9\textwidth]{e02/TB610}
     \end{figure}
    \end{minipage}
      \begin{align} 
        f=\frac{1}{T}=\frac{1}{12 \cdot 10^{-6}s}=83,3 \cdot 10^{3} Hz= 83,3 kHz
\end{align}
\end{frame}

\begin{frame}
  \frametitle{Wechselspannungen}
    \begin{minipage}{0.4\textwidth}
      \begin{figure}
	\includegraphics[width=.5\textwidth]{e02/Wechselspannungsformen.png}
	\attribcaption{Formen von Wechselspannung}{Saure}{https://commons.wikimedia.org/wiki/File:Wechselspannungsformen.svg}{\ccbysa}
     \end{figure}
    \end{minipage}
    \begin{minipage}{0.4\textwidth}	
      \begin{itemize}
        \item Verschiedene Formen von Wechselspannung
        \item Beim Morsen bestenfalls ein Sinus
        \item Ganz verschiedene Frequenzen
        \item Stromnetz im Hause: $50Hz$ 
        \item $70cm$ AFu Band $435.000.000Hz$
      \end{itemize}
    \end{minipage}
\end{frame}

\begin{frame}
  \frametitle{Effektivwert vs Spitzen Wert}
  \begin{align}
    \int \! sin^2(\omega t) \, \mathrm{d}t & = \frac{t}{2} - \frac{1}{4\omega}sin^2(2\omega t) + c \\
    U_{eff}^2 \cdot \int_0^T \! sin^2(\omega t) \, \mathrm{d}t & = \frac{T}{2} \cdot \frac{u_{spitze}^2}{T} \\
    U_{eff} & = \frac{1}{\sqrt{2}} u_{spitze}
  \end{align}
  \begin{center}
    \begin{figure}
      \includegraphics[width=.6\textwidth,height=.3\textheight,keepaspectratio]{e02/EffektivwertSinus.png}\\
      \attribcaption{Effektivwert-Sinus}{Saure}{https://de.wikipedia.org/wiki/Datei:Effektivwert-Sinus.svg}{\ccbysa}
    \end{figure}
  \end{center}
\end{frame}

\begin{frame}
    \frametitle{Effektivwert vs. Spitzenwert}
    \begin{align} 
		U_{eff} & = \frac{1}{\sqrt{2}} u_{spitze}
	\end{align}
	\begin{alertblock}{Hausaufgabe}
	    Aufgabe: Berechne den Spitzenwert des ``Hausstroms'' mit $U_{eff} = 240V$
	\end{alertblock}
\end{frame}

\section*{Referenzen}

\begin{frame}
    \frametitle{Referenzen/Links}
    
    \footnotesize
    \begin{itemize}
        \item Moltrecht E 02: \\
              \url{http://www.darc.de/referate/ajw/ausbildung/darc-online-lehrgang/technik-klasse-e/technik-e02/}
    \end{itemize}

\end{frame}

% Hier könnte noch eine Kontaktfolie stehen

\end{document}

