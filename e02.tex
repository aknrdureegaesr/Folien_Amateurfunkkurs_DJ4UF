% Foliensatz: "AFu-Kurs nach DJ4UF" von DK0TU, Amateurfunkgruppe der TU Berlin
% Lizenz: CC BY-NC-SA 3.0 de (http://creativecommons.org/licenses/by-nc-sa/3.0/de/)
% Autoren: Felix Baum DB4UM <baum@campus.tu-berlin.de>
% Überarbeitung: Hendrik Boerma
% Korrekturen: Lars Weiler <dc4lw@darc.de>

preamble.dk0tu.tex
\subtitle{Technik 02: \\
          Spannung und Strom, Wechselspannung \\[2em]}
\date{Stand 18.09.2017}
 \begin{document}

\begin{frame}
    \titlepage
    \vfill
    \begin{center}
        \ccbyncsaeu\\
        {\tiny This work is licensed under the \em{Creative Commons Attribution-NonCommercial-ShareAlike 3.0 License}.}\\[0.5ex]
         \tiny Amateurfunkgruppe der Technische Universität Berlin (AfuTUB), DKØTU
         %\includegraphics[scale=0.5]{img/DK0TU_Logo.pdf}
    \end{center}
\end{frame}


\section*{Einleitung}

\begin{frame}
  \frametitle{Strom -- Spannung}
  \begin{itemize}
    \item Was ist Spannung/Strom (elektrische)?
    \item Woher bekommt man Spannung/Strom?
  \end{itemize}
\end{frame}

\begin{frame}
  \frametitle{Spannung}
  \begin{center}
    \begin{figure}
      \includegraphics[width=.6\textwidth,height=.75\textheight,keepaspectratio]{e02/ladung.png}
      \caption{Elektrische Spannung wird erzeugt durch die Trennung von Ladungen \cite{charge}}
      \label{fig_charge}
      %\attribcaption{Elektrische Spannung wird erzeugt durch die Trennung von Ladungen}{Geek3}{https://commons.wikimedia.org/wiki/File:VFPt_charges_plus_minus_thumb.svg}{\ccbysa}
    \end{figure}
  \end{center}
\pnote{Statisch vs. Quelle: Aufrechterhaltung}
\end{frame}


\section*{Spannung}

\begin{frame}
  \frametitle{Spannung}
  \begin{center}
    Elektrische Spannung $U$ wird in [$V$] Volt angegeben
  \end{center}
\end{frame}

\begin{frame}
  \frametitle{Batterie}
  \begin{center}
    \begin{figure}
      \includegraphics[width=\textwidth,height=.5\textheight,keepaspectratio]{e02/batterieEagle.png}
      \caption{Schaltungszeichen einer Batterie / Batteriezelle}
    \end{figure}
  \end{center}
\end{frame}

\begin{frame}
  \frametitle{9V Batterie}
  \begin{center}
    \begin{figure}
      \includegraphics[width=\textwidth,height=.5\textheight,keepaspectratio]{e02/9vBatEagle.png}
      \caption{Schaltungszeichen einer 9V Batterie. Pro Zelle $1,8V$. Werden bei Reihenschaltung addiert.}
    \end{figure}
  \end{center}
\end{frame}

\begin{frame}
  \frametitle{Spannung bestimmen}
  \begin{center}
    \begin{figure}
      \includegraphics[width=.5\textwidth,height=.75\textheight,keepaspectratio]{e02/Spannung.png}
      \caption{Spannungsquellen in einem Netzwerk mit möglichen Messpunkten}
    \end{figure}
  \end{center}
\end{frame}

\begin{frame}
  \frametitle{Digitales Multimeter}
  \begin{center}
    \begin{figure}
      \includegraphics[width=.25\textwidth,height=.75\textheight,keepaspectratio]{e02/digitalmultimeter.jpg}
      \caption{Digitales Multimeter \cite{multimeter}}
      \label{fig_multimeter}
      %\attribcaption{Digitales Multimeter}{MichaelHaeckel}{https://commons.wikimedia.org/wiki/File:Digitalmultimeter.jpg}{\ccpd}
    \end{figure}
  \end{center}
\end{frame}

\begin{frame}
  \frametitle{Was wo anschließen?}
  \begin{minipage}{0.5\textwidth}
    \begin{figure}
      \includegraphics[width=1\textwidth,height=.75\textheight,keepaspectratio]{e02/digitalmultimeterMess.jpg}
      \caption{Digitales Multimeter (Ausschnitt) \cite{multimeter}}
	  \label{fig_multimeter_crop}
      %\attribcaption{Digitales Multimeter}{MichaelHaeckel (modifiziert)}{https://commons.wikimedia.org/wiki/File:Digitalmultimeter.jpg}{\ccpd}
    \end{figure}
  \end{minipage}
$   \begin{minipage}{0.4\textwidth}
    \begin{itemize}
      \item Was kann alles gemessen werden?
      \item Wo zum Strom messen anschließen?
      \item Wo zum Spannung messen anschließen?
      \item Welcher Messbereich?
    \end{itemize}
  \end{minipage} $
  \pnote{Beliebter Fehler: Spannung messen mit Testleitung am Stromeingang}
\end{frame}

\begin{frame}
  \frametitle{Spannung bestimmen}
  \begin{center}
    \begin{figure}
      \includegraphics[width=.4\textwidth,height=.75\textheight,keepaspectratio]{e02/SpannungMess.png}
      \caption{Zusatzfrage: Welchen Betrag zeigt das Messgerät beim Messen von $A$ nach $C$ an?}
    \end{figure}
  \end{center}
\end{frame}

\begin{frame}
  \frametitle{Potential}
  \begin{block}{}
    Spannung ist der Potentialunterschied an zwei Punkten.
  \end{block}
  \begin{block}{}
    Ein Multimeter misst den Potentialunterschied.
  \end{block}
  \begin{block}{}
    Nullpotential ist ein theoretisches Modell.
  \end{block}
\end{frame}

\section*{Strom}

\begin{frame}
  \frametitle{Strom}
  \begin{center}
    \begin{block}{}
      Elektrischer Strom ist die gerichtete Bewegung von Ladungsträgern.
    \end{block}
    \begin{block}{}
      Elektrischer Strom $I$ wird in [$A$] Ampere angegeben
    \end{block}
  \end{center}
\end{frame}

\begin{frame}
  \frametitle{Physikalische vs. Technische Stromrichtung}
  \begin{center}
    \begin{figure}
      \includegraphics[width=.7\textwidth,height=.75\textheight,keepaspectratio]{e02/Current_notation.png}
      \caption{Nur Elektronen können sich bewegen \cite{stromrichtung}}
      \label{fig_stromrichtung}
      %\attribcaption{Nur Elektronen können sich bewegen}{User:Flekstro (Conventional\_Current.png by User:Romtobbi)}{https://commons.wikimedia.org/wiki/File:Current_notation.svg}{\ccby}
    \end{figure}
  \end{center}
\end{frame}

\begin{frame}
  \frametitle{Strom messen}
  \begin{center}
    \begin{figure}
      \includegraphics[width=.5\textwidth,height=.75\textheight,keepaspectratio]{e02/reiheAmpare.png}
      \caption{Strom wird in Reihe gemessen}
    \end{figure}
  \end{center}
\end{frame}

\begin{frame}
  \frametitle{Wie sollte gemessen werden?}
  \begin{figure}
    \includegraphics[width=\textwidth,height=.75\textheight,keepaspectratio]{e02/stromSpannung.png}
    \caption{Fragenkatalog Bundesnetzagentur Klasse E TJ201}
  \end{figure}
	\pnote{Nicht ganz korrket: Spannungsabfall Amperemeter}
\end{frame}

\section*{Ladung}

\begin{frame}
  \frametitle{Ladung}
	  \begin{minipage}{0.4\textwidth}
	    \begin{figure}
	      \includegraphics[width=\textwidth,height=.85\textheight,keepaspectratio]{e02/ladung.png}
	      \caption{Geladene Teilchen \cite{charge}}
	      %\attribcaption{$Q = I \cdot t$}{Geek3}{https://commons.wikimedia.org/wiki/File:VFPt_charges_plus_minus_thumb.svg}{\ccbysa}
	    \end{figure}
	  \end{minipage}
  \begin{minipage}{0.5\textwidth}
  	\begin{itemize}
  		\item Ladung des Elektrons: Elementarladung
  		\item Einheit: Coulomb [C], Formelzeichen Q
  		\item Bei konstantem Strom: $Q = I*t$
  	\end{itemize}
  \end{minipage}
\end{frame}

%\begin{frame}
%  \frametitle{Prüfungsfrage}
%  \begin{tabular}{l||p{.8\textwidth}}\hline
%    \textbf{TB205} & \textbf{Wie lange könnte man mit einem voll geladenen Akku mit 55\,Ah einen Amateurfunk-Empfänger betreiben, der einen Strom von 0,8\,Ampere aufnimmt?} \\ \hline\hline
%    A & 68 Stunden und 75\,Minuten \\ \hline
%    B & Genau 44\,Stunden \\ \hline
%    C & 6 Stunden 52\,min und 30\,s \\\hline
%    D \only<2>\checkmark & 68\,Stunden und 45\,Minuten \\\hline
%  \end{tabular}
%  \pause
%  \begin{align}
%    t = \frac{Q}{I} = \frac{55 Ah}{0,8 A} = 68,75 h = 68 h 45 min
%  \end{align}
%\end{frame}


\section*{Wechsel\-spannung}

\begin{frame}
  \frametitle{Wechselspannungen}
  \begin{minipage}{0.4\textwidth}
  	\begin{itemize}
  		\item Verschiedene Formen von Wechselspannung
  		\item Beim Morsen bestenfalls ein Sinus
  		\item Stromnetz im Hause: $50Hz$
  		\item $70cm$ AFu Band $435.000.000Hz$
  	\end{itemize}
  \end{minipage}
  \begin{minipage}{0.4\textwidth}
    \begin{figure}
      \includegraphics[height=0.85\textheight,keepaspectratio]{e02/Wechselspannungsformen_trunc.png}
      \caption{Wechselspannungsformen \cite{ac}}
      \label{fig_ac}
      %\attribcaption{Wechselspannungsformen}{Saure}{https://commons.wikimedia.org/wiki/File:Wechselspannungsformen.svg}{\ccbysa}
    \end{figure}
  \end{minipage}
  
\end{frame}

\begin{frame}
	\frametitle{Charakteristika einer Sinusschwingung}
	\begin{minipage}{0.4\textwidth}
		\begin{figure}
			\includegraphics[height=0.75\textheight,keepaspectratio]{e02/Wechselspannung.png}
			\caption{Sinusschwingung \cite{sinus}}
			\label{fig_sinus}
		\end{figure}
	\end{minipage}
	\begin{minipage}{0.5\textwidth}
			\begin{itemize}
				\item 1: Scheitelwert
				\item 2: Spitze-Spitze-Spannung
				\item 3: Effektivwert
				\item 4: Periodendauer
			\end{itemize}
	\end{minipage}
\end{frame}
%\begin{frame}
%  \frametitle{Effektivwert vs Scheitelwert}
%  \begin{align}
%    \int \! sin^2(\omega t) \, \mathrm{d}t & = \frac{t}{2} - \frac{1}{4\omega}sin^2(2\omega t) + c \\
%    U_{eff}^2 \cdot \int_0^T \! sin^2(\omega t) \, \mathrm{d}t & = \frac{T}{2} \cdot \frac{u_{spitze}^2}{T} \\
%    U_{eff} & = \frac{1}{\sqrt{2}} u_{spitze}
%  \end{align}
%  \begin{center}
%    \begin{figure}
%      \includegraphics[width=.6\textwidth,height=.3\textheight,keepaspectratio]{e02/EffektivwertSinus.png}\\
      %\attribcaption{Effektivwert-Sinus}{Saure}{https://de.wikipedia.org/wiki/Datei:Effektivwert-Sinus.svg}{\ccbysa}
%    \end{figure}
%  \end{center}
%\end{frame}

\begin{frame}
  \frametitle{Umrechnung Effektivwert/Scheitelwert}
  \begin{align}
    U_{eff} & = \frac{1}{\sqrt{2}} u_{s}
  \end{align}
  %\begin{alertblock}{Hausaufgabe}
  %  Aufgabe: Berechne den Spitzenwert des ``Hausstroms'' mit $U_{eff} = 240V$
  %\end{alertblock}
\end{frame}

\section*{Referenzen}

\begin{frame}
  \frametitle{Referenzen/Links}

  \footnotesize

	\begin{thebibliography}{}
		\bibitem{dj4uf} Moltrecht E 02: \\
		\url{https://www.darc.de/der-club/referate/ajw/lehrgang-te/e02/}
		
		\bibitem{charge}  Abbildung \ref{fig_charge}: Elektrisches Feld zwischen Ladungen \\
		\url{https://commons.wikimedia.org/wiki/File:VFPt_charges_plus_minus_thumb.svg}\\
		
		\bibitem{multimeter}  Abbildung \ref{fig_multimeter}: Digitales Multimeter \\
		\url{https://commons.wikimedia.org/wiki/File:Digitalmultimeter.jpg}\\
		
		\bibitem{stromrichtung}  Abbildung \ref{fig_stromrichtung}: Stromrichtung \\
		\url{https://commons.wikimedia.org/wiki/File:Current_notation.svg}\\
		
		\bibitem{ac}  Abbildung \ref{fig_ac}: Wechselspannungsformen (Ausschnitt)\\
		\url{https://commons.wikimedia.org/wiki/File:Wechselspannungsformen.svg}\\
		
		\bibitem{sinus}  Abbildung \ref{fig_sinus}: Charakteristika einer Sinusspannung\\
		\url{https://commons.wikimedia.org/wiki/File:Sinusspannung.svg}\\

		
	\end{thebibliography}

\end{frame}

% Hier könnte noch eine Kontaktfolie stehen

\end{document}

