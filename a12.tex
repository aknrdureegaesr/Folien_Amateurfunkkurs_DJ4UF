% Foliensatz: "AFu-Kurs nach DJ4UF" von DK0TU, Amateurfunkgruppe der TU Berlin
% Lizenz: CC BY-NC-SA 3.0 de (http://creativecommons.org/licenses/by-nc-sa/3.0/de/)
% Autoren: Sebastian Lange <dl7bst@dk0tu.de>

preamble.dk0tu.tex
\subtitle{Technik Klasse A 12: \\
          Modulation und Demodulation \\[2em]}
\date{Stand 11.06.2015}
 \begin{document}

\begin{frame}
    \titlepage
    \vfill
    \begin{center}
        \ccbyncsaeu\\
        {\tiny This work is licensed under the \em{Creative Commons Attribution-NonCommercial-ShareAlike 3.0 License}.}\\[0.5ex]
         \tiny Amateurfunkgruppe der Technische Universität Berlin (AfuTUB), DKØTU
         %\includegraphics[scale=0.5]{img/DK0TU_Logo.pdf}
    \end{center}
\end{frame}


\section{Überblick}

\begin{frame}
    \frametitle{Überblick}

    Wiederholung: Was ist Modulation? \bigskip

    Nennt die Prinzipien von:

    \begin{itemize}
        \item AM
        \item SSB
        \item FM
    \end{itemize}

\end{frame}

\section{AM}

\subsection{Modulation}

\begin{frame}
    \frametitle{AM-Modulation}

%    \begin{center}
%        \includegraphics[width=0.8\textwidth]{a13/TG103a.png}
%        \tiny (TG103)
%    \end{center}

\end{frame}

\subsubsection{Modulationsgrad}

\begin{frame}
    \frametitle{AM-Modulationsgrad}

    \begin{center}
        \includegraphics[width=0.5\textwidth]{a12/TE111.png}
        \tiny (TE111)
    \end{center}
   
    \begin{equation*}
        m = \frac{\hat{U}_{mod}}{\hat{U}_T}
    \end{equation*}

\end{frame}

\subsubsection{Leistung}

\subsection{Demodulation}

\begin{frame}
    \frametitle{AM-Demodulation}

%    \begin{center}
%        \includegraphics[width=0.5\textwidth]{a13/TG226.png}
%        \tiny (TG226)
%    \end{center}

\end{frame}

\subsubsection{Audion}

\begin{frame}
    \frametitle{Audion}

%    \begin{center}
%        \includegraphics[width=0.5\textwidth]{a13/TG226.png}
%        \tiny (TG226)
%    \end{center}

\end{frame}

\subsection{Trägerunterdrückung}

\begin{frame}
    \frametitle{Trägerunterdrückung}

%    \begin{center}
%        \includegraphics[width=0.5\textwidth]{a13/TG106.png}
%        \tiny (TG106)
%    \end{center}

\end{frame}

\section{SSB}

\begin{frame}
    \frametitle{SSB}

%    \begin{center}
%        \includegraphics[width=0.8\textwidth]{a13/TD701.png}
%        \tiny (TD701)
%    \end{center}

\end{frame}

\subsection{Modulation}

\begin{frame}
    \frametitle{SSB-Modulation}

%    \begin{center}
%        \includegraphics[width=0.8\textwidth]{a13/TD701.png}
%        \tiny (TD701)
%    \end{center}

\end{frame}

\subsection{Demoduluation}

\begin{frame}
    \frametitle{SSB-Demodulation}

%    \begin{center}
%        \includegraphics[width=0.8\textwidth]{a13/TG110.png}
%        \tiny (TG110)
%    \end{center}

\end{frame}

\subsubsection{Produktdetektor}

\begin{frame}
    \frametitle{Produktdetektor}

\end{frame}

\section{FM}

\begin{frame}
    \frametitle{FM}

%    \begin{center}
%        \includegraphics[width=0.5\textwidth]{a13/TF206.png}
%        \tiny (TF206)
%    \end{center}

\end{frame}

\subsection{Modulation}

\begin{frame}
    \frametitle{FM-Modulation}

%    \begin{center}
%        \includegraphics[width=0.8\textwidth]{a13/TF205b.png}
%        \tiny (TF205b)
%    \end{center}

\end{frame}

\subsection{Demodulation}

\begin{frame}
    \frametitle{FM-Demodulation}

%    \begin{center}
%        \includegraphics[width=0.8\textwidth]{a13/TF209b.png}
%        \tiny (TF209b)
%    \end{center}

\end{frame}

\renewcommand{\refname}{Referenzen}

\hypertarget{refs}{}
\textcolor{white}{} \\ %\vspace{} geht nicht
\Large Referenzen/Links
\footnotesize

\begin{thebibliography}{}
    \bibitem{darc}  DARC Online-Lehrgang Lektion A12:
                    \url{http://www.darc.de/referate/ajw/ausbildung/darc-online-lehrgang/technik-klasse-a/technik-a12/}
\end{thebibliography} 

% Hier könnte noch eine Kontaktfolie stehen

\end{document}

