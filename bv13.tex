% Foliensatz: "AFu-Kurs nach DJ4UF" von DK0TU, Amateurfunkgruppe der TU Berlin
% Lizenz: CC BY-NC-SA 3.0 de (http://creativecommons.org/licenses/by-nc-sa/3.0/de/)
% Autoren: Sebastian Lange <dl7bst@dk0tu.de>

preamble.dk0tu.tex
\subtitle{Betriebstechnik/Vorschriften 13: \\
          RST-System, UTC, Logbuch, QSL-Karte \\[2em]}
\date{Stand 29.01.2015}
 \begin{document}

\begin{frame}
    \titlepage
    \vfill
    \begin{center}
        \ccbyncsaeu\\
        {\tiny This work is licensed under the \em{Creative Commons Attribution-NonCommercial-ShareAlike 3.0 License}.}\\[0.5ex]
         \tiny Amateurfunkgruppe der Technische Universität Berlin (AfuTUB), DKØTU
         %\includegraphics[scale=0.5]{img/DK0TU_Logo.pdf}
    \end{center}
\end{frame}


\section{Einleitung}

\begin{frame}
    \frametitle{Einleitung}

    Für den eigentlichen Spaß des Funkbetriebes braucht es etwas ''back
    office'': 

    \begin{itemize}
        \item Logbuchführung
        \item QSL-Karten
    \end{itemize}

    \vspace{1cm}

    Benötigt: \emph{RST}-System und Umgang mit \emph{UTC}...

\end{frame}

\section{RST-System}

\begin{frame}
    \frametitle{RST-System}

    \textbf{R}eadability, Signal \textbf{S}trength, \textbf{T}one

    \begin{center}
        \includegraphics[width=0.8\textwidth]{e10/S-Meter.jpg}
        \tiny \hyperlink{refs}{\cite{wc}}
    \end{center}

    \begin{itemize}
        \item Zweck: Empfangsbeurteilung
        \item \emph{T} nur in Telegrafie
        \item Verwendung seit 1930er
    \end{itemize}

\end{frame}

\begin{frame}
    \frametitle{RST-System / Tabelle}

    \begin{center}
    \footnotesize
    \begin{tabular}{|l|l|l|l|}\hline
          & \textbf{R}eadability        & \textbf{S}trength & \textbf{T}one \\ \hline \hline
        1 & nicht lesbar                & $-48 dB$          & äußerst roh   \\ \hline
        2 & zeitweise lesbar            & $-42 dB$          & sehr roh      \\ \hline
        3 & mit Schwierigkeiten lesbar  & $-36 dB$          & roh           \\ \hline
        4 & ohne Schwierigkeiten lesbar & $-30 dB$          & leicht roh    \\ \hline
        5 & einwandfrei lesbar          & $-24 dB$          & musikalisch   \\ \hline
        6 &                             & $-18 dB$          & moduliert     \\ \hline
        7 &                             & $-12 dB$          & instabil      \\ \hline
        8 &                             & $-6 dB$           & etwas Brumm   \\ \hline
        9 &                             & $0dB$             & rein          \\ \hline
        9+x &                           & $+x dB$ über S9   &               \\ \hline
    \end{tabular}
    \end{center}

    S-Stufe 9 als Referenzspannung am $50 \Omega$-Antenneneingang:
    \begin{itemize}
        \item KW: $50\mu V$
        \item UKW: $5\mu V$
    \end{itemize}

\end{frame}

\begin{frame}
    \frametitle{RST-System / Abgeleitete Systeme}

    \begin{itemize}
        \item R: Sprechfunk (Relais)
        \item RS: Sprechfunk (p2p)
        \item RSV: \textbf{V}ideo - SSTV/ATV
        \item RSQ: \textbf{Q}uality - digitalen Betriebsarten
        \item RSA: \textbf{A}urora - Tonrapport wäre quatsch $\rightarrow$ ''A'' geben
    \end{itemize}

\end{frame}

\begin{frame}
    \frametitle{RST-System / S-Stufe}

    Bedenke: Eine S-Stufe = $6 dB$

    \begin{itemize}
        \item gemessen an Spannung (Feldgröße) $\rightarrow$ Verdopplung
        \item Leistungsgröße $\rightarrow$ Vervierfachung!
    \end{itemize}

    \vspace{1cm}

    Beispiel\footnote{Prüfungskatalog Frage BB303}:    

    \begin{block}{Um wie viel S-Stufen müsste die S-Meter- Anzeige Ihres
                  Empfängers steigen, wenn Ihr Partner die Sendeleistung von 100
                  Watt auf 400 Watt erhöht?}
        \only<2>{$\rightarrow$ Um eine S-Stufe}
    \end{block}

\end{frame}

\begin{frame}
    \frametitle{RST-System / S-Stufe}

    Noch ein Beispiel\footnote{Prüfungskatalog Frage BB307}:    

    \begin{block}{Durch ''Fading'' sinkt die S-Meter-Anzeige von S9 auf S8. Auf
                  welchen Wert sinkt dabei die Empfänger-Eingangsspannung ab,
                  wenn bei S9 am Empfängereingang $50\mu V$ anliegen? Die
                  Empfänger-Eingangsspannung sinkt auf}
        \only<2>{$\rightarrow 25\mu V$}
    \end{block}

\end{frame}

\section{UTC}

\begin{frame}
    \frametitle{UTC}

    \textbf{U}niversal \textbf{T}ime, \textbf{C}oordinated

    \begin{center}
        \includegraphics[width=0.8\textwidth]{bv13/Standard_time_zones_of_the_world.png}
        \tiny \hyperlink{refs}{\cite{wc}}
    \end{center}

    \begin{itemize}
        \item Bezugsachse: lokale ''Sonnenzeit'' am Nullmeridian durch
              London-Greenwich\footnote{früher übliche Bezeichnung
              \emph{Greenwich Mean Time (GMT)}}
    \end{itemize}

\end{frame}

\begin{frame}
    \frametitle{UTC}

    \begin{center}
        \includegraphics[width=0.9\textwidth]{bv13/Standard_time_zones_of_the_world.png}
        \tiny \hyperlink{refs}{\cite{wc}}
    \end{center}

    \begin{itemize}
        \item weit mehr als 24 Zeitzonen: UTC-12 bis UTC+14 (Kartenzoom lohnt sich)
              - nicht immer volle Stunden
        \item Deutschland:
        \begin{itemize}
            \item UTC+1 = CET/MEZ
            \item UTC+2 = CEST/MESZ
        \end{itemize}
    \end{itemize}

    Da AFu international $\rightarrow$ alleinige Verwendung von UTC!

\end{frame}

\section{Logbook}

\begin{frame}
    \frametitle{Logbook}

    ''Stationstagebuch, das ein Funkamateur freiwillig führt oder in besonderen
    Fällen führen muss.''

    %todo aktuelles Papierlogbild?
    \begin{center}
        \includegraphics[width=0.8\textwidth]{bv13/DK0TU_LOG_KW_1972-02_Auszug.jpg}
        \tiny \footnote{DK0TU Kurzwellenlog Feb. 1972 (Auszug, nach Wasserschaden)}
    \end{center}

\end{frame}

\begin{frame}
    \frametitle{Logbook}

    \begin{center}
        \includegraphics[width=0.5\textwidth]{bv13/DK0TU_LOG_KW_1972-02_Auszug.jpg}
    \end{center}

    Je Aussendung:
    
    \begin{itemize}
        \item Tag, Uhrzeit (immer UTC!)
        \item Frequenz
        \item Rufzeichen
        \item Rapporte
        \item ggf. Aufzeichnungen über Bedingungen, Sendeleistung, Antenne, ...
    \end{itemize}

    Praktisch für sich selbst oder auch einen \emph{EMV}-Fall.

\end{frame}

\subsection{Digitallog}

\begin{frame}
    \frametitle{Logbook / Digitallog}

    Papierlog oder Computerlog: Minimalistisch vs. einfache Handhabung und
    Katalogisierung:

    \begin{center}
        \includegraphics[width=1\textwidth]{bv13/tucnak_Hf-10.png}
        \tiny \footnote{Tucnak Contest-Log \hyperlink{refs}{\cite{tucn}}}
    \end{center}

\end{frame}

\begin{frame}
    \frametitle{Logbook / Digitallog}

    \begin{center}
        \includegraphics[height=0.25\textheight]{bv13/DK0TU_LOG_KW_1972-02_Auszug.jpg}
        $\rightarrow$
        \includegraphics[height=0.25\textheight]{bv13/tucnak_Hf-10.png}
    \end{center}
    %\begin{center}
    %    \includegraphics[width=0.5\textwidth]{bv13/tucnak_Hf-10.png}
    %    \tiny \hyperlink{refs}{\cite{tucn}}
    %\end{center}

    \begin{itemize}
        \item Uhrzeit automatisch
        \item QRG vom TRX übertragen
        \item QTF\footnote{Antennenrichtung}-Anzeige oder -Steuerung
        \item Call-Datenbanken
        \item DX-Cluster
        \item Sync im LAN/WAN für Multi-Op
        \item ...
    \end{itemize}

\end{frame}

\subsection{Beispiele}

\begin{frame}
    \frametitle{Beispiel: Papierlog}

    \begin{center}
        \includegraphics[width=1\textwidth]{bv13/dk0tu_log_kw.png}
    \end{center}

\end{frame}

\begin{frame}
    \frametitle{Beispiele: Software}
    
    \begin{center}

    \Large Demo... \\[3em]

    \normalsize $\Rightarrow$ Export: \url{http://www.dk0tu.de/Logbook}

    \end{center}

\end{frame}

\subsection{Angeordnete Logbuchführung}

\begin{frame}
    \frametitle{Logbuchführung / Angeordnete Logbuchführung}

    Heutzutage ist die Logbuchführung freiwillig - im Falle einer Störung nicht.
    Die ''zuständigen Behörde'' (\emph{BNetzA}) kann das anordnen.

    \begin{center}
        \includegraphics[width=0.3\textwidth]{bv13/Bundesnetzagentur_logo_709px.png}
        \tiny \hyperlink{refs}{\cite{wp}}
    \end{center}

    \begin{itemize}
        \item Log muss angeordnete Zeit aufgehoben werden
        \item digitales Log zulässig
        \item bei Softwarewechsel muss die alte verfügbar bleiben
    \end{itemize}

\end{frame}

\section{QSL-Karte}

\begin{frame}
    \frametitle{QSL-Karte}

    \emph{QSL} - wir erinnern uns aus \texttt{BV03} - Empfangsbestätigung \\[1em]

    \begin{center}
        \includegraphics[width=0.6\textwidth]{bv13/DK0TU_1.jpg}
    \end{center}

    $\rightarrow$ traditionell auch mit einer ''Ansichtskarte'' bei KW oder UKW-DX

\end{frame}

\begin{frame}
    \frametitle{QSL-Karte}

    \begin{center}
        \includegraphics[width=0.5\textwidth]{bv13/DK0TU_0.jpg}
    \end{center}

    \begin{itemize}
        \item für SWL lange Zeit einziger Rückkanal
        \item schriftl. Bestätigung der Angaben
        %\item Beantragung Amateurfunk-Diplome
        \item individuell sehr verschieden gestaltet
    \end{itemize}

\end{frame}

\begin{frame}
    \frametitle{QSL-Karte / Informationen}

    \begin{block}{Welche Angaben sollten mind. enthalten sein?}
        \only<1>{$\rightarrow$ Tafel} 
        \only<2>{
        Im Prinzip Daten aus dem \textbf{Logbook}:
        \begin{itemize}
            \item Rufzeichen
            \item Rufzeichen der Gegenstation
            \item Datum
            \item Uhrzeit (UTC)
            \item Frequenz
            \item Betriebsart
            \item Signal-Rapport
            \item Unterschrift des Op
        \end{itemize}
        }
    \end{block}

    \only<2>{
        Besonderheit\footnote{z.B. bei \emph{DXpeditionen}}:
        \emph{''QSL via [Call]''} $\rightarrow$ QSL-Manager
    }

\end{frame}

\begin{frame}
    \frametitle{QSL-Karte / Informationen}

    \begin{center}
        \includegraphics[width=1\textwidth]{bv13/dk0tu_qsl2.pdf}
    \end{center}

\end{frame}

\begin{frame}
    \frametitle{QSL-Karte / Versand}

    \begin{itemize}
        \item kostenfrei \emph{''via bureau''}
        \item direkte Zusendung durch Adresse aus \emph{int. Callbook} oder Internet
    \end{itemize}

    Für besonders Eilige oder Länder ohne Radio Clubs: Mit Versand der Karte
    Antwortbriefumschlag (\emph{SAE}\footnote{Self-Addressed Envelope}) und
    \emph{IRC}\footnote{International Reply Coupon}s zwei 1\$-Noten beilegen. \\[2em]

\end{frame}

\subsection{eQSL}

\begin{frame}
    \frametitle{QSL-Karte / eQSL}

    im Prinzip wie \textbf{e}Mail

    \begin{center}
        \includegraphics[width=0.6\textwidth]{bv13/eQSL_DK0TU-EA3ARE.jpg}
        \tiny \hyperlink{refs}{\cite{wc}}
    \end{center}

    \begin{itemize}
        \item minimaler Kostenaufwand
        \item Ausdruck nur bei Bedarf
        \item weniger ''Flair''
    \end{itemize}

    Wie bei E-Mail gibt es diverse Webdienste oder Programme.

\end{frame}

\subsection{SSTV}

\begin{frame}
    \frametitle{QSL-Karte / SSTV}

    Manchmal wird das empfangene SSTV-Bild zur Bestätigung per E-Mail
    zurückgesendet.

    \begin{center}
        \includegraphics[width=0.6\textwidth]{bv13/QSL_SSTV_2014-11-24_162753.png}
    \end{center}

\end{frame}

\section{Sonderstationen}

\begin{frame}
    \frametitle{Sonderstationen}

    Einige HAMs sammeln besondere Stationen:

    \begin{center}
        \includegraphics[width=0.6\textwidth]{bv13/DK0TU_SDOK_FSB13.jpg}
    \end{center}

    \begin{itemize}
        \item Calls zu bestimmten Veranstaltungen, z.B.
              \emph{DK0IFA}\footnote{zur Internationale Funkausstellung (IFA)}
        \item Sonder-DOK\footnote{Distrikts-Ortsverbands-Kenner}s, z.B.
              \emph{FSB13}\footnote{zum Jubiläum der ehemaligen Field Station
              Berlin (Teufelsberg)}
    \end{itemize}

\end{frame}

\section{Diplome}

\begin{frame}
    \frametitle{Diplome}

    %todo Foto ARRL Diplom

    Auszeichnung\footnote{meist Urkundenpapier, aber auch diverse andere
    Gegenstände} für besondere Leistungen im Amateurfunk.

    Herausgeber z.B.

    \begin{itemize}
        \item Radioclubs
        \item Ortsverbände
        \item Länder
    \end{itemize}
    
\end{frame}

\begin{frame}
    \frametitle{Diplome / Beispiele}

    Bekannteste Diplome z.B.

    \begin{itemize}
        \item DXCC\footnote{DX century club}-Diplom für $>100$ Länder
        \item DLD\footnote{Deutschland Diplom} für $>100$ \emph{OV}s
        \item ... ''All \$Achievement Award''
    \end{itemize}

    Beantragung vom Operator selbst. Nachweis: QSL-Karten, relevante
    Logbuchauszüge oder Bestätigung durch andere \emph{HAM}s.

\end{frame}

\renewcommand{\refname}{Referenzen}

\hypertarget{refs}{}
\textcolor{white}{} \\ %\vspace{} geht nicht
\Large Referenzen/Links
\footnotesize

\begin{thebibliography}{}

    \bibitem{dj4uf} Moltrecht B/V 13: \\
                    \url{http://www.darc.de/referate/ajw/ausbildung/darc-online-lehrgang/betriebstechnikvorschriften/kapitel-bv13/}
    \bibitem{wp}    Wikipedia DE: \\
                    \url{https://de.wikipedia.org/wiki/RST-System}\\
                    \url{https://de.wikipedia.org/wiki/RSQ-System}\\
                    \url{https://de.wikipedia.org/wiki/Koordinierte_Weltzeit}\\
                    \url{https://de.wikipedia.org/wiki/Bundesnetzagentur}\\
    %\bibitem{we}    Wikipedia EN: \\
    \bibitem{wc}    Wikimedia Commons: \\
                    \url{https://commons.wikimedia.org/wiki/File:S-Meter.jpg}\\
                    \url{https://commons.wikimedia.org/wiki/File:Standard_time_zones_of_the_world.png}\\
    \bibitem{tucn}  \url{http://tucnak.nagano.cz/wiki/File:Hf-10.png}\\
\end{thebibliography} 

% Hier könnte noch eine Kontaktfolie stehen

\end{document}

