% Foliensatz: "AFu-Kurs nach DJ4UF" von DK0TU, Amateurfunkgruppe der TU Berlin
% Lizenz: CC BY-NC-SA 3.0 de (http://creativecommons.org/licenses/by-nc-sa/3.0/de/)
% Autoren: Sebastian Lange <dl7bst@dk0tu.de>

preamble.dk0tu.tex
\subtitle{Technik Klasse A 13: \\
          Frequenzaufbereitung \\[2em]}
\date{Stand 18.06.2015}
 \begin{document}

\begin{frame}
    \titlepage
    \vfill
    \begin{center}
        \ccbyncsaeu\\
        {\tiny This work is licensed under the \em{Creative Commons Attribution-NonCommercial-ShareAlike 3.0 License}.}\\[0.5ex]
         \tiny Amateurfunkgruppe der Technische Universität Berlin (AfuTUB), DKØTU
         %\includegraphics[scale=0.5]{img/DK0TU_Logo.pdf}
    \end{center}
\end{frame}



% FIXME Blockschaltbilder sollten in Klasse E schon irgendwo rein

\section{Überblick}

\begin{frame}
    \frametitle{Überblick}

    Wozu Frequenzaufbereitung? Wo liegen die klassischen Kurzwellenbänder und
    was fällt dabei auf?

\end{frame}

\begin{frame}
    \frametitle{Überblick}

    Bereits bekannt aus Kapitel \emph{E 15}: \\[2em]

    Signal im Basisband muss für die Übertragung auf einen HF-Träger
    "`aufgeprägt"' werden.

\end{frame}

\begin{frame}
    \frametitle{Überblick}

    Kurzwellenbänder: \\[2em]

    3,5 - 7 - 14 - 21 - 28 MHz (80m, 40m, 20m, 15m, 10m)

\end{frame}

\section{Sender}

\subsection{Vervielfacher}

\begin{frame}
    \frametitle{Frequenzvervielfacher}

    Neben Vermeidung von Oberwellenstörungen kommerzieller Funkdienste,
    besser für Senderbau durch Vervielfacher:

    \begin{center}
        \includegraphics[width=0.7\textwidth]{a13/TG103a.png}
        \tiny (TG103)
    \end{center}

    Welches Band mit VFO auf 3,51 MHz? \only<2>{\textbf{Lösung:} 14,04 MHz (20m)}

\end{frame}

\begin{frame}
    \frametitle{Frequenzvervielfacher}

    Bei FM darauf achten, dass sich auch der Hub vervielfacht!

    \begin{center}
        \includegraphics[width=0.8\textwidth]{a13/TG102.png}
        \tiny (TG102)
    \end{center}

    ... z.B. durch Antiparallelschaltung von Dioden.\footnote{Einstellung durch
    den Widerstand}

\end{frame}

\subsection{Mischer}

\subsubsection{Einfachmischer}

\begin{frame}
    \frametitle{Einfachmischer}

    Vervielfachung ist nur bei CW und FM sinnvoll. Hub zeigt, dass Modulation
    "`auseinandergezogen wird"' - die NF in SSB würde viel zu breit werden.

    \begin{center}
        \includegraphics[width=0.5\textwidth]{a13/TG226.png}
        \tiny (TG226)
    \end{center}

    Mischer sind bereits aus der Modulation bekannt. Anwendung hier:
    Up-/Downconversion.
    
    \only<2>{\vspace{2em} \textbf{Lösung:} 10,7 und 52,7 MHz}

\end{frame}

\begin{frame}
    \frametitle{Einfachmischer}

    \begin{center}
        \includegraphics[width=0.5\textwidth]{a13/TG226.png}
        \tiny (TG226)
    \end{center}

    Es entstehen immer zwei Frequenzen, sodass immer eine Filterung notwendig
    ist.

\end{frame}

\subsubsection{Balance-Mischer}

\begin{frame}
    \frametitle{Balance-Mischer}

    SSB-Aufbereitung mit einem 9-MHz-Quarzfilter (balancierter Ringmischer)

    \begin{center}
        \includegraphics[width=0.5\textwidth]{a13/TG106.png}
        \tiny (TG106)
    \end{center}

    Für USB: 8,9985 MHz

\end{frame}

\begin{frame}
    \frametitle{Balance-Mischer}

    Der gesamte Sendepfad würde so aussehen:

    \begin{center}
        \includegraphics[width=0.8\textwidth]{a13/TG101.png}
        \tiny (TG101)
    \end{center}

    Was wird nach der SSB-Mischstufe benötigt?
    
    \only<2>{\vspace{2em} \textbf{Lösung:} Ein Quarzfilter als Seitenbandsperre.}

\end{frame}

\subsection{VCO-PLL}

\begin{frame}
    \frametitle{VCO-PLL}

    Phasenregelschleifen dienen zum Angleichen zweier Phasen.

    \begin{center}
        \includegraphics[width=0.8\textwidth]{a13/TD701.png}
        \tiny (TD701)
    \end{center}

    Stabiler Zustand: Die Frequenzen an A und B sind gleich.

    A = 12,5 kHz. Ausgang 12 bis 14 MHz. Teilerfaktor?
    \only<2>{\textbf{Lösung:} 960 bis 1120}

\end{frame}

\subsubsection{Mehrfach-Mischer}

\begin{frame}
    \frametitle{Mehrfach-Mischer}

    % TODO Muendl. oder Folien warum Mehrfachmischung

    \begin{center}
        \includegraphics[width=0.6\textwidth]{a13/TG110.png}
        \tiny (TG110)
    \end{center}

    Frage: Welche Frequenz erzeugt der Sender, wenn VCO1 auf 2,651 MHz
    eingestellt und VCO2 auf 6 MHz eingerastet ist? \only<2>{\textbf{Lösung:} 3,651 MHz}

\end{frame}

\section{Empfänger}

\begin{frame}
    \frametitle{Empfängerprinzipien}

    Doppelsuper: Niedrige zweite ZF gute Trennschärfe. \\[2em]

    Bei heutigen TRX: Die 1. ZF liegt höher als das Doppelte der maximalen
    Empfangsfrequenz. Nach der Filterung im Roofing-Filter (1. ZF) wird auf die
    2. ZF im Bereich um 9 bis 10 MHz heruntergemischt. \\[2em]

    Erste ZF-Filterbandbreite mind. so groß wie höchste benötigte Bandbreite.

\end{frame}

\subsection{Doppelsuper}

\begin{frame}
    \frametitle{Doppelsuper}

    \begin{center}
        \includegraphics[width=0.5\textwidth]{a13/TF206.png}
        \tiny (TF206)
    \end{center}

    RX 145,6 MHz, CO 134,9 MHz. Wo mögliche Spiegelstörungen?
    
    \only<2>{\textbf{Lösung:} 124,2 MHz - deshalb Vorselektion wichtig}

\end{frame}

\begin{frame}
    \frametitle{Doppelsuper}

    \begin{center}
        \includegraphics[width=0.8\textwidth]{a13/TF205b.png}
        \tiny (TF205b)
    \end{center}

    Erste ZF von 10,7 MHz und zweite ZF von 460 kHz. Empfangsfrequenz 28 MHz
    sein. VFO, CO?
    
    \only<2>{\textbf{Lösung:} Der VFO muss bei 38,70 MHz und der CO bei 11,16 MHz schwingen.}
    % FIXME Slide 15: 17,3 und 10,24 moeglich


\end{frame}

\begin{frame}
    \frametitle{Doppelsuper}

    \begin{center}
        \includegraphics[width=0.8\textwidth]{a13/TF209b.png}
        \tiny (TF209b)
    \end{center}

    Vorteil von Kurzwellen-Empfängern mit sehr hoher ZF-Frequenz (z.B. 50 MHz):
    Spiegelfrequenz liegt sehr weit außerhalb des Empfangsbereichs.

\end{frame}


\subsection{Direktüberlagerungsempfänger}

\begin{frame}
    \frametitle{Direktüberlagerungsempfänger}

    Auch Produktdetektor genannt. \\[2em]

    Prinzip wie der Einfachmischer beim TX. Eine Mischstufe mit VFO in nächster
    Nähe zur Empfangsfrequenz.

\end{frame}

\subsection{PLL}

\begin{frame}
    \frametitle{PLL}

    \begin{center}
        \includegraphics[width=0.75\textwidth]{a13/TF213.png}
        \tiny (TF213)
    \end{center}

    PLL-Frequenzaufbereitung: \\
    VCO1: 67,5 MHz, VCO2: 85,5 MHz

\end{frame}

\subsection{Konverter}

\begin{frame}
    \frametitle{Konverter}

    \begin{center}
        \includegraphics[width=0.8\textwidth]{a13/TF204.png}
        \tiny (TF204)
    \end{center}

    2m-Konverter für einen KW-Empfänger

\end{frame}

\subsection{Transverter}

\begin{frame}
    \frametitle{Transverter}

    \begin{center}
        \includegraphics[width=0.75\textwidth]{a13/TF209.png}
        \tiny (TF209)
    \end{center}

    Transverter für das 2m-Band

\end{frame}

\renewcommand{\refname}{Referenzen}

\hypertarget{refs}{}
\textcolor{white}{} \\ %\vspace{} geht nicht
\Large Referenzen/Links
\footnotesize

\begin{thebibliography}{}
    \bibitem{darc}  DARC Online-Lehrgang Lektion A13:
                    \url{http://www.darc.de/referate/ajw/ausbildung/darc-online-lehrgang/technik-klasse-a/technik-a13/}
\end{thebibliography} 

% Hier könnte noch eine Kontaktfolie stehen

\end{document}

