% Foliensatz: "AFu-Kurs nach DJ4UF" von DK0TU, Amateurfunkgruppe der TU Berlin
% Lizenz: CC BY-NC-SA 3.0 de (http://creativecommons.org/licenses/by-nc-sa/3.0/de/)
% Autoren: Sebastian Lange <dl7bst@dk0tu.de>

preamble.dk0tu.tex
\subtitle{Technik Klasse A 18: \\
          Gerätetechnik \\[2em]}
\date{Stand 09.07.2015}
 \begin{document}

\begin{frame}
    \titlepage
    \vfill
    \begin{center}
        \ccbyncsaeu\\
        {\tiny This work is licensed under the \em{Creative Commons Attribution-NonCommercial-ShareAlike 3.0 License}.}\\[0.5ex]
         \tiny Amateurfunkgruppe der Technische Universität Berlin (AfuTUB), DKØTU
         %\includegraphics[scale=0.5]{img/DK0TU_Logo.pdf}
    \end{center}
\end{frame}


\section{Überblick}

\begin{frame}
    \frametitle{Überblick}

    Themen aus dem Kapitel \emph{E15 - Sender- und Empfängertechnik} werden hier
    weiterführend behandelt.

    \bigskip

    % TODO Drake Foto?

    Woraus bestehen Sender und Empfänger und welche Haupteigenschaften fallen
    euch ein?

\end{frame}

\section{Empfindlichkeit}

\begin{frame}
    \frametitle{Wiederholung SNR}

    \begin{center}
        \includegraphics[width=0.8\textwidth]{a18/Received_message.jpg}
        \tiny \hyperlink{refs}{\cite{wc}}
    \end{center}

    \begin{center}
    \includegraphics[width=0.8\textwidth]{a18/Analyse_thermo_gravimetrique_bruit.png}
        \tiny \hyperlink{refs}{\cite{wc}}
    \end{center}

\end{frame}

\begin{frame}
    \frametitle{Empfindlichkeit}

    Empfindlichkeit gibt an, wie stark Eingangssignal sein muss um über dem
    thermischen Eigenrauschen zu liegen. Einfacher: \\

    Fähigkeit des Empfängers, schwache Signale zu empfangen.

    \bigskip

    Wahrnehmungsschwelle bei ca. $6 dB$

\end{frame}

\begin{frame}
    \frametitle{Empfindlichkeit}

    Rauschleistung (Formelsammlung): $P_R = k \cdot T_0 \cdot b$
    \footnote{mit Boltzmannkonstante $k$, Temperatur $T_0$ und Bandbreite $b$}

    \bigskip

    Wichtig ist: Die Leistung eines gleichmäßig über einen Frequenzbereich
    verteilten Rauschens ist proportional zur Bandbreite!

    \bigskip

    \begin{block}{Wie verhält sich bei sonst gleich bleibenden Bedingungen die
                  Rauschleistung nach Umschaltung von SSB auf CW?}
        \only<2>{$\frac{2,5 kHz}{0,5 kHz} \approx \frac{1}{5}$}
    \end{block}

\end{frame}

\subsection{Rauschzahl}

\begin{frame}
    \frametitle{Rauschzahl}

    Noise Figure $F$ bzw. Rauschzahl ist der Faktor um den theoretische
    Rauschformeln gerätespezifisch erweitert werden.

    \bigskip

    Angaben sind als Faktor oder in $dB$ möglich.

    \bigskip

    Beispiele:

    \begin{itemize}
        \item $F=1,8 dB \rightarrow$ Ausgang $1,8dB$ geringeres SNR als das Eingang
        \item $F=2 \rightarrow$ Ausgang \only<1>{\textbf{?}} \only<2>{$3dB$ geringeres} SNR als Eingang
    \end{itemize}

\end{frame}

\begin{frame}
    \frametitle{Rauschzahl}

    Für Kurzwelle und niedrigere Frequenzen spielt die Noise Figure keine Rolle,
    da QRN und QRM den SNR bestimmen.

    \bigskip

    Empfindlichkeiten werden bei HF z.B. mit $0,25 \mu V$ Eingangsspannung für
    $S/N=10 dB$ angegeben.

\end{frame}

\begin{frame}
    \frametitle{Rauschzahl}

    VHF-Vorverstärker möglichst direkt an der Antenne. Warum?

    \bigskip

    Empfindlichkeit kann auch durch starke HF-Signale auf einer nahen Frequenz
    beeinträchtigt werden $\rightarrow$ Selektivität.

\end{frame}

\subsection{Selektivität}

\begin{frame}
    \frametitle{Selektivität}

    Fähigkeit Signale mit steilen Filterflanken zu selektieren. Deshalb auch:
    Trennschärfe.

    \begin{center}
    \includegraphics[width=0.5\textwidth]{a18/TF410.png}
        \tiny (TF410/411)
    \end{center}

    Grenzbandbreite bei -60 dB? Für welche Signale geeignet?
    \only<2>{A: 4 kHz, SSB}

\end{frame}

\begin{frame}
    \frametitle{Selektivität}

    Für einen steilen und schmalen Bandpass eignen sich am besten
    Quarzkristalle.

    Welche Filterbandbreiten würdet ihr für J3E, F1B (RTTY Shift 170 Hz), F3E nutzen?
    \only<2>{A: 2,2 kHz, 500 Hz, 12 kHz}

\end{frame}

\section{HF-Regelung}

\subsection{AGC}

\begin{frame}
    \frametitle{AGC}

    \emph{Automatic Gain Control} sorgt für konstante NF auch bei schwankendem
    HF-Eingang.

    \bigskip

    $\rightarrow$ Bei starkem Eingang reduzieren der Verstärkung der HF-und ZF-Stufen.

\end{frame}

\subsection{Squelch}

\begin{frame}
    \frametitle{Squelch}

    Steuert die ZF- oder NF-Signale, um Grundrauschen auszublenden.

    \bigskip

    Einstellung des Levels etwas über den Rauschen und unterhalb des erwarteten
    Eingangssignals.

\end{frame}

\section[Störungsverm.]{Störungsverminderung}

\begin{frame}
    \frametitle{Störungsverminderung}

    Es werden kurz Beispiele zur Störungsverminderung angerissen.

    \bigskip

    Prüfungsrelevant ist lediglich der Notchfilter.

\end{frame}

\subsection{Passband-Tuning}

\begin{frame}
    \frametitle{Passband-Tuning}

    Durch IF/ZF-Shift wird die Filterkurve soweit verschoben, dass das
    Störsignal ausgeblendet wird.

    % FIXME Gutes Bild finden oder bauen

\end{frame}

\subsection{Bandwidth-Tuning}

\begin{frame}
    \frametitle{Bandwidth-Tuning}

    Übereinanderschieben von steilflankigen Filtern, sodass der
    Durchlassbereicht kleiner wird.

    % FIXME Gutes Bild finden oder bauen
    
    \bigskip

    Wie verhält sich der SNR?
    \only<2>{Proportional zur Brandbreite. Remember?}

\end{frame}

\subsection{Notchfilter}

\begin{frame}
    \frametitle{Notchfilter}

    Auch: Kerbfilter mit ''Loch'' im IF-Durchlassbereich zum Ausblenden
    schmalbandiger Störungen. Frequenzverlauf:

    \begin{center}
    \includegraphics[width=0.5\textwidth]{a18/TF326A.png}
        \tiny (TF326)
    \end{center}

\end{frame}

\subsection{Störbegrenzer/-austaster}

\begin{frame}
    \frametitle{Störbegrenzer/-austaster}

    Störbegrenzer schneidet Spitzenspannungen ab gewissem NF-Pegel ab
    $\rightarrow$ Clipping.

    \bigskip

    Störaustaster regelt bei Störungen IF oder NF komplett herunter
    $\rightarrow$ Noise Blanker.

    % TODO Zeichnungen?

\end{frame}

\section{Großsignalfestigkeit}

\begin{frame}
    \frametitle{Großsignalfestigkeit}

    Starke Signale führen zu Intermodulations- oder Kreuzmodulationsprodukten,
    auch wenn sie außerhalb des Afu-Bandes liegen.

    \bigskip

    Entstehung: Nichtlinearität realer Bauteile.

\end{frame}

\begin{frame}
    \frametitle{Interception Point}

    Ungeradzahlige Intermodulationsprodukte sollen so gering wie möglich
    auftreten $\rightarrow$ Intermodulationsabstand. Aufhebung zu Null an den
    ''Interception Points.''

    % FIXME Skizze

    Beurteilung Intermodulation: Meist mit Interception Point $IP_3$

    \bigskip

    Bei fehlender Großsignalfestigkeit kann Dämpfungsglied am
    Empfängereingang helfen.

\end{frame}

\section{Transceiver}

\begin{frame}
    \frametitle{Transceiver}

    In diesem Teil wird kurz auf praktische Merkmale von Transceivern
    eingegangen.

\end{frame}

\begin{frame}
    \frametitle{Leistung}

    \Large QRP ... QRO

    % <10W ... 750W

\end{frame}

\begin{frame}
    \frametitle{Betriebsarten}

    USB, LSB, FM, RTTY, CW, ...

\end{frame}

\begin{frame}
    \frametitle{Frequenzbereiche}

    HF-TRX meist 160m - 10m

    \bigskip

    UKW-TRX meist 2m + 70cm, ggf. 23cm

\end{frame}

\begin{frame}
    \frametitle{Frequenzanzeige}

    Ältere Empfänger können mit einem einem quarzgesteuerten
    Frequenzmarken-Generator geprüft werden - heute nicht mehr notwendig.

    \bigskip

    Ansonsten wie bei der Messtechnik immer an die Ambivalenz digitaler Anzeigen
    denken: Auflösung $\neq$ Anzeigegenauigkeit.

\end{frame}

\begin{frame}
    \frametitle{RIT}

    Receiver Incremental Tuning zur geringfügigen Veränderung der
    Empfangsfrequenz gegenüber der Sende-QRG.

    \bigskip

    z.B. TX-Drift oder Pile-Up.

\end{frame}

\begin{frame}
    \frametitle{Kompressor}

    Audiokompression der Stimme, damit sie ''satter'' rüberkommt und
    verständlicher wird, damit allerdings ihre Färbung verliert.

\end{frame}

\begin{frame}
    \frametitle{Clipper}

    Signal wird voll ausgesteuert und vom Clipper begrenzt. Zu Vermeidung von
    Oberwellen weitere Tiefpassfilterung.

    \bigskip

    Dies geschieht natürlich alles auf Kosten von Audioinformationen.

\end{frame}

\begin{frame}
    \frametitle{DSP}

    Digital Signal Processing als Überbegriff für jegliche digitale
    Audioverarbeitung. Alle o.g. Beispiele werden heute durch DSPs umgesetzt.

    \bigskip

    Das Signal muss vor dem DSP digitalisiert und anschließend wieder in eine
    analoges Signal umgeformt werden.

\end{frame}

\begin{frame}
    \frametitle{PTT und VOX}

    Push To Talk wurde bereits oft angesprochen - es handelt sich um eine
    einfache Sende-/Empfangsumschaltung.

    \bigskip

    Mit einer Voice Control (VOX) lässt sich die Umschaltung durch den NF-Pegel
    der eigenen Sprache auslösen. Nachteile: Störgeräusche können ggf. auch
    umschalten und meist dadurch ein Delay.

\end{frame}

\renewcommand{\refname}{Referenzen}

\hypertarget{refs}{}
\textcolor{white}{} \\ %\vspace{} geht nicht
\Large Referenzen/Links
\footnotesize

\begin{thebibliography}{}
    \bibitem{darc}  DARC Online-Lehrgang Lektion A18:
                    \url{http://www.darc.de/referate/ajw/ausbildung/darc-online-lehrgang/technik-klasse-a/technik-a18/}
    \bibitem{wm}    Wikimedia:
                    \url{https://commons.wikimedia.org/wiki/File:Received_message.jpg}
                    \url{https://commons.wikimedia.org/wiki/File:Analyse_thermo_gravimetrique_bruit.png}
\end{thebibliography} 

% Hier könnte noch eine Kontaktfolie stehen

\end{document}

