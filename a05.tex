% Foliensatz: "AFu-Kurs nach DJ4UF" von DK0TU, Amateurfunkgruppe der TU Berlin
% Lizenz: CC BY-NC-SA 3.0 de (http://creativecommons.org/licenses/by-nc-sa/3.0/de/)
% Autoren:
% Felix Baum DB4UM <baum@campus.tu-berlin.de>
% Lars Weiler <dc4lw@darc.de>

preamble.dk0tu.tex
\subtitle{Technik A05: \\
  Die Diode und ihre Anwendungen \\[2em]}
\date{Stand 18.05.2016}
 \begin{document}

\begin{frame}
    \titlepage
    \vfill
    \begin{center}
        \ccbyncsaeu\\
        {\tiny This work is licensed under the \em{Creative Commons Attribution-NonCommercial-ShareAlike 3.0 License}.}\\[0.5ex]
         \tiny Amateurfunkgruppe der Technische Universität Berlin (AfuTUB), DKØTU
         %\includegraphics[scale=0.5]{img/DK0TU_Logo.pdf}
    \end{center}
\end{frame}


%fixme Referenzen/Fußnoten-Systematik vereinheitlichen

\section*{Wiederholung}

\begin{frame}
  \frametitle{Halbleiter}
  \begin{itemize}
    \item Was ist ein Halbleiter? (Erinnern aus alter Technik)
    \item Wann leitet eine Siliziumdiode?
    \item Wann leitet eine Germaniumdiode?
    \item Was bedeutet p-dotiert?
  \end{itemize}
\end{frame}

\begin{frame}
  \frametitle{Leitet die Siliziumdiode?}
  \begin{center}
    \includegraphics[width=1\textwidth,height=.85\textheight,keepaspectratio]{a05/Leit_Diode.png}
  \end{center}
\end{frame}

\begin{frame}
  \frametitle{Diode Raumladungszone}
  \begin{center}
    \includegraphics[width=1\textwidth,height=.7\textheight,keepaspectratio]{a05/diode_with_electrical_symbol.png}
    \tiny \hyperlink{refs}{\cite{wm}} \\[3em]
    \large Wie, wo und warum bildet sich eine Raumladungszone (RLZ), auch Verarmungszone genannt, in der pn-Diode aus?
  \end{center}
\end{frame}

\begin{frame}
  \frametitle{Leitet die Siliziumdiode?}
  \begin{center}
    \includegraphics[width=1\textwidth,height=.65\textheight,keepaspectratio]{a05/Pn_Junction_Diffusion_and_Drift.pdf}\\
    \emph{Durch Ausgleichsbewegungen (Diffusionsstrom) zwischen dem p- und dem n-Gebiet wird die RLZ gebildet.\\
    Dort gibt es keine frei beweglichen Ladungsträger mehr.}\
  \end{center}
\end{frame}

\begin{frame}
  \frametitle{Diodenkennlinien}
  \begin{columns}[c]
    \column[c]{.5\textwidth}
    \includegraphics[width=1\textwidth,height=.85\textheight,keepaspectratio]{a05/tc511.png}\\
    {\tiny Prüfungsfrage TC511}
    \column{.45\textwidth}
    Welcher Diodentyp hat welche Kennlinie?
    \begin{description}
      \item[1 $\rightarrow$] \only<2>{Schottkydiode}
      \item[2 $\rightarrow$] \only<2>{Germaniumdiode}
      \item[3 $\rightarrow$] \only<2>{Siliziumdiode}
    \end{description}
  \end{columns}
\end{frame}

\section*{Verschiedene Spezialdioden}

\subsection*{Schottky-Diode}
\begin{frame}
  \frametitle{Schottky-Diode}
  \begin{columns}[c]
    \column[c]{.5\textwidth}
    \begin{center}
      \includegraphics[width=1\textwidth,height=.3\textheight,keepaspectratio]{a05/Diode-Schottky-EN_A-K.png}\\
      \includegraphics[width=0.8\textwidth,height=.6\textheight,keepaspectratio]{a05/AusfuerungsformenSchottkyDiode.png}
      \tiny \hyperlink{refs}{\cite{wm}}
    \end{center}
    \column{.45\textwidth}
    \begin{itemize}
      \item Sehr gut geeignet für HF-Schaltungen
      \item Raumladungszone baut sich schneller auf und ab
      \item RLZ zwischen Metall und N-gebiet
      \item Geringe Durchlassspannung ($0.25V$)
    \end{itemize}
  \end{columns}
\end{frame}

\subsection*{Kapazitätsdiode}
\begin{frame}
  \frametitle{Kapazitätsdiode}
  \begin{columns}[c]
    \column[c]{.5\textwidth}
    \begin{center}
      \includegraphics[width=1\textwidth,height=.3\textheight,keepaspectratio]{a05/Varicap_symbol.png}\\
      \includegraphics[width=0.73\textwidth,height=.55\textheight,keepaspectratio]{a05/Varactor_function.png}
      \tiny \hyperlink{refs}{\cite{wm}}
    \end{center}
    \column{.45\textwidth}
    \begin{itemize}
      \item auch \emph{Varicap}, \emph{Varaktor} oder \emph{Abstimmdiode}
      \item Kapazität sinkt mit steigender Spannung
      \item in fast allen VCOs verbaut
    \end{itemize}
  \end{columns}
\end{frame}

\subsection*{Z-Diode}
\begin{frame}
  \frametitle{Z-Diode}
  \begin{columns}[c]
    \column[c]{.5\textwidth}
    \begin{center}
      \includegraphics[width=.8\textwidth,height=.3\textheight,keepaspectratio]{a05/z-diode.png}
    \end{center}
    \column{.45\textwidth}
    \begin{itemize}
      \item Meist zur Spannungsstabilisierung
      \item Einbau in Sperrichtung
      \item Zur Sicherheit immer mit Vorwiderstand einbauen
    \end{itemize}
  \end{columns}
\end{frame}

\begin{frame}
  \frametitle{Z-Diode}
  \begin{center}
    \includegraphics[width=.7\textwidth,height=.75\textheight,keepaspectratio]{a05/Kennlinie_Z-Diode.png}
    \footnote{\tiny \url{https://commons.wikimedia.org/wiki/File:Kennlinie_Z-Diode.svg}}
  \end{center}
\end{frame}

\subsection*{Fotodiode}
\begin{frame}
  \frametitle{Fotodiode}
  \begin{columns}[c]
    \column[c]{.5\textwidth}
    \begin{center}
      \includegraphics[width=1\textwidth,height=.3\textheight,keepaspectratio]{a05/Symbol_Photodiode.png}\\
      \includegraphics[width=0.8\textwidth,height=.55\textheight,keepaspectratio]{a05/Fotodio.jpg}
      \tiny \hyperlink{refs}{\cite{wm}}
    \end{center}
    \column{.45\textwidth}
    \begin{itemize}
      \item Ändert Widerstand abhängig von Lichteinfall
      \item Wird z.B. als Helligkeitssensor verwendet
    \end{itemize}
  \end{columns}
\end{frame}

\subsection*{Solarzelle}
\begin{frame}
  \frametitle{Solarzelle}
  \begin{center}
    \includegraphics[width=1\textwidth,height=.85\textheight,keepaspectratio]{a05/Solarzelle_Funktionsprinzip2.png}\\
    \tiny \hyperlink{refs}{\cite{wm}}
  \end{center}
\end{frame}

\begin{frame}
  \frametitle{Kennlinie Solar + Fotodiode}
  \begin{center}
    \includegraphics[width=1\textwidth,height=.85\textheight,keepaspectratio]{a05/Kennlinie_Photodiode_1.png}\\
    \tiny \hyperlink{refs}{\cite{wm}}
  \end{center}
\end{frame}

\subsection*{LED}
\begin{frame}
  \frametitle{Leuchtdiode (LED)}
  \begin{columns}[c]
    \column[c]{.5\textwidth}
    \begin{center}
      \includegraphics[width=1\textwidth,height=.3\textheight,keepaspectratio]{a05/Symbol_LED.png}\\
      \includegraphics[width=0.8\textwidth,height=.55\textheight,keepaspectratio]{a05/Verschiedene_LEDs.jpg}
      \tiny \hyperlink{refs}{\cite{wm}}
    \end{center}
    \column{.45\textwidth}
    \begin{itemize}
      \item Leuchtet, wenn in Durchlassrichtung betrieben
      \item Betriebsspannung je nach LED-Farbe von etwa $1.5V$ bis $3.2V$
    \end{itemize}
  \end{columns}
\end{frame}

\subsection*{Optokoppler}
\begin{frame}
  \frametitle{Optokoppler}
  \begin{columns}[c]
    \column[c]{.5\textwidth}
    \begin{center}
      \includegraphics[width=1\textwidth,height=.4\textheight,keepaspectratio]{a05/Optokoppler_Aus.png}\\
      \includegraphics[width=1\textwidth,height=.4\textheight,keepaspectratio]{a05/Optokoppler_An.png}
      \tiny \hyperlink{refs}{\cite{wm}}
    \end{center}
    \column{.45\textwidth}
    \begin{itemize}
      \item LED und Photodiode in einem Bauelement
      \item Galvanische Trennung von zwei Schaltkreisen
      \item Probleme bei Hochfrequenzschaltungen
    \end{itemize}
  \end{columns}
\end{frame}

\section*{Anwendungen der Diode}

\subsection*{Spannungs\-begrenzung}
\begin{frame}
  \frametitle{Spannungsbegrenzung}
  \begin{columns}[c]
    \column[c]{.5\textwidth}
    \begin{center}
      \includegraphics[width=1\textwidth]{a05/spannungsBegrenz.png}\\
      \tiny \hyperlink{refs}{\cite{wm}}
    \end{center}
    \column{.45\textwidth}
    \begin{itemize}
      \item Schneidet Signale oberhalb und unterhalb der Durchlassspannung ab
      \item Gut als Überspannungsschutz
    \end{itemize}
  \end{columns}
\end{frame}


\subsection*{Entkopplung}
\begin{frame}
  \frametitle{Entkopplung - Prüfungsfrage TC527}
  \begin{columns}[c]
    \column[c]{.5\textwidth}
    \begin{center}
      \includegraphics[width=1\textwidth]{a05/TC527.png}\\
      \tiny \hyperlink{refs}{\cite{bna}}
    \end{center}
    \column{.45\textwidth}
    Der Sonnenkollektor liefert $U_1 = 14.9V$. \\[0.2em]
    Der Akkumulator hat $U_2 = 13.9V$. \\[0.2em]
    Das Netzteil ist auf $U_3 = 13.5V$ eingestellt. \\[1.2em]
    Welche Dioden leiten und welche nicht?
  \end{columns}
\end{frame}

\section*{Gleichrichter}

\begin{frame}
  \frametitle{Gleichrichter}
  \begin{columns}[c]
    \column[c]{.5\textwidth}
    \begin{center}
      \includegraphics[width=1\textwidth,height=.85\textheight,keepaspectratio]{a05/Brueckengleichrichter_Bilder.jpg}\\
      \tiny \hyperlink{refs}{\cite{wm}}
    \end{center}
    \column{.45\textwidth}
    \begin{itemize}
      \item Gleichrichter machen aus Wechselspannung Gleichspannung
      \item Mit Hilfe von Dioden wird die obere und untere Halbwelle getrennt
      \item Ohne Dioden sehr schwierig
    \end{itemize}
  \end{columns}
\end{frame}

\begin{frame}
  \frametitle{Einweggleichrichtung}
  \begin{center}
    \includegraphics[width=1\textwidth,height=.6\textheight,keepaspectratio]{a05/Halfwave_rectifier.png}\\
    \tiny \hyperlink{refs}{\cite{wm}} \\[1em] \large
    \begin{itemize}
      \item Nur obere Halbwelle, also sehr hohe Verluste
      \item Einfach aufzubauen
      \item Wäre mit Kondensator ein AM-Hüllkurvendemodulator (siehe Siebung)
    \end{itemize}
  \end{center}
\end{frame}

\begin{frame}
  \frametitle{Vollweggleichrichtung}
  \begin{center}
    \includegraphics[width=1\textwidth,height=.6\textheight,keepaspectratio]{a05/Fullwave_rectifier.png}\\
    \tiny \hyperlink{refs}{\cite{wm}} \\[1em] \large
    \begin{itemize}
      \item Obere und untere Halbwelle, also deutlich weniger Verluste
      \item Schwieriger aufzubauen
      \item Verdoppelt die Abtastrate
      \item Benötigt Transformator mit Mittelpunktanzapfung
    \end{itemize}
  \end{center}
\end{frame}

\begin{frame}
  \frametitle{Brückengleichrichtung}
  \begin{center}
    \includegraphics[width=1\textwidth,height=.6\textheight,keepaspectratio]{a05/Gratz_rectifier.png}\\
    \tiny \hyperlink{refs}{\cite{wm}} \\[1em] \large
    \begin{itemize}
      \item Obere und untere Halbwelle, also deutlich weniger Verluste
      \item Schwieriger aufzubauen
      \item Verdoppelt Frequenz
      \item Benötigt vier Dioden
    \end{itemize}
  \end{center}
\end{frame}

\begin{frame}
  \frametitle{Siebung}
  \begin{columns}[c]
    \column[c]{.5\textwidth}
    \begin{center}
      \includegraphics[width=1\textwidth]{a05/Smoothed_ripple.png}\\
      \tiny \hyperlink{refs}{\cite{wm}}
    \end{center}
    \column{.45\textwidth}
    \begin{itemize}
      \item Glättung durch Tiefpass hinter dem Gleichrichter
      \item Durch verdoppelte Frequenz noch einfacher umzusetzen
      \item LC-TP oder RC-TP nutzbar
      \item In der Realität oft Elektrolytkondensatoren mit mehreren $mF$
    \end{itemize}
  \end{columns}
\end{frame}

\begin{frame}
  \begin{tabular}{l|p{.8\textwidth}}\hline
    \textbf{TF317} &
    \begin{tabular}{c}
      \includegraphics[width=\textwidth,height=.5\textheight,keepaspectratio]{a05/tf317.png}
    \end{tabular}
    \textbf{Bei der Schaltung handelt es sich um einen... ?}
  \end{tabular}
\end{frame}

\begin{frame}
  \frametitle{Gleichspannungsrückgewinnung}
  \begin{center}
    \includegraphics[width=1\textwidth,height=.6\textheight,keepaspectratio]{a05/Positive_Voltage_Clamping_Circuit.png}\\
    \tiny \hyperlink{refs}{\cite{wm}} \\[1em] \large
    \begin{itemize}
      \item Wird benutzt um das Signal anzuheben
      \item Wird z.B. bei Verstärkerschaltungen benötigt
    \end{itemize}
  \end{center}
\end{frame}

\renewcommand{\refname}{Referenzen}

\hypertarget{refs}{}
\textcolor{white}{} \\ %\vspace{} geht nicht
\Large Referenzen/Links
\footnotesize

\begin{thebibliography}{}
  \bibitem{darc}  DARC Online-Lehrgang Lektion A05:
    \url{https://www.darc.de/der-club/referate/ajw/lehrgang-ta/a05/}
  \bibitem{wm}  Wikimedia:
    \url{https://en.wikipedia.org/wiki/File:PN_diode_with_electrical_symbol.svg}
    \url{https://commons.wikimedia.org/wiki/File:Pn_Junction_Diffusion_and_Drift.svg}
    \url{http://commons.wikimedia.org/wiki/File:AusfuerungsformenSchottkyDiode.png}
    \url{http://de.wikipedia.org/wiki/Datei:Diode-Schottky-EN_A-K.svg}
    \url{http://commons.wikimedia.org/wiki/File:Varicap_symbol.svg}
    \url{http://commons.wikimedia.org/wiki/File:Varactor_function.svg}
    \url{http://commons.wikimedia.org/wiki/File:Solarzelle_Funktionsprinzip2.svg}
    \url{https://commons.wikimedia.org/wiki/File:Kennlinie_Photodiode_1.png}
    \url{https://commons.wikimedia.org/wiki/File:Verschiedene_LEDs.jpg}
    \url{http://commons.wikimedia.org/wiki/File:Optokoppler.gif}
    \url{https://commons.wikimedia.org/wiki/File:Brueckengleichrichter_IMGP5380.jpg}
    \url{https://de.wikipedia.org/wiki/Datei:Halfwave.rectifier.en.svg}
    \url{https://commons.wikimedia.org/wiki/File:Gratz.rectifier.en.svg}
    \url{https://commons.wikimedia.org/wiki/File:Smoothed_ripple.svg}
    \url{https://en.wikipedia.org/wiki/File:Positive_Voltage_Clamping_Circuit.svg}
    \url{}
    \url{}
  \bibitem{wp}    Wikipedia - Die freie Enzyklopädie:
    \url{https://de.wikipedia.org/wiki/Gleichrichter}
  \bibitem{bna}   Fragenkatalog Bundesnetzagentur Technik Klasse A:
    \url{https://www.bundesnetzagentur.de/SharedDocs/Downloads/DE/Sachgebiete/Telekommunikation/Unternehmen_Institutionen/Frequenzen/Amateurfunk/Fragenkatalog/TechnikFragenkatalogKlasseAf252rId9014pdf.pdf?__blob=publicationFile&v=3}
  \bibitem{fi}    Freie Inhalte (DK0TU):
    \url{http://www.dk0tu.de/Projekte/Freie_Inhalte/}
  \bibitem{yu}    Youtube Video:
    \url{https://www.youtube.com/watch?v=IcrBqCFLHIY}
\end{thebibliography}

% Hier könnte noch eine Kontaktfolie stehen

\end{document}

