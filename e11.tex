% Foliensatz: "AFu-Kurs nach DJ4UF" von DK0TU, Amateurfunkgruppe der TU Berlin
% Lizenz: CC BY-NC-SA 3.0 de (http://creativecommons.org/licenses/by-nc-sa/3.0/de/)
% Autoren: Felix Baum <baum@campus.tu-berlin.de>
% Korrekturen: Lars Weiler <dc4lw@darc.de>

preamble.dk0tu.tex
\subtitle{Technik 11: \\
  Antennentechnik \\[2em]}
\date{Stand 18.09.2017}
 \begin{document}

\begin{frame}
    \titlepage
    \vfill
    \begin{center}
        \ccbyncsaeu\\
        {\tiny This work is licensed under the \em{Creative Commons Attribution-NonCommercial-ShareAlike 3.0 License}.}\\[0.5ex]
         \tiny Amateurfunkgruppe der Technische Universität Berlin (AfuTUB), DKØTU
         %\includegraphics[scale=0.5]{img/DK0TU_Logo.pdf}
    \end{center}
\end{frame}


\section*{Einleitung}

\begin{frame}
  \frametitle{Antenne}
  \begin{center}
    \begin{figure}
      \includegraphics[width=.5\textwidth,height=.7\textheight,keepaspectratio]{e11/Traqueur_acquisition.JPG}
      \attribcaption{Satellite tracking-aquisition antenna}{Kingbastard}{https://commons.wikimedia.org/wiki/File:Traqueur_acquisition.JPG}{\ccbysa}
    \end{figure}
  \end{center}
\end{frame}

\begin{frame}
  \frametitle{Einleitung}
  \begin{minipage}{0.49\textwidth}
    \begin{figure}
      \includegraphics[width=0.9\textwidth,height=.8\textheight,keepaspectratio]{e11/HEINRICH_HERTZ.jpg}
      \attribcaption{Heinrich Hertz}{}{https://commons.wikimedia.org/wiki/File:HEINRICH_HERTZ.JPG}{\ccpd}
    \end{figure}
  \end{minipage}
  \begin{minipage}{0.49\textwidth}
    \begin{itemize}
      \item Heinrich Hertz (1857-1894)
      \item Nachweis elektromagnetischer Wellen 1888
      \item Hertz'sche Dipol
    \end{itemize}
  \end{minipage}
\end{frame}

\begin{frame}
  \frametitle{Schwingkreis}
  \begin{center} \huge
    $$f = \frac{1}{2  \pi \cdot \sqrt{LC}}$$
    \begin{figure}
      \includegraphics[width=.5\textwidth,height=.3\textheight,keepaspectratio]{e11/Schwingkreis.png}
      \caption{Schwingkreis}
    \end{figure}
  \end{center}
\end{frame}

\begin{frame}
  \frametitle{Dipol}
  \begin{center}
    \begin{figure}
      \begin{figure}
        \includegraphics[width=1\textwidth,height=.8\textheight,keepaspectratio]{e11/dipol_entstehung.png}
        \caption{Dipolentstehung}
      \end{figure}
    \end{figure}
  \end{center}
\end{frame}

\begin{frame}
  \frametitle{E- und H-Feld}
  \begin{center}
    \begin{figure}
      \includegraphics[width=0.85\textwidth,height=.75\textheight,keepaspectratio]{e11/Felder_um_Dipol.png}
      \attribcaption{Felder um Dipol}{Averse}{https://commons.wikimedia.org/wiki/File:Felder_um_Dipol.jpg}{\ccbysa}
    \end{figure}
  \end{center}
\end{frame}

\begin{frame}
  \frametitle{Allgemeines}
  \begin{itemize}
    \item Jeder ungeschirmte Draht ist eine Antenne
    \item Grundsätzlich sind Sende- und Empfangsantennen ähnlich
    \item Alle guten Kurzwellen Sendeantennen sind auch gute Empfangsantennen
    \item Der Antennengewinn bei Sendeantennen gilt ebenso für den Empfang
    \item Umgekehrt nicht immer der Fall (z.B. Ferritantenne)
  \end{itemize}
\end{frame}

\section*{Dipol}

\begin{frame}
  \frametitle{Dipol}
  \begin{center}
    \begin{figure}
      \includegraphics[width=1\textwidth,height=.75\textheight,keepaspectratio]{e11/Faltdipol.png}
      \attribcaption{Faltdipol}{Averse}{https://commons.wikimedia.org/wiki/File:Faltdipol.png}{\ccbysa}
    \end{figure}
  \end{center}
\end{frame}


\begin{frame}
  \frametitle{Wellenlänge}
  \begin{center} \huge
    $$\lambda = \frac{c}{f}$$
    mit $c = 299\,792\,458 \frac{m}{s}$
  \end{center}
\end{frame}

\begin{frame}
  \frametitle{Wellenlänge einfacher}
  \begin{center} \huge
    $$\lambda [m] \approx \frac{300}{f[MHz]}$$
  \end{center}
\end{frame}

%\begin{frame}

%    \frametitle{$\lambda / 2$ Dipol berechnen}
%    \begin{center}
% \begin{itemize} \Large
%  \item Theoretische Strahlerlänge eines $\lambda / 2$ Dipols für 7MHz berechnen
%  \item $\lambda[m] = \frac{300}{f[MHz]}$
%    \end{itemize}
%    \end{center}
%\end{frame}

\begin{frame}
  \frametitle{Verkürzungsfaktor beachten}
  \begin{center}
    \begin{itemize} \Large
      \item Innerhalb von Feststoffen breiten sich EM-Wellen nicht mit Lichtgeschwindigkeit aus
      \item Deshalb: $\lambda[m] = \frac{300}{f[MHz]} \cdot $ Verkürzungsfaktor
      \item Beispiele:
      \begin{itemize}
        \item Kupfer 0,95
        \item RG-213 Koaxialkabel 0,66
        \item Glasfaser 0,67
      \end{itemize}
    \end{itemize}
  \end{center}
\end{frame}

\begin{frame}
  \frametitle{Fertig?}
  \begin{center}
    \begin{figure}
      \includegraphics[width=.8\textwidth,height=.75\textheight,keepaspectratio]{e11/cat-antenna.jpg}
      \attribcaption{Cat Antenna}{Jodi Summers}{http://www.socalgreenrealestateblog.com/control-your-homes-energy-use-with-these-apps/cat-antenna/}{}
    \end{figure}
  \end{center}
\end{frame}

\begin{frame}
  \frametitle{Strom und Spannungsverteilung}
  \begin{itemize}
    \item Antennenlänge: $2\cdot\frac{\lambda}{4}$ Strahler
    \item Stromknoten und Spannungsbauch an den Enden (unendlich großer Widerstand)
    \item Spannungsknoten und Strombauch in der Mitte (fast $0 \Omega$ Widerstand)
  \end{itemize}
  \begin{center}
    \begin{figure}
      \includegraphics[width=0.85\textwidth,height=.5\textheight,keepaspectratio]{e11/DipolUI.png}
      \attribcaption{Strom- (rot) und Spannungsverlauf (blau) entlang der Stäbe eines Halbwellendipols}{Averse in Zusammenarbeit mit Ulfbastel}{https://commons.wikimedia.org/wiki/File:Lineare_antennen.svg}{\ccbysa}
    \end{figure}
  \end{center}
\end{frame}

\begin{frame}
  \frametitle{Fußpunktwiderstand}
  \begin{center}
    \begin{itemize}
      \item Fußpunktwiderstand/Impedanz/Speisewiderstand
      \item Beim Dipol im Freien Raum: $70 \Omega$
      \item Je nach Dipolhöhe zwischen $40 \Omega$ und $80 \Omega$
      \item Bei $0,15 \lambda$ Höhe $50 \Omega$ Speisewiderstand
    \end{itemize}
  \end{center}
\end{frame}

%\begin{frame}
%    \frametitle{Prüfungsfrage}
%    \begin{center}
%    \begin{tabular}{l||l}\hline
%        TH206 & Ein Halbwellendipol wird auf der  \\
%         " "  & Grundfrequenz in der Mitte \\ \hline\hline
%         A & spannungsgespeist.\\\hline
%         B & stromgespeist. \\\hline
%         C & endgespeist. \\ \hline
%         D & parallel gespeist.\\\hline
%    \end{tabular}
%  \end{center}
%\end{frame}

%\begin{frame}
%    \frametitle{Prüfungsfrage}
%
%    \begin{center}
%    \begin{tabular}{l||l}\hline
%        TH206 & Ein Halbwellendipol wird auf der  \\
%         " "  & Grundfrequenz in der Mitte \\ \hline\hline
%         " " & spannungsgespeist.\\\hline
%         X & stromgespeist. \\\hline
%         " " & endgespeist. \\ \hline
%         " " & parallel gespeist.\\\hline
%    \end{tabular}
%        \includegraphics[width=0.85\textwidth]{e11/DipolUI.png}
%        \footnote{\tiny \url{https://commons.wikimedia.org/wiki/%File:Lineare_antennen.svg}}
%  \end{center}
%\end{frame}


\begin{frame}
  \frametitle{Fußpunktwiderstand}
  \begin{center}
    \begin{itemize}
      \item Erwünschter Widerstand: Real $50 \Omega$ Imaginär $O \Omega$
      \item SWR-Meter (Standing wave Ratio) möglichst 1:1
    \end{itemize}
  \end{center}
\end{frame}

\begin{frame}
  \frametitle{SWR}
  \begin{center}
    \begin{itemize}
      \item Aussage über Verhältnis Widerstand am Kabelende zu Widerstand an der Antenne
      \item Informationen darüber wie viel Leistung an die Antenne abgegeben wird
      \item Nicht an die Antenne gegebene Leistung wird zurück in die Endstufe reflektiert
      \item Keine Aussage über Abstrahleigenschaften der Antenne ($50 \Omega$ R ist perfekt)
      \item Typische Werte für eine reale Antenne im WLAN-Bereich liegen etwa bei 2:1 bis 2,5:1.
      \item Für rad1o werden Antennen benötigt mit SWR besser als 2:1
    \end{itemize}
  \end{center}
\end{frame}

\begin{frame}
  \frametitle{Schwingkreis}
  \begin{center}
    \begin{figure}
      \includegraphics[width=.7\textwidth,height=.75\textheight,keepaspectratio]{e11/Measurement0010.png}
      \caption{Messung einer selbstgebauten Yagi für $10m$ bei DK\O TU}
    \end{figure}
  \end{center}
\end{frame}


\section*{Richtdiagramm}

\begin{frame}
  \frametitle{Richtdiagramm Dipol}
  \begin{center}
    \begin{figure}
      \includegraphics[width=0.6\textwidth,height=.75\textheight,keepaspectratio]{e11/Richt-Dipol.png}
      \caption{DB4UM Programm: cocoaNec 2.0}
    \end{figure}
  \end{center}
\end{frame}

\section*{Gewinn}

\begin{frame}
  \frametitle{ERP und EIRP}
  \begin{itemize}
    \item ERP
      \begin{itemize}
        \item Effective Radiated Power
        \item Bezug auf Dipol
        \item $P_{ERP} = G_{Dipol} \cdot (P_{Sender} - P_{Verlust})$
      \end{itemize}
    \item EIRP
      \begin{itemize}
        \item Equivalent Isotropically Radiated Power
        \item Bezug auf Isotropstrahler
        \item $dB_{ERP} = 2,15 + dB_{EIRP}$
        \item $P_{EIRP} = 1,64 \cdot P_{ERP}$
      \end{itemize}
  \end{itemize}
\end{frame}

\begin{frame}
  \frametitle{Isotropstrahler}
  \begin{center}
    \begin{figure}
      \includegraphics[width=0.65\textwidth,height=.75\textheight,keepaspectratio]{e11/Spherical_wave2.png}
      \attribcaption{Sphärische Welle}{Oleg Alexandrov}{https://commons.wikimedia.org/wiki/File:Spherical_wave2.gif}{\ccpd}
    \end{figure}
  \end{center}
\end{frame}

\section*{Multiband}

\begin{frame}
  \frametitle{Multiband Dipol}
  Gewinn: 2,15dBi
  \begin{center}
    \begin{figure}
      \includegraphics[width=0.9\textwidth,height=.75\textheight,keepaspectratio]{e11/Multiband.jpg}
      \caption{Antenne EA-1015204080 von EAntenna}
    \end{figure}
  \end{center}
\end{frame}

\begin{frame}
  \frametitle{G5RV}
  \begin{center}
    \begin{figure}
      \includegraphics[width=0.9\textwidth,height=.75\textheight,keepaspectratio]{e11/G5RV_Antenna.png}
      \attribcaption{G5RV Antenne}{Gerolf Ziegenhain}{https://commons.wikimedia.org/wiki/File:G5RV_Antenna.svg}{\ccbysa}
    \end{figure}
  \end{center}
\end{frame}


\section*{Yagi-Uda}

\begin{frame}
  \frametitle{Yagi-Uda}
  $5dBi$-$30dBi$
  \begin{center}
    \begin{figure}
      \includegraphics[width=0.36\textwidth,height=.7\textheight,keepaspectratio]{e11/Yagi_3_element.png}
      \attribcaption{3 Element Yagi Antenne}{Sankeytm}{https://commons.wikimedia.org/wiki/File:Yagi_3_element.svg}{\ccbysa}
    \end{figure}
  \end{center}
\end{frame}


\begin{frame}
  \frametitle{Richtdiagramm Yagi}
  \begin{center}
    \begin{figure}
      \includegraphics[width=0.55\textwidth,height=.75\textheight,keepaspectratio]{e11/yagi_gain.png}
      \caption{DK\O TU 10m Yagi 28.1 MHz von DL2JAS Programm: EZNEC}
    \end{figure}
  \end{center}
\end{frame}

\begin{frame}
  \frametitle{Yagi - Richtung erkennen}
  \begin{center}
    \begin{figure}
      \includegraphics[width=.75\textwidth,height=.75\textheight,keepaspectratio]{e11/yagi.jpg}
      \caption{10M Yagi bei DK\O TU von DK9GD}
    \end{figure}
  \end{center}
\end{frame}


\section*{Groundplane}

\begin{frame}
  \frametitle{Groundplane}
  $2.15dBi$ - Jeder Draht $\frac{\lambda}{4}$
  \begin{center}
    \begin{figure}
      \includegraphics[width=0.6\textwidth,height=.7\textheight,keepaspectratio]{e11/GP-DB4UM.png}
      \caption{DB4UM mit cocoaNec 2.0}
    \end{figure}
  \end{center}
\end{frame}

\begin{frame}
  \frametitle{Spiegelladung}
  \begin{center}
    \begin{figure}
      \includegraphics[width=0.7\textwidth,height=.75\textheight,keepaspectratio]{e11/Spiegelladung.png}
      \attribcaption{Spiegelladung einer positiven Ladung an einer Metallfläche}{Paeng}{https://de.wikipedia.org/wiki/Datei:Spiegelladung.svg}{\ccbysa}
    \end{figure}
  \end{center}
\end{frame}

\begin{frame}
  \frametitle{Spiegelladung}
  \begin{center}
    \begin{figure}
      \includegraphics[width=0.55\textwidth,height=.75\textheight,keepaspectratio]{e11/Antena_marconi501.png}
      \attribcaption{Marconi Antenne}{n/a}{https://commons.wikimedia.org/wiki/File:Antena_marconi501.png}{\ccpd}
    \end{figure}
  \end{center}
\end{frame}

\begin{frame}
  \frametitle{Magnetfuss-Antenne}
  \begin{center}
    \begin{figure}
      \includegraphics[height=0.8\textheight,height=.75\textheight,keepaspectratio]{e11/magnetfuss.jpg}
      \caption{Magnetfuss-Antenne}
    \end{figure}
  \end{center}
\end{frame}

\section*{Magnetic Loop}

\begin{frame}
  \frametitle{Magnetic Loop}
  \begin{center}
    \begin{figure}
      \includegraphics[width=0.89\textwidth,height=.75\textheight,keepaspectratio]{e11/Magloop.jpg}
      \caption{Magloop bei DK\O TU von DB4UM}
    \end{figure}
  \end{center}
\end{frame}

\section*{Portabelfunk}

\begin{frame}
  \frametitle{Portabel Kurzwellenfunk}
  \begin{center}
    \begin{figure}
      \includegraphics[height=.83\textheight,height=.75\textheight,keepaspectratio]{e11/portabelFunk.png}
      \caption{C-Pole Antenne für 20m im Mauerpark \#berlinUrbanHamRadio}
    \end{figure}
  \end{center}
\end{frame}

\section*{Polarisierung}

\begin{frame}
  \frametitle{Polarisierung}
  \begin{itemize}
    \item Welche Antennen sind vertikal, welche horizontal polarisiert?
    \item Wie ist die Antenne unten polarisiert?
  \end{itemize}
  \begin{center}
    \begin{figure}
      \includegraphics[width=.3\textwidth,height=.75\textheight,keepaspectratio]{e11/kreutzYagi.jpg}
      \caption{DK\O TU Fieldday 2014 von DL7BUR}
    \end{figure}
  \end{center}
\end{frame}

%\begin{frame}
%    \frametitle{Jetzt Ihr}
%    \begin{center}  Baut, messt und benutzt eure selbstgebauten Antennen
%        \includegraphics[width=1\textwidth]{e11/hamPro.jpg}
% \end{center}
%\end{frame}

%\begin{frame}
%    \frametitle{Kontakt}
%     \begin{itemize}
%  \item Twitter: [at]b4um oder [at]dk0tu%
%  \item baum AT campus.tu-berlin.de
%  \item GSM: B4UM
%  \item Amateur Radio Dorf
%    \end{itemize}
%\end{frame}


\section*{Referenzen}

\begin{frame}
  \frametitle{Referenzen/Links}

  \footnotesize
  \begin{itemize}
    \item Moltrecht E 11: \\
      \url{https://www.darc.de/der-club/referate/ajw/lehrgang-te/e11/}
    \item Strahlungsdiagramm (Youtube): \\
      \url{https://www.youtube.com/watch?v=gBqqp7rnZ64}
  \end{itemize}

\end{frame}


% Hier könnte noch eine Kontaktfolie stehen

\end{document}

