% Foliensatz: "AFu-Kurs nach DJ4UF" von DK0TU, Amateurfunkgruppe der TU Berlin
% Lizenz: CC BY-NC-SA 3.0 de (http://creativecommons.org/licenses/by-nc-sa/3.0/de/)
% Autoren: Sebastian Lange <dl7bst@dk0tu.de>
% Korrekturen: Lars Weiler <dc4lw@darc.de>

preamble.dk0tu.tex
\subtitle{Betriebstechnik/Vorschriften 02:         \\
  Das ``Internationale Buchstabieralphabet'' \\[2em]}
\date{Stand 18.09.2017}
 \begin{document}

\begin{frame}
    \titlepage
    \vfill
    \begin{center}
        \ccbyncsaeu\\
        {\tiny This work is licensed under the \em{Creative Commons Attribution-NonCommercial-ShareAlike 3.0 License}.}\\[0.5ex]
         \tiny Amateurfunkgruppe der Technische Universität Berlin (AfuTUB), DKØTU
         %\includegraphics[scale=0.5]{img/DK0TU_Logo.pdf}
    \end{center}
\end{frame}


\section*{Einleitung}

\begin{frame}
  \frametitle{Einleitung / Buchstabieralphabete}
  \begin{center}
    \Large{Wozu braucht man das?} \\
    \Large{Welche kennt ihr?}
  \end{center}
\end{frame}

\begin{frame}
  \frametitle{Einleitung / Buchstabieralphabete}

  Vermeidung von Verwechslungen (bei englischer Aussprache):

  \begin{itemize}
    \item D vs. G vs. T vs. E
    \item J vs. A
    \item Q vs. U
    \item Y vs. I
    \item je nach Dialekt und Aussprache einige mehr
  \end{itemize}

  Viele verschiende nationale und internationale Buchstabieralphabete. Wir
  verwenden nicht das deutsche.

\end{frame}

\section*{Buch\-stabier\-tafel}

\begin{frame}
  \frametitle{Buchstabiertafel / Zusammentragen}

  Laut Radio Regulations (RR) verwenden wir das internationale Alphabet
  (ITU\footnote{\tiny International Telecommunication
  Union}/ICAO\footnote{\tiny International Civil Aviation
  Organization}/NATO\footnote{\tiny North Atlantic Treaty Organization}).

  Nicht nur für die Prüfung wichtig, sondern für die Praxis. Die Benutzung
  trainiert.

  \begin{exampleblock}{Buchstabierwörter}
    Aufgabe: Gemeinsames Zusammentragen der Buchstabierwörter!
  \end{exampleblock}

  Audio Check: \emph{NATO Phonetic Alphabet reading}
  \footnote{\tiny \url{http://commons.wikimedia.org/wiki/File:NATO_Phonetic_Alphabet_reading.ogg}}

\end{frame}

\begin{frame}
  \frametitle{Buchstabiertafel / gesamte Tabelle}

  \begin{center}
    \scriptsize
    \begin{tabular}{|l|l|||l|l|}\hline
      Zeichen           & Phonetik      & Zeichen          & Phonetik       \\ \hline \hline
      \textbf{A}lfa     & (AL-FAH)      & \textbf{S}ierra  & (SEE-AIR-RAH)  \\ \hline
      \textbf{B}ravo    & (BRAH-VOH)    & \textbf{T}ango   & (TANG-GO)      \\ \hline
      \textbf{C}harlie  & (SHAR-LEE)    & \textbf{U}niform & (YOU-NEE-FORM) \\ \hline
      \textbf{D}elta    & (DELL-TAH)    & \textbf{V}ictor  & (VIK-TAH)      \\ \hline
      \textbf{E}cho     & (ECK-OH)      & \textbf{W}hiskey & (WISS-KEY)     \\ \hline
      \textbf{F}oxtrot  & (FOKS-TROT)   & \textbf{X}ray    & (ECKS-RAY)     \\ \hline
      \textbf{G}olf     & (GOLF)        & \textbf{Y}ankee  & (YANG-KEY)     \\ \hline
      \textbf{H}otel    & (HOH-TEL)     & \textbf{Z}ulu    & (ZOO-LOO)      \\ \hline
      \textbf{I}ndia    & (IN-DEE-AH)   & \textbf{0} Zero  & (ZEE-RO)       \\ \hline
      \textbf{J}uliett  & (JEW-LEE-ETT) & \textbf{1} One   & (WUN)          \\ \hline
      \textbf{K}ilo     & (KEY-LOH)     & \textbf{2} Two   & (TOO)          \\ \hline
      \textbf{L}ima     & (LEE-MAH)     & \textbf{3} Three & (TREE)         \\ \hline
      \textbf{M}ike     & (MIKE)        & \textbf{4} Four  & (FOW-ER)       \\ \hline
      \textbf{N}ovember & (NO-VEM-BER)  & \textbf{5} Five  & (FIFE)         \\ \hline
      \textbf{O}scar    & (OSS-CAH)     & \textbf{6} Six   & (SIX)          \\ \hline
      \textbf{P}apa     & (PAH-PAH)     & \textbf{7} Seven & (SEV-EN)       \\ \hline
      \textbf{Q}uebec   & (KEH-BECK)    & \textbf{8} Eight & (AIT)          \\ \hline
      \textbf{R}omeo    & (ROW-ME-OH)   & \textbf{9} Nine  & (NIN-ER)       \\ \hline
    \end{tabular}
  \end{center}

\end{frame}


\begin{frame}
  \frametitle{Buchstabiertafel / Anmerkungen}

  Gerade im deutschen Raum Vorsicht mit: \\
  \textbf{Y}ankee (kein J) und \textbf{Z}ulu (kein S). \\[2em]

  Für die \textbf{Praxis} im Hinterkopf behalten: \\
  Verwechslungen können dennoch möglich sein. Es bietet sich an einige
  Alternativen/Ergänzungen zu kennen und ggf. zu verwenden z.B.
  \begin{itemize}
    \item \textbf{H}onolulu
    \item \textbf{I}taly
    \item \textbf{R}adio
    \item \textbf{S}ugar
    \item \textbf{T}okyo
    \item \textbf{U}nited
  \end{itemize}

\end{frame}

\begin{frame}
  \frametitle{Alternativen / Beispiel DKØTU}

  \only<1>{Buchstabiert?}

  \pause

  Oft verstanden als \emph{Delta Kilo Zero \textbf{Echo} Uniform} \\[2em]

  \pause

  Alternativen: \emph{Tokyo/Texas United}
  \footnote{Disclaimer: Nicht von alten/alternativen
  Buchstabieralphabeten verwirren lassen: Romeo, Radio, Russia...!}\\[2em]
  Problematisch kann auch z.B. Y vs. W sein.

\end{frame}


\section*{Übung}

\begin{frame}

  \begin{center}
    \begin{figure}
      \includegraphics[width=\textwidth,height=.9\textheight,keepaspectratio]{bv02/na_bravo_meme.jpg}
      \attribcaption{Na Bravo}{“Der Postillon”}{http://www.der-postillon.com/2013/02/newsticker-414.html}{}
    \end{figure}
  \end{center}

\end{frame}

\begin{frame}
  \frametitle{Übung: Buchstabieren von Pangrammen\footnote{https://de.wikipedia.org/wiki/Pangramm}}

  \begin{exampleblock}{Aufgabe: Verwendet das internat. Buchstabieralphabet!}

    \only<1>{\uppercase{The quick brown fox jumps over the lazy dog}}

    \only<2>{\uppercase{Franz jagt im komplett verwahrlosten Taxi quer durch Bayern}}

    \only<3>{\uppercase{Falsches Üben von Xylophonmusik quält jeden größeren Zwerg}}

    \only<4>{\uppercase{Prall vom Whisky flog Quax den Jet zu Bruch}}

    \only<5>{\uppercase{The five boxing wizards jump quickly.}}

    \only<6>{\uppercase{MS ``Fliegerkosmonaut der DDR Sigmund
    Jähn''}\footnote{ehem. Frachtschiff der DDR und
    beliebt den Funker zum Buchstabieren aufzufordern}}

    % Vogel Quax zwickt Johnys Pferd Bim
    % Zwölf Boxkämpfer jagen Viktor quer über den großen Sylter Deich

  \end{exampleblock}

\end{frame}

\begin{frame}
  \frametitle{Übung: Buchstabieren und Aufschreiben}

  \begin{alertblock}{Hausaufgabe}
    Das Internationale Buchstabieralphabet nebenbei auswendig lernen:
    Einfach alles möglich sich selbst buchstabieren. Zum Beispiel
    Kfz-Kennzeichen, Werbung in der U-Bahn, etc.
  \end{alertblock}

  \begin{alertblock}{Aufgabe: Buchstabiert euren Namen!}
    Den eigenen Namen komplett buchstabieren können -- beim nächsten Mal
    wird zufällig jemand ausgewählt und alle anderen müssen mitschreiben.
  \end{alertblock}

\end{frame}

\section*{Referenzen}

\begin{frame}
  \frametitle{Referenzen/Links}

  \footnotesize
  \begin{itemize}
    \item Moltrecht B/V 02: \\
      \url{https://www.darc.de/der-club/referate/ajw/lehrgang-bv/bv02/}
    \item Wikipedia DE: \\
      \url{http://de.wikipedia.org/wiki/Buchstabiertafel}
    \item Wikipedia EN: \\
      \url{http://en.wikipedia.org/wiki/NATO_phonetic_alphabet}
  \end{itemize}

\end{frame}

% Hier könnte noch eine Kontaktfolie stehen

\end{document}

