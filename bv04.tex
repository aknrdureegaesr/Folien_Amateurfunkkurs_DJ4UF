% Foliensatz: "AFu-Kurs nach DJ4UF" von DK0TU, Amateurfunkgruppe der TU Berlin
% Lizenz: CC BY-NC-SA 3.0 de (http://creativecommons.org/licenses/by-nc-sa/3.0/de/)
% Autoren: Sebastian Lange <dl7bst@dk0tu.de>
% Korrekturen: Lars Weiler <dc4lw@darc.de>

preamble.dk0tu.tex
\subtitle{Betriebstechnik/Vorschriften 04:         \\
  Betriebliche Abkürzungen \\[2em]}
\date{Stand 18.09.2017}
 \begin{document}

\begin{frame}
    \titlepage
    \vfill
    \begin{center}
        \ccbyncsaeu\\
        {\tiny This work is licensed under the \em{Creative Commons Attribution-NonCommercial-ShareAlike 3.0 License}.}\\[0.5ex]
         \tiny Amateurfunkgruppe der Technische Universität Berlin (AfuTUB), DKØTU
         %\includegraphics[scale=0.5]{img/DK0TU_Logo.pdf}
    \end{center}
\end{frame}


\section*{Einleitung}

\begin{frame}
  \frametitle{Einleitung / Abkürzungen}
  \begin{center}
    \Large{Wozu braucht man diese?} \\
    \Large{Welche kennt ihr bereits? ($\Rightarrow$ ggf. Tafel)}
  \end{center}
\end{frame}

\section*{Abkürzungen}

\begin{frame}
  \frametitle{Abkürzungen}

  Es folgen Abkürzungen, die in Prüfungsfragen eine Rolle spielen. \\[1em]

  Vollständige Liste in der Lektion
  \href{http://www.amateurfunkpruefung.de/lehrg/bv04/bv04.html}{\ExternalLink \texttt{B/V 04} des Moltrechts}. \\[1em]

  Don't Panic! Es geht um eine Übersicht - in den einzelnen Lektionen werden
  die Begriffe behandelt. Als Abschlusskapitel dient es noch einmal der
  Wiederholung.

\end{frame}

\section*{Technik}

%todo nach Moltrecht-Themen sortieren und verlinken: Betriebstechnik, EMV, ...

\subsection*{Frequenzen}
\begin{frame}
  \frametitle{Bänder, Wellen/Wellenausbreitung}

  \begin{center}
    \footnotesize
    \begin{tabular}{|l|l|}\hline
      \textbf{Abkürzung} & \textbf{Bedeutung}                         \\ \hline \hline
      NF    & Niederfrequenz (3-30000 Hz; AF = Audio Frequency)       \\ \hline
      HF    & High Frequency (3-30 MHz)                               \\ \hline
      VHF   & Very High Frequency (30-300MHz)                         \\ \hline
      UHF   & Ultra High Frequency (300-3000 MHz)                     \\ \hline
      SHF   & Super High Frequency (3-30 GHz)                         \\ \hline
      LUF   & Lowest Usable Frequency                                 \\ \hline
      MUF   & Maximum Usable Frequency                                \\ \hline
    \end{tabular}
  \end{center}

\end{frame}


\subsection*{Betriebsarten}
\begin{frame}
  \frametitle{Betriebsarten}

  \begin{center}
    \footnotesize
    \begin{tabular}{|l|l|}\hline
      \textbf{Abkürzung} & \textbf{Bedeutung}                         \\ \hline \hline
      AFSK  & Audio Frequency-Shift Keying                            \\ \hline
      AM    & Amplitude Modulation                                    \\ \hline
      APRS  & Automatic Packet Reporting System                       \\ \hline
      ATV   & Amateur TeleVision                                      \\ \hline
      BBS   & Bulletin Board System (Telnet Mailbox)                  \\ \hline
      CW    & Continuous Wave (carrier wave modulation, Morse code)   \\ \hline
      DSB   & Double-SideBand transmission                            \\ \hline
      FAX   & Facsimile, Fax                                          \\ \hline
      FM    & Frequency Modulation                                    \\ \hline
      FSK   & Frequency-Shift Keying                                  \\ \hline
      LSB   & Lower SideBand                                          \\ \hline
      PM    & Phase Modulation                                        \\ \hline
      PSK   & Phase-Shift Keying                                      \\ \hline
      RTTY  & RadioTeleTYpe                                           \\ \hline
      SSB   & Single Side Band modulation                             \\ \hline
      USB   & Upper SideBand                                          \\ \hline
    \end{tabular}
  \end{center}

\end{frame}

\subsection*{Sende-/Empfangstechnik}
\begin{frame}
  \frametitle{Sende-/Empfangstechnik}

  \begin{center}
    \footnotesize
    \begin{tabular}{|l|l|}\hline
      \textbf{Abkürzung} & \textbf{Bedeutung}                         \\ \hline \hline
      AGC   & Automatic Gain Control (RX)                             \\ \hline
      ALC   & Automatic Level Control (TX)                            \\ \hline
      BFO   & Beat Frequency Oscillator                               \\ \hline
      DSP   & Digital Signal Processing                               \\ \hline
      FET   & Field-Effect Transistor                                 \\ \hline
      LPM   & Letters Per Minute                                      \\ \hline
      PA    & Power Amplifier                                         \\ \hline
      PLL   & Phase-Locked Loop                                       \\ \hline
      PWR   & PoWeR                                                   \\ \hline
      RX    & Receiver                                                \\ \hline
      TX    & Transmitter                                             \\ \hline
      TRX   & Transceiver (TX + RX)                                   \\ \hline
      TNC   & Terminal Node Controller                                \\ \hline
      VCO   & Voltage-Controlled Oscillator                           \\ \hline
      VFO   & Variable-Frequency Oscillator                           \\ \hline
      VOX   & Voice Operated eXchange (voice-operated switch)         \\ \hline
    \end{tabular}
  \end{center}

\end{frame}

\section*{Vorschriften}

\subsection*{Institutionen}
\begin{frame}
  \frametitle{Institutionen}

  \begin{center}
    \footnotesize
    \begin{tabular}{|l|l|}\hline
      \textbf{Abkürzung}   & \textbf{Bedeutung}                             \\ \hline \hline
      \multirow{2}*{CCIR}  & Comité Consultatif International des Radiocommunications \\
      & (heute ITU-R; von 1927 bis 1992)               \\ \hline
      \multirow{3}*{CCITT} & Comité Consultatif International               \\
      & Téléphonique et Télégraphique                  \\
      & (heute ITU-T; 1865 bis 1993)                   \\ \hline
      \multirow{2}*{CEPT}  & Conférence Européenne des Administrations      \\
      & des Postes et des Télécommunications           \\ \hline
      IARU                 & International Amateur Radio Union              \\ \hline
      ITU                  & International Telecommunication Union          \\ \hline
      WARC                 & World Administrative Radio Conference          \\ \hline
    \end{tabular}
  \end{center}

\end{frame}

\subsection*{EMV, Störung}
\begin{frame}
  \frametitle{EMV, Störung}

  \begin{center}
    \footnotesize
    \begin{tabular}{|l|l|}\hline
      \textbf{Abkürzung}  & \textbf{Bedeutung}                                 \\ \hline \hline
      EMV                 & ElektroMagnetische Verträglichkeit                 \\ \hline
      \multirow{2}*{EMVG} & Gesetz über die elektromagnetische Verträglichkeit \\
      & von Betriebsmitteln                                \\ \hline
      EMVU                & Elektromagnetische Umweltverträglichkeit           \\ \hline
      ERP                 & Effective Radiated Power                           \\ \hline
      PEP                 & Peak Envelope Power                                \\ \hline
      TVI                 & TeleVision Interference                            \\ \hline
    \end{tabular}
  \end{center}

\end{frame}


\section*{Betriebstechnik}

\begin{frame}
  \frametitle{AFu-Betriebstechnik}

  \begin{center}
    \footnotesize
    \begin{tabular}{|l|l|}\hline
      \textbf{Abkürzung} & \textbf{Bedeutung}                         \\ \hline \hline
      DX    & Distance                                                \\ \hline
      IRC   & International Reply Coupon                              \\ \hline
      OSCAR & Orbiting Satellite Carrying Amateur Radio               \\ \hline
      SAE   & Stamped Addressed Envelope                              \\ \hline
      SASE  & Self-Addressed Stamped Envelope                         \\ \hline
      UTC   & Universal Time, Coordinated                             \\ \hline
      WX    & Weather                                                 \\ \hline
    \end{tabular}
  \end{center}

\end{frame}

\section*{Telegrafie}

\begin{frame}
  \frametitle{Telegrafie}

  Hier ein paar wichtige und für die Prüfung relevante Abkürzungen:

  \begin{center}
    \footnotesize
    \begin{tabular}{|l|l|}\hline
      \textbf{Abkürzung} & \textbf{Bedeutung}                         \\ \hline \hline
      CQ de DK0TU & Allgemeiner Anruf von DK0TU                       \\ \hline
      CQ DL de ...& Verbindung zu einer DL-Station gewünscht          \\ \hline
      R           & Received                                          \\ \hline
      K           & Aufforderung zum Senden                           \\ \hline
      BK          & Signal zur Unterbrechung der Sendung              \\ \hline
      73          & Viele Grüße                                       \\ \hline
    \end{tabular}
  \end{center}

  Schaut euch die umfangreicheren Listen im Moltrecht an...

\end{frame}

\subsection*{Übung}
\begin{frame}
  \frametitle{Übung: Kurzer Beispieltext aus einem CW-QSO}

  \begin{exampleblock}{Aufgabe: Was könnten folgende Zeilen bedeuten?}
    CQ DL CQ DL de HB9ZZ pse k \\
    HB9ZZ de DK0TU DK0TU pse k \\
    DK0TU de HB9ZZ rst 599 k\\
    HB9ZZ de DK0TU rst 599 bk ... condx vy good k \\
    DK0TU de HB9ZZ tks fer qso - 73 k \\
    HB9ZZ de DK0TU gn et cul sk \\
  \end{exampleblock}

\end{frame}

\section*{Referenzen}

\begin{frame}
  \frametitle{Referenzen/Links}

  \footnotesize
  \begin{itemize}
    \item Moltrecht B/V 04: \\
      \url{https://www.darc.de/der-club/referate/ajw/lehrgang-bv/bv04/}
  \end{itemize}

\end{frame}

% Hier könnte noch eine Kontaktfolie stehen

\end{document}

