% Foliensatz: "AFu-Kurs nach DJ4UF" von DK0TU, Amateurfunkgruppe der TU Berlin
% Lizenz: CC BY-NC-SA 3.0 de (http://creativecommons.org/licenses/by-nc-sa/3.0/de/)
% Autoren: Sebastian Lange <dl7bst@dk0tu.de>

preamble.dk0tu.tex
\subtitle{Betriebstechnik/Vorschriften 07: \\
          Funkbetrieb im Ausland \\[2em]}
\date{Stand 27.11.2014}
 \begin{document}

\begin{frame}
    \titlepage
    \vfill
    \begin{center}
        \ccbyncsaeu\\
        {\tiny This work is licensed under the \em{Creative Commons Attribution-NonCommercial-ShareAlike 3.0 License}.}\\[0.5ex]
         \tiny Amateurfunkgruppe der Technische Universität Berlin (AfuTUB), DKØTU
         %\includegraphics[scale=0.5]{img/DK0TU_Logo.pdf}
    \end{center}
\end{frame}


%fixme Referenzen/Fußnoten-Systematik vereinheitlichen

\section{Einleitung}

\begin{frame}
    \frametitle{Einleitung: Wiederholung Rufzeichen/Landeskenner}

    \begin{center}
        \Large{Wie erkennt man eine Station in einem Gastland?}
    \end{center}

\end{frame}

\begin{frame}
    \frametitle{Einleitung: Generell ...}

    ... muss für den vorübergehenden Funkbetrieb im Ausland eine so genannte
    \textbf{Gastzulassung} (auch: Gastlizenz) beantragen oder gar eine Prüfung
    ablegen. \\[1em]

    Wie bei den \emph{Gesetzen, Vorschriften und Regelungen} kennengelernt, hat
    der Staat das Vorrecht den Funk in seinem Land zu regulieren.

\end{frame}

\section{CEPT-Empfehlungen}

\begin{frame}
    \frametitle{CEPT-Empfehlungen}

    \textbf{CEPT}, bekannt aus \texttt{BV05} (Gesetze, Vorschriften und Regelungen): \\[1em]

    \begin{center}
        \emph{Conférence Européenne des Administrations des Postes et des
        Télécommunications}
    \end{center}

\end{frame}

\subsection{T/R 61-01}

\begin{frame}
    \frametitle{T/R 61-01}

    Erlaubt den Betrieb einer Amateurfunkstation im Ausland unter folgenden
    Bedingungen:

    \begin{itemize}
        \item gültig in allen EU-Staaten sowie solche die den Empfehlungen \textbf{beigetreten} sind
        \item \textbf{vorübergehend}er Aufenthalt im Gastland - maximal \textbf{drei Monate}.
        \item \textbf{Landeskenner} des Gastlandes voranstellen, z.B. \emph{HB9/DK0TU} - ggf. CEPT-Klasse beachten
        \item \textbf{Bestimmungen} des Gastlandes beachten! Z.B. kein \emph{6m}
    \end{itemize}

\end{frame}

\subsection{T/R 61-02 (HAREC)}

\begin{frame}
    \frametitle{T/R 61-02 (HAREC)}

    Längere Aufenthalte erfordern eine Funklizenz (oft mit Prüfung) im
    entsprechenden Land. \\[1em]

    Ausnahme: Anerkennung der Empfehlung \texttt{T/R 61-02}. \\[1em]

    Diese regelt das \textbf{Harmonized Amateur Radio Examination
    Certificate (HAREC)} (= AFu-Zeugnis Klasse A).

\end{frame}

\subsection{ECC-(05)06}

\begin{frame}
    \frametitle{ECC-(05)06}

    Legt \textbf{Novice-Licence} (= AFu-Zeugnis Klasse E) fest. \\[2em]

    ECC: \emph{Electronic Communications Committee (ECC)} (Ausschuss der CEPT)

\end{frame}

\section{Lernhinweise}

\begin{frame}
    \frametitle{Lernhinweise}

    \begin{itemize}
        \item zeitnah die Moltrecht-Lektion \texttt{B/V 07} durchlesen - die
        Abschlussübungen sind interessant und aufschlussreich
        \item "Buzzwords" merken:
        \begin{itemize}
            \item CEPT
            \item T/R 61-01
            \item T/R 61-02 (HAREC)
            \item ECC (05)06
        \end{itemize}
        \item Aufgaben in Ruhe lesen und nicht verwirren lassen
        \item Ausschlussverfahren
        \item in der Woche vor der Prüfung nochmal alles mit dem Simulator durchklicken
    \end{itemize}

\end{frame}

\section{Referenzen}

\begin{frame}
    \frametitle{Referenzen/Links}
    
    \footnotesize
    \begin{itemize}
        \item Moltrecht B/V 07: \\
              \url{http://www.amateurfunkpruefung.de/lehrg/bv07/bv07.htm}
        \item Wikipedia DE: \\
              \url{http://de.wikipedia.org/wiki/CEPT} \\
              \url{http://de.wikipedia.org/wiki/HAREC} \\
              \url{http://de.wikipedia.org/wiki/Amateurfunkzeugnis} \\
              \url{http://de.wikipedia.org/wiki/Regelungen_im_Amateurfunkdienst} \\
    \end{itemize}

\end{frame}

% Hier könnte noch eine Kontaktfolie stehen

\end{document}

