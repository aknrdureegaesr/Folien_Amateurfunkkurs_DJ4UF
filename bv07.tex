% Foliensatz: "AFu-Kurs nach DJ4UF" von DK0TU, Amateurfunkgruppe der TU Berlin
% Lizenz: CC BY-NC-SA 3.0 de (http://creativecommons.org/licenses/by-nc-sa/3.0/de/)
% Autoren: Sebastian Lange <dl7bst@dk0tu.de>
% Korrekturen: Lars Weiler <dc4lw@darc.de>

preamble.dk0tu.tex
\subtitle{Betriebstechnik/Vorschriften 07: \\
  Funkbetrieb im Ausland \\[2em]}
\date{Stand 18.09.2017}
 \begin{document}

\begin{frame}
    \titlepage
    \vfill
    \begin{center}
        \ccbyncsaeu\\
        {\tiny This work is licensed under the \em{Creative Commons Attribution-NonCommercial-ShareAlike 3.0 License}.}\\[0.5ex]
         \tiny Amateurfunkgruppe der Technische Universität Berlin (AfuTUB), DKØTU
         %\includegraphics[scale=0.5]{img/DK0TU_Logo.pdf}
    \end{center}
\end{frame}


\section{Einleitung}

\begin{frame}
  \frametitle{Einleitung: Wiederholung Rufzeichen/Landeskenner}

  \begin{center}
    \Large{Wie erkennt man eine Station in einem Gastland?}
  \end{center}

\end{frame}

\begin{frame}
  \frametitle{Einleitung: Generell \ldots}

  \ldots muss für den vorübergehenden Funkbetrieb im Ausland eine so genannte
  \textbf{Gastzulassung} (auch: Gastlizenz) beantragen oder gar eine Prüfung
  abgelegt werden. \\[1em]

  Wie bei den \emph{Gesetzen, Vorschriften und Regelungen} kennengelernt, hat
  der Staat das Vorrecht den Funk in seinem Land zu regulieren.

\end{frame}

\section{CEPT-Empfehlungen}

\begin{frame}
  \frametitle{CEPT-Empfehlungen}

  \textbf{CEPT}, bekannt aus \texttt{BV05} (Gesetze, Vorschriften und Regelungen): \\[1em]

  \begin{center}
    \emph{Conférence Européenne des Administrations des Postes et des
    Télécommunications}
  \end{center}

\end{frame}

\subsection{T/R 61-01}

\begin{frame}
  \frametitle{CEPT ECC\footnote{Electronic Communications Committee (Ausschuss
  der CEPT)} Recommendation T/R 61-01}

  Erlaubt den Betrieb einer Amateurfunkstation im Ausland unter folgenden
  Bedingungen:

  \begin{itemize}
    \item \textbf{beigetreten} sind derzeit 42 europäische und 9
      außereuropäische Staaten
    \item \textbf{vorübergehend}er Aufenthalt im Gastland -- maximal
      \textbf{drei Monate}.
    \item \textbf{Landeskenner} des Gastlandes voranstellen, z.\,B.
      \emph{HB9/DL7BST} -- ggf. CEPT-Klasse beachten
    \item \textbf{Bestimmungen} des Gastlandes beachten! Z.\,B. kein
      \emph{6m}\footnote{\scriptsize\ExternalLink\url{http://www.darc.de/fileadmin/filemounts/referate/ausland/CEPT-Laenderliste_2016.pdf}}
  \end{itemize}

  Genaueres steht in der T/R 61-01.

\end{frame}

\subsection{T/R 61-02 (HAREC)}

\begin{frame}
  \frametitle{CEPT ECC Recommendation T/R 61-02 (HAREC)}

  Längere Aufenthalte erfordern eine Funklizenz (oft mit Prüfung) im
  entsprechenden Land. \\[1em]

  Ausnahme: Anerkennung der Empfehlung \texttt{T/R 61-02}. \\[1em]

  Gilt in 40 europäischen und 7 außereuropäischen Staaten. \\[1em]

  Diese regelt das \textbf{Harmonized Amateur Radio Examination Certificate
  (HAREC)} (= AFu-Zeugnis Klasse A). Es stehen sogar die Prüfungsinhalte drin.

\end{frame}

\subsection{ECC-(05)06}

\begin{frame}
  \frametitle{CEPT ECC Recommendation (05)06}

  Legt \textbf{Novice Licence} (= AFu-Zeugnis Klasse E) fest. \\[2em]

  Wird aktuell in 21 Ländern anerkannt.\\[2em]

  In manchen Ländern ist das Präfix der Novice Class voran zu stellen. In
  Deutschland DO/, in der Schweiz HB3/. Also z.B. HB3/DO1ABC. Genaueres steht
  in der Empfehlung.

\end{frame}

\section{Beispiele}

\begin{frame}
  \frametitle{Beispiele}
  \begin{exampleblock}{Gastland-Präfixe}
    \begin{itemize}
      \item \textbf{F/}DC4LW
        \only<2>{$\rightarrow$ DC4LW zu Besuch in Frankreich}
      \item \textbf{DL/}M\O INN
        \only<2>{$\rightarrow$ M\O INN zu Besuch in Deutschland}
      \item \textbf{OE/}DD6YI
        \only<2>{$\rightarrow$ DD6YI zu Besuch in Österreich}
      \item \textbf{MM/}DK\O CCC
        \only<2>{$\rightarrow$ Clubstation DK\O CCC zu Besuch in Schottland\footnote{nicht erlaubt, da eine Clubstation nur für einen festen Betriebsort in Deutschland ausgestellt wird}}
      \item \textbf{HB3/}DO3THK
        \only<2>{$\rightarrow$ DO3THK mit Klasse E zu Besuch in der Schweiz}
      \item \textbf{HB\O Y/}DO1PYL/am
        \only<2>{$\rightarrow$ DO1PYL mit Klasse E im Überflug über Liechtenstein}
    \end{itemize}
  \end{exampleblock}
\end{frame}

\section{Lernhinweise}

\begin{frame}
  \frametitle{Lernhinweise}

  \begin{itemize}
    \item zeitnah die Moltrecht-Lektion \texttt{B/V 07} durchlesen -- die
      Abschlussübungen sind interessant und aufschlussreich
    \item ``Buzzwords'' merken:
      \begin{itemize}
        \item CEPT
        \item T/R 61-01
        \item T/R 61-02 (HAREC)
        \item ECC (05)06
      \end{itemize}
    \item Aufgaben in Ruhe lesen und nicht verwirren lassen
    \item Ausschlussverfahren
    \item in der Woche vor der Prüfung nochmal alles mit dem Simulator durchklicken
  \end{itemize}

\end{frame}

\section{Referenzen}

\begin{frame}
  \frametitle{Referenzen/Links}

  \footnotesize
  \begin{itemize}
    \item Moltrecht B/V 07: \\
      \url{https://www.darc.de/der-club/referate/ajw/lehrgang-bv/bv07/}
    \item Wikipedia DE: \\
      \url{http://de.wikipedia.org/wiki/CEPT} \\
      \url{http://de.wikipedia.org/wiki/HAREC} \\
      \url{http://de.wikipedia.org/wiki/Amateurfunkzeugnis} \\
      \url{http://de.wikipedia.org/wiki/Regelungen_im_Amateurfunkdienst} \\
    \item DARC CEPT Übersicht (Stand 2016): \\
      \url{http://www.darc.de/fileadmin/filemounts/referate/ausland/CEPT-Laenderliste_2016.pdf}
  \end{itemize}

\end{frame}

% Hier könnte noch eine Kontaktfolie stehen

\end{document}

