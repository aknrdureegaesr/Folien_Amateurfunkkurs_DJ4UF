% Foliensatz: "AFu-Kurs nach DJ4UF" von DK0TU, Amateurfunkgruppe der TU Berlin
% Lizenz: CC BY-NC-SA 3.0 de (http://creativecommons.org/licenses/by-nc-sa/3.0/de/)
% Autoren: 
% (Martin Deutschmann)
% Sebastian Lange <dl7bst@dk0tu.de>
% Lars Weiler <dc4lw@darc.de>

preamble.dk0tu.tex
\subtitle{Technik Klasse E 10: \\
          Dezibel, Dämpfung \& Kabel \\[2em]}
\date{Stand 10.12.2015}
 \begin{document}

\begin{frame}
    \titlepage
    \vfill
    \begin{center}
        \ccbyncsaeu\\
        {\tiny This work is licensed under the \em{Creative Commons Attribution-NonCommercial-ShareAlike 3.0 License}.}\\[0.5ex]
         \tiny Amateurfunkgruppe der Technische Universität Berlin (AfuTUB), DKØTU
         %\includegraphics[scale=0.5]{img/DK0TU_Logo.pdf}
    \end{center}
\end{frame}


%fixme Referenzen/Fußnoten-Systematik vereinheitlichen

\section*{Dämpfung}

\begin{frame}
\frametitle{Dämpfungsfaktor}
\begin{center}
\begin{minipage}{0.3\textwidth}
	\huge{$ D = \cfrac{P_{in}}{P_{out}}$}
\end{minipage}
\begin{minipage}{0.6\textwidth}
\begin{itemize}
	\item Dämpfungsfaktor D gibt Verhältnis zwischen Eingangsleistung $P_{in}$
          und der Ausgangsleistung $P_{out}$.
\end{itemize}	
\end{minipage}
\vspace{1cm}	
\begin{figure}
% FIXME Uebertragungsstrecke von Martin war totaler Quatsch - korrigiert wie im Moltrecht, muss aber neu
\includegraphics[scale=0.35]{e10/ubertragung.png}
\caption{Übertragungswege einer Funkstation}
\end{figure}
\end{center}
\end{frame}


\begin{frame}
\frametitle{Dämpfungsmaß dB}
  \begin{itemize}
    \item Dämpfungsmaß ist die Logarithmierung des Dämpfungsfaktors
    \item Einheit Bel $\rightarrow dB$ 
    \item durch Logarithmusgesetze: Multiplikation $\rightarrow$ Addition
  \end{itemize}
  \vspace{0.5cm}
  \begin{block}{Dämpfungsmaß}
    \Huge{$a = 10 \cdot log(\frac{P_{in}}{P_{out}})$ dB}
  \end{block}
\end{frame}

\begin{frame}
\frametitle{Verstärkung in dB}
  \begin{itemize}
    \item generell wird eher mit Verstärkung und nicht mit Dämpfung gerechnet
    \item Dämpfung ist eine negative Verstärkung
  \end{itemize}
  \vspace{1cm}
  \begin{block}{Verstärkung}
    \Huge{$g = 10 \cdot log(\frac{P_{out}}{P_{in}})$ dB}
  \end{block}
\end{frame}

\begin{frame}
\frametitle{Verstärkung in dB}
\begin{center}
	\begin{Large}
	\begin{tabular}{|c|c|}
		\hline
		dB & $\approx$ Leistungsfaktor \\
		\hline \hline
		0    & 1                 \\ \hline
		1,5  & $\sqrt{2}$ = 1,41 \\ \hline
		2,15 & 1,64              \\ \hline
		3    & 2                 \\ \hline
		6    & 4                 \\ \hline
		10   & 10                \\ \hline
		20   & 100               \\ \hline
	\end{tabular}
	\end{Large}		
	\end{center}
\end{frame}

\begin{frame}
\frametitle{Spannungs\-dämpfungs\-maß}
  \begin{block}{Spannungsdämpfungsmaß}
      \Huge{$a_{U} = 20 \cdot log(\frac{U_{1}}{U_{2}})$ dB}
  \end{block}
\vspace{2cm}
Wird in der A-Technik näher erläutert.
\end{frame}

\begin{frame}
\frametitle{Spannungsfaktor \& Leistungsfaktor}
\begin{center}
\begin{Huge}
\begin{minipage}{0.3\textwidth}
	\begin{tabular}{|c|c|}
		\hline
		\textbf{dB} & $a_{U}$ \\
		\hline \hline
		\alert{6dB}  & \alert{2}  \\ \hline
		12dB & 4  \\ \hline
		\alert{20dB} & \alert{10} \\ \hline
	\end{tabular}
\end{minipage}
\hspace{2cm}
\begin{minipage}{0.3\textwidth}
	\begin{tabular}{|c|c|}
		\hline
		\textbf{dB} & $a_{P}$ \\
		\hline \hline
		\alert{3dB}  & \alert{2}  \\ \hline
		6dB  & 4  \\ \hline
		\alert{10dB} & \alert{10} \\ \hline
	\end{tabular}
\end{minipage}
\end{Huge}
\end{center}
\end{frame}

\section*{S-Stufen}
\begin{frame}
\frametitle{S-Stufen}
\begin{center}
\begin{figure}
\includegraphics[scale=1.2]{e10/S-Meter.jpg}
 \attribcaption{S-Meter}{Cqdx}{https://commons.wikimedia.org/wiki/File:S-Meter.jpg}{\ccbysa}
\end{figure}
\begin{itemize}
	\item eine S-Stufe entspricht $6 dB \rightarrow$ Faktor?
	\item Wird beim RST-System verwendet
	\item gibt einen bestimmten Spannungswert an einem $50\Omega$ Widerstand an
	\item Kurzwelle: S9 = $50\mu V$ bei $50\Omega$
	\item UKW: S9 = $5\mu V$ bei $50\Omega$
\end{itemize}
\end{center}
\end{frame}

\section{Pegel}

\begin{frame}
    \frametitle{Pegel}

    \begin{itemize}
        \item Pegel ist auf einen Grundwert $P_0$ bezogen
        \item Grundwert wird auch Normal oder Nullwert genannt
    \end{itemize}

    \begin{block}{Leistungspegel}
        \centering \Large $L_P = 10 lg \frac{P}{P_0} dB$\alert{x}
    \end{block}

\begin{center}
\begin{minipage}{0.3\textwidth}
	\begin{tabular}{|c|c|}
		\hline
		$dBm$ & $P_0 = 1 mW$ \\
		\hline \hline
		20dBm  & 100mW  \\ \hline
		10dBm  & 10mW   \\ \hline
		0dBm   & 1mW    \\ \hline
		-10dBm & 0,1mW  \\ \hline
	\end{tabular}
\end{minipage}
\hspace{2cm}
\begin{minipage}{0.3\textwidth}
	\begin{tabular}{|c|c|}
		\hline
		$dBW$ & $P_0 = 1 W$ \\
		\hline \hline
		20dBW  & 100W  \\ \hline
		10dBW  & 10W   \\ \hline
		0dBW   & 1W    \\ \hline
		-10dBW & 0,1W  \\ \hline
	\end{tabular}
\end{minipage}
\vspace{0.5cm}
\end{center}
\end{frame}


\section*{Leiter}
\begin{frame}
\frametitle{Hochfrequenzleitungen}
%FIXME Erläuterungen der Bilder fehlen
\begin{center}
  \begin{columns}
    \column{.55\textwidth}
      \begin{center}
	\begin{figure}
	\includegraphics[width=1\textwidth,height=.5\textwidth,keepaspectratio]{e10/parallel.png}
	       \attribcaption{Paralleldrahtleitung}{SpinningSpark, Inductiveload, Wdwd}
                        {https://commons.wikimedia.org/wiki/File:Twin-lead_cable_dimension.svg}{\ccbysa}
      \end{figure}
      \end{center}
    \column{.4\textwidth}
      \begin{center}
	\begin{figure}
	\includegraphics[width=1\textwidth,height=.5\textwidth,keepaspectratio]{e10/coax.png}
	 \attribcaption{Koaxialkabel}{Tkgd2007, Fleshgrinder}
                  {https://commons.wikimedia.org/wiki/File:Coaxial_cable_cutaway_new.svg}{\ccby}
      \end{figure}
      \end{center}
  \end{columns}
\end{center}
\end{frame}

\begin{frame}
\frametitle{Hochfrequenzleitungen}
\begin{center}
\begin{figure}
\includegraphics[width=1\textwidth,height=.7\textheight,keepaspectratio]{e10/hohl.jpg}
  \attribcaption{Hohlleiter}{Averse}{https://commons.wikimedia.org/wiki/File:Elli_holl.jpg}{\ccbysa}
\end{figure}
%\end{minipage}
\end{center}
\end{frame}

\subsection*{Wellen\-widerstand}
\begin{frame}
\frametitle{Wellenwiderstand}
\begin{figure}
\includegraphics[width=1\textwidth,height=.3\textheight,keepaspectratio]{e10/wellenesb.png}
\caption{Ersatzschaltbild}
\end{figure}
%\vspace{1cm}\\
\begin{figure}
\includegraphics[width=1\textwidth,height=.3\textheight,keepaspectratio]{e10/wellenesbex.png}
\caption{Genaues Ersatzschaltbild eines Koxialkabels}
\end{figure}
\end{frame}

\begin{frame}
\frametitle{Wellenwiderstand}
\begin{itemize}
	\item \Huge{ $Z_W = \sqrt{\frac{L'}{C'}}$}
	 \normalsize \item Paralleldrahtleitungen: $Z_W = 150 \Omega \cdots 600 \Omega$
	\item Koaxialleitungen: $Z_W =  50 \Omega \cdots 95 \Omega$ -- verbreitet:
    \begin{itemize}
        \item $50 \Omega$ !
        \item $60 \Omega$
        \item $75 \Omega$
    \end{itemize}
	\item Wellenwiderstand entspricht dem Abschlusswiderstand einer Leitung, bei
          dem keine stehenden Wellen auftreten $\rightarrow SWR$
\end{itemize}
\end{frame}

\begin{frame}
\frametitle{Dämpfungsberechnung}
\begin{Large}
Formelsammlung, Diagramm auf der letzten Seite.
\end{Large}
\end{frame}

\begin{frame}
\frametitle{Gemeinsame Rechnung (optional)}
\begin{exampleblock}{Übung 1}
  \begin{description}
 	\item[Kabel] RG58
 	\item[Länge] 10 m
 	\item[Frequenz] 145 MHz
  \end{description}
\end{exampleblock}
\end{frame}

\begin{frame}
\frametitle{Gemeinsame Rechnung (optional)}
\begin{exampleblock}{Übung 2}
  \begin{description}
 	\item[Kabel] Aircell7
 	\item[Länge] 40 m
 	\item[Frequenz] 29 MHz
  \end{description}
\end{exampleblock}
\end{frame}

\begin{frame}
\frametitle{Gemeinsame Rechnung (optional)}
\begin{exampleblock}{Übung 3}
  \begin{description}
 	\item[Kabel] RG174
 	\item[Länge] 5 m
 	\item[Frequenz] 1296 MHz
  \end{description}
\end{exampleblock}
\end{frame}


\subsection*{Anpassung}
\begin{frame}
\frametitle{Anpassung}
\begin{center}
\begin{figure}
\includegraphics[width=1\textwidth,height=.7\textheight,keepaspectratio]{e10/Anpassung.png}
\attribcaption{Anpassung}{Frank Murmann}
                    {https://commons.wikimedia.org/wiki/File:EingangswiderstandAusgangswiderstandA.svg}{\ccpd}
\end{figure}
\end{center}
\end{frame}

\begin{frame}
\frametitle{Stehwellenverhältnis}
\begin{itemize}
	\item ist ein Maß für die Anpassung
	\item ist das Verhältnis von vorlaufender zu zurücklaufender Welle
\end{itemize}
\begin{block}{}
  \centering $SWR = \cfrac{u_{vorlaufend}+u_{ruecklaufend}}{u_{vorlaufend}-u_{ruecklaufend}}$ 
\end{block}
\begin{center}
  \begin{figure}
  %\animategraphics[loop,width=\textwidth,height=0.3\textheight,keepaspectratio,controls]{10}{e10/Stehwelle-}{0}{80}
  \includegraphics[width=\textwidth,height=0.3\textheight,keepaspectratio]{e10/Stehwelle-10.png}
  \footnote{Animation: Stehwelle.gif\cite{stehwelle}}
   \attribcaption{\centering Stehwelle mit SWR 4 \newline
           {\footnotesize (blau: vorlaufende Welle, grün: reflektierte Welle,
           rot: Überlagerung)}}{Pyrometer}
           {https://commons.wikimedia.org/wiki/File:Standing_wave_SWR_4_(forward,_reflected)_open.gif}{\cczero}
  \end{figure}
\end{center}
\end{frame}

\begin{frame}
\frametitle{Symmetrierung}
\begin{minipage}{0.3\textwidth}
  \begin{figure}
	\includegraphics[width=1\textwidth,height=.8\textheight,keepaspectratio]{e10/balun.png}
	 \attribcaption{Balun}{Wolfmankurd}{https://commons.wikimedia.org/wiki/File:Cdbalun2.svg}{\ccbysa}
      \end{figure}

\end{minipage}
\begin{minipage}{0.6\textwidth}
	\begin{itemize}
		\item Wird bei Verbindungen zwischen symmetrischen und unsymmetrischen Punkten verwendet
		\item Koaxialkabel ist unsymmetrisch
		\item Paralleldraht ist symmetrisch
		\item Alle Dipole sind symmetrisch
		\item Alle Antennen die gegen Erde erregt werden sind unsymmetrisch
	\end{itemize}
\end{minipage}
\end{frame}

\section*{Stecker und Adapter}
\begin{frame}
\frametitle{Stecker und Adapter}
\begin{center}
  \begin{figure}
	\includegraphics[width=.6\textwidth,height=.6\textheight,keepaspectratio]{e10/pl.jpg}
	\attribcaption{UHF- oder PL-Stecker}{Appaloosa}
                    {https://commons.wikimedia.org/wiki/File:UHF_PL_Connector.jpg}{\ccbysa}
      \end{figure}
	\begin{itemize}
		\item UHF im Namen, aber ungeeignet dafür
		\item Kurzwelle, auch 2-Meter-Band
        \item ``geschirmte Banane''
	\end{itemize}
\end{center}
\end{frame}

\begin{frame}
\frametitle{Stecker und Adapter}
\begin{center}
  \begin{figure}
	\includegraphics[width=.6\textwidth,height=.6\textheight,keepaspectratio]{e10/n.jpg}
	 \attribcaption{N-Stecker}{Swift.Hg}
                    {https://commons.wikimedia.org/wiki/File:Male_type_N_connector.jpg}{\ccbysa}
      \end{figure}
	\begin{itemize}
		\item N
		\item HF \dots 70-cm und höher
	\end{itemize}	 
\end{center}
\end{frame}

\begin{frame}
\frametitle{Stecker und Adapter}
\begin{center}
  \begin{figure}
	\includegraphics[width=.6\textwidth,height=.7\textheight,keepaspectratio]{e10/bnc.jpg}
	    \attribcaption{BNC-Stecker und -Buchse}{Kaback}
                    {https://commons.wikimedia.org/wiki/File:BNC_50_75_Ohm.jpg}{\ccbysa}
      \end{figure}
	\begin{itemize}
		\item BNC
		\item HF \dots 70-cm und höher
	\end{itemize}	

\end{center}
\end{frame}

\begin{frame}
\frametitle{Stecker und Adapter}
\begin{center}
  \begin{figure}
	\includegraphics[width=.6\textwidth,height=.6\textheight,keepaspectratio]{e10/sma.jpg}
	      \attribcaption{RP-SMA-Stecker}{Peter Trieb}
                    {https://commons.wikimedia.org/wiki/File:BNC_50_75_Ohm.jpg}{\ccpd}
      \end{figure}
	\begin{itemize}
		\item SMA
		\item VHF-/UHF-Handfunkgeräte
	\end{itemize}	 
\end{center}
\end{frame}

\section*{Referenzen}
\begin{frame}
    \frametitle{Referenzen/Links}
    
    \footnotesize
    \begin{itemize}
        \item Moltrecht E 08 : \\
              \url{http://www.darc.de/referate/ajw/ausbildung/darc-online-lehrgang/technik-klasse-e/technik-e08/}      
    \end{itemize}

\end{frame}

\section*{Referenzen}
\begin{frame}
  \frametitle{Referenzen/Links}

    \footnotesize
    \begin{itemize}
      \item Moltrecht E 10: \\
        \url{https://www.darc.de/der-club/referate/ajw/lehrgang-te/e10/}
    \end{itemize}

\end{frame}

% Hier könnte noch eine Kontaktfolie stehen

\end{document}

