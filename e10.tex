% Foliensatz: "AFu-Kurs nach DJ4UF" von DK0TU, Amateurfunkgruppe der TU Berlin
% Lizenz: CC BY-NC-SA 3.0 de (http://creativecommons.org/licenses/by-nc-sa/3.0/de/)
% Autoren: Martin Deutschmann

preamble.dk0tu.tex
\subtitle{Technik Klasse E 10: \\
          Dezibel, D\"ampfung \& Kabel \\[2em]}
\date{Stand 01.12.2014}
 \begin{document}

\begin{frame}
    \titlepage
    \vfill
    \begin{center}
        \ccbyncsaeu\\
        {\tiny This work is licensed under the \em{Creative Commons Attribution-NonCommercial-ShareAlike 3.0 License}.}\\[0.5ex]
         \tiny Amateurfunkgruppe der Technische Universität Berlin (AfuTUB), DKØTU
         %\includegraphics[scale=0.5]{img/DK0TU_Logo.pdf}
    \end{center}
\end{frame}


%fixme Referenzen/Fußnoten-Systematik vereinheitlichen

\section*{D\"ampfungsfaktor}
\begin{frame}
\frametitle{D\"ampfungsfaktor}
\begin{center}
\begin{minipage}{0.3\textwidth}
	\huge{$ D = \frac{P_{1}}{P_{2}}$}
\end{minipage}
\begin{minipage}{0.6\textwidth}
\begin{itemize}
	\item  D\"ampfungsfaktor D gibt Verh\"altnis zwischen Eingangsleistung $P_{1}$ und der Ausgangsleistung $P_{2}$.
\end{itemize}	
\end{minipage}
\vspace{1cm}	
\includegraphics[scale=0.35]{e10/ubertragung.png}\\
Abb. 1: Übertragungswege einer Funkstation
\end{center}
\end{frame}



\section*{D\"ampfungsma{\ss} dB}
\begin{frame}
\frametitle{D\"ampfungsma{\ss} dB}
\begin{center}
	\begin{itemize}
		\item D\"ampfungsma{\ss} ist die logarithmierung des D\"ampfungsfaktor
		\item logarithmische Werte kann man addieren
	\end{itemize}
	\vspace{0.5cm}
	\fbox{
		\Huge{$a_{P} = 10 \cdot log(\frac{P_{1}}{P_{2}})$ dB}
	}
\end{center}
\end{frame}

\section{Verst\"arkung in dB}
\begin{frame}
\frametitle{Verst\"arkung in dB}
\begin{itemize}
	\item im Amateurfunk wird mit Verstärkung und nicht mit D\"ampfung gerechnet
	\item D\"ampfung ist eine negative Verst\"arkung
\end{itemize}
\vspace{1cm}
	\fbox{
		\Huge{$g = 10 \cdot log(\frac{P_{2}}{P_{1}})$ dB}
	}
\end{frame}

\begin{frame}
\frametitle{Verst\"arkung in dB}
\begin{center}
	\begin{Large}
	\begin{tabular}{|c|c|}
		\hline
		dB & Leistungsfaktor \\
		\hline \hline
		0    & 1                 \\ \hline
		1,5  & $\sqrt{2}$ = 1,41 \\ \hline
		2,15 & 1,64              \\ \hline
		3    & 2                 \\ \hline
		6    & 4                 \\ \hline
		10   & 10                \\ \hline
		20   & 100               \\ \hline
	\end{tabular}
	\end{Large}		
	\end{center}
\end{frame}

\section*{Spannungsd\"ampfungsma{\ss}}
\begin{frame}
\frametitle{Spannungsd\"ampunfgsma{\ss}}
\fbox{
		\Huge{$a_{U} = 20 \cdot log(\frac{U_{1}}{U_{2}})$ dB}
	}
\vspace{2cm}
\begin{itemize}
	\item Wird in der A-Technik n\"aher erl\"autert.
\end{itemize}
\end{frame}

\begin{frame}
\frametitle{Spannungsfaktor \& Leistungsfaktor}
\begin{center}
\begin{Huge}
\begin{minipage}{0.3\textwidth}
	\begin{tabular}{|c|c|}
		\hline
		dB & $a_{U}$ \\
		\hline \hline
		6dB  & 2  \\ \hline
		12dB & 4  \\ \hline
		20dB & 10 \\ \hline
	\end{tabular}
\end{minipage}
\hspace{2cm}
\begin{minipage}{0.3\textwidth}
	\begin{tabular}{|c|c|}
		\hline
		dB & $a_{P}$ \\
		\hline \hline
		3dB  & 2  \\ \hline
		6dB  & 4  \\ \hline
		10dB & 10 \\ \hline
	\end{tabular}
\end{minipage}
\end{Huge}
\end{center}
\end{frame}

\section*{S-Stufen}
\begin{frame}
\frametitle{S-Stufen}
\begin{center}
\includegraphics[scale=1.2]{e10/S-Meter.jpg}\\
Abb. 2: S-Meter
\footnote{\url{http://commons.wikimedia.org/wiki/File:S-Meter.jpg}}
\begin{itemize}
	\item eine S-Stufe entspricht 6 dB
	\item Wird beim RST-System verwendet
	\item gibt einen bestimmten Spannungswert an einem 50$\Omega$ Widerstand an
	\item Kurzwelle: S9 = 50$\mu$V bei 50$\Omega$
	\item UKW: S9 = 5$\mu$V bei 50$\Omega$
\end{itemize}
\end{center}
\end{frame}

\section*{Pegel}
\begin{frame}
\frametitle{Pegel}
\begin{center}
\begin{large}
\begin{minipage}{0.3\textwidth}
	\begin{tabular}{|c|c|}
		\hline
		dBm & mW \\
		\hline \hline
		20dBm  & 100mW  \\ \hline
		10dBm  & 10mW   \\ \hline
		0dBm   & 1mW    \\ \hline
		-10dBm & 0,1mW  \\ \hline
	\end{tabular}
\end{minipage}
\hspace{2cm}
\begin{minipage}{0.3\textwidth}
	\begin{tabular}{|c|c|}
		\hline
		dBpW & mW \\
		\hline \hline
		20dBpW  & 100pW  \\ \hline
		10dBpW  & 10pW   \\ \hline
		0dBpW   & 1pW    \\ \hline
		-10dBpW & 0,1pW  \\ \hline
	\end{tabular}
\end{minipage}
\vspace{0.5cm}
\end{large}
\begin{itemize}
	\item Pegel ist auf einen bestimmten Grundwert bezogen
	\item Grundwert wird auch Normal oder Nullwert genannt
\end{itemize}
\end{center}
\end{frame}


\section*{Hochfrequenzleitung}
\begin{frame}
\frametitle{Hochfrequenzleitungen}
\begin{center}
%\begin{minipage}{0.4\textwidth}
\includegraphics[scale=0.25]{e10/parallel.png}\\
\footnote{\url{http://de.wikipedia.org/wiki/Datei:Twin-lead_cable_dimension.svg}}
Abb. 3: Paralleldrahtleitung
%\end{minipage}\\
\\
%\begin{minipage}{0.4\textwidth}
\includegraphics[scale=0.4]{e10/coax.png}\\
\footnote{\url{http://commons.wikimedia.org/wiki/File:Coaxial_cable_cutaway_new.svg}}
Abb. 4: Koaxialkabel
%\end{minipage}
\end{center}
\end{frame}

\section*{Hochfrequenzleitung}
\begin{frame}
\frametitle{Hochfrequenzleitungen}
\begin{center}
\includegraphics[scale=0.4]{e10/hohl.jpg}\\
\footnote{\url{http://de.wikipedia.org/wiki/Datei:Elli_holl.jpg}}
Abb. 5: Hohlleiter
%\end{minipage}
\end{center}
\end{frame}

\section*{Wellenwiderstand}
\begin{frame}
\frametitle{Wellenwiderstand}
\includegraphics[scale=0.8]{e10/wellenesb.png}\\
Abb. 6: ESB 
\vspace{1cm}\\
\includegraphics[scale=0.65]{e10/wellenesbex.png}\\
Abb. 7: Genaues Ersatzschaltbild eines Koxialkabels
\end{frame}

\begin{frame}
\frametitle{Wellenwiderstand}
\begin{itemize}
	\item \Huge{ $Z_W = \sqrt{\frac{L'}{C'}}$}
	 \normalsize \item Paralleldrahtleitungen: Zw = 150 $\Omega$ bis 600 $\Omega$
	\item Koaxialleitungen: Zw =  50 $\Omega$ bis 95 $\Omega$
	\item Der Wellenwiderstand entspricht dem Abschlusswiderstand einer Leitung, bei dem keine stehenden Wellen auftreten.
\end{itemize}
\end{frame}

\section*{D\"ampfungsberechnung}
\begin{frame}
\frametitle{D\"ampfungsberechnung}
\begin{Large}
Ich hoffe ihr habt eure Formelsammlung dabei C:
\end{Large}
\end{frame}

\begin{frame}
\frametitle{Let's calculate together}
 \begin{itemize}
 	\item RG58\\
 	\item 40 m\\
 	\item 28 MHz
 \end{itemize}
\end{frame}

\begin{frame}
\frametitle{Let's calculate together}
 \begin{itemize}
 	\item Aircell7\\
 	\item 40 m\\
 	\item 28 MHz
 \end{itemize}
\end{frame}

\begin{frame}
\frametitle{Let's calculate together}
 \begin{itemize}
 	\item RG174\\
 	\item 40 m\\
 	\item 28 MHz
 \end{itemize}
\end{frame}


\section*{Anpassung}
\begin{frame}
\frametitle{Anpassung}
\includegraphics[scale=0.35]{e10/Anpassung.png}\\
Abb. 8: Anpassung
\footnote{\url{http://de.wikipedia.org/wiki/Datei:EingangswiderstandAusgangswiderstandA.svg}}
\end{frame}

\section*{Stehwellenverh\"altnis}
\begin{frame}
\frametitle{Stehwellenverh\"altnis}
\begin{Large}
\begin{itemize}
	\item ist ein Maß für die Anpassung
	\item \Huge $SWR = \frac{u_{v}+u_{r}}{u_{v}-u_{r}}$ \Large
	\item ist das Verhältnis von vorlaufender zu zurücklaufender Welle
	\item \href{http://commons.wikimedia.org/wiki/File:Stehwelle_(Animation).gif}{Animierte Darstellung}
\end{itemize}
\end{Large}
\end{frame}

\section*{Symmetrierung}
\begin{frame}
\frametitle{Symmetrierung}
\begin{minipage}{0.3\textwidth}
	\includegraphics[scale=0.5]{e10/balun.png}\\
	Abb. 9: Balun

\end{minipage}
	\footnote{\url{http://commons.wikimedia.org/wiki/File:Cdbalun2.svg}}
\begin{minipage}{0.6\textwidth}
	\begin{itemize}
		\item Wird bei Verbindungen zwischen symmetrischen und unsymmetrischen Punkten verwendet
		\item Koaxialkabel ist unsymmetrisch
		\item Paralleldraht ist symmetrisch
		\item Alle Dipole sind symmetrisch
		\item Alle Antennen die gegen Erde erregt werden sind unsymmetrisch
	\end{itemize}
\end{minipage}
\end{frame}

\section*{Stecker und Adapter}
\begin{frame}
\frametitle{Stecker und Adapter}
\begin{center}
	\includegraphics[scale=0.25]{e10/pl.jpg}\\
	Abb. 10: PL-Stecker
	\begin{itemize}
		\item UHF
		\item Kurzwelle,auch 2-Meter-Band
	\end{itemize}
	\footnote{\url{http://de.wikipedia.org/wiki/Datei:UHF_PL_Connector.jpg}}
\end{center}
\end{frame}

\begin{frame}
\frametitle{Stecker und Adapter}
\begin{center}
	\includegraphics[scale=0.3]{e10/n.jpg}\\
	Abb. 11: N-Stecker
	\begin{itemize}
		\item N
		\item 70-cm-Band, auch 2-Meter-Band
	\end{itemize}	 
	\footnote{\url{http://de.wikipedia.org/wiki/	Datei:Male_type_N_connector.jpg}}
\end{center}
\end{frame}

\begin{frame}
\frametitle{Stecker und Adapter}
\begin{center}
	\includegraphics[scale=0.5]{e10/bnc.jpg}\\
	Abb. 12: BNC-Stecker und -Buchse
	\begin{itemize}
		\item BNC
		\item Kurzwelle,auch 2-Meter-Band
	\end{itemize}	
	\footnote{\url{http://commons.wikimedia.org/wiki/File:BNC_50_75_Ohm.jpg}}

\end{center}
\end{frame}

\begin{frame}
\frametitle{Stecker und Adapter}
\begin{center}
	\includegraphics[scale=0.25]{e10/sma.jpg}\\
	Abb. 13: SMA-Stecker
	\begin{itemize}
		\item SMA
		\item VHF-/UHF- Handfunkgeräte
	\end{itemize}	 
	\footnote{\url{http://de.wikipedia.org/wiki/Datei:Reverse_sma_stecker.jpg}}
\end{center}
\end{frame}

\section*{Referenzen}
\begin{frame}
    \frametitle{Referenzen/Links}
    
    \footnotesize
    \begin{itemize}
        \item Moltrecht E 08 : \\
              \url{http://www.darc.de/referate/ajw/ausbildung/darc-online-lehrgang/technik-klasse-e/technik-e08/}      
    \end{itemize}

\end{frame}

% Hier könnte noch eine Kontaktfolie stehen

\end{document}

