% Foliensatz: "AFu-Kurs nach DJ4UF" von DK0TU, Amateurfunkgruppe der TU Berlin
% Lizenz: CC BY-NC-SA 3.0 de (http://creativecommons.org/licenses/by-nc-sa/3.0/de/)
% Autoren: Sebastian Lange <dl7bst@dk0tu.de>
% Korrekturen: Lars Weiler <dc4lw@darc.de>

preamble.dk0tu.tex
\subtitle{Betriebstechnik/Vorschriften 05:  \\
  Gesetze, Vorschriften, Regelungen \\[2em]}
\date{Stand 18.09.2017}
 \begin{document}

\begin{frame}
    \titlepage
    \vfill
    \begin{center}
        \ccbyncsaeu\\
        {\tiny This work is licensed under the \em{Creative Commons Attribution-NonCommercial-ShareAlike 3.0 License}.}\\[0.5ex]
         \tiny Amateurfunkgruppe der Technische Universität Berlin (AfuTUB), DKØTU
         %\includegraphics[scale=0.5]{img/DK0TU_Logo.pdf}
    \end{center}
\end{frame}


\section{Einleitung}

\begin{frame}
  \frametitle{Einleitung / Disclaimer}

  \begin{center}
    \Large{Das Kapitel ist stark zusammengekürzt, da dieser Kurs eher
    praktischer Natur sein soll.}
  \end{center}

  \normalsize

  \begin{itemize}
    \item Es werden nur Vorschriften behandelt, die in Prüfungsfragen eine Rolle spielen.
    \item Zur Vollständigkeit lest bitte selbstständig die Lektion \texttt{B/V 05} im
      \emph{Moltrecht}!
  \end{itemize}

\end{frame}


\begin{frame}
  \frametitle{Einleitung / Vorschriften}

  \begin{center}
    \Large{Wozu brauchen Amateure Regelungen?} \\
    \Large{Welche kennt ihr bereits und wie unterscheiden sie sich? ($\Rightarrow$~Tafel)}
  \end{center}

\end{frame}

\begin{frame}
  \frametitle{Gesetze, Verordnungen, Internationale Vereinbarungen}

  \textbf{Gesetze}: Allgemein verbindliche Rechtsnormen eines Staates durch
  das Parlament (Legislative) \\[1em]

  \textbf{Verordnungen}: Rechtsnormen, die durch ein Regierungs- oder
  Verwaltungsorgan (Exekutive) erlassen werden \\[1em]

  \textbf{Internationale Vereinbarungen}: Aufgrund alleiniger Souveränität der
  Staaten notwendig um internationale Regeln aufzustellen. Wichtige Vereinbarungen
  werden als Gesetz oder Verordnung umgesetzt.

\end{frame}


\section{RR}

\begin{frame}
  \frametitle{Radio Regulations (RR)}

  Radio Regulations (\textbf{RR}), deutsch: Vollzugsordnung für den Funkdienst
  (\textbf{VO Funk}), \emph{int. Vereinbarung}

  \begin{itemize}
    \item regelt international im Rahmen des Völkerrechts Funkdienste und
      die Funkfrequenznutzung
    \item gehört zu den Grundsatzdokumenten der ITU
    \item reguliert elektromagnetisches Spektrum von $9kHz$ bis $275GHz$
  \end{itemize}

\end{frame}

\section{AFuG}

\begin{frame}
  \frametitle{Amateurfunkgesetz (AFuG)}

  Das Amateurfunkgesetz (\textbf{AFuG}) regelt Voraussetzungen und Bedingungen
  für die Teilnahme am Amateurfunkdienst. Umsetzung der \emph{RR} auf
  staatlicher Ebene.

  \vspace{2em}
  \emph{Fun Fact}: In Kraft seit 23.03.1949, also vor dem Grundgesetz
  (23.05.1949). Zu dem Zeitpunkt einziges Gesetz welches das Fernmeldewesen
  regelte. Siehe auch Stichwort \emph{Backsteinaktion}.

\end{frame}

\begin{frame}
  \frametitle{AFuG in Kürze}

  \begin{itemize}
    \item Begriffsbestimmungen: Funkamateur, Amateurfunkdienst,
      Amateurfunkstelle.
    \item wichtigste Regelungen:
      \begin{itemize}
        \item Wer
        \item Rufzeichen
        \item Frequenznutzungsplan
        \item nicht-gewerblich
        \item Notfunk
        \item EMV-Verfahren
        \item Gebühren/Bußgelder
      \end{itemize}
  \end{itemize}

  Allgemeine Regelungen des \emph{AFuG} werden durch die \emph{AFuV} ergänzt.

\end{frame}

\section{AFuV}

\begin{frame}
  \frametitle{AFuV}

  Amateurfunkverordnung (\textbf{AFuV}), genauer \emph{Verordnung zum Gesetz
  über den Amateurfunk} regelt Details zum Amateurfunkgesetz
  \footnote{Verantwortlich: Deutsche Bundespost $\rightarrow$ Deutsche
  Bundespost Telekom $\rightarrow$ Deutsche Telekom AG $\rightarrow$
  Bundesamt für Post und Telekommunikation $\rightarrow$ Regulierungsbehörde
  für Telekommunikation und Post (RegTP) $\rightarrow$
  \textbf{Bundesnetzagentur (BNetzA)}}, z.B.:

  \begin{itemize}
    \item Lizenzklassen
    \item Prüfungen
    \item zugelassene Betriebsarten
  \end{itemize}

\end{frame}

\section{TKG}

\begin{frame}
  \frametitle{TKG}

  Telekommunikationsgesetz (\textbf{TKG}) heute Bundesgesetz zur Regelung des
  Wettbewerbes nach Ende des staatlichen Monopols. \\[1em]

  Gesetzesinhalte, die den Amateurfunk betreffen:

  \begin{itemize}
    \item Anmeldepflicht (Frequenzzuteilung)
    \item Abhören von Nachrichten (Fernmeldegeheimnis)
    \item Ordnungswidrigkeiten
  \end{itemize}

\end{frame}

\section{FTEG}

\begin{frame}
  \frametitle{FTEG}

  Gesetz über Funkanlagen und Telekommunikationsendeinrichtungen
  (\textbf{FTEG}): \\[2em]

  \begin{quote}
    [\ldots] Regelungen über das Inverkehrbringen, den freien Verkehr
    und die Inbetriebnahme von Funkanlagen und
    Telekommunikationsendeinrichtungen [\ldots]
  \end{quote}

  \begin{itemize}
    \item gilt für \textbf{alle} serienmäßig hergestellte Fernmeldegeräte
    \item CE-Kennzeichnung
    \item Begleitpapiere für bestimmungsgemäßen Betrieb
  \end{itemize}

\end{frame}

\section{Lernhinweise}

\begin{frame}
  \frametitle{Lernhinweise}

  \begin{itemize}
    \item zeitnah die Moltrecht-Lektion \texttt{B/V 05} durcharbeiten
    \item bei allen Vorschriften gilt: Konzentriert euch nur auf die
      richtigen Antworten -- sucht euch Stichworte und achtet auf
      Formulierungen
    \item nicht fertig machen lassen und in der Woche vor der Prüfung
      nochmal alles mit dem Simulator durchklicken
  \end{itemize}

\end{frame}

\section{Referenzen}

\begin{frame}
  \frametitle{Referenzen/Links}

  \footnotesize
  \begin{itemize}
    \item Moltrecht B/V 05: \\
      \url{https://www.darc.de/der-club/referate/ajw/lehrgang-bv/bv05/}
    \item Wikipedia DE: \\
      \url{http://de.wikipedia.org/wiki/VO\_Funk} \\
      \url{http://de.wikipedia.org/wiki/Amateurfunkgesetz}
      \url{http://de.wikipedia.org/wiki/Amateurfunkverordnung}
      \url{http://de.wikipedia.org/wiki/Telekommunikationsgesetz}
    \item \emph{Gesetze im Internet} des \emph{BMJV}
      \url{http://www.gesetze-im-internet.de/fteg/}
  \end{itemize}

\end{frame}

% Hier könnte noch eine Kontaktfolie stehen

\end{document}

