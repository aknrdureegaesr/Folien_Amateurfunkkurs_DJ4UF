% Foliensatz: "AFu-Kurs nach DJ4UF" von DK0TU, Amateurfunkgruppe der TU Berlin
% Lizenz: CC BY-NC-SA 3.0 de (http://creativecommons.org/licenses/by-nc-sa/3.0/de/)
% Autoren: Martin Deutschmann <martin.deutschmann@campus.tu-berlin.de>, Lars Weiler <dc4lw@darc.de>
% Überarbeitung: Hendrik Boerma

preamble.dk0tu.tex
\subtitle{Technik Klasse E 03 \\
          Ohmsches Gesetz, Leistung \& Arbeit \\[2em]}
\date{Stand 27.10.2016}
 \begin{document}

\begin{frame}
    \titlepage
    \vfill
    \begin{center}
        \ccbyncsaeu\\
        {\tiny This work is licensed under the \em{Creative Commons Attribution-NonCommercial-ShareAlike 3.0 License}.}\\[0.5ex]
         \tiny Amateurfunkgruppe der Technische Universität Berlin (AfuTUB), DKØTU
         %\includegraphics[scale=0.5]{img/DK0TU_Logo.pdf}
    \end{center}
\end{frame}


%fixme Referenzen, Link funktioniert auch nicht

\section{Das ohmsche Gesetz}

\begin{frame}
  \frametitle{Was ist das ohmsche Gesetz?}
  \begin{itemize}
    \item Das ohmsche Gesetz ist folgendes: $U = I \cdot R$
  \end{itemize}
  \begin{center}
    \begin{figure}
      \includegraphics[width=\textwidth,height=.7\textheight,keepaspectratio]{e03/Ohm_law_triangle.png}
      \attribcaption{Das ohmsche Dreieck}{Eirik}{https://commons.wikimedia.org/wiki/File:Ohm's_law_triangle.PNG}{\ccpd}
    \end{figure}
 	\end{center}
 	\begin{itemize}
 		\item	Aber was bedeutet das nun für uns?
 	\end{itemize}
\end{frame}

\begin{frame}
	\frametitle{Das ohmsche Gesetz}
	\begin{itemize}
		\item	Das ohmsche Gesetz gibt die Abhängigkeiten zwischen Spannung, Strom \& ohmschen Widerstand an.
		\item	Dadurch wissen wir, dass sich zum Beispiel der Strom an einem konstanten Widerstand proportional zur Spannung ändert.
	\end{itemize}
\end{frame}

\begin{frame}
  \frametitle{Ein kleines Gedankenexperiment}
  \begin{itemize}
    \item Wir stellen uns folgenden Aufbau vor:
    \item Der Widerstand der Lampe soll 1 $\Omega$ betragen
  \end{itemize}
  \begin{columns}
    \column{0.35\textwidth}
    \begin{center}
      \begin{figure}
        \includegraphics[width=\textwidth,height=.6\textheight,keepaspectratio]{e03/Strom_Spannung_Abh.png}
        \caption{Helligkeit einer Lampe bei verschiedenen Spannungen}
      \end{figure}
    \end{center}
    \column{0.6\textwidth}
    \begin{exampleblock}{Aufgabe}
      Wird die Lampe heller oder dunkler, wenn man sie mit $1,5 V$ anstatt mit $4,5 V$ betreibt?
    \end{exampleblock}
  \end{columns}
\end{frame}

\begin{frame}
	\frametitle{wie funktioniert das Ohmsche Dreieck}
	\begin{itemize}
		\item	Nun wollen wir wissen, wie viel Strom in beiden Fällen unseres Gedankenexperimentes fließt
		\item	Dazu nehmen wir uns das Ohmsche Dreieck zu Hilfe
	\end{itemize}
	\begin{center}
 		   \begin{figure}
    				\includegraphics[width=\textwidth,height=.2\textheight,keepaspectratio]{e03/Ohm_law_triangle.png}
      				\attribcaption{Das ohmsche Dreieck}{Eirik}{https://commons.wikimedia.org/wiki/File:Ohm's_law_triangle.PNG}{\ccpd}
    \end{figure}
 	\end{center}
 	\begin{itemize}
 		\item	Decken wir den Wert, den wir ermitteln wollen ab, dann kann man die Formel aus dem Dreieck ablesen.
 		\item	Somit lautet die Formel für den Strom:
 	\end{itemize}
 	\begin{equation}
 		I = \frac{U}{R}
 		\label{equ:Strom}
 	\end{equation}
\end{frame}

\begin{frame}
	\frametitle{Das Ergebnis unseres Experiments}
	\begin{itemize}
		\item	Setzen wir nun die Werte in diese Formel ein, so erhalten wir:
	\end{itemize}
	\begin{equation}
		I_{4,5 V} = \frac{4,5 V}{1 \Omega} = 4,5 A
	\end{equation}
	\begin{itemize}
		\item Sowie
	\end{itemize}
	\begin{equation}
		I_{1,5 V} = \frac{1,5 V}{1 \Omega} = 1,5 A
	\end{equation}
	\begin{itemize}
		\item Somit fließt bei einer angelegten Spannung von 4,5V der größere Strom durch die Lampe.
	\end{itemize}
\end{frame}

\begin{frame}
	\frametitle{Der Innenwiderstand}
	\begin{itemize}
		\item	Oftmals bemerken wir einen Spannungsabfall, zwischen einer Maschine im Leerlauf und der gleichen Maschine bei Belastung.
		\item	Dies können wir auf den Innenwiderstand der Machine zurückführen.
	\end{itemize}
	\begin{center}
 		\begin{figure}
      			\includegraphics[scale=1.4]{e03/Innenwiderstand.png}
     			 \caption{Innenwiderstand einer Batterie}
   		 \end{figure}
 	\end{center}
\end{frame}

\begin{frame}
	\frametitle{Der Innenwiderstand}
	\begin{itemize}
		\item	Um den Innenwiderstand zu ermitteln, nutzen wir wieder das ohmsche Gesetz
		\item	Dabei gilt es zu beachten, dass diesmal die Differenzen der Spannungen und des Stromes zwischen dem Leerlauf und dem belasteten Fall verrechnet werden
		\item	Es gilt:
	\end{itemize}
	\begin{equation}
		R_{innen} = \frac{\Delta U}{\Delta I}
	\end{equation}
	\begin{itemize}
		\item Um den Wert nicht zu sehr zu verfälschen sollten \textbf{Spannungsquellen einen niedrigen} und \textbf{Stromquellen einen hohen Innenwiderstand} besitzen
	\end{itemize}
\end{frame}

\begin{frame}
	\begin{small}	
	\begin{tabular}{|l|l|l|}
	\hline
		\multicolumn{3}{|c|}{\textbf{TD302:} Die Leerlaufspannung einer Gleichspannungsquelle}\\
		\multicolumn{3}{|c|}{beträgt 13,5 V. Wenn die Spannungsquelle einen Strom von 1 A}\\
		\multicolumn{3}{|c|}{abgibt, sinkt die Klemmenspannung auf 12,4 V. Wie groß ist der}\\			\multicolumn{3}{|c|}{Innenwiderstand der Spannungsquelle?}\\
		\hline
		A & $1,1 \Omega$ & ??? \\ \hline
		B & $1,2 \Omega$ & ??? \\ \hline
		C & $12,4 \Omega$ & ??? \\ \hline
		D & $13,5 \Omega$ & ??? \\ \hline 		
	\end{tabular}
	\end{small}
\end{frame}

\begin{frame}
	\begin{small}	
	\begin{tabular}{|l|l|l|}
	\hline
		\multicolumn{3}{|c|}{\textbf{TD302:} Die Leerlaufspannung einer Gleichspannungsquelle}\\
		\multicolumn{3}{|c|}{beträgt 13,5 V. Wenn die Spannungsquelle einen Strom von 1 A}\\
		\multicolumn{3}{|c|}{abgibt, sinkt die Klemmenspannung auf 12,4 V. Wie groß ist der}\\			\multicolumn{3}{|c|}{Innenwiderstand der Spannungsquelle?}\\
		\hline
		A & $1,1 \Omega$ & richtig \\ \hline
		B & $1,2 \Omega$ & " "\\ \hline
		C & $12,4 \Omega$ & " "\\ \hline
		D & $13,5 \Omega$ & " " \\ \hline 		
	\end{tabular}
	\end{small}
	\vspace{1cm}
	\begin{equation}
		R = \frac{\Delta U}{\Delta I} = \frac{13,5V - 12,4V}{1A} = \frac{1,1V}{1A} = 1,1\Omega
	\end{equation}
\end{frame}

\begin{frame}
	\begin{small}	
	\begin{tabular}{|l|l|l|}
	\hline
		\multicolumn{3}{|c|}{\textbf{TD301:} Welche Eigenschaften sollten Strom- und}\\
		\multicolumn{3}{|c|}{Spannungsquellen aufweisen?}\\
		\hline
		A &  Strom- und Spannungsquellen sollten einen möglichst  & ??? \\
		" " & niedrigen Innenwiderstand haben. & " "\\ \hline
		B &  Strom- und Spannungsquellen sollten einen möglichst  & ??? \\
		" " & hohen Innenwiderstand haben. & " "\\ \hline
		C & Spannungsquellen sollten einen möglichst hohen & ??? \\
		" " & Innenwiderstand und Stromquellen einen möglichst  & " " \\ 
		" " & niedrigen Innenwiderstand haben.  & "" \\ \hline
		D & Spannungsquellen sollten einen möglichst niedrigen & ??? \\
		" " & Innenwiderstand und Stromquellen einen möglichst & " " \\ 
		" " & hohen Innenwiderstand haben.  & " " \\ \hline 		
	\end{tabular}
	\end{small}
\end{frame}

\begin{frame}
	\begin{small}	
	\begin{tabular}{|l|l|l|}
	\hline
		\multicolumn{3}{|c|}{\textbf{TD301:} Welche Eigenschaften sollten Strom- und}\\
		\multicolumn{3}{|c|}{Spannungsquellen aufweisen?}\\
		\hline
		A &  Strom- und Spannungsquellen sollten einen möglichst  & " " \\
		" " & niedrigen Innenwiderstand haben. & " "\\ \hline
		B &  Strom- und Spannungsquellen sollten einen möglichst  & " " \\
		" " & hohen Innenwiderstand haben. & " "\\ \hline
		C & Spannungsquellen sollten einen möglichst hohen & " " \\
		" " & Innenwiderstand und Stromquellen einen möglichst  & " " \\ 
		" " & niedrigen Innenwiderstand haben.  & "" \\ \hline
		D & Spannungsquellen sollten einen möglichst niedrigen & richtig \\
		" " & Innenwiderstand und Stromquellen einen möglichst & " " \\ 
		" " & hohen Innenwiderstand haben.  & " " \\ \hline 		
	\end{tabular}
	\end{small}
\end{frame}

\section{Die elektrische Leistung}
\begin{frame}
	\frametitle{Die elektrische Leistung}
	\begin{itemize}
		\item	Fließt ein Strom durch einen Widerstand, so entsteht Verlustleistung in Form von Wärme an dem Widerstand
		\item	Dies findet Anwendung bei Heizspiralen in Elektroherd, oder im Bügeleisen
		\item	In einer elektronischen Schaltung sollte diese Wärmeleistung aber so gering wie möglich sein
		\item	Um die Leistung zu ermitteln nutzen wir folgende Formel:
	\end{itemize}
	\begin{equation}
		P = U \cdot I
	\end{equation}
	\begin{itemize}
		\item	Die Einheit der Leistung ist das Watt $[W]$
	\end{itemize}
	\begin{equation}
		 1[W] = 1[V] \cdot 1[A]
	\end{equation}
\end{frame}

\begin{frame}
  \begin{columns}
    \column{.43\textwidth}
      \begin{center}
	 \begin{figure}
        \includegraphics[width=\textwidth,height=.8\textheight,keepaspectratio]{e03/electricity.jpg}\\
        \attribcaption{Electricity explained the Escher way}{swocket}{https://twitter.com/swocket/status/646649716869046272}{}
      \end{figure}
      \end{center}
    \column{.57\textwidth}
      \pause
      \begin{exampleblock}{Widerstand}
	Was passiert bei Änderung von $\ell$ oder $a$? Was bei $\rho$\footnote{\tiny spezifischer elektrischer Widerstand}?
      \end{exampleblock}
      \pause
      \begin{exampleblock}{Strom}
	Was passiert bei Änderung von $C$ oder $s$?
      \end{exampleblock}
      \pause
      \begin{exampleblock}{Spannung}
	Was passiert bei Änderung von $A$ oder $\Omega$?
      \end{exampleblock}
      \pause
      \begin{exampleblock}{Leistung}
	Was passiert bei Änderung von $A$ oder $V$?
      \end{exampleblock}
  \end{columns}
\end{frame}

\begin{frame}
	\begin{small}	
	\begin{tabular}{|l|l|l|}
	\hline
		\multicolumn{3}{|c|}{\textbf{TB906:} Eine Glühlampe hat einen Nennwert von 12 V }\\
		\multicolumn{3}{|c|}{und 48 W. Bei einer 12-V-Versorgung beträgt die Stromentnahme}\\
		\hline
		A & $36 A $ & ??? \\ \hline
		B & $250 mA $ & ??? \\ \hline
		C & $750 mA $ & ??? \\ \hline
		D & $4 A $ & ??? \\ \hline	
	\end{tabular}
	\end{small}
\end{frame}

\begin{frame}
	\begin{small}	
	\begin{tabular}{|l|l|l|}
	\hline
		\multicolumn{3}{|c|}{\textbf{TB906:} Eine Glühlampe hat einen Nennwert von 12 V }\\
		\multicolumn{3}{|c|}{und 48 W. Bei einer 12-V-Versorgung beträgt die Stromentnahme}\\
		\hline
		A & $36 A $ & " " \\ \hline
		B & $250 mA $ & " " \\ \hline
		C & $750 mA $ & " " \\ \hline
		D & $4 A $ & richtig \\ \hline	
	\end{tabular}
	\end{small}
	\vspace{1cm}
	\begin{equation}
		I = \frac{P}{U} = \frac{48W}{12V} = 4A
	\end{equation}
\end{frame}

\section{Die elektrische Arbeit}

\begin{frame}
	\frametitle{Die elektrische Arbeit}
	\begin{itemize}
		\item	Die elektrische Arbeit ist definiert als die Leistung, die in einer bestimmten Zeit erbracht wurde
		\item	Es gilt also:
	\end{itemize}
	\begin{equation}
		W = P \cdot t
	\end{equation}
	\begin{itemize}
		\item	Wir können die Leistung auch als das Produkt von Strom und Spannung aufschreiben und erhalten dann:
	\end{itemize}
	\begin{equation}
		W = U \cdot I \cdot t
	\end{equation}
	\begin{itemize}
		\item	Die Einheit der elektrischen Arbeit ist die Wattsekunde $[Ws]$
	\end{itemize}
\end{frame}

\begin{frame}
	\begin{small}	
	\begin{tabular}{|l|l|l|}
	\hline
		\multicolumn{3}{|c|}{\textbf{TB905:} Eine Stromversorgung nimmt bei 230 Volt einen Strom }\\
		\multicolumn{3}{|c|}{von 0,63 Ampere auf. Welche elektrische Arbeit  wird }\\
		\multicolumn{3}{|c|}{bei einer Betriebsdauer von 7 Stunden verbraucht?}\\
		\hline
		A & $1,01 kWh$ & ??? \\ \hline
		B & $0,1 kWh $ & ??? \\ \hline
		C & $2,56 kWh$ & ??? \\ \hline
		D & $20,7 kWh$ & ??? \\ \hline	
	\end{tabular}
	\end{small}
\end{frame}

\begin{frame}
	\begin{small}	
	\begin{tabular}{|l|l|l|}
	\hline
		\multicolumn{3}{|c|}{\textbf{TB905:} Eine Stromversorgung nimmt bei 230 Volt einen Strom }\\
		\multicolumn{3}{|c|}{von 0,63 Ampere auf. Welche elektrische Arbeit  wird }\\
		\multicolumn{3}{|c|}{bei einer Betriebsdauer von 7 Stunden verbraucht?}\\
		\hline
		A & $1,01 kWh$ & richtig \\ \hline
		B & $0,1 kWh $ & " " \\ \hline
		C & $2,56 kWh$ & " " \\ \hline
		D & $20,7 kWh$ & " " \\ \hline	
	\end{tabular}
	\end{small}
	\vspace{1cm}
	\begin{equation}
		W = P \cdot t = U \cdot I \cdot t = 230V \cdot 0,63A \cdot 7h = 1,01kWh
	\end{equation}
\end{frame}

\begin{frame}
  \begin{alertblock}{Hausaufgabe}
    Aus Fragenkatalog Klasse E Kapitel 1.2.9 ``Ohmsches Gesetz, Leistung und Energie'' (TB901--TB911) durcharbeiten.
  \end{alertblock}
\end{frame}

\renewcommand{\refname}{Referenzen}

\hypertarget{refs}{}
\textcolor{white}{} \\ %\vspace{} geht nicht
\Large Referenzen/Links
\footnotesize

\begin{thebibliography}{}
    \bibitem{e03}   Moltrecht E 03: \\
                    \url{http://www.darc.de/referate/ajw/ausbildung/darc-online-lehrgang/technik-klasse-e/technik-e03/}
    \bibitem{wp}    Wikipedia DE: \\
                    \url{http://de.wikipedia.org/wiki/Ohmsches_Gesetz}\\ 
                    \url{http://de.wikipedia.org/wiki/Elektrische_Leistung}\\ 
                    \url{http://de.wikipedia.org/wiki/lektrische_Energie#Elektrische_Energie_in_einem_elektrischen_Feld}\\ 
    \bibitem{wmde}	Wikimedia DE:\\
    				\url{http://commons.wikimedia.org/wiki/File:Ohmsches_Gesetz.svg?uselang=de}\\
   	\bibitem{wmen}	Wikimedia EN:\\
   					\url{http://commons.wikimedia.org/wiki/File:Ohm\%27s_law_triangle.PNG}\\
\end{thebibliography} 

% Hier könnte noch eine Kontaktfolie stehen

\end{document}

