% Foliensatz: "AFu-Kurs nach DJ4UF" von DK0TU, Amateurfunkgruppe der TU Berlin
% Lizenz: CC BY-NC-SA 3.0 de (http://creativecommons.org/licenses/by-nc-sa/3.0/de/)
% Autoren: Martin Deutschmann

preamble.dk0tu.tex
\subtitle{Technik Klasse E 05: \\
          Der Kondensator und seine Schaltungsarten \\[2em]}
\date{Stand 5.11.2014}
 \begin{document}

\begin{frame}
    \titlepage
    \vfill
    \begin{center}
        \ccbyncsaeu\\
        {\tiny This work is licensed under the \em{Creative Commons Attribution-NonCommercial-ShareAlike 3.0 License}.}\\[0.5ex]
         \tiny Amateurfunkgruppe der Technische Universität Berlin (AfuTUB), DKØTU
         %\includegraphics[scale=0.5]{img/DK0TU_Logo.pdf}
    \end{center}
\end{frame}


\section*{Einleitung}

\begin{frame}
    \frametitle{Einleitung / Kondensator}
    \begin{center}
        \Large{Wie sieht er aus?}\\
        \Large{Was tut der?}         
    \end{center}
\end{frame}


\begin{frame}
    \frametitle{Einleitung / Kondensator}

    \begin{center}
        \includegraphics[width=0.8\textwidth]{e05/Kondensator01.jpg}
        \footnote{Abb.1: Verschiedene Kondensatoren \cite{wmen}}
    \end{center}
 	

\end{frame}

\begin{frame}
    \frametitle{Einleitung / Kondensator}

    \begin{center}
        \includegraphics[width=0.8\textwidth]{e05/Kondensator02.jpg}
        \footnote{Abb.2: Verschiedene kleine Kondensatoren \cite{wmen}}
    \end{center}
 	

\end{frame}


\section*{Kapazität}

\begin{frame}
    \frametitle{Kapazität}
	
	\begin{center}
        \includegraphics[width=0.8\textwidth]{e05/c-aufbau.png}
        \footnote{Abb.3: Interner Aufbau eines Plattenkondensators \cite{wp}}
    \end{center}
 	

\end{frame}

\begin{frame}
	\frametitle{ein paar Formeln}
	\begin{center}
		\huge{$Q = C \cdot U$}\\
		\vspace{2cm}
		\huge{$C= \frac{\varepsilon_{0} \cdot \varepsilon_{r} \cdot A}{d}$}
	\end{center}
   
   
\end{frame}

\section*{Parallelschaltung}

\begin{frame}
    \frametitle{Parallelschaltung von Kondensatoren}
	
	\begin{center}
        \includegraphics[width=0.8\textwidth]{e05/c-parallel.png}
        \footnote{\tiny By Martin Deutschmann mit Eagle}
    \end{center}
 	\huge{$C_{ges} = C_{1} + C_{2} + C_{3} + C_{4} + C_{5}$} 

\end{frame}
	
\begin{frame}
	\begin{center}
			\huge{TC206: Drei Kondensatoren mit den Kapazitäten $C1 = 0,1 \mu F, C2 = 150 nF$ und $C3 = 50000 pF$ werden parallel geschaltet. Wie groß ist die Gesamtkapazität?}
	\end{center}
\end{frame}

\section*{Reihenschaltung}

\begin{frame}
    \frametitle{Reihenschaltung von Kondensatoren}
    \begin{center}
		\huge{$\frac{1}{C_{ges}} = \frac{1}{ C_{1}} + \frac{1}{C_{2}} + \frac{1}{C_{3}} + \frac{1}{C_{4}} + \frac{1}{C_{5}}$}\\
		\vspace{1cm}
		\huge{$C_{ges} = \frac{1}{\frac{1}{ C_1} + \frac{1}{C_2} + \frac{1}{C_3} + \frac{1}{C_4} + \frac{1}{C_5}}$}\\
		\vspace{1cm}	
        \includegraphics[width=0.8\textwidth]{e05/c-reihe.png}
        \footnote{\tiny By Martin Deutschmann mit Eagle}
    \end{center}
	
    
\end{frame}
\begin{frame}
	\begin{center}
		\huge{Zwei Kondensatoren von 100 pF und 150 pF sind hintereinander (in Serie) geschaltet. Berechnen Sie die Gesamtkapazität.}
	\end{center}
\end{frame}

\section*{Widerstand eines Kondensators}
\begin{frame}
    \frametitle{Widerstand eines Kondensators}
	
	\begin{center}
     \huge{$X_c = \frac{1}{2 \cdot \pi \cdot f \cdot C}$}
    \end{center}

\end{frame}

\begin{frame}
	\begin{small}
	
	
	
	\begin{tabular}{|l|l|l|}
	\hline
		\multicolumn{3}{|c|}{\textbf{TC208: Mit zunehmender Frequenz...}}\\
		\hline
		A & steigt der Wechselstromwiderstand des Kondensators. & ??? \\ \hline
		B & sinkt der Wechselstromwiderstand des Kondensators. & ??? \\ \hline
		C & steigt der Wechselstromwiderstand des Kondensators  & ??? \\ 
		" " & bis zu einem Maximum und sinkt dann wieder. & " " \\ \hline
		D & sinkt der Wechselstromwiderstand des Kondensators & ??? \\
		" " & bis zu einem Minimum und steigt dann wieder. & " " \\ \hline 	
	
	\end{tabular}
	\end{small}
\end{frame}

\begin{frame}
	\begin{small}
	
	
	
	\begin{tabular}{|l|l|l|}
	\hline
		\multicolumn{3}{|c|}{\textbf{TC208: Mit zunehmender Frequenz...}}\\
		\hline
		A & steigt der Wechselstromwiderstand des Kondensators. & ??? \\ \hline
		B & sinkt der Wechselstromwiderstand des Kondensators. & Right \\ \hline
		C & steigt der Wechselstromwiderstand des Kondensators  & ??? \\ 
		" " & bis zu einem Maximum und sinkt dann wieder. & " " \\ \hline
		D & sinkt der Wechselstromwiderstand des Kondensators & ??? \\
		" " & bis zu einem Minimum und steigt dann wieder. & " " \\ \hline 	
	
	\end{tabular}
	\end{small}
\end{frame}

\renewcommand{\refname}{Referenzen}

\hypertarget{refs}{}
\textcolor{white}{} \\ %\vspace{} geht nicht
\Large Referenzen/Links
\footnotesize

\begin{thebibliography}{}
    \bibitem{e03}   Moltrecht E 05: \\
                    \url{http://www.darc.de/referate/ajw/ausbildung/darc-online-lehrgang/technik-klasse-e/technik-e05/}
    \bibitem{wp}    Wikipedia DE: \\
                    \url{http://de.wikipedia.org/wiki/Kondensator_(Elektrotechnik)}\\ 
    \bibitem{wmde}	Wikimedia DE:\\
    				\url{http://de.wikipedia.org/wiki/Datei:Plate_Capacitor_DE.svg}\\
   	\bibitem{wmen}	Wikimedia EN:\\
   					\url{http://commons.wikimedia.org/wiki/File:Capacitors_(7189597135).jpg}\\
   					\url{http://commons.wikimedia.org/wiki/File:Condensators.JPG}
\end{thebibliography} 

% Hier könnte noch eine Kontaktfolie stehen

\end{document}

