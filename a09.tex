% Foliensatz: "AFu-Kurs nach DJ4UF" von DK0TU, Amateurfunkgruppe der TU Berlin
% Lizenz: CC BY-NC-SA 3.0 de (http://creativecommons.org/licenses/by-nc-sa/3.0/de/)
% Autoren: Felix Baum DB4UM <baum@campus.tu-berlin.de>
% Korrekturen: Lars Weiler <dc4lw@darc.de>

preamble.dk0tu.tex
\subtitle{Technik A09: \\
  Antennentechnik  \\[2em]}
\date{Stand 06.02.2017}
 \begin{document}

\begin{frame}
    \titlepage
    \vfill
    \begin{center}
        \ccbyncsaeu\\
        {\tiny This work is licensed under the \em{Creative Commons Attribution-NonCommercial-ShareAlike 3.0 License}.}\\[0.5ex]
         \tiny Amateurfunkgruppe der Technische Universität Berlin (AfuTUB), DKØTU
         %\includegraphics[scale=0.5]{img/DK0TU_Logo.pdf}
    \end{center}
\end{frame}


\section*{Wiederholung}

\begin{frame}
  \frametitle{Antenne}
  \begin{center}
    \begin{figure}
      \includegraphics[width=.6\textwidth,height=.75\textheight,keepaspectratio]{e11/Traqueur_acquisition.JPG}
      \attribcaption{Satellite tracking-aquisition antenna}{Kingbastard}{https://commons.wikimedia.org/wiki/File:Traqueur_acquisition.JPG}{\ccbysa}
    \end{figure}
  \end{center}
\end{frame}

\begin{frame}
  \frametitle{Strom und Spannungsverteilung}
  \begin{center}
    \begin{figure}
      \includegraphics[width=0.8\textwidth,height=.75\textheight,keepaspectratio]{a09/Felder_um_Dipol.png}
      \attribcaption{Felder um Dipol}{Averse}{https://commons.wikimedia.org/wiki/File:Felder_um_Dipol.jpg}{\ccbysa}
    \end{figure}
  \end{center}
\end{frame}

\begin{frame}
  \frametitle{Stromverteilung Dipol}
  \begin{center}
    \Large Wie verteilen sich Strom und Spannung auf einem Dipol? \\ (Bäuche und Knoten)
  \end{center}
\end{frame}

\begin{frame}
  \frametitle{Strom und Spannungsverteilung}
  \begin{center}
    \begin{figure}
      \includegraphics[width=0.8\textwidth,height=.75\textheight,keepaspectratio]{a09/DipolUI.png}
      \attribcaption{Strom- (rot) und Spannungsverlauf (blau) entlang der Stäbe eines Halbwellendipols}{Averse in Zusammenarbeit mit Ulfbastel}{https://commons.wikimedia.org/wiki/File:Lineare_antennen.svg}{\ccbysa}
    \end{figure}
  \end{center}
\end{frame}

\begin{frame}
  \frametitle{Verschiedene Frequenzen}
  \begin{center}
    \begin{figure}
      \includegraphics[width=0.8\textwidth,height=.75\textheight,keepaspectratio]{a09/TH108.png}
      \attribcaption{TH108}{BNetzA}{https://www.bundesnetzagentur.de/amateurfunk/}{}
    \end{figure}
    Die Wellen zeigen die Stromverteilung am Dipol. \\Welche Frequenz ist welcher Buchstabe?
  \end{center}
\end{frame}


\section*{Antennen"=impedanz}

\begin{frame}
  \frametitle{Antennenimpedanz Dipol}
  \begin{center}
    \begin{itemize}
      \item Dipol hat je nach Höhe über dem Boden unterschiedliche Impedanz \\[1em]
      \item Lambda-Halbe Dipol hat eine Impedanz von ca 60 - 75 Ohm \\[1em]
      \item Beim Lambda-Halbe Dipol rein reeller Widerstand \\[1em]
      \item Welche Impedanz hat ein Lambda-Halbe-Dipol unterhalb und oberhalb seiner Grundfrequenz? \\[1em]
    \end{itemize}
  \end{center}
\end{frame}

\begin{frame}
  \frametitle{Komplexe Eingangsimpedanzen}
  \begin{center}
    \begin{itemize}
      \item Welche Impedanz hat ein Lambda-Halbe-Dipol unterhalb und oberhalb seiner Grundfrequenz? \\[1em]
      \item Unterhalb der Grundfrequenz ist die Impedanz kapazitiv \\[1em]
      \item Oberhalb der Grundfrequenz ist die Impedanz induktiv \\[1em]
      \item Dipol ist wie Schwingkreis
    \end{itemize}
  \end{center}
\end{frame}

\begin{frame}
  \frametitle{Antenne verlängern und kürzen}
  \begin{columns}[c]
    \column[c]{5cm}
    \begin{center}
      \begin{figure}
        \includegraphics[width=0.8\textwidth,height=.75\textheight,keepaspectratio]{a09/Verkurzte_10m_Antenne.jpg}
        \attribcaption{verkürzte 10m Antenne}{DB4UM}{}{}
      \end{figure}
    \end{center}
    \column{5cm} \large
    \begin{center}
      \begin{itemize}
        \item Mechanisch zu kurze Antennen können durch Induktivitäten verlängert werden.
        \item Zu lange Antennen können durch Kapazitäten elektrisch verkürzt werden.
      \end{itemize}
    \end{center}
  \end{columns}
\end{frame}

\begin{frame}
  \frametitle{Realer Dipol}
  \begin{center}
    \begin{itemize}
      \item Da Antennendraht nicht unendlich dünn ist und immer kapazitive Kopplung zum Erdboden besteht, verstimmt sich die Antenne zu niedrigeren Frequenzen
      \item Beim Bau einer realen Antenne mit Verkürzungsfaktoren so um $0.90-0.98$ rechnen.
    \end{itemize}
  \end{center}
\end{frame}

\section*{Richtdiagramm}

\begin{frame}
  \frametitle{Richtdiagramme erkennen}
  \begin{center}
    \begin{figure}
      \includegraphics[width=0.95\textwidth,height=.75\textheight,keepaspectratio]{a09/Abstrahl.png}
      \attribcaption{Welches Richtdiagramm ist welche Antenne?}{BNetzA}{https://www.bundesnetzagentur.de/amateurfunk/}{}
    \end{figure}
  \end{center}
\end{frame}

\begin{frame}
  \frametitle{Halbwertsbreite}
  \begin{center}
    \begin{figure}
      \includegraphics[width=0.7\textwidth,height=.75\textheight,keepaspectratio]{a09/TH213.png}
      \attribcaption{TH213}{BNetzA}{https://www.bundesnetzagentur.de/amateurfunk/}{}
    \end{figure}
    \large Die Halbwertsbreite einer Antenne ist der Winkelbereich, innerhalb dem
    die Feldstärke auf nicht weniger als den 0,707-fachen Wert der maximalen Feldstärke absinkt.\\[1em] Wie viel Grad Öffnungswinkel hat dieser Richtstrahler?
  \end{center}
\end{frame}

\section*{Gewinn}

\begin{frame}
  \frametitle{Gewinn einer Antenne}
  \begin{itemize}
    \item Zum Messen werden Testaussendungen im Labor gemacht
    \item Vergleich oft mit Dipol
    \item \href{https://www.youtube.com/watch?v=gBqqp7rnZ64}{\ExternalLink Youtube-Video}\\[2em]
  \end{itemize}
  \begin{block}{Gewinn}
    \begin{center}
      $$g_D = 20 \cdot lg(\cfrac{P_{test}}{P_{dipol}}) = g_D[dBd]$$
      $$g_I = g_D + 2.15dB$$
    \end{center}
    Der Gewinn eines Dipols gegenüber dem Kugelstrahler ist immer 2,15 dB.
  \end{block}
\end{frame}

\begin{frame}
  \frametitle{Vor-Rück Verhältnis, dBi, dBd}
  \begin{center}
    \begin{figure}
      \includegraphics[width=0.8\textwidth,height=.75\textheight,keepaspectratio]{a09/TH206.png}
      \attribcaption{Was ist das Vor-Rück Verhältnis, in dBd und dBi?}{BNetzA}{https://www.bundesnetzagentur.de/amateurfunk/}{}
    \end{figure}
  \end{center}
\end{frame}


\section*{ERP, EIRP}

\begin{frame}
  \frametitle{ERP und EIRP}
  \begin{block}{ERP und EIRP}
    \begin{center}
      \Large $$P_{ERP} = g_{D} \cdot P_{sender}$$ \\
      $$P_{EIRP} = g_{I} \cdot P_{sender}$$ \\
      wobei $g_D + 2.15dB = g_I$
    \end{center}
  \end{block}
\end{frame}

\section*{Antennentypen}

\begin{frame}
  \frametitle{Fuchskreis}
  \begin{center}
    \begin{figure}
      \includegraphics[width=1\textwidth,height=.75\textheight,keepaspectratio]{a09/1000px-Endgespeiste_Antenne.png}
      \attribcaption{Dipol mit Anpassschaltung für Koaxkabel}{Herbertweidner}{https://commons.wikimedia.org/wiki/File:Endgespeiste_Antenne.svg}{\ccbysa}
    \end{figure}
  \end{center}
\end{frame}

\section*{Multiband}

\begin{frame}
  \frametitle{Multibanddipol}
  \begin{center}
    \begin{figure}
      \includegraphics[width=0.9\textwidth,height=.75\textheight,keepaspectratio]{a09/Multiband.jpg}
      \caption{Antenne EA-1015204080 von EAntenna}
    \end{figure}
  \end{center}
\end{frame}

\begin{frame}
  \frametitle{G5RV}
  \begin{center}
    \begin{figure}
      \includegraphics[width=0.9\textwidth,height=.75\textheight,keepaspectratio]{a09/G5RV_Antenna.png}
      \attribcaption{G5RV Antenne}{Gerolf Ziegenhain}{https://commons.wikimedia.org/wiki/File:G5RV_Antenna.svg}{\ccbysa}
    \end{figure}
  \end{center}
\end{frame}

\begin{frame}
  \frametitle{W3DZZ}
  \begin{center}
    \begin{figure}
      \includegraphics[width=1\textwidth,height=.75\textheight,keepaspectratio]{a09/W3DZZ.png}
      \attribcaption{TH134}{BNetzA}{https://www.bundesnetzagentur.de/amateurfunk/}{}
    \end{figure}
  \end{center}
\end{frame}

\begin{frame}
  \frametitle{Traps}
  \begin{center}
    \begin{figure}
      \includegraphics[width=1\textwidth,height=.75\textheight,keepaspectratio]{a09/W3DZZ.png}
      \attribcaption{TH134}{BNetzA}{https://www.bundesnetzagentur.de/amateurfunk/}{}
    \end{figure}
    \begin{itemize}
      \item Wenn man diese Mehrband-Antenne auf 3,5\,MHz erregt, dann wirken die LC-Resonanzkreise wie?
      \item Wenn man diese Mehrband-Antenne auf 7,05\,MHz erregt, dann wirken die LC-Resonanzkreise wie?
    \end{itemize}
  \end{center}
\end{frame}

\begin{frame}
  \frametitle{Windom}
  \begin{center}
    \begin{figure}
      \includegraphics[width=1\textwidth,height=.75\textheight,keepaspectratio]{a09/WINDOM_TH130.png}
      \attribcaption{TH130}{BNetzA}{https://www.bundesnetzagentur.de/amateurfunk/}{}
    \end{figure}
  \end{center}
  Eindrahtspeiseleitung mit $500 \Omega$; heutzutage kaum noch verwendet wegen zu vielen Umwelteinflüßen auf die Speiseleitung
\end{frame}

\begin{frame}
  \frametitle{Yagi}
  \begin{center}
    \begin{figure}
      \includegraphics[width=0.5\textwidth,height=.75\textheight,keepaspectratio]{a09/Yagi_3_element.png}
      \attribcaption{3 Element Yagi-Uda Antenne}{Sankeytm}{https://commons.wikimedia.org/wiki/File:Yagi_3_element.svg}{\ccbysa}
    \end{figure}
  \end{center}
\end{frame}

\begin{frame}
  \frametitle{Yagi - Richtung erkennen}
  \begin{center}
    \begin{figure}
      \includegraphics[width=.9\textwidth,height=.75\textheight,keepaspectratio]{a09/yagi.jpg}
      \attribcaption{10m 5-Element-Yagi bei DK\O TU}{DK9GD}{}{}
    \end{figure}
  \end{center}
\end{frame}

\begin{frame}
  \frametitle{Mehrband Beam}
  \begin{center}
    \begin{figure}
      \includegraphics[width=.9\textwidth,height=.6\textheight,keepaspectratio]{a09/P160924-1657050.jpg}
      \attribcaption{Cushcraft A-3S Beam für 10-15-20m mit Erweiterungssatz A-743 (kapazitive und induktive Endglieder) für 30-40m}{DH8GHH}{http://db0smg.afug.uni-goettingen.de/~dh8ghh/bilder/2016/fielddayd23-2016/}{}
    \end{figure}
    \begin{itemize}
      \item Kreuzung aus dem Dipol mit Traps und der Yagi-Antenne
      \item Bereich mit guter Resonanz ist schmaler, aber Antenne kürzer
    \end{itemize}
  \end{center}
\end{frame}

\section*{Vertikal"=antennen}

\begin{frame}
  \frametitle{Groundplane oder Marconi-Antenne}
  \begin{center}
    \begin{figure}
      \includegraphics[width=0.7\textwidth,height=.75\textheight,keepaspectratio]{a09/GP-DB4UM.png}
      \attribcaption{Groundplane-Antenne}{DB4UM mit cocoaNec 2.0}{}{}
    \end{figure}
  \end{center}
\end{frame}

\begin{frame}
  \frametitle{Dipol mit Ground}
  \begin{center}
    \begin{figure}
      \includegraphics[width=0.7\textwidth,height=.75\textheight,keepaspectratio]{a09/A6-3EN.jpg}
      \attribcaption{Viertelwellen Monopol Antenne durch Ground Plane reflektiert}{LP}{https://commons.wikimedia.org/wiki/File:A6-3EN.jpg}{\ccbysa}
    \end{figure}
    Der Boden bewirkt wie eine Spiegelung elektrisch invertierte des Strahlers
  \end{center}
\end{frame}

\begin{frame}
  \frametitle{Spezille Antennen}
  \begin{center}
    \begin{figure}
      \includegraphics[width=1\textwidth,height=.75\textheight,keepaspectratio]{a09/patch.png}
      \attribcaption{Patch Antenne auf Platine}{Urheber unbekannt}{}{}
    \end{figure}
  \end{center}
\end{frame}

\begin{frame}
  \frametitle{Magnetic Loop}
  \begin{center}
    \begin{figure}
      \includegraphics[width=1\textwidth,height=.75\textheight,keepaspectratio]{a09/Magloop.jpg}
      \attribcaption{Magloop bei DK\O TU}{DB4UM}{}{}
    \end{figure}
  \end{center}
\end{frame}

\begin{frame}
  \frametitle{Spiegel Antenne}
  \begin{center}
    \begin{figure}
      \includegraphics[width=0.4\textwidth,height=.75\textheight,keepaspectratio]{a09/Parabolic_antenna_types.png}
      \attribcaption{Verschiedene Arten von Parabolantennen}{Chetvorno}{https://commons.wikimedia.org/wiki/File:Parabolic_antenna_types2.svg}{\ccpd}
    \end{figure}
  \end{center}
\end{frame}


\begin{frame}
  \frametitle{Helix Antenne}
  \begin{center}
    \begin{figure}
      \includegraphics[width=0.6\textwidth,height=.75\textheight,keepaspectratio]{a09/Traqueur_acquisition.JPG}
      \attribcaption{Satellite tracking-aquisition antenna}{Kingbastard}{https://commons.wikimedia.org/wiki/File:Traqueur_acquisition.JPG}{\ccbysa}
    \end{figure}
  \end{center}
\end{frame}

\begin{frame}
  \frametitle{Portabelfunken in Berlin}
  \begin{columns}[c]
    \column[c]{5cm}
    \begin{center}
      \begin{figure}
        \includegraphics[width=0.82\textwidth,height=.75\textheight,keepaspectratio]{a09/db4um_dm1ri_portabel.jpg}
        \attribcaption{C-Pole im Springpfuhl}{DB4UM}{}{}
      \end{figure}
    \end{center}
    \column{5cm} \large
    \begin{center}
      \begin{figure}
        \includegraphics[width=0.82\textwidth,height=.75\textheight,keepaspectratio]{a09/db4um_Drachenberg.jpg}
        \attribcaption{Groundplane auf dem Drachenberg}{DB4UM}{}{}
      \end{figure}
    \end{center}
  \end{columns}
\end{frame}

\begin{frame}
  \frametitle{Kurzwellensender Nauen}
  \begin{center}
    \begin{figure}
      \includegraphics[width=0.93\textwidth,height=.75\textheight,keepaspectratio]{a09/Nauen_Dipolarray.jpg}
      \attribcaption{Kurzwellensenderantennen in Nauen}{DB4UM}{}{}
    \end{figure}
  \end{center}
\end{frame}



\renewcommand{\refname}{Referenzen}

\hypertarget{refs}{}
\textcolor{white}{} \\ %\vspace{} geht nicht
\Large Referenzen/Links
\footnotesize

\begin{thebibliography}{}
  \bibitem{darc}  DARC Online-Lehrgang Lektion A09:
    \url{https://www.darc.de/der-club/referate/ajw/lehrgang-ta/a09/}
  \bibitem{wp}    Wikipedia - Die freie Enzyklopädie:
    \url{https://de.wikipedia.org/wiki/Dipolantenne}
  \bibitem{bna}   Fragenkatalog Bundesnetzargentur Technik Klasse A:
    \url{https://www.bundesnetzagentur.de/SharedDocs/Downloads/DE/Sachgebiete/Telekommunikation/Unternehmen_Institutionen/Frequenzen/Amateurfunk/Fragenkatalog/TechnikFragenkatalogKlasseAf252rId9014pdf.pdf?__blob=publicationFile&v=3}
\end{thebibliography}

% Hier könnte noch eine Kontaktfolie stehen

\end{document}

