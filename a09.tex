% Foliensatz: "AFu-Kurs nach DJ4UF" von DK0TU, Amateurfunkgruppe der TU Berlin
% Lizenz: CC BY-NC-SA 3.0 de (http://creativecommons.org/licenses/by-nc-sa/3.0/de/)
% Autoren: Felix Baum DB4UM <baum@campus.tu-berlin.de>

preamble.dk0tu.tex
\subtitle{Technik A09: \\
           Antennentechnik  \\[2em]}
\date{Stand 01.06.2015}
 \begin{document}

\begin{frame}
    \titlepage
    \vfill
    \begin{center}
        \ccbyncsaeu\\
        {\tiny This work is licensed under the \em{Creative Commons Attribution-NonCommercial-ShareAlike 3.0 License}.}\\[0.5ex]
         \tiny Amateurfunkgruppe der Technische Universität Berlin (AfuTUB), DKØTU
         %\includegraphics[scale=0.5]{img/DK0TU_Logo.pdf}
    \end{center}
\end{frame}


\section*{Wiederholung}

\begin{frame}
    \frametitle{Antenne}
      \begin{center}
        \includegraphics[width=.6\textwidth]{e11/Traqueur_acquisition.JPG}
        \tiny \hyperlink{refs}{\cite{wm}}
    \end{center}
\end{frame}

\begin{frame}
    \frametitle{Strom und Spannungsverteilung}
    \begin{center}
        \includegraphics[width=0.8\textwidth]{a09/Felder_um_Dipol.png}
        \tiny \hyperlink{refs}{\cite{wm}} \\[1em] \large
    \end{center}
\end{frame}

\begin{frame}
    \frametitle{Stromverteilung Dipol}
    \begin{center}
    \Large Wie verteilen sich Strom und Spannung auf einem Dipol? \\ (Bäuche und Knoten)
    \end{center}
\end{frame}

\begin{frame}
    \frametitle{Strom und Spannungsverteilung}
    \begin{center}
        \includegraphics[width=0.8\textwidth]{a09/DipolUI.png}
        \tiny \hyperlink{refs}{\cite{wm}} \\[1em] \large
    \end{center}
\end{frame}

\begin{frame}
    \frametitle{Verschiedene Frequenzen}
        \begin{center}
        \includegraphics[width=0.8\textwidth]{a09/TH108.png} \\
        \tiny \hyperlink{refs}{\cite{bna}} TH108\\[1em] \large
        Die Wellen zeigen die Stromverteilung am Dipol. \\Welche Frequenz ist welcher Buchstabe?
    \end{center}
\end{frame}


\section*{Antennenimpedanz}

\begin{frame}
    \frametitle{Antennenimpedanz Dipol}
    \begin{center}
    	\begin{itemize}
		\item Dipol hat je nach Höhe über dem Boden unterschiedliche Impedanz \\[1em]
        \item Lambda-Halbe Dipol hat eine Impedanz von ca 60 - 75 Ohm \\[1em]
        \item Beim Lambda-Halbe Dipol rein reeller Widerstand \\[1em]
       	\item Welche Impedanz hat ein Lambda-Halbe-Dipol unterhalb und oberhalb seiner Grundfrequenz? \\[1em]
    	\end{itemize}
	\end{center}
\end{frame}

\begin{frame}
    \frametitle{Komplexe Eingangsimpedanzen}
        \begin{center}
    	\begin{itemize}
		\item Welche Impedanz hat ein Lambda-Halbe-Dipol unterhalb und oberhalb seiner Grundfrequenz? \\[1em]
        \item Unterhalb der Grundfrequenz ist die Impedanz kapazitiv \\[1em]
        \item Oberhalb der Grundfrequenz ist die Impedanz induktiv \\[1em]
       	\item Dipol ist wie Schwingkreis
    	\end{itemize}
	\end{center}
\end{frame}

\begin{frame}
    \frametitle{Antenne verlängern und Kürzen}
    \begin{columns}[c]
        \column[c]{5cm}
        \begin{center}
        \includegraphics[width=0.8\textwidth]{a09/Verkurzte_10m_Antenne.jpg}\\
        \tiny Bild von DB4UM
    \end{center}
    \column{5cm} \large
        \begin{center}
        \begin{itemize}
		\item Mechanisch zu kurze Antennen können durch Induktivitäten verlängert werden.
        \item Zu lange Antennen können durch Kapazitäten elektrisch verkürtzt werden.
    	\end{itemize}
    \end{center}
    \end{columns}
\end{frame}

\begin{frame}
    \frametitle{Realer Dipol}
        \begin{center}
    	\begin{itemize}
		\item Da Antennendraht nicht unendlich dünn ist und immer Kapazitive Kopplung zum Erdboden besteht verstimmt sich die Antenne zu niedrigeren Frequenzen
		\item Beim Bau einer Realen Antenne mit Verkürzungsfaktoren so um $0.90-0.98$ rechnen.
    	\end{itemize}
	\end{center}
\end{frame}

\section*{Richtdiagramm}

\begin{frame}
    \frametitle{Richtdiagramme erkennen}
    \begin{center}
        \includegraphics[width=0.95\textwidth]{a09/Abstrahl.png}
        \tiny \hyperlink{refs}{\cite{bna}} \\[1em] \large Welches Richtdiagramm ist welche Antenne?
    \end{center}
\end{frame}

\begin{frame}
    \frametitle{Halbwertsbreite}
        \begin{center}
        \includegraphics[width=0.7\textwidth]{a09/TH213.png}\\
        \tiny \hyperlink{refs}{\cite{bna}} TH213 \\[1em] 
		\large Die Halbwertsbreite einer Antenne ist der Winkelbereich, innerhalb dem 
die Feldstärke auf nicht weniger als den 0,707-fachen Wert der maximalen Feldstärke absinkt.\\[1em] Wie viel Grad Öffnungswinkel hat dieser Richtstrahler?
    \end{center}
\end{frame}

\section*{Gewinn}

\begin{frame}
    \frametitle{Gewinn einer Antenne}
    \begin{center}
		\large Zum Messen Testaussendungen im Labor\\
		Vergleich oft mit Dipol\\
		\url{https://www.youtube.com/watch?v=gBqqp7rnZ64}\\[2em]
		$$g_D = 20 \cdot lg(\frac{P_{test}}{P_{dipol}}) = g_D[dBd]$$	
		$$g_I = g_D + 2.5dB$$	
    \end{center}
\end{frame}

\begin{frame}
    \frametitle{Vor-Rück Verhältnis, dBi, dBd}
    \begin{center}
        \includegraphics[width=0.8\textwidth]{a09/TH206.png}
        \tiny \hyperlink{refs}{\cite{bna}} \\[1em] \large Was ist das Vor-Rück Verhältnis, dBd und dBi ?
    \end{center}
\end{frame}


\section*{ERP, EIRP}

\begin{frame}
    \frametitle{ERP und EIRP}
    \begin{center}
    \Large $$P_{ERP} = g_{D} \cdot P_{sender}$$ \\
    $$P_{EIRP} = g_{I} \cdot P_{sender}$$ \\ 
    wobei $g_D + 2.15dB = g_I$
    \end{center}
\end{frame}

\section*{Antennentypen}

\begin{frame}
    \frametitle{Fuchskreis}
    \begin{center}
        \includegraphics[width=1\textwidth]{a09/1000px-Endgespeiste_Antenne.png}
        \footnote{\tiny \url{https://commons.wikimedia.org/wiki/File:Endgespeiste_Antenne.svg}}
	\end{center}
\end{frame}

\section*{Multiband}

\begin{frame}
    \frametitle{Multibanddipol}
    \begin{center}
        \includegraphics[width=0.9\textwidth]{a09/Multiband.jpg}
        \footnote{\tiny Antenne EA-1015204080 von EAntenna}
	\end{center}
\end{frame}

\begin{frame}
    \frametitle{G5RV}
    \begin{center}
        \includegraphics[width=0.9\textwidth]{a09/G5RV_Antenna.png}
        \footnote{\tiny \url{https://commons.wikimedia.org/wiki/File:G5RV_Antenna.svg}}
	\end{center}
\end{frame}

\begin{frame}
    \frametitle{W3DZZ}
    \begin{center}
        \includegraphics[width=1\textwidth]{a09/W3DZZ.png}
        \tiny \hyperlink{refs}{\cite{bna}} TH134
	\end{center}
\end{frame}

\begin{frame}
    \frametitle{Traps}
    \begin{center}
        \includegraphics[width=1\textwidth]{a09/W3DZZ.png}
        \tiny \hyperlink{refs}{\cite{bna}} TH134 \\[1em] \large Wenn man diese Mehrband-Antenne auf 3,5 MHz erregt, dann wirken die LC-Resonanzkreise wie? \\[2em]
        Wenn man diese Mehrband-Antenne auf 7,05 MHz erregt, dann wirken die LC-Resonanzkreise wie?
	\end{center}
\end{frame}

\begin{frame}
    \frametitle{Windom}
    \begin{center}
        \includegraphics[width=1\textwidth]{a09/WINDOM_TH130.png}
        \tiny \hyperlink{refs}{\cite{bna}} TH134 \\[1em] \large Eindrehtspeiseleitung mit $500 \Omega$ \\[2em] Heutzutage kaum noch verwendet wegen zu vielen Umwelteinflüßen auf die Speiseleitung
	\end{center}
\end{frame}

\begin{frame}
    \frametitle{Yagi}
    \begin{center}
        \includegraphics[width=0.5\textwidth]{a09/Yagi_3_element.png}
        \footnote{\tiny \url{https://commons.wikimedia.org/wiki/File:Yagi_3_element.svg}}
	\end{center}
\end{frame}

\begin{frame}
    \frametitle{Yagi - Richtung erkennen}
    \begin{center}
        \includegraphics[width=.9\textwidth]{a09/yagi.jpg}
        \footnote{\tiny 10M Yagi bei DK0TU von DK9GD}
	\end{center}
\end{frame}

\begin{frame}
    \frametitle{Mehrband Beam}
    \begin{center} \large
        \begin{itemize}
		\item Kreuzung aus dem Dipol mit Traps und der Yagi-Antenne
        \item Bereich mit guter Resonanz ist schmaler, aber Antenne kürzer.
    	\end{itemize}
    \end{center}
\end{frame}

\section*{Vertikalantennen}

\begin{frame}
    \frametitle{Groundplane oder Marconi-Antenne}
    \begin{center}
        \includegraphics[width=0.7\textwidth]{a09/GP-DB4UM.png}
        \footnote{\tiny DB4UM mit cocoaNec 2.0}
	\end{center}
\end{frame}

\begin{frame}
    \frametitle{Dipol mit Ground}
    \begin{center}
        \includegraphics[width=0.7\textwidth]{a09/A6-3EN.jpg}
        \tiny \hyperlink{refs}{\cite{wm}} \\[2em] \large
        Der Boden bewirkt wie eine Spiegelung elektrisch invertierte des Strahlers
	\end{center}
\end{frame}

\begin{frame}
    \frametitle{Spezille Antennen}
    \begin{center}
        \includegraphics[width=1\textwidth]{a09/patch.png}
        
	\end{center}
\end{frame}

\begin{frame}
    \frametitle{Magnetic Loop}
    \begin{center}
        \includegraphics[width=1\textwidth]{a09/Magloop.jpg}
        \footnote{\tiny Magloop bei DK0TU von DB4UM}
	\end{center}
\end{frame}

\begin{frame}
    \frametitle{Spiegel Antenne}
    \begin{center}
        \includegraphics[width=0.4\textwidth]{a09/Parabolic_antenna_types.png}
        \tiny \hyperlink{refs}{\cite{wm}}
	\end{center}
\end{frame}


\begin{frame}
    \frametitle{Helix Antenne}
    \begin{center}
        \includegraphics[width=0.6\textwidth]{a09/Traqueur_acquisition.JPG}
        \tiny \hyperlink{refs}{\cite{wm}}
	\end{center}
\end{frame}

\begin{frame}
    \frametitle{Portabelfunken in Berlin}
    \begin{columns}[c]
        \column[c]{5cm}
        \begin{center}
        \includegraphics[width=0.82\textwidth]{a09/db4um_dm1ri_portabel.jpg}\\
        \tiny C-Pole im Springpfuhl Bild: DB4UM
    \end{center}
    \column{5cm} \large
        \begin{center}
        \includegraphics[width=0.82\textwidth]{a09/db4um_Drachenberg.jpg}\\
        \tiny Groundplane auf dem Drachenberg Bild: DB4UM
    \end{center}
    \end{columns}
\end{frame}

\begin{frame}
    \frametitle{Kurzwellensender Nauen}
    \begin{center}
        \includegraphics[width=0.93\textwidth]{a09/Nauen_Dipolarray.jpg}\\
        \tiny Kurzwellensenderantennen in Nauen Bild: DB4UM
	\end{center}
\end{frame}



\renewcommand{\refname}{Referenzen}

\hypertarget{refs}{}
\textcolor{white}{} \\ %\vspace{} geht nicht
\Large Referenzen/Links
\footnotesize

\begin{thebibliography}{}
    \bibitem{darc}  DARC Online-Lehrgang Lektion A08:
                    \url{https://www.darc.de/der-club/referate/ajw/lehrgang-ta/a09/}
    \bibitem{wm} 	Wikimedia:
                    \url{https://upload.wikimedia.org/wikipedia/commons/3/32/Traqueur_acquisition.JPG}
                    \url{https://commons.wikimedia.org/wiki/File:Felder_um_Dipol.jpg}
                    \url{https://commons.wikimedia.org/wiki/File:Lineare_antennen.svg}
                    \url{https://commons.wikimedia.org/wiki/File:Lineare_antennen.svg}
                    \url{https://commons.wikimedia.org/wiki/File:Dipole_Antenna.svg}
                    \url{https://commons.wikimedia.org/wiki/File:G5RV_Antenna.svg}
                    \url{https://commons.wikimedia.org/wiki/File:Endgespeiste_Antenne.svg}
                    \url{https://commons.wikimedia.org/wiki/File:Yagi_3_element.svg}
                    \url{https://commons.wikimedia.org/wiki/File:FWHM.svg}
                    \url{https://commons.wikimedia.org/wiki/File:Parabolic_antenna_types2.svg}
                    \url{https://commons.wikimedia.org/wiki/File:A6-3EN.jpg}
                    \url{}
    \bibitem{beam}  Beamkarten Generator von NS6T:
                    \url{http://ns6t.net/azimuth/azimuth.html}
    \bibitem{wp}    Wikipedia - Die freie Enzyklopädie:
                    \url{https://de.wikipedia.org/wiki/Dipolantenne}
	\bibitem{bna}   Fragenkatalog Bundesnetzargentur Technik Klasse A:                   
                    \url{https://www.bundesnetzagentur.de/SharedDocs/Downloads/DE/Sachgebiete/Telekommunikation/Unternehmen_Institutionen/Frequenzen/Amateurfunk/Fragenkatalog/TechnikFragenkatalogKlasseAf252rId9014pdf.pdf?__blob=publicationFile&v=3}
\end{thebibliography} 

% Hier könnte noch eine Kontaktfolie stehen

\end{document}

